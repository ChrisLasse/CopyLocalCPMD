\documentclass[twoside,10pt,titlepage,a4paper]{article}
%
\usepackage{amsmath}
\usepackage{amssymb}
\usepackage{makeidx}
\usepackage{multicol}
\sloppy%\nonstopmode
%Version CPMD
\newcommand{\cpmdversion}{4.3.0}
\newcommand{\cpmd}{CPMD \cpmdversion}
\newcommand{\sgn}{\mathrm{sgn}}

\usepackage[bookmarks,bookmarksopen,colorlinks,
            pdftitle={Manual of \cpmd},
            pdfsubject={How to use \cpmd},
            pdfauthor={The CPMD consortium},
            pdfkeywords={\cpmd, ab initio},
            linkcolor={blue},
            citecolor={blue},
            urlcolor={red},
            hyperindex]{hyperref}
%%%%%% MAKE FILE SETUP %%%%%%%%
%  HTML TAG -  DO NOT REMOVE  %
%%%%%%%%%%%%%%%%%%%%%%%%%%%%%%%

\makeindex

% Definition of the page
\setlength{\paperheight}{297mm}
\setlength{\paperwidth}{210mm}
\setlength{\textheight}{230mm}
\setlength{\textwidth}{150mm}
\setlength{\evensidemargin}{-0.4mm}
\setlength{\oddsidemargin}{9.6mm}
\setlength{\topmargin}{-0.4mm}
\setlength{\overfullrule}{2pt}
\setlength{\parindent}{0pt}
\setlength{\parskip}{0pt}
\renewcommand{\baselinestretch}{1.0}

% Optimize output for different media (links, summaries)
% Call pdflatex with pdflatex '\def\faqsum{include} \input manual'
% if you want to get a summary of each FAQ section with links
\newif\ifhtmlmanual \htmlmanualfalse
\newif\ifpdfmanual  \pdfmanualfalse
\newif\iffaqsum     \ifx\undefined\faqsum\faqsumfalse\else\faqsumtrue\fi

\ifx\undefined\pdfoutput
\else
  \ifnum\pdfoutput=1 \pdfmanualtrue\fi
\fi
\ifpdfmanual
  \htmlmanualfalse
\else
  \faqsumtrue
  \htmlmanualtrue
\fi

\ifpdfmanual
  \typeout{This manual is optimized for PDF output}
  \newcommand{\referto}[2]{\hyperlink{#1}{#2}}
  \newcommand{\reflabel}[1]{\hypertarget{#1}}
\else
  \typeout{This manual is optimized for HTML output}
  \faqsumtrue
  \newcommand{\referto}[2]{\htmlref{#2}{#1}}
  \newcommand{\reflabel}[1]{\label{#1}}
\fi

\ifpdfmanual
  \newcommand{\htref}[2]{\href{#1}{#2}}
\fi
\ifhtmlmanual
  \newcommand{\htref}[2]{\htmladdnormallink{#2}{#1}}
\fi

\newcommand{\reffaqquestion}[2]{\referto{faq#1}{#2}}
\newcommand{\faqquestion}[1]{\vspace{2ex}\reflabel{faq#1}{{\bf Q:\ }}}
\newcommand{\faqanswer}{\vspace{1ex}{{\bf A:\ }}}

% Arguments: name contribution_in_index hypertargetname
\newcommand{\keyword}[5]{%
\vspace{1.0cm}
\begin{minipage}{15cm}
%\hypertarget{#1}{\textbf{\large #1}}% Name; printed
\reflabel{#1}{\textbf{\large #1}}% Name; printed
\index{#1}% Index
%\index{Keywords!#1}% Index
\ \textbf{#2}% Optional arguments
\ \textbf{#3}% Exclusive arguments
\ \textit{#4}% Notes
     \hfill\\\smallskip
     {Section: #5}
     \hfill\\\smallskip\vskip 10pt
\end{minipage}
}%
\newcommand{\spekeyword}[6]{%
\vspace{1.0cm}
\begin{minipage}{15cm}
%\hypertarget{#6}{\textbf{\large #1}}% Name; printed
\reflabel{#6}{\textbf{\large #1}}% Name; printed
\index{#6@#1}% Index
%\index{Keywords!#1}% Index
\ \textbf{#2}% Optional arguments
\ \textbf{#3}% Exclusive arguments
\ \textit{#4}% Notes
     \hfill\\\smallskip
     {Section: #5}
     \hfill\\\smallskip\vskip 10pt
\end{minipage}
}%

% Arguments: name hypertargetname
\newcommand{\refkeyword}[1]{%
%\hyperlink{#1}{\textbf{#1}}%
\referto{#1}{\textbf{#1}}%
\index{#1}% Index
%\index{Keywords!#1}% Index
}%
\newcommand{\refspekeyword}[2]{%
%\hyperlink{#2}{\textbf{#1}}%
\referto{#2}{\textbf{#1}}%
\index{#2@#1}% Index
%\index{Keywords!#1}% Index
}%

% Arguments: options options notes
\newcommand{\options}[3]{%
\ \textbf{#1}% Optional arguments
\ \textbf{#2}% Exclusive arguments
\ \textit{#3}% Notes
     \hfill\\\smallskip
}%

% Arguments: text
\ifpdfmanual
   \newcommand{\desc}[1]{%
   \hspace*{\fill} \parbox{130mm}{\sloppy
                          {#1}% all the text
                             }
     \hfill\\\smallskip
   }%
\else
   \newcommand{\desc}[1]{#1\vspace{1ex}}
\fi

\newcommand{\shellcommand}[1]{%
  \vspace*{3mm}
  \noindent
  \texttt{\# #1}
  \vspace*{3mm}
}

\newcommand{\defaultvalue}[1]{%
  \textbf{#1}
}

\begin{document}
%---------------------------------------------------------------------

%---------------------------------------------------------------------
\setcounter{page}{-3}
\pagestyle{empty}
   \rule{0in}{1in}
   \begin{center}
       {\Huge\bf CPMD \\}
       \vspace{5ex}
       {\huge Car-Parrinello Molecular Dynamics \\}
       \vspace{5ex}
       {\Large An {\sl ab initio} Electronic Structure and \\
               \vspace{1ex}
               Molecular Dynamics Program} \\
       \vspace{10ex}
       {\large
          The CPMD consortium                 \smallskip \\
          WWW: \htref{http://www.cpmd.org/}{http://www.cpmd.org/} \smallskip \\
          Mailing list: \htref{mailto:cpmd-list@cpmd.org}{cpmd-list@cpmd.org} \smallskip  \\
          E-mail: \htref{mailto:cpmd@cpmd.org}{cpmd@cpmd.org}  \smallskip  \\ }
       \vspace{5ex}
       {\large \today\\}
       \vspace{10ex}
       Send comments and bug reports to \medskip \\
       \htref{mailto:cpmd@cpmd.org}{cpmd@cpmd.org} \\
       \vfill
       {\bf This manual is for CPMD version \cpmdversion}
       \end{center}

\clearpage

\vfill

\cpmd  \hfill  \today

\clearpage

\tableofcontents

\clearpage

\pagestyle{myheadings}
\markboth{\cpmd}{\cpmd}

% Freedom for page bottom in order to avoid underfull \vbox
\raggedbottom
%---------------------------------------------------------------------
\part{Overview}
\section{About this manual}\label{about}

  Many members of the CPMD consortium
(\htref{http://www.cpmd.org/}{http://www.cpmd.org/}) contributed to this manual.
This version of the manual is based on a compilation by Barbara Kirchner,
Ari P.~Seitsonen and J\"urg Hutter working at the Physical Chemistry Institute
of the University of Zurich. Recent updates by Mauro Boero, Alessandro Curioni,
J\"urg Hutter, Axel Kohlmeyer, Nisanth Nair and Wolfram Quester.

  If you want to contribute or have constructive criticism please contact
\htref{mailto:cpmd@cpmd.org}{cpmd@cpmd.org}.

\vspace{5mm}

\section{Citation}\label{citation}
%
Publications of results obtained with CPMD should acknowledge its
use by an appropriate citation of the following kind:
\vspace{5mm}

\noindent
\textit{CPMD, http://www.cpmd.org/,\\
 Copyright IBM Corp 1990-2019,\\
 Copyright MPI f\"ur Festk\"orperforschung Stuttgart 1997-2001}.
%

\vspace{5mm}

\section{Important constants and conversion factors}
Input and output are in Hartree atomic units (a.u.),
unless otherwise explicitly mentioned.

\subsubsection*{IMPORTANT NOTICE:}
 As of CPMD version 3.15.1 all constants and conversion factors
 have been consolidated and updated to the CODATA 2006 data 
 set\cite{codata2006}. For details see the file {\tt cnst.inc} and
\htref{http://physics.nist.gov/constants}%
{http://physics.nist.gov/constants}.


\begin{tabbing}
1234567890123456789012 \= 1234567890123456789012  \kill
\textit{quantity} \> \textit{conversion factor} \\
%AK: these are the old values:
%time step         \> 1 a.u.\ = 0.0241888428 fs \\
%coordinates       \> 1 Bohr = 1 a$_0$ = 0.529177249 {\AA} \\
%velocity          \> 1 a.u.\ = 1 Bohr / 1 a.t.u.\ = 2188491.52 m/s \\
%energy            \> 1 Ha = 27.21161 eV = 627.5095 kcal/mol = 2625.5 kJ/mol \\
%plane wave cutoff \> 1 Ry = 1/2 Ha = 13.6058 eV \\
time step         \> 1 a.u.\ =  0.02418884326505 fs \\
coordinates       \> 1 Bohr = 1 a$_0$ = 0.52917720859 {\AA} \\
velocity          \> 1 a.u.\ = 1 Bohr / 1 a.t.u.\ = 2187691.2541 m/s \\
energy            \> 1 Ha = 27.21138386 eV = 627.5094706 kcal/mol = 2625.4996251 kJ/mol \\
plane wave cutoff \> 1 Ry = 1/2 Ha = 13.60569193 eV \\
dipole moment     \> 1 a.u.\ = 2.5417462289 Debye\\
atomic mass       \> 1 a.u.\ = 0.00054857990943 a.m.u\\
\end{tabbing}
%---------------------------------------------------------------------
\clearpage
%---------------------------------------------------------------------

\clearpage
\section{Recommendations for further reading}\label{intro:further}
\begin{itemize}
\item {\bf General Introduction to Theory and Methods}\\
  Jorge Kohanoff, "Electronic Structure Calculation for Solids and Molecules",\\
  Cambridge University Press, 2006, ISBN-13 978-0-521-81591-8\\
  \htref{http://www.cambridge.org/9780521815918}%
{http://www.cambridge.org/9780521815918}
\item {\bf General introduction to Car-Parrinello simulation}\\
  D. Marx and J. Hutter, "Ab Initio Molecular Dynamics - Basic Theory
  and Advanced Methods'',\\
  Cambridge University Press, 2009\\
  D. Marx and J. Hutter, "Modern Methods and Algorithms of Quantum Chemistry",\\
  Forschungszentrum J\"ulich, NIC Series, Vol. 1 (2000), 301-449 \\
  W. Andreoni and A. Curioni,\\
 "New Advances in Chemistry and Material Science with CPMD and Parallel Computing",\\
{ \em Parallel Computing}, 26 (2000) 819.
\item {\bf Electronic Structure Theory}\\
  Richard M.~Martin, "Electronic Structure: Basic Theory and Practical Methods'",\\
  Cambridge University Press, 2004, ISBN-13: 978-0-521-78285-2 \\
  \htref{http://electronicstructure.org}{http://electronicstructure.org}
\item{\bf General overview about quantum simulation techniques}\\
  J. Grotendorst, D. Marx, and A. Muramatsu,
  {\em Quantum Simulations of Complex Many--Body Systems:
  From Theory to Algorithms},
  (John von Neumann Institute for Computing,
  Forschungszentrum J\"ulich 2002);
  \newline
  Printed Version:
  ISBN 3--00--009057--6
  \newline
  Electronic Version:
  \htref{http://www.fz-juelich.de/nic-series/volume10/}%
  {http://www.fz-juelich.de/nic-series/volume10/}
  \newline
  Audio--Visual Version:
  \htref{http://www.fz-juelich.de/video/wsqs/}%
  {http://www.fz-juelich.de/video/wsqs/}

\item {\bf Molecular dynamics simulation}\\
     M.~P.~Allen and D.~J.~Tildesley, {\em Computer Simulation of Liquids}
     (Clarendon Press, Oxford, 1987; reprinted 1990).\\
     D.~Frenkel and B.~Smit, {\em Understanding Molecular Simulation --
     From Algorithms to Applications} (Academic Press, San Diego, 1996).\\
     M.~E.~Tuckerman and G.~J.~Martyna, J.~Phys.~Chem.~B {\bf 104} 159 (2000).

\item {\bf Pseudopotentials}\\
\htref{http://www.cpmd.org/documentation/useful-links}%
{http://www.cpmd.org/documentation/useful-links}\\
\htref{http://www.cpmd.org/cpmd_download.html}%
{http://www.cpmd.org/cpmd\_download.html}\\
\htref{http://cvs.berlios.de/cgi-bin/viewcvs.cgi/cp2k/potentials/Goedecker/cpmd/}%
{http://cvs.berlios.de/cgi-bin/viewcvs.cgi/cp2k/potentials/Goedecker/cpmd/}\\
\htref{http://www.fhi-berlin.mpg.de/th/fhi98md/fhi98PP/}%
{http://www.fhi-berlin.mpg.de/th/fhi98md/fhi98PP/}\\
\htref{http://www.physics.rutgers.edu/~dhv/uspp/}%
{http://www.physics.rutgers.edu/\~{}dhv/uspp/}\\
\htref{http://www.pwscf.org/pseudo.htm}%
{http://www.pwscf.org/pseudo.htm}

\item{\bf Parallelization \& Performance}\\
     J.~Hutter and A.~Curioni,
     Parallel Computing {\bf 31}, 1 (2005). \\
     J.~Hutter and A.~Curioni,
     ChemPhysChem {\bf 6}, 1788-1793 (2005). \\
     C.~Bekas and A.Curioni,
     Comp.~Phys.~Comm.{\bf 181}, 1057 (2010). \\
\end{itemize}

\clearpage

\section{History}\label{intro}

\subsection{CPMD Version 1}

  In summer 1993 a project was started to combine the two different ab initio
molecular dynamics codes~\cite{Allen87} that were used in the group for computational physics
of the IBM Research Laboratory in R\"uschlikon. There was the IBM-AIX version
(ported by J.~Kohanoff and F.~Buda)  of the IBM-VM version (by W.~Andreoni and P.~Ballone) 
of the original Car-Parrinello~\cite{CP85} code and a version of the code by K.~Laasonen 
and F.~Buda that could handle ultra-soft pseudopotentials~\cite{Vanderbilt}.
Further goals were to provide a common
platform for future developments, as new integration techniques or
parallelization. The original Car--Parrinello code~\cite{CP85,Galli91} was about 8000 lines of
Fortran. A first parallel version using the IBM MPL library was finished in
1993. Many people contributed to this effort in different ways: M.~Parrinello,
J.~Hutter, W.~Andreoni, A.~Curioni, P.~Giannozzi, E.~Fois, D.~Marx and M.~Tuckerman.

\subsection{CPMD Version 2}

\subsubsection{Version 2.0}

  The first major update of the code was finished in summer 1993. New
features of the code included a keyword driven input, an initial guess from
atomic pseudo-wavefunctions, a module for geometry optimization, several new
types of molecular dynamics, Nos\'e--Hoover~\cite{Nose84,Hoover85} thermostats and a 
diagonalization routine to get Kohn-Sham energies~\cite{KS}. This code had 17'000 lines.

\subsubsection{Version 2.5}

  In 1994 many additions were made to the code. The communication was improved
and a library interface for MPI was introduced. The code reached its most
stable version at the end of the year with version number 2.5.
At this stage a working version of {\em ab initio} path integrals~\cite{Marx94,Marx96} 
based on a one level parallelization was implemented in a separate branch
of the code by Dominik Marx.

\subsection{CPMD Version 3}

\subsubsection{Version 3.0}

  This major update included changes to improve the portability of the code to
other platforms. Most notable was the shmem interface for optimal parallel
performance on Cray computers. New features of this version were constant
pressure molecular dynamics using the Parrinello-Rahman Lagrangian~\cite{parrah,parrah2}, 
the possibility for symmetry constraints and Stefan Goedecker's dual space
pseudopotentials~\cite{goe_pp}. The library concept for the pseudopotentials had been
changed. The code had grown to 55'000 lines.

\subsubsection{Version 3.1}

  Only minor updates were made for this version. However, it served as a
starting point for two major new developments. The free energy functional~\cite{Alavi94} 
code with {\bf k} points was developed by Ali Alavi and Thierry Deutsch in Belfast. An
efficient path integral version using two level parallelism was put together 
by  Mark Tuckerman, Dominik Marx, and J\"urg Hutter~\cite{Tuckerman96} .

\subsubsection{Version 3.2}

  This version included several new algorithms. Some of these were lost in the
transfer to the next version.

\subsubsection{Version 3.3}

  This version was developed using the free energy functional version (based on
3.1) as a basis. The path integral version was fully included but only part of
the changes from the "main" version 3.2 were taken over. The QM/MM interface to
the EGO code was included~\cite{egoqmmm}. 
Development of the linear response~\cite{LinRes} parts of the code started.
Maximally localized Wannier functions~\cite{Marzari97} were implemented.
This version was finished in 1998, the code was about 115'000 lines
long.

\subsubsection{Version 3.4}

  The most notable change to this version was the inclusion of the QM/MM
interface developed by Alessandro Laio, Joost VandeVondele and Ursula
R\"othlisberger~\cite{qmmm02,qmmm03,qmmm04}. 
Besides that only minor changes to the functionality of the
code were done. This version included mostly bug fixes and was finished in
2000.

\subsubsection{Version 3.5}

  This was the first version made generally available at www.cpmd.org in early
2002. Many bugs were fixed, most notably the code for the ultra-soft
pseudopotentials was working again. The new size of the code was 136'000 lines.

\subsubsection{Version 3.6}

  This developers version included the final versions of the linear response
methods for the calculation of the polarizability and the chemical NMR shifts
developed by Anna Putrino and Daniel Sebastiani~\cite{apdsmp,apmp,dsmp}. 
Marcella Iannuzzi contributed a ${\bf k} \cdot {\bf p}$ module~\cite{mimp}.
Time-dependent density functional response theory was
implemented and forces for excited state energies programmed.
Salomon Billeter, Alessandro Curioni and Wanda Andreoni implemented
new linear scaling geometry optimizers that allow to locate geometrical
transition states in a clean way~\cite{LSCAL}.
Fine grained parallelism with OpenMP was added (by Alessandro Curioni
and J\"urg Hutter) and can be used together with the distributed
memory MPI version.

\subsubsection{Version 3.7}

  The stable version of the developers code was made publicly available in
early 2003. The code has 150'000 lines.

\subsubsection{Version 3.8}

 Developer's version.

\subsubsection{Version 3.9}

  Many new developments, improvements, cleanups, and bug fixes have been added
  since the last public version of the code.
  Most notably, the methodology for reactive
  Car-Parrinello metadynamics~\cite{alaio,iannuzzi}
  is made available in this version.\\
  Other new functionality includes
  G-space localization of wavefunctions,
  Hockney-type Poisson Solver~\cite{Hockney70}
  for slabs with influence function in G-Space,
  code to determine molecular KS states from Wannier functions,
  code for trajectory analysis, calculation of dipole moments using
  the Berry phase and in real space,
  transition matrix elements between orbitals,
  growth function for constraints and restraints,
  new code for applying static electrical fields,
  periodic or final diagonalization of WF, van der Waals force field
  according to Elstner's formula~\cite{Elstner}
  and dumping files for PDOS.\\
  Improvements of the code include
  performance and OpenMP improvements,
  improved code for keeping wavefunction in real space,
  updated TDDFT, SAOP TDDFT functional,
  a much improved configure script,
  bug fixes for HF exchange, screened exchange,
  cleanup of memory management,
  more checks on unsupported options,
  fixed constraints in geometry optimization. \\
  Modified ROKS~\cite{Frank98}, 
  Ports to MacOS-X/PPC, Cray X1, and Intel EM64T,
  k-points with swapfiles are working again on many platforms,
  detection of incompatible Vanderbilt pseudopotentials.
%

\subsubsection{Version 3.10}

  Developer's version.

\subsubsection{Version 3.11}
  Many improvements, cleanups, bug fixes and some new features
  have been added since the last public version of the code.
  New functionalities include calculation of the electric field 
  gradient tensor along MD trajectory, EPR calculations, efficient 
  wavefunction extrapolation for BOMD, distance screening for HFX 
  calculation and hybrid functional with PBC, interaction perturbation method,
  molecular states in TDDFT calculations, analytic second derivatives of 
  gradient corrected functionals~\cite{xcder}, Born charge tensor during 
  finite difference vibrational analysis, Gromacs  QM/MM 
  interface~\cite{gmxqmmm}, and distributed linear algebra support. \\
  New supported platforms include, IBM Blue Gene/L~\cite{bgl}, Cray XT3,
  NEC-SX6 Earth Simulator (Vector-Parallel) and Windows NT/XT using GNU Gfortran.
  Performance tunings for existing platforms include FFTW interface,
  16 Byte memory, alignment for Blue Gene/L, extension of the taskgroup 
  implementation to cartesian taskgroups (Blue Gene/L),
  parallel distributed linear algebra, 
  alltoall communication in either single (to reduce
  communication bandwidth) or double precision, special parallel OPEIGR,
  improved OpenMP support \cite{mixed}, and improved metadynamics.


\subsubsection{Version 3.12}

  Developer's version.

\subsubsection{Version 3.13}
  Several improvements, cleanups, bug fixes and a few new features
  have been added since the last public version of the code.
  New functionalities include additional distributed linear algebra code 
  for initialization, final wavefunction projection and Friesner 
  diagonalization, mean free energy path search method, multiscale 
  shock method~\cite{shock}, Langevin integrator for metadynamics 
  with extended Lagrangian, calculation of non-adiabatic couplings, 
  Landau-Zener Surface hopping, ROKS-based Slater transition-state
  density, linear-response DFPT with a ROKS-based reference 
  state~\cite{GrimmJCP2003}, simplified van der Waals correction 
  according to Grimme~\cite{Grimme06}, simplified ROKS input options
  with hard-wired variants of modified Goedecker algorithms for ROKS,
  PBEsol functional, ports to IBM Blue Gene/P, MacOS-X/x86 and PACS-CS / T2K,
  support for fftw-3, improved  ultrasoft pseudopotential parallel code (VDB) 
  (MPI and OpenMP), optimizations for scalar CPUs, new collective 
  variables for metadynamics, variable cell support in DCD output, 
  isotropic and zflexible cell for Parrinello-Rahman dynamics, 
  damped dynamics and Berendsen thermostats for electrons, ions and cell,
  path-integral support for BO-MD, support for completely reproducible 
  outputs for CPMD TestSuite, consistent and updated unit conversions 
  throughout the code, spin-density Mulliken analysis,
  aClimax format output of vibrational frequencies, optimization scheme 
  for Goedecker pseudopotential parameters for use as link atoms in 
  QM/MM applications, support for QUENCH BO with PCG MINIMIZE when using 
  VDB potentials, corrections for a number of serious bugs in the Gromos 
  QM/MM code, use of PDB format coordinate files for Amber2Gromos,
  Taskgroup support for Gromos QM/MM with SPLIT option, BO-MD with 
  EXTRAPOLATE WFN fully restartable, access to QM and MM energy in 
  QM/MM calculations, and improvements of the manual.

\subsubsection{Version 3.14}

  Developer's version.

\subsubsection{Version 3.15}
New features, performance improvement and bug fixes have been added
since the latest version of the code. 
These include a new efficient orthogonalization scheme~\cite{be_cur},
Constrained Density Functional Theory~\cite{ober}, force matching in
QM/MM runs, the new generalized Langevin thermostat ~\cite{ceriotti},
the screened hybrid functional HSE06 from the group of 
Scuseria ~\cite{HSE06a,HSE06b}, 
multiple walkers in metadynamics, surface hopping
dynamics with non-adiabatic coupling vectors in TDDFT~\cite{taver},
extensions of Grimme vdW corrections and initial support for the Ehrenfest Molecular
Dynamics~\cite{taver1}, and Kerker mixing for handling metallic 
systems~\cite{Kerker}.  
The new version includes also ports to IBM
POWER7, Fujitsu-Primergy BX900, several Linux updates and an updated
pseudopotential library.

\subsubsection{Version 3.17}
%
New features, scalability improvements and bug fixes have been added since the
latest version of the code.
%
Among the several novel implementations we number a new highly parallel scheme
for the evaluation of the Hartree--Fock exact exchange with an efficient
thresholding, support for parallel I/O in writing and reading the RESTART file,
the introduction of a second parallelization layer over the molecular states
which replaces the original TASKGROUP parallelization, increasing further the
scalability together with the already available parallelization over
plane-waves.
%
Finally, improved OpenMP 3.0 support has been introduced both in the main code
and in the QM/MM interface.  This allows for a substantial speedup of the code
in a hybrid OMP/MPI distribution (see note below).
%
Between the novel methodologies  we mention the availability of the {\it
ab-initio} vdW corrections, of a fully functional Ehrenfest Dynamics module and
of several improvements to the FO-DFT scheme.  Additional developments include
novel schemes for treating  QM/MM link-atoms and the possibility of
post-processing Nos\'e-Hoover trajectories and energies.
%
The new version includes also porting and optimization to the IBM BlueGene/Q
and different updates to the Linux architecture files.


\emph{Note on OMP 3.0}: The {\it collapse(n)} clause, although powerful in well
nested loops, may not work in old compilers and is known to have troubles e.g.
in old Intel Fortran Compilers (versions 11 and former ones). 
%
Please, check carefully your OS and your compiler berfore compiling blindly and
refer to the discussion in
http://openmp.org/forum/viewtopic.php?f=3\&t=1023\&start=10.

\subsection{CPMD Version 4}

\subsubsection{Version 4.0}
%
A new development version started in 2012 (and developed concurrently with
version 3.17.1) with the purpose to refactor CPMD.


\subsubsection{Version 4.1} 
%
Release 4.1 is the new version made available after almost two years from the
last public release.
%
It sets a turning point to CPMD and its new position as a modern software
(object oriented) while still retaining all the features and functionalities
from the previous versions.


Among the major technical breakthroughs: all memory allocations are converted
to dynamical allocations using standard fortran; extensive usage of modules so
to promote a code re-usage policy; fixed several instabilities due to
arithmetical exceptions;  all floats operations are now converted to arbitrary
precision; support of few more compilers; all COMMONs have been converted to
TYPES; extensive usage of IMPLICIT NONE; removed a massive amounts of unused
variables and procedures. Finally, we have build-up a regression tester that
covers more than half of the functionalities (700 tests and keeps increasing
day by day), providing a quality control for developers and users.


Beyond cleanup and bug fixes improving the stability of all functionalities, we
also report the implementation of new features such as: Spin-Orbit-Coupling,
local control, inter-system-crossing SH dynamics together, Adiabatic Bohmian
dynamics; coupling with IPHIGENIE to perform QM/MM calculations with
polarizable schemes and a wider usage of CP\_GROUPS to improve scalability of
different other functionalities.  The new version includes also porting,
optimisation and different updates for various architectures. The work
initiated with version 4.0 will continue tirelessly on the forthcoming version.

\subsubsection{Version 4.3} 
%
With the version 4.3, we have ported some of the core-procedures to gpu. The XC procedures
have been refactored and it was added the possibility to link to libxc.
Coulomb-attenuated functionals (CAM) and the Minnesota functionals (M05, M06, M08 and
M11 families) are now available as internal xc procedures.
A new development for Eherenfest dynamics with ionized states is deployed.
Few novel implementations are available: a multiple time step scheme for MD,
accelerated exact exchange for isolated systems, 
the stress tensor, LSD and UltraSoft Vanderbilt pseudopotentials and NPT.
Also a new bicanonical ensemble method is made available with version 4.3.
Extensive cleanup and bug fixes and strenghtening of the regression test suite.

%%%
%
% Just listing stuff here so nothing gets forgotten at the next release.
%
\subsubsection{Version 4.5} 
The Minnesota non-separable exchange functional families (MN12, MN15) as well as the revM06-L functional
are now available.

% XXX tutorial added.
\clearpage
%---------------------------------------------------------------------
\section{Installation}\label{installation}
%
This version of CPMD is equipped with a shell script to create 
a Makefile for a number of given platforms. If you run the shell 
script \texttt{configure.sh} (NOTE: this script was previously
named \texttt{config.sh}) without any options it will tell you 
what platforms are available. Choose the label for a target 
platform close to what your machine is and run the script again.

\shellcommand{./configure.sh PLATFORM }

\textbf{NOTE:} Due to filesystem implementation limitations,
compilation under MacOS X, and Windows NT/XP/Vista requires the
compilation outside of the \textbf{src} directory. See below.

Most likely the generated makefile with \textbf{not} match the
setup on your machine and you have to adapt the various definitions 
of compilers, optimization flags, library locations and so on.
To display additional information about a configuration type:

\shellcommand{./configure.sh -i PLATFORM}

The executable can then be compiled using the \texttt{make} command. 
To see all possible options use

\shellcommand{./configure.sh -help}

A common problem is that the default names of the libraries and the path to
the libraries are not correct in the \texttt{Makefile}. In this case you
have to change the corresponding entries in the \texttt{Makefile} manually.
If you are changing the preprocessor flags (the CPPFLAGS entry), 
e.g.\ going from a serial to a parallel compilation, you have to delete 
%all "\texttt{.f}" and "\texttt{.o}" files first, preferably by executing:
all preprocessed, module and object files first, preferably by executing:

\shellcommand{make clean}

\vspace{3mm}
  Alternatively you can compile CPMD outside the source directory.
  This is highly recommended, if you need to compile several executables
  concurrently, e.g. if you are doing development on several platforms.
  This is done by creating a directory for each platform
  (e.g. by 'mkdir ./bin/cpmd-pc-pgi; mkdir ./bin/cpmd-pc-pgi-mpi'  ) and then
  create a makefile for each of those directories and pointing to the
  original source directory with with \texttt{SRC} and \texttt{DEST}
  flags. For the above examples this would be:\\
\shellcommand{./configure.sh -m -SRC=\$PWD -DEST=./bin/cpmd-pc-pgi PC-PGI}\\
\shellcommand{./configure.sh -m -SRC=\$PWD -DEST=./bin/cpmd-pc-pgi-mpi  PC-PGI-MPI}\\
  Now you can do development in the original source directory and only
  need to recompile the altered modules by typing 'make' in the
  respective subdirectories.

\textbf{NOTE:} For compilation under Mac OS-X this procedure is
  currently \textbf{required}.


\vspace{3mm}
  Compiling CPMD on Linux platforms can be particularly tricky,
  since there are several Fortran compilers available and there
  are no standard locations for supporting libraries (which on top
  of that usually have to be compiled with the same or a compatible
  compiler). If you run into problems, you may want to check out the
  CPMD Mailing list archives at

  \centerline{\htref{http://www.cpmd.org/pipermail/cpmd-list/}{http://www.cpmd.org/pipermail/cpmd-list/}}

  to see, whether your specific problem has already been dealt with.

\vspace{7mm}
  Please also note that only recent versions of the GNU gfortran compiler
  (4.1 and later) are sufficient to compile CPMD. The now obsolete GNU 
  Fortran 77 compiler, \texttt{g77}, and the G95 Fortran compiler, 
  \texttt{g95}, are {\em not} able to compile CPMD.
  \clearpage
%---------------------------------------------------------------------
\section{Running CPMD}\label{runcpmd}
%
The CPMD program is started with the following command:

\shellcommand{cpmd.x {\sl file.in} [{\sl PP\_path}] $>$ {\sl file.out}}

Running \texttt{cpmd.x} requires the following files:
\begin{itemize}
  \item an {\bf input file}\ \ {\sl file.in}\ (see section \ref{inputfile})

  \item {\bf pseudopotential files} for all atomic species specified
        in {\sl file.in} (see section \ref{S_Pseudopotentials}).
\end{itemize}

\bigskip

The path to the pseudopotential library can be given in different ways:
\begin{itemize}

\item The second command line argument \ [{\sl PP\_path}]\ \ is set.

\item If the second command line argument \ [{\sl PP\_path}]\ \ is not
  given, the program checks the environment variables\\ {\bf
    CPMD\_PP\_LIBRARY\_PATH and PP\_LIBRARY\_PATH}.

\item If neither the environment variables nor the second command line
  argument are set, the program assumes that the pseudopotential files are
  in the current directory
\end{itemize}

During the run\ cpmd.x\ creates different outputs:
\begin{itemize}
\item Various status messages to monitor the correct operation of the program is
  printed to standard output (in our case redirected to the file {\sl file.out}).

\item
  Detailed data is written into different files (depending on the keywords
  specified in the input {\sl file.in}). An overview on them is given in section
  \ref{FILES}. Large files are written either to the current directory, the directory
  specified by the environment variable {\bf CPMD\_FILEPATH}, or the directory
  specified in the input file using the keyword \refkeyword{FILEPATH}.
\end{itemize}

In case CPMD quits with a non-zero exit status, an error message is written to a file called
\emph{LocalError-X-X-X.log} by every task, indicating the procedure in which the error occured
and the call stack. In certain cases (mostly when processing the input), more ample error information
is written to the output unit as well.
\\

  Jobs can be stopped at the next breakpoint by creating a file:

\medskip

\begin{center}
{\em EXIT}
\end{center}
                
\noindent
in the run-directory.
\index{stopping a run}
%AK: this is also more confusing than helping.
%\medskip
%
%For IBM, DEC, SGI, and SUN machines
%there is the possibility to use the command:
%
%\medskip
%\centerline{\sl kill -30 pid}
%\begin{center}
%\sl kill -30 pid
%\end{center}
%\noindent
%where pid is the process ID.
%
\clearpage
%---------------------------------------------------------------------
\section{Files}\label{FILES}
%

Incomplete list of the files used or created by CPMD:\par

\begin{tabbing}
1234567890123456789012 \= 1234567890123456789012  \kill
\textsl{File}   \> \textsl{Contents}                     \\
\\
RESTART         \> Restart file                          \\
RESTART.x       \> Old/New restart files                 \\
LATEST          \> Info file on the last restart file    \\
\\
GEOMETRY        \> Current ionic positions and velocities \\
GEOMETRY.xyz    \> Current ionic positions and velocities in \AA\\
GEOMETRY.scale  \> Current unit cell vectors in \AA and atomic units \\
                 \> and ionic positions in scaled coordinates \\
GSHELL          \> {\bf G}$^2$ (NOT normalized), G-shells $|{\bf G}|$ in a.u.\\
                \> and related shell index \\
\\
TRAJECTORY      \> All ionic positions and velocities along the trajectory \\
TRAJEC.xyz      \> All ionic positions along the trajectory in xyz-format\\
TRAJEC.dcd      \> All ionic positions along the trajectory in dcd-format\\
GEO\_OPT.xyz     \> All ionic positions along the geometry optimization\\
                \> in xyz-format\\
ENERGIES        \> All energies along the trajectory     \\
MOVIE           \> Atomic coordinates in Movie format    \\
\\
STRESS          \> The "trajectory" of stress tensors    \\
CELL            \> The "trajectory" of the unit cell     \\
NOSE\_ENERGY    \> The kinetic, potential and total energies of the Nose-Hoover thermostat(s) \\
NOSE\_TRAJEC    \> The "trajectory" of the Nose-Hoover variables and velocities \\
CONSTRAINT      \> The "trajectory" of constraint/restraint forces \\
METRIC          \> The "trajectory" of collective variable metric (restraints only) \\
DIPOLE          \> The "trajectory" of dipole moments    \\
dipole.dat      \> The dipole file produced in the spectra calculation using the propagation TDDFT scheme \\
\\
DENSITY.x       \> Charge density in Fourier space       \\
SPINDEN.x       \> Spin density in Fourier space         \\
ELF             \> Electron localization function in Fourier space \\
LSD\_ELF        \> Spin polarized electron localisation function \\
ELF\_ALPHA      \> Electron localisation function of the alpha density \\
ELF\_BETA       \> Electron localisation function of the beta density \\
WAVEFUNCTION    \> Wavefunction instead of density is stored \\
\\
HESSIAN         \> Approximate Hessian used in geometry optimization \\
FINDIF          \> Positions and gradients for finite    \\
                \> difference calculations               \\
VIBEIGVEC       \> Eigenvectors of Hessian               \\
MOLVIB          \> The matrix of second derivatives,     \\
                \> as used by the program MOLVIB         \\
VIB1.log        \> Contains the modes 4\,--\,3N in a style\\
                \> similar to the gaussian output for    \\
                \> visualization with MOLDEN, MOLEKEL,...\\
VIB2.log        \> Contains the modes 1\,--\,3N-3        \\
ENERGYBANDS     \> Eigenvalues for each k points         \\
KPTS\_GENERATION\> Output of k points generation         \\
WANNIER\_CENTER \> Centers of the Wannier functions      \\
WC\_SPREAD      \> Spread of the Wannier functions      \\
WC\_QUAD        \> Second moments of the Wannier functions as : \\
                \> ISTEP QXX QXY QXZ QYY QYZ QZZ              \\
IONS+CENTERS.xyz\> Trajectory of ionic positions and WFCs in \AA\\
WANNIER\_DOS    \> Projection of the Wannier functions   \\
                \> onto the Kohn-Sham states             \\
WANNIER\_HAM    \> KS Hamiltonian in the Wannier states representation \\
WANNIER\_1.x    \> Wannier orbitals or orbital densities \\
wavefunctions   \> Containes the complex wavefunctions for Ehrenfest dynamics\\
HARDNESS        \> Orbital hardness matrix               \\
\\
RESTART.NMR     \> Files needed to restart a NMR/EPR calculation \\
RESTART.EPR      \\
RESTART.L\_x     \\
RESTART.L\_y     \\
RESTART.L\_z     \\
RESTART.p\_x     \\
RESTART.p\_y     \\
RESTART.p\_z     \\
j$\alpha$\_B$\beta$.cube \> Files in .CUBE format that contain the \\
B$\alpha$\_B$\beta$.cube \> induced current densities and magnetic fields \\
                 \> in an NMR calculation, respectively
                    ($\alpha,\beta$=x,y,z)\\
\\
PSI\_A.$i$.cube  \> Files in .CUBE format that contain (spinpolarized) \\
PSI\_B.$i$.cube  \> orbitals and densities.\\
RHO\_TOT.cube    \\
RHO\_SPIN.cube   \\
\\

QMMM\_ORDER  \> Relation between the various internal atom order lists in QM/MM runs.\\
QM\_TEMP     \> ``local'' temperatures of the QM/MM subsystems.\\
CRD\_INI.grm \> Positions of all atoms of first step in Gromos format (QM/MM only).\\
CRD\_FIN.grm \> Positions of all atoms of last step in Gromos format (QM/MM only).\\
MM\_TOPOLOGY \> New Gromos format topology file (QM/MM only).\\
ESP          \> Contains the ESP charges of the QM atoms (QM/MM only).\\
EL\_ENERGY   \> Contains the electrostatic interaction energy (QM/MM only).\\
MULTIPOLE    \> Contains the dipole- and the quadrupole-moment of the quantum system.\\
MM\_CELL\_TRANS \> Contains the trajectory of the offset of the QM cell (QM/MM only).\\
\\
OCCMAT       \> The "trajectory" of the Occupation Matrix (DFT+U only)\\
\end{tabbing}

In case of path integral runs
every replica $s=\{1, \dots , P\}$ gets its own
\texttt{RESTART\_$s$},
\texttt{RESTART\_$s$.x},
\texttt{DENSITY\_$s$.x},
\texttt{ELF\_$s$.x},
and
\texttt{GEOMETRY\_$s$} file.

\smallskip

In contrast, in case of mean free energy path searches, each replica uses 
its own directory for nearly all files. Exception are the file 
\texttt{LATEST} and the geometry files named as for path integrals: 
\texttt{GEOMETRY\_$s$}. The directories are built from the 
\refkeyword{FILEPATH} path (or else from the environment variable 
{\bf CPMD\_FILEPATH}) with the suffixes \texttt{\_$s$}, $s=\{1, \dots , P\}$.

\medskip


In case of bicanonical Ensemble MD 
\begin{tabbing}
1234567890123456789012 \= 1234567890123456789012  \kill
\textsl{File}   \> \textsl{Contents}                     \\
\\
OUTPUT\_CNF2          \> standard output of the smaller canonical system \\
RESTART\_CNF1         \> Restart file of the larger canonical system \\
RESTART\_CNF2         \> Restart file of the smaller canonical system \\
RESTART\_CNF1.x       \> Old/New restart files                 \\
RESTART\_CNF2.x       \> Old/New restart files                 \\
LATEST\_CNF1          \> Info file on the last restart file    \\
LATEST\_CNF2          \> Info file on the last restart file    \\
ENERGIES             \> Chemical potential, bicanonical weight, Kohn-Sham energy difference along the trajectory     \\
ENERGIES\_CNF1        \> All energies along the trajectory  larger canonical system    \\
ENERGIES\_CNF2        \> All energies along the trajectory  smaller canonical system   \\
GEOMETRY\_CNF1        \> Current ionic positions and velocities larger canonical system \\
GEOMETRY\_CNF2        \> Current ionic positions and velocities smaller canonical system\\
GEOMETRY\_CNF1.xyz    \> Current ionic positions and velocities in \AA larger canonical system \\
GEOMETRY\_CNF2.xyz    \> Current ionic positions and velocities in \AA smaller canonical system\\
TRAJECTORY\_CNF1      \> All ionic positions and velocities along the trajectory larger canonical system\\
TRAJECTORY\_CNF2      \> All ionic positions and velocities along the trajectory smaller canonical system\\
TRAJEC\_CNF1.xyz      \> All ionic positions along the trajectory in xyz-format larger canonical system\\
TRAJEC\_CNF2.xyz      \> All ionic positions along the trajectory in xyz-format smaller canonical system\\
TRAJEC\_CNF1.dcd      \> All ionic positions along the trajectory in dcd-format larger canonical system\\
TRAJEC\_CNF2.dcd      \> All ionic positions along the trajectory in dcd-format smaller canonical system\\
\end{tabbing}

\medskip

In general, existing files are {\bf overwritten}!

\medskip

  Exceptions are ``trajectory'' type files (\texttt{TRAJECTORY},
\texttt{ENERGIES}, \texttt{MOVIE}, \texttt{STRESS}, ...), in them data are {\bf
appended}.

\clearpage
\part{Reference Manual}
%---------------------------------------------------------------------
\section{Input File Reference}\label{inputfile}
%---------------------------------------------------------------------
The following sections \textbf{try} to explain the various keywords and
syntax of a CPMD input file. It is not meant to teach how to
create good CPMD input files, but as a reference manual.

\subsection{Basic rules}
\begin{itemize}
  \item \textbf{Warning:} Do not expect the input to be logical.
    The programmers logic may be different from yours.

  \item \textbf{Warning:} This input description may not refer to the actual
    version of the program you are using. Therefore the ultimate and
    authoritative input guide is the source code. Most of the input
    file is read via the code in the files\\ 
    \texttt{control\_utils.mod.F90},  \texttt{sysin\_utils.mod.F90},    \texttt{dftin\_utils.mod.F90}, 
    \texttt{ratom\_utils.mod.F90},    \texttt{recpnew\_utils.mod.F90},  \texttt{detsp\_utils.mod.F90},
    \texttt{proppt\_utils.mod.F90},   \texttt{setbasis\_utils.mod.F90}, \texttt{vdwin\_utils.mod.F90},
    \texttt{respin\_p\_utils.mod.F90}, \texttt{lr\_in\_utils.mod.F90},  \texttt{orbhard\_utils.mod.F90},
    \texttt{egointer\_utils.mod.F90}, \texttt{pi\_cntl\_utils.mod.F90}, \texttt{cl\_init\_utils.mod.F90}, 
    \texttt{cplngs\_utils.mod.F90},  \texttt{mts\_utils.mod.F90}


  \item The input is free format except when especially stated

  \item In most cases, only the first 80 characters of a line are read
   (exceptions are lists, that have to be on one line).

  \item Lines that do not match a keyword are treated as comments
    and thus ignored.\\
    \textbf{Warning:} For most sections there will be a report of
    unrecognized keywords. For the \verb+&ATOMS+ this is not possible,
    so please check the atom coordinates in the output with particular
    care.

  \item Take warnings seriously. There are a few warnings, that can
   be safely ignored under specific circumstances, but usually warnings
   are added to a program for a good reason.

  \item The order of the keywords is arbitrary unless it is explicitly
    stated. For keywords that select one of many alternatives (e.g. the
    algorithm for wavefunction optimization), the last one `wins'.

  \item Only keywords with capital letters match

  \item Lists inclosed in \textbf{\{ \}} imply that you have to choose {\bf
    exactly one} of the items

  \item Lists inclosed in \textbf{[ ]} imply that you can choose {\bf any
    number} of items on the same line

  \item Arguments to a keyword are given on the following line(s) if not explicitly stated otherwise

  \item The full keyword/input line has to be within columns 1 to 80

  \item There are exceptions to those rules.

\end{itemize}
%
\clearpage
\subsection{Input Sections}\label{sections}
%
The input file is composed of different sections. Each section is started by
\verb+&SECTIONNAME+ and ended by \verb+&END+. All input outside the
sections is ignored.

\begin{tabbing}
12345 \= 1234567890123456789 \= 12345678 \= 12345 \= \kill
\> \&INFO ...   \> \&END  \>  $\leftrightarrow$ \>
   A place to put comments about the job.\\
\> The contents of this
   section will be copied to the output file at the beginning of
   the calculation.\\
\\
\> \&CPMD ...   \> \&END  \>  $\leftrightarrow$ \>
   General control parameters for calculation (\textbf{required}).\\
\\
\> \&SYSTEM ... \> \&END  \>  $\leftrightarrow$ \>
   Simulation cell and plane wave parameters (\textbf{required}).\\
\\
\> \&PIMD ...   \> \&END  \>  $\leftrightarrow$ \>
   Path integral molecular dynamics (PIMD)\\
\> This section is only evaluated if the \refkeyword{PATH INTEGRAL}
   keyword is given in the \&CPMD section.\\
\\
\> \&PATH ...   \> \&END  \>  $\leftrightarrow$ \>
   Mean free energy path calculation (MFEP)\\
\> This section is only evaluated if the \refkeyword{PATH MINIMIZATION}
   keyword is given in the \&CPMD section.\\
\\
\> \&ATOMS ...  \> \&END  \>  $\leftrightarrow$ \>
   Atoms and pseudopotentials and related parameters (\textbf{required}).\\
\> Section \ref{S_Pseudopotentials} explains the usage of
   pseudopotentials in more detail.\\
\\
\> \&DFT ...    \> \&END  \>  $\leftrightarrow$ \>
   Exchange and correlation functional and related parameters.\\
\\
\> \&PROP ...   \> \&END  \>  $\leftrightarrow$ \>
   Calculation of properties\\
\> This section is only fully evaluated if the \refkeyword{PROPERTIES} keyword
   is given in the \&CPMD section.\\
\\
\> \&BASIS ...  \> \&END  \>  $\leftrightarrow$ \>
   Atomic basis sets for properties or initial guess\\
\\
\> \&RESP ...  \> \&END  \>  $\leftrightarrow$ \>
   Response calculations \\
\> This section is always evaluated, even if it is not used.\\
\\
\> \&PTDDFT ...  \> \&END  \>  $\leftrightarrow$ \>
   Propagation TDDFT for Ehrenfest dynamics and spectra \\
\\
\> \&LINRES ...  \> \&END  \>  $\leftrightarrow$ \>
   General input for HARDNESS and TDDFT calculations \\
\\
\> \&HARDNESS ...  \> \&END  \>  $\leftrightarrow$ \>
   Input for HARDNESS calculations \\
\> This section is only evaluated if the
   \refkeyword{ORBITAL HARDNESS}~\textbf{LR}\\
\> keyword is given in the \&CPMD section.\\
\\
\> \&TDDFT ...  \> \&END  \>  $\leftrightarrow$ \>
   Input for TDDFT calculations \\
\\
\> \&QMMM ...  \> \&END  \>  $\leftrightarrow$ \>
   Input for Gromos QM/MM interface (see section \ref{sec:qmmm}).\\
\> \textbf{Required} if the \refkeyword{QMMM} keyword is given in the \&CPMD section \\
\\
\> \&CLAS ...  \> \&END  \>  $\leftrightarrow$ \>
   Simple classical code interface \\
\\
\> \&EXTE ...  \> \&END  \>  $\leftrightarrow$ \>
   External field definition for EGO QM/MM interface \\
\\
\> \&VDW ...    \> \&END  \>  $\leftrightarrow$ \>
   Settings associated to van der Waals-correction schemes. \\
\> This section is only evaluated if either the keyword \refkeyword{DCACP}, \refkeyword{VDW CORRECTION}, \\
\> or \refkeyword{VDW WANNIER} is given in
   the \&CPMD section.\\
\\
\> \&MTS ...  \> \&END  \>  $\leftrightarrow$ \>
   Parameters for the Multiple Time-Step MD scheme. \\
\end{tabbing}

  A detailed discussion of the different keywords will be given in the
following section.

\clearpage
%
%---------------------------------------------------------------------
\subsection{List of Keywords by Sections}
%
\subsubsection[\&CPMD ... \&END]{\&CPMD $\ldots$ \&END}
\parindent=0pt
\options{}{}{}
\refkeyword{ALEXANDER MIXING}\options{}{}{}
\refkeyword{ALLTOALL}\options{\{SINGLE,DOUBLE\}}{}{}
\refkeyword{ANDERSON MIXING}\options{}{}{}
\refkeyword{ANNEALING}\options{\{IONS,ELECTRONS,CELL\}}{}{}
\refkeyword{BENCHMARK}\options{}{}{}
\refkeyword{BERENDSEN}\options{\{IONS,ELECTRONS,CELL\}}{}{}
\refkeyword{BFGS}\options{}{}{}
\refkeyword{BICANONICAL ENSEMBLE}\options{\{CHEMICALPOTENTIAL,\,XWEIGHT\}}{[INFO]}{}
\refkeyword{BLOCKSIZE STATES}\options{}{}{}
\refkeyword{BOGOLIUBOV CORRECTION}\options{}{[OFF]}{}
\refkeyword{BOX WALLS}\options{}{}{}
\refkeyword{BROYDEN MIXING}\options{}{}{}
\refkeyword{CDFT}\options{}{[NEWTON, DEKKER],[SPIN, ALL, PCGFI, RESWF, NOCCOR, HDA[AUTO, PHIOUT, PROJECT]]}{}
\refkeyword{CENTER MOLECULE}\options{}{[OFF]}{}
\refkeyword{CHECK MEMORY}\options{}{}{}
\refkeyword{CLASSTRESS}\options{}{}{}
\refkeyword{CMASS}\options{}{}{}
\refkeyword{COMBINE SYSTEMS}\options{}{[REFLECT,NONORTH,SAB]}{}
\refkeyword{COMPRESS}\options{\{WRITEnn\}}{}{}
\refkeyword{CONJUGATE GRADIENTS}\options{\{ELECTRONS, IONS\}}{}{}
\refkeyword{CONVERGENCE}\options{}{[ORBITALS, GEOMETRY, CELL]}{}
\refspekeyword{CONVERGENCE}{CONVERGENCE 2}\options{}{[ADAPT, ENERGY, CALFOR, RELAX, INITIAL]}{}
\refspekeyword{CONVERGENCE}{CONVERGENCE 3}\options{}{[CONSTRAINT]}{}
\refspekeyword{CP\_GROUPS}{CP GROUPS}\options{}{}{}
\refkeyword{CZONES}\options{}{[SET]}{}
\refkeyword{DAMPING}\options{\{IONS,ELECTRONS,CELL\}}{}{}
\refkeyword{DAVIDSON DIAGONALIZATION}\options{}{}{}
\refkeyword{DAVIDSON PARAMETER}\options{}{}{}
\refkeyword{DCACP}\options{}{}{}
\refkeyword{DEBUG FILEOPEN}\options{}{}{}
\refkeyword{DEBUG FORCES}\options{}{}{}
\refkeyword{DEBUG MEMORY}\options{}{}{}
\refkeyword{DEBUG NOACC}\options{}{}{}
\refkeyword{DIIS MIXING}\options{}{}{}
\refkeyword{DIPOLE DYNAMICS}\options{\{SAMPLE,WANNIER\}}{}{}
\refkeyword{DISTRIBUTED LINALG}\options{\{ON,OFF\}}{}{}
\refkeyword{DISTRIBUTE FNL}\options{}{}{}
\refkeyword{ELECTRONIC SPECTRA}\options{}{}{}
\refkeyword{ELECTROSTATIC POTENTIAL}\options{}{[SAMPLE=nrhoout]}{}
\refkeyword{ELF}\options{}{[PARAMETER]}{}
\refkeyword{EMASS}\options{}{}{}
\refkeyword{ENERGYBANDS}\options{}{}{}
\refkeyword{EXTERNAL POTENTIAL}\options{\{ADD\}}{}{}
\refkeyword{EXTRAPOLATE WFN}\options{\{STORE\}}{}{}
\refkeyword{EXTRAPOLATE CONSTRAINT}\options{}{}{}
% MPB: removed with FFTW I guess
% \refkeyword{FFTW WISDOM}\options{}{[ON,OFF]}{}
\refkeyword{FILE FUSION}\options{}{}{}
\refkeyword{FILEPATH}\options{}{}{}
\refkeyword{FINITE DIFFERENCES}\options{}{}{}
\refkeyword{FIXRHO UPWFN}\options{}{}{}
\refkeyword{FREE ENERGY FUNCTIONAL}\options{}{}{}
\refkeyword{GDIIS}\options{}{}{}
\refkeyword{GSHELL}\options{}{}{}
\refkeyword{HAMILTONIAN CUTOFF}\options{}{}{}
\refkeyword{HARMONIC REFERENCE SYSTEM}\options{}{[OFF]}{}
\refkeyword{HESSCORE}\options{}{}{}
\refkeyword{HESSIAN}\options{}{[DISCO,SCHLEGEL,UNIT,PARTIAL]}{}
\refkeyword{IMPLICIT NEWTON RAPHSON}\options{options}{}{}
\refkeyword{INITIALIZE WAVEFUNCTION}\options{\{RANDOM, ATOMS\}}{[PRIMITIVE]}{}
\refkeyword{INTERFACE}\options{\{EGO,GMX\}}{\{[MULLIKEN, LOWDIN, ESP, HIRSHFELD],PCGFIRST\}}{}
\refkeyword{INTFILE}{[READ,WRITE,FILENAME]}\options{}{}{}
\refkeyword{ISOLATED MOLECULE}\options{}{}{}
\refkeyword{KOHN-SHAM ENERGIES}\options{}{[OFF,NOWAVEFUNCTION]}{}
\refkeyword{KSHAM}\options{}{[MATRIX,ROUT,STATE]}{}
\refkeyword{LANCZOS DIAGONALIZATION}\options{\{ALL\}}{}{}
\refspekeyword{LANCZOS DIAGONALIZATION}{LANCZOS DIAGONALIZATION OPT}\options{\{OPT,RESET=n\}}{}{}
\refkeyword{LANCZOS PARAMETER}\options{[N=n]}{[ALL]}{}
\refkeyword{LANGEVIN}\options{\{WHITE, CPMD, OPTIMAL, SMART, CUSTOM\}}{[MOVECOM]}{}
\refkeyword{LBFGS}\options{}{[NREM, NTRUST, NRESTT, TRUSTR]}{}
\refkeyword{LINEAR RESPONSE}\options{}{}{}
\refkeyword{LSD}\options{}{}{}
\refkeyword{LOCAL SPIN DENSITY}\options{}{}{}
\refkeyword{MAXRUNTIME}\options{}{}{}
\refkeyword{MAXITER}\options{}{}{}
\refkeyword{MAXSTEP}\options{}{}{}
\refkeyword{MEMORY}\options{}{[SMALL, BIG]}{}
\refkeyword{MIRROR}\options{}{}{}
\refkeyword{MIXDIIS}\options{}{}{}
\refkeyword{MIXSD}\options{}{}{}
\refkeyword{MODIFIED GOEDECKER}\options{[PARAMETERS]}{}{}
\refkeyword{MOLECULAR DYNAMICS}\options{\{CP, BO, EH, PT, CLASSICAL, FILE [XYZ, NSKIP=N, NSAMPLE=M]\}}{}{}
\refkeyword{MOVERHO}\options{}{}{}
\refkeyword{MOVIE}\options{}{[OFF, SAMPLE]}{}
\refkeyword{NOGEOCHECK}\options{}{}{}
\refkeyword{NONORTHOGONAL ORBITALS}\options{}{[OFF]}{}
\refkeyword{NOSE}\options{\{IONS, ELECTRONS, CELL\}}{[ULTRA,MASSIVE,CAFES]}{}
\refkeyword{NOSE PARAMETERS}\options{}{}{}
\refkeyword{ODIIS}\options{}{[NOPRECONDITIONING,NO\_RESET=nreset]}{}
\refkeyword{OPTIMIZE GEOMETRY}\options{}{[XYZ, SAMPLE]}{}
\refkeyword{OPTIMIZE WAVEFUNCTION}\options{}{}{}
\refkeyword{ORBITAL HARDNESS}\options{\{LR,FD\}}{}{}
\refkeyword{ORTHOGONALIZATION}\options{}{[LOWDIN, GRAM-SCHMIDT]}{[MATRIX]}
\refspekeyword{PARA\_BUFF\_SIZE}{PARA BUFF SIZE}\options{}{}{}
\refspekeyword{PARA\_STACK\_BUFF\_SIZE}{PARA STACK BUFF SIZE}\options{}{}{}
\refspekeyword{PARA\_USE\_MPI\_IN\_PLACE}{PARA USE MPI IN PLACE}\options{}{}{}
\refkeyword{PARRINELLO-RAHMAN}\options{\{NPT,SHOCK\}}{}{}
\refkeyword{PATH INTEGRAL}\options{}{}{}
\refkeyword{PATH MINIMIZATION}\options{}{}{}
\refkeyword{PATH SAMPLING}\options{}{}{}
\refkeyword{PCG}\options{}{[MINIMIZE,NOPRECONDITIONING]}{}
\refkeyword{PRFO}\options{}{[MODE, MDLOCK, TRUSTP, OMIN, PRJHES, DISPLACEMENT, HESSTYPE]}{}
\refspekeyword{PRFO}{PRFO NVAR}\options{}{[NVAR, CORE, TOLENV, NSMAXP]}{}
\refkeyword{PRFO NSVIB}\options{}{}{}
\refkeyword{PRINT}\options{\{ON,OFF\}}{options}{}
\refkeyword{PRINT ENERGY}\options{ \{ON, OFF\}} {options}{}
\refkeyword{PRNGSEED}\options{}{}{}
\refkeyword{PROJECT}\options{\{NONE, DIAGONAL, FULL\}}{}{}
\refkeyword{PROPAGATION SPECTRA}\options{}{}{}
\refkeyword{PROPERTIES}\options{}{}{}
\refkeyword{QUENCH}\options{}{[IONS, ELECTRONS, CELL, BO]}{}
\refkeyword{RANDOMIZE}\options{}{[COORDINATES, WAVEFUNCTION], [DENSITY,
    CELL]}{}
\refkeyword{RATTLE}\options{}{}{}
\refkeyword{REAL SPACE WFN KEEP}\options{}{[SIZE]}{}
\refkeyword{RESCALE OLD VELOCITIES}\options{}{}{}
\refkeyword{RESTART}\options{}{}{[{options}]}
\refkeyword{RESTFILE}\options{}{}{}
\refkeyword{REVERSE VELOCITIES}\options{}{}{}
\refkeyword{RFO ORDER=nsorder}\options{}{}{}
\refkeyword{RHOOUT}\options{}{[BANDS,SAMPLE=nrhoout]}{}
\refkeyword{ROKS}\options{}{\{SINGLET, TRIPLET\},\{DELOCALIZED, LOCALIZED, GOEDECKER\}}{}
\refkeyword{SCALED MASSES}\options{}{[OFF]}{}
\refkeyword{SHIFT POTENTIAL}\options{}{}{}
\refkeyword{SOC}\options{}{}{}
\refkeyword{SPLINE}\options{}{[POINTS, QFUNCTION, INIT, RANGE]}{}
\refkeyword{SSIC}\options{}{}{}
\refkeyword{STEEPEST DESCENT}\options{}{[ELECTRONS, IONS, CELL,
    NOPRECONDITIONING, LINE]}{}
\refkeyword{STORE}\options{ \{OFF\}} {[WAVEFUNCTIONS, DENSITY, POTENTIAL]}{}
\refspekeyword{STRESS TENSOR}{STRESS TENSOR CPMD}\options{}{}{}
\refkeyword{STRUCTURE}\options{}{[BONDS, ANGLES, DIHEDRALS, SELECT]}{}
\refkeyword{SUBTRACT}\options{}{[COMVEL, ROTVEL]}{}
\refkeyword{SURFACE HOPPING}\options{}{}{}
%\refkeyword{TASKGROUPS}\options{}{[MINIMAL,MAXIMAL,CARTESIAN]}{}
\refkeyword{TDDFT}\options{}{}{}
\refkeyword{TEMPCONTROL}\options{}{{IONS, ELECTRONS, CELL}}{}
\refkeyword{TEMPERATURE}\options{}{[RAMP]}{}
\refkeyword{TEMPERATURE ELECTRON}\options{}{}{}
\refkeyword{TIMESTEP}\options{}{}{}
\refkeyword{TIMESTEP ELECTRONS}\options{}{}{}
\refkeyword{TIMESTEP IONS}\options{}{}{}
\refkeyword{TRACE}\options{}{[ALL,MASTER]}{}
\refspekeyword{TRACE\_PROCEDURE}{TRACE PROCEDURE}\options{}{}{}
\refspekeyword{TRACE\_MAX\_DEPTH}{TRACE MAX DEPTH}\options{}{}{}
\refspekeyword{TRACE\_MAX\_CALLS}{TRACE MAX CALLS}\options{}{}{}
\refkeyword{TRAJECTORY}\options{}{[OFF, XYZ, DCD, SAMPLE, BINARY, RANGE, FORCES]}{}
\refkeyword{TROTTER FACTOR}\options{}{}{}
\refkeyword{TROTTER FACTORIZATION OFF}\options{}{}{}
\refspekeyword{USE\_IN\_STREAM}{USE IN STREAM}\options{}{}{}
\refspekeyword{USE\_OUT\_STREAM}{USE OUT STREAM}\options{}{}{}
\refspekeyword{USE\_MPI\_IO}{USE MPI IO}\options{}{}{}
\refspekeyword{USE\_MTS}{USE MTS}\options{}{}{}
\refkeyword{QMMM}\options{}{[QMMMEASY]}{}{}{}
%_FM[
\refkeyword{FORCEMATCH}\options{}{}{}
%_FM]
\refkeyword{VGFACTOR}\options{}{}{}
\refkeyword{VIBRATIONAL ANALYSIS}\options{}{[FD, LR, IN], [GAUSS, SAMPLE, ACLIMAX]}{}
\refkeyword{VMIRROR}\options{}{}{}
\refkeyword{VDW CORRECTION}\options{}{[ON, OFF]}{}
\refkeyword{VDW DCACP}\options{}{}{}
\refkeyword{VDW WANNIER}\options{}{}{}
\refkeyword{WANNIER DOS}\options{}{}{}
\refkeyword{WANNIER MOLECULAR}\options{}{}{}
\refkeyword{WANNIER NPROC}\options{}{}{}
\refkeyword{WANNIER OPTIMIZATION}\options{\{SD,JACOBI,SVD\}}{}{}
\refkeyword{WANNIER PARAMETER}\options{}{}{}
\refkeyword{WANNIER REFERENCE}\options{}{}{}
\refspekeyword{WANNIER RELOCALIZE\_EVERY}{WANNIER RELOCALIZE EVERY}\options{}{}{}
\refspekeyword{WANNIER RELOCALIZE\_IN\_SCF}{WANNIER RELOCALIZE IN SCF}\options{}{}{}
\refkeyword{WANNIER SERIAL}\options{}{}{}
\refkeyword{WANNIER TYPE}\options{\{VANDERBILT,RESTA\}}{}{}
\refkeyword{WANNIER WFNOUT}\options{}{[ALL,PARTIAL,LIST,DENSITY]}{}
\refkeyword{WOUT}\options{}{[FULL]}{}
%
%
\subsubsection[\&SYSTEM ... \&END]{\&SYSTEM $\ldots$ \&END}
%
\options{}{}{}
\refkeyword{ACCEPTOR}\options{}{[HDASINGLE,WMULT]}{}
\refkeyword{ANGSTROM}\options{}{}{}
\refkeyword{CELL}\options{}{[ABSOLUTE, DEGREE, VECTORS]}{}
\refkeyword{CHARGE}\options{}{}{}
\refkeyword{CHECK SYMMETRY}\options{}{[OFF]}{}
\refkeyword{CLASSICAL CELL}\options{}{[ABSOLUTE, DEGREE]}{}
\refkeyword{CLUSTER}\options{}{}{}
\refkeyword{CONSTANT CUTOFF}\options{}{}{}
\refkeyword{COUPLINGS}\options{\{FD,PROD\}}{[NAT]}{}
\refkeyword{COUPLINGS LINRES}\options{\{BRUTE FORCE,NVECT\}}{[THR,TOL]}{}
\refkeyword{COUPLINGS NSURF}\options{}{}{}
\refkeyword{CUTOFF}\options{}{[SPHERICAL,NOSPHERICAL]}{}
\refkeyword{DENSITY CUTOFF}\options{}{[NUMBER]}{}
\refkeyword{DONOR}\options{}{}{}
\refkeyword{DUAL}\options{}{}{}
\refkeyword{ENERGY PROFILE}\options{}{}{}
\refkeyword{EXTERNAL FIELD}\options{}{}{}
\refkeyword{HFX CUTOFF}\options{}{}{}
\refkeyword{ISOTROPIC CELL}\options{}{}{}
\refkeyword{KPOINTS}\options{}{}{options}
\refkeyword{LOW SPIN EXCITATION}\options{}{}{}
\refkeyword{LOW SPIN EXCITATION LSETS}\options{}{}{}
\refkeyword{LSE PARAMETERS}\options{}{}{}
\refkeyword{MESH}\options{}{}{}
\refkeyword{MULTIPLICITY}\options{}{}{}
\refkeyword{OCCUPATION}\options{}{[FIXED]}{}
\refkeyword{NSUP}\options{}{}{}
\refkeyword{POINT GROUP}\options{}{[MOLECULE], [AUTO], [DELTA=delta]}{}
\refkeyword{POISSON SOLVER}\options{ \{HOCKNEY, TUCKERMAN, MORTENSEN\}}{[PARAMETER]}{}
\refkeyword{POLYMER}\options{}{}{}
\refkeyword{PRESSURE}\options{}{}{}
\refkeyword{REFERENCE CELL}\options{}{[ABSOLUTE, DEGREE, VECTORS]}{}
\refkeyword{SCALE}\options{}{[CARTESIAN]}{[S=sascale] [SX=sxscale] [SY=syscale]
  [SZ=szscale]}
\refkeyword{STATES}\options{}{}{}
\refspekeyword{STRESS TENSOR}{STRESS TENSOR SYSTEM}\options{}{}{}
\refkeyword{SURFACE}\options{}{}{}
\refkeyword{SYMMETRIZE COORDINATES}\options{}{}{}
\refkeyword{SYMMETRY}\options{}{}{}
\refkeyword{TESR}\options{}{}{}
\refkeyword{WGAUSS}\options{}{}{NWG}
\refkeyword{WCUT}\options{}{}{CUT}
\refkeyword{ZFLEXIBLE CELL}\options{}{}{}
%
%
\subsubsection[\&PIMD ... \&END]{\&PIMD $\ldots$ \&END}
%
\options{}{}{}
\refkeyword{CENTROID DYNAMICS}\options{}{}{}
\refkeyword{CLASSICAL TEST}\options{}{}{}
\refkeyword{DEBROGLIE}\options{}{[CENTROID]}{}
\refkeyword{FACMASS}\options{}{}{}
\refkeyword{GENERATE REPLICAS}\options{}{}{}
\refkeyword{INITIALIZATION}\options{}{}{}
\refkeyword{NORMAL MODES}\options{}{}{}
\refkeyword{OUTPUT}\options{}{[ALL, GROUPS, PARENT]}{}
\refkeyword{PRINT LEVEL}\options{}{}{}
\refkeyword{PROCESSOR GROUPS}\options{}{}{}
\refkeyword{READ REPLICAS}\options{}{}{}
\refkeyword{STAGING}\options{}{}{}
\refkeyword{TROTTER DIMENSION}\options{}{}{}
%
%
\subsubsection[\&PATH ... \&END]{\&PATH $\ldots$ \&END}
%
This section contains specific information for Mean Free Energy Path 
searches\cite{Eijnden06}. However, the space of collective variables in 
which the search will be performed has to be defined using {\bf restraints} 
in the \&ATOMS$\ldots$\&END section (see \ref{sec:cnstr}). The initial string 
in this collective variable space is read in an external file 
{\em STRING.0}. 
%{\em string.inp}. 
This file contains one line per replica, each giving first the replica index and then 
the list of collective variables values. The basename for directories where each replica 
is ran should also be specified using the \refkeyword{FILEPATH}, or else the environment 
variable {\bf CPMD\_FILEPATH}.
The code writes the files CONSTRAINT.x and METRIC.x during the execution, where x is
the string number. 

\options{}{}{}
\refkeyword{REPLICA NUMBER}\options{}{}{}
\refkeyword{NLOOP}\options{}{}{} 
\refkeyword{NEQUI}\options{}{}{}
\refkeyword{NPREVIOUS}\options{}{}{}
\refkeyword{FACTOR}\options{}{}{}
\refkeyword{ALPHA}\options{}{}{}
\refkeyword{OUTPUT}\options{}{[ALL, GROUPS, PARENT]}{}
\refkeyword{PRINT LEVEL}\options{}{}{}
\refkeyword{PROCESSOR GROUPS}\options{}{}{}
%
%
\subsubsection[\&PTDDFT ... \&END]{\&PTDDFT $\ldots$ \&END}
%
This section contains specific information for Ehrenfest dynamics and the
spectra calculation computed using the propagation of the perturbed KS orbitals
(Fourier transform of the induced dipole fluctuation).
searches\cite{taver_eh,taver1}. 

\options{}{}{}
\refkeyword{ACCURACY}\options{}{}{}
\refspekeyword{PROP\_TSTEP}{PROP-TSTEP}\options{}{}{}
\refspekeyword{EXT\_PULSE}{EXT-PULSE}\options{}{}{}
\refspekeyword{EXT\_POTENTIAL}{EXT-POTENTIAL}\options{}{}{}
\refspekeyword{N\_CYCLES}{N-CYCLES}\options{}{}{}
\refspekeyword{PERT\_TYPE}{PERT-TYPE}\options{}{}{}
\refspekeyword{PERT\_AMPLI}{PERT-AMPLI}\options{}{}{}
\refspekeyword{PERT\_DIRECTION}{PERT-DIRECTION}\options{}{}{}
\refkeyword{RESTART}\options{}{}{}
\refspekeyword{TD\_POTENTIAL}{TD-POTENTIAL}\options{}{}{}
\refkeyword{PIPULSE}\options{}{}{}
%
\subsubsection[\&ATOMS ... \&END]{\&ATOMS $\ldots$ \&END}
%
This section also contains the nuclear coordinates and information on the pseudopotentials to be used. See
section \ref{S_Pseudopotentials} for more details on this.
%The following \options gives proper indentation, but it is a bad solution

\options{}{}{}
\refkeyword{ATOMIC CHARGES}\options{}{}{}
\refkeyword{CHANGE BONDS}\options{}{}{}
\refkeyword{CONFINEMENT POTENTIAL}\options{}{}{}
\refkeyword{CONSTRAINTS ... END CONSTRAINTS}\options{}{}{}
\refkeyword{METADYNAMICS ... END METADYNAMICS}\options{}{}{}
\refkeyword{DUMMY ATOMS}\options{}{}{}
\refkeyword{GENERATE COORDINATES}\options{}{}{}
\refkeyword{ISOTOPE}\options{}{}{}
\refkeyword{MOVIE TYPE}\options{}{}{}
\refkeyword{VELOCITIES ... END VELOCITIES}\options{}{}{}
%
%
\subsubsection[\&DFT ... \&END]{\&DFT $\ldots$ \&END}
%
\options{}{}{}
\refkeyword{ACM0}\options{}{}{}
\refkeyword{ACM1}\options{}{}{}
\refkeyword{ACM3}\options{}{}{}
\refspekeyword{ANALYTICAL\_DIV}{ANALYTICAL DIV}\options{}{}{}
\refkeyword{BECKE BETA}\options{}{}{}
\refkeyword{COULOMB ATTENUATION}\options{}{}{}
\refspekeyword{CP\_LIBRARY\_ONLY}{CP LIBRARY ONLY}\options{}{}{}
\refkeyword{EXCHANGE CORRELATION TABLE}\options{}{[NO]}{}
\refkeyword{FUNCTIONAL}\options{}{}{functional(s)}
\refkeyword{HUBBARD}\options{[NORM,ORTHO,NUATM=$nuatm$,OCCMAT=$printfreq$,VERB]}{}{}
\refkeyword{HARTREE}\options{}{}{}
\refkeyword{HARTREE-FOCK}\options{}{}{}
\refspekeyword{HFX\_BLOCK\_SIZE}{HFX BLOCK SIZE}\options{}{}{}
\refspekeyword{HFX\_DISTRIBUTION}{HFX DISTRIBUTION}\options{}{[BLOCK\_CYCLIC,DYNAMIC]}{}
\refkeyword{HFX SCREENING}\options{\{WFC,DIAG,EPS\_INT,RECOMPUTE\_TWO\_INT\_LIST\_EVERY\}}{}{}
\refkeyword{SCEX}\options{}{}{}
\refkeyword{SCALED EXCHANGE}\options{}{}{}
\refkeyword{GC-CUTOFF}\options{}{}{}
\refkeyword{GRADIENT CORRECTION}\options{}{}{functionals}
\refkeyword{LIBRARY}\options{}{}{library\_for\_functional\_1, library\_for\_functional\_2, \dots}
\refspekeyword{KERNEL\_LIBRARY}{KERNEL LIBRARY}\options{}{}{library\_for\_kernel\_1, library\_for\_kernel\_2, \dots}
\refspekeyword{LIBXC\_ONLY}{LIBXC ONLY}\options{}{}{}
\refkeyword{LDA CORRELATION}\options{}{}{functional}
\refkeyword{LR KERNEL}\options{}{}{functionals}
\refkeyword{NEWCODE}\options{}{}{}
\refspekeyword{NUMERICAL\_DIV}{NUMERICAL DIV}\options{}{}{}
\refkeyword{OLDCODE}\options{}{}{}
\refspekeyword{OLD\_DEFINITIONS}{OLD DEFINITIONS}\options{}{}{}
\refspekeyword{PBE\_FLEX\_KAPPA}{PBE FLEX KAPPA}\options{}{}{}
\refspekeyword{PBE\_FLEX\_GAMMA}{PBE FLEX GAMMA}\options{}{}{}
\refspekeyword{PBE\_FLEX\_MU}{PBE FLEX MU}\options{}{}{}
\refspekeyword{PBE\_FLEX\_BETA}{PBE FLEX BETA}\options{}{}{}
\refspekeyword{PBE\_FLEX\_UEG\_CORRELATION}{PBE FLEX UEG CORRELATION}\options{}{}{functional}
\refkeyword{SCALES}\options{}{}{scaling\_for\_functional\_1, scaling\_for\_functional\_2, \dots}
\refspekeyword{HFX\_SCALE}{HFX SCALE}\options{}{}{scaling\_for\_hfx}
\refspekeyword{KERNEL\_SCALES}{KERNEL SCALES}\options{}{}{scaling\_for\_kernel\_1, scaling\_for\_kernel\_2, \dots}
\refspekeyword{KERNEL\_HFX\_SCALE}{KERNEL HFX SCALE}\options{}{}{scaling\_for\_hfx\_in\_kernel}
\refkeyword{SCREENED EXCHANGE}\options{\{ASHCROFT,CAM,ERFC,EXP\}}{}{}
\refkeyword{SLATER}\options{}{[NO]}{}
\refkeyword{SMOOTH}\options{}{}{}
\refkeyword{REFUNCT}\options{}{}{functionals}
\refspekeyword{XC\_DRIVER}{XC DRIVER}\options{}{}{}\options{}{}{}
\refspekeyword{XC\_KERNEL}{XC KERNEL}\options{}{}{functionals}
\refspekeyword{MTS\_LOW\_FUNC}{MTS LOW FUNC}\options{}{}{functionals}
\refspekeyword{MTS\_HIGH\_FUNC}{MTS HIGH FUNC}\options{}{}{functionals}


%
%
\subsubsection[\&PROP ... \&END]{\&PROP $\ldots$ \&END}
%
The keyword \refkeyword{PROPERTIES} has to be present in the \&CPMD-section of
the input-file if this section shall be evaluated.
%The following \options gives proper indentation, but it is a bad solution

\options{}{}{}
\refkeyword{CHARGES}\options{}{}{}
\refkeyword{CONDUCTIVITY}\options{}{}{}
\refkeyword{CORE SPECTRA}\options{}{}{}
\refkeyword{CUBECENTER}\options{}{}{}
\refkeyword{CUBEFILE}\options{\{ORBITALS,DENSITY\}}{[HALFMESH]}{}
\refkeyword{DIPOLE MOMENT}\options{}{[BERRY,RS]}{}
\refkeyword{EXCITED DIPOLE}\options{}{}{}
\refkeyword{LDOS}\options{}{}{}
\refkeyword{LOCALIZE}\options{}{}{}
\refkeyword{OPTIMIZE SLATER EXPONENTS}\options{}{}{}
\refkeyword{LOCAL DIPOLE}\options{}{}{}
\refkeyword{NOPRINT ORBITALS}\options{}{}{}
\refkeyword{POLARIZABILITY}\options{}{}{}
\refkeyword{POPULATION ANALYSIS}\options{}{[MULLIKEN, DAVIDSON, n-CENTER]}{}
\refkeyword{PROJECT WAVEFUNCTION}\options{}{}{}
\refkeyword{TRANSITION MOMENT}\options{}{}{}
\refkeyword{n-CENTER CUTOFF}\options{}{}{}
\refkeyword{AVERAGED POTENTIAL}\options{}{}{}
%
%
\subsubsection[\&RESP ... \&END]{\&RESP $\ldots$ \&END}
%
\options{}{}{}
\refkeyword{CG-ANALYTIC}\options{}{}{}
\refkeyword{CG-FACTOR}\options{}{}{}
\refspekeyword{CONVERGENCE}{CONVERGENCE RESP}\options{}{}{}
\refkeyword{DISCARD}\options{}{[OFF, PARTIAL, TOTAL, LINEAR]}{}
\refkeyword{EIGENSYSTEM}\options{}{}{}
\refkeyword{EPR}\options{}{}{options}
\refkeyword{FUKUI}\options{}{[N=nf, COEFFICIENTS]}{}
\refspekeyword{HAMILTONIAN CUTOFF}{HAMILTONIAN CUTOFF RESP}\options{}{}{}
\refkeyword{HARDNESS}\options{}{}{}
\refkeyword{INTERACTION}\options{}{}{}
\refkeyword{KEEPREALSPACE}\options{}{}{}
\refkeyword{KPERT}\options{}{}{options}
\refkeyword{LANCZOS}\options{}{ [CONTINUE,DETAILS]}{}
\refkeyword{NMR}\options{}{}{options}
\refkeyword{NOOPT}\options{}{}{}
\refkeyword{OACP}\options{}{[DENSITY, REF\_DENSITY, FORCE]}{}
\refkeyword{PHONON}\options{}{}{}
\refkeyword{POLAK}\options{}{}{}
\refkeyword{RAMAN}\options{}{}{}
\refkeyword{TIGHTPREC}\options{}{}{}
%
%
\subsubsection[\&LINRES ... \&END]{\&LINRES $\ldots$ \&END}
\reflabel{inputkw:linres}{}
%
\options{}{}{}
\refspekeyword{CONVERGENCE}{CONVERGENCE LINRES}\options{}{}{}
\refkeyword{DIFF FORMULA}\options{}{}{}
\refkeyword{HTHRS}\options{}{}{}
\refspekeyword{MAXSTEP}{MAXSTEP LINRES}\options{}{}{}
\refkeyword{OPTIMIZER}\options{}{[SD,DIIS,PCG,AUTO]}{}
\refspekeyword{QS\_LIMIT}{QS-LIMIT}\options{}{}{}
\refkeyword{STEPLENGTH}\options{}{}{}
\refkeyword{THAUTO}\options{}{}{}
\refspekeyword{XC\_ANALYTIC}{XC-ANALYTIC}\options{}{}{}
\refspekeyword{XC\_DD\_ANALYTIC}{XC-DD-ANALYTIC}\options{}{}{}
\refspekeyword{XC\_EPS}{XC-EPS}\options{}{}{}
\refkeyword{ZDIIS}\options{}{}{}
\refkeyword{GAUGE}\options{\{PARA,GEN,ALL\}}{}{}
%
%
\subsubsection[\&TDDFT ... \&END]{\&TDDFT $\ldots$ \&END}
\reflabel{inputkw:tddft}{}
%
\options{}{}{}
\refspekeyword{DAVIDSON PARAMETER}{DAVIDSON PARAMETER TDDFT}\options{}{}{}
\refkeyword{DAVIDSON RDIIS}\options{}{}{}
\refkeyword{DIAGONALIZER}\options{\{DAVIDSON,NONHERMIT,PCG\}}{[MINIMIZE]}{}
\refkeyword{FORCE STATE}\options{}{}{}
\refkeyword{LOCALIZATION}\options{}{}{}
\refkeyword{MOLECULAR STATES}\options{}{}{}
\refkeyword{LZ-SHTDDFT}\options{}{}{}
\refkeyword{LR-TDDFT}\options{}{}{}
\refkeyword{PCG PARAMETER}\options{}{}{}
\refkeyword{PROPERTY}\options{\{ STATE \}}{}{}
\refspekeyword{RANDOMIZE}{RANDOMIZE TDDFT}\options{}{}{}
\refkeyword{REORDER}\options{}{}{}
\refkeyword{REORDER LOCAL}\options{}{}{}
\refkeyword{ROTATION PARAMETER}\options{}{}{}
\refspekeyword{STATES}{STATES TDDFT}\options{}{[MIXED,SINGLET,TRIPLET]}{}
\refkeyword{T-SHTDDFT}\options{}{}{}
\refkeyword{TAMM-DANCOFF}\options{}{[SUBSPACE,OPTIMIZE]}{}
\refspekeyword{TD\_METHOD\_A}{TD METHOD A}\options{}{[ \em functionals ]}{}%
%
\subsubsection[\&HARDNESS ... \&END]{\&HARDNESS $\ldots$ \&END}
%
%
\options{}{}{}
\refkeyword{DIAGONAL}\options{}{[OFF]}{}
\refspekeyword{LOCALIZE}{LOCALIZE HARDNESS}\options{}{}{}
\refkeyword{ORBITALS}\options{}{}{}
\refkeyword{REFATOM}\options{}{}{}
%
%
\subsubsection[\&CLASSIC ... \&END]{\&CLASSIC $\ldots$ \&END}
%
\options{}{}{}
\refkeyword{FORCE FIELD ... END FORCE FIELD}\options{}{}{}
\refkeyword{FREEZE QUANTUM}\options{}{}{}
\refkeyword{FULL TRAJECTORY}\options{}{}{}
\refkeyword{PRINT COORDINATES}\options{}{}{}
\refkeyword{PRINT FF}\options{}{}{}
%
\subsubsection[\&VDW ... \&END]{\&VDW $\ldots$ \&END}
%
\options{}{}{}
\refkeyword{DCACP Z=z}\options{}{}{}
\refspekeyword{NO\_CONTRIBUTION}{NO CONTRIBUTION}\options{}{}{}
\refspekeyword{INCLUDE\_METALS}{INCLUDE METALS}\options{}{}{}
\refkeyword{VDW PARAMETERS}\options{}{}{}
\refkeyword{VDW-CUTOFF}\options{}{}{}
\refkeyword{VDW-CELL}\options{}{}{}
\refkeyword{VDW WANNIER}\options{}{}{}
%
\subsubsection[\&QMMM ... \&END]{\&QMMM $\ldots$ \&END}
%
\options{}{}{}
\refkeyword{COORDINATES}\options{}{}{}
\refkeyword{INPUT}\options{}{}{}
\refkeyword{TOPOLOGY}\options{}{}{}
\refspekeyword{ADD\_HYDROGEN}{ADD-HYDROGEN}\options{}{}{}
\refkeyword{AMBER}\options{}{}{}
\refkeyword{ARRAYSIZES ... END ARRAYSIZES}\options{}{}{}
\refkeyword{BOX TOLERANCE}\options{}{}{}
\refkeyword{BOX WALLS}\options{}{}{}
\refkeyword{CAPPING}\options{}{}{}
\refspekeyword{CAP\_HYDROGEN}{CAP-HYDROGEN}\options{}{}{}
%\refkeyword{CHARGE...}\options{}{}{}
\refkeyword{ELECTROSTATIC COUPLING}\options{[LONG RANGE]}{}{}
\refkeyword{ESPWEIGHT}\options{}{}{}
\refkeyword{EXCLUSION}\options{\{GROMOS,LIST\{NORESP\}\}}{}{}
\refkeyword{FLEXIBLE WATER}\options{[ALL,BONDTYPE]}{}{}
%_FM[
\refkeyword{FORCEMATCH ... END FORCEMATCH}\options{}{}{}
%_FM[
\refkeyword{GROMOS}\options{}{}{}
\refkeyword{HIRSHFELD}\options{[ON,OFF]}{}{}
\refkeyword{MAXNN}\options{}{}{}
\refkeyword{NOSPLIT}\options{}{}{}
\refspekeyword{RCUT\_NN}{RCUT-NN}\options{}{}{}
\refspekeyword{RCUT\_MIX}{RCUT-MIX}\options{}{}{}
\refspekeyword{RCUT\_ESP}{RCUT-ESP}\options{}{}{}
\refkeyword{RESTART TRAJECTORY}\options{[FRAME \{num\},FILE '\{fname\}',REVERSE]}{}{}
\refkeyword{SAMPLE INTERACTING}\options{[OFF,DCD]}{}{}
\refkeyword{SPLIT}\options{}{}{}
\refkeyword{TIMINGS}\options{}{}{}
\refkeyword{UPDATE LIST}\options{}{}{}
\refkeyword{VERBOSE}\options{}{}{}
\refkeyword{WRITE LOCALTEMP}\options{[STEP \{nfi\_lt\}]}{}{}
%
\subsubsection[\&MTS ... \&END]{\&MTS $\ldots$ \&END}
%
\options{}{}{}
\refspekeyword{TIMESTEP\_FACTOR}{TIMESTEP FACTOR}\options{}{}{}
\refspekeyword{PRINT\_FORCES}{PRINT FORCES}\options{}{}{}
\refspekeyword{LOW\_LEVEL\_FORCES}{LOW LEVEL FORCES}\options{\{DFT, EXTERNAL\}}{}{}
\refspekeyword{HIGH\_LEVEL\_FORCES}{HIGH LEVEL FORCES}\options{\{DFT, EXTERNAL\}}{}{}
%
\clearpage
%
%---------------------------------------------------------------------
\subsection{Alphabetic List of Keywords}
{\bf Note~1:} Additional components of CPMD input files that do
not fit into the following list are explained in the succeeding
section \ref{further_input}.

{\bf Note~2:} Keywords for the \&QMMM section of the CPMD/Gromos
QM/MM-Interface code are not listed here but in
section \ref{sec:qmmm-input}.

\keyword{ACM0}{}{}{}{\&DFT}
  \desc{Add exact exchange to the specified \refkeyword{FUNCTIONAL} according
    to the adiabatic connection method 0.~\cite{acm0,adamo2000}
    % MPB: Not necessarily true anymore...
    % This only works for isolated systems
    % and should only be used if an excessive amount of CPU time is available.
    }

\keyword{ACM1}{}{}{}{\&DFT}
  \desc{Add exact exchange to the specified \refkeyword{FUNCTIONAL} according
    to the adiabatic connection method 1.~\cite{adamo2000,acm1}
    The parameter is read from the next line.
    % MPB: Not necessarily true anymore...
    % This only works for isolated systems and should only be used if an
    % excessive amount of CPU time is available.
    }

\keyword{ACM3}{}{}{}{\&DFT}
  \desc{Add exact exchange to the specified \refkeyword{FUNCTIONAL} according
    to the adiabatic connection method 3.~\cite{adamo2000,acm3}
    The three needed parameters are read from the next line.
    % MPB: Not necessarily true anymore...
    % This only works for isolated systems and should only be
    % used if an excessive amount of CPU time is available.
    }

\keyword{ACCEPTOR}{}{[HDASINGLE,WMULT]}{}{\&SYSTEM}
  \desc{Set the \refkeyword{CDFT} acceptor atoms. Parameter NACCR must be 
    specified next to the keyword. NACCR $\in [1,2,...,N]$ is the number of acceptor 
    Atoms ($N$ being the total number of atoms). The indices of NACCR atoms 
    separated by whitespaces are read from the next line.\\
    {\bf HDASINGLE} \defaultvalue{off} if set together with CDFT HDA, CPMD 
    performs a constrained HDA calculation with only an ACCEPTOR group weight 
    but different constraint values $N_\text{c}$.
    {\bf WMULT} \defaultvalue{off} if set together with CDFT HDA, CPMD 
    performs a constrained HDA calculation with two different an ACCEPTOR group
    weights for the two states.\\
    {\bf HDASINGLE} and {\bf WMULT} are mutually exclusive.}
   
\keyword{ACCURACY}{}{}{}{\&PTDDFT}
  \desc{Specifies the accuracy to be reached in the Cayley propagation scheme
   used in Ehrenfest type of dynamics and spectra calculation.}

\keyword{ALEXANDER MIXING}{}{}{}{\&CPMD}
  \desc{Mixing used during optimization of geometry or molecular dynamics.
    Parameter read in the next line.\\
%
    \textbf{Default} value is \defaultvalue{0.9}}

\keyword{ALPHA}{}{}{}{\&PATH}
  \desc{Smoothing parameter for iterating the string (see \cite{Eijnden06}).\\
    \textbf{Default} value is \defaultvalue{0.2}}


\keyword{ALLTOALL}{\{SINGLE,DOUBLE\}}{}{}{\&CPMD}
  \desc{Perform the matrix transpose (AllToAll communication) in the
        3D FFT using single/double precision numbers. Default is
        to use double precision numbers.}

\spekeyword{ANALYTICAL\_DIV}{}{}{}{\&DFT}{ANALYTICAL DIV}
  \desc{Calculate the $\mathbf{G}=0$ term for the Coulomb-attenuated exact
        exchange using an analytical description of the integral term.
        This is the default in combination with all internal hybrid xc functionals.}

\keyword{ANDERSON MIXING}{}{$N=n$}{}{\&CPMD}
  \desc{Anderson mixing for the electronic density during self-consistent
    iterations. In the next line the parameter (between 0 and 1) for the
    Anderson mixing is read.\\
%
    \textbf{Default} is \defaultvalue{0.2}.\\
%
    With the additional option $N=n$ a mixing parameter can be specified for
    different threshold densities. $n$ different thresholds can be set. The
    program reads $n$ lines, each with a threshold density and an Anderson
    mixing parameter.}

\keyword{ANGSTROM}{}{}{}{\&SYSTEM}
  \desc{The atomic coordinates and the supercell parameters and several
    other parameters are read in {\AA}ngs\-troms.\\
%
    {\bf Default} is {\bf atomic units} which are always used
    internally.
    Not supported for \refkeyword{QMMM} calculations.
}

\keyword{ANNEALING}{\{IONS,ELECTRONS,CELL\}}{}{}{\&CPMD}
  \desc{Scale the ionic, electronic, or cell velocities every
    time step. The scaling factor is read from the next line.}

\keyword{ATOMIC CHARGES}{}{}{}{\&ATOMS}
  \desc{Changes the default charge (0) of the atoms for the initial guess to
    the values read from the next line. One value per atomic species has to be given.}

\keyword{AVERAGED POTENTIAL}{}{}{}{\&PROP}
  \desc{ Calculate averaged electrostatic potential in spheres of radius Rcut
around the atomic positions.  \\
Parameter Rcut is read in from next line.}

\keyword{BECKE BETA}{}{}{}{\&DFT}
  \desc{Change the $\beta$ parameter in Becke's exchange functional~\cite{Becke88} to the
    value given on the next line.}

\keyword{BENCHMARK}{}{}{}{\&CPMD}
  \desc{This keyword is used to control some special features related to
    benchmarks. If you want to know more, have a look in the source code.}

\keyword{BERENDSEN}{\{IONS,ELECTRONS,CELL\}}{}{}{\&CPMD}
  \desc{Use a simple Berendsen-type thermostat\cite{Berendsen84} 
    to control the respective temperature of ions, electrons, or cell.
    The target temperature and time constant $\tau$ (in a.u.)
    are read from the next line. 

    These thermostats are a gentler alternative to the 
    \refkeyword{TEMPCONTROL} mechanism to thermalize a system.
    For production runs, please use the corresponding \refkeyword{NOSE}
    or \refkeyword{LANGEVIN} thermostats, as the Berendsen scheme does 
    not represent any defined statistical mechanical ensemble.
  }

\keyword{BFGS}{}{}{}{\&CPMD}
  \desc{Use a quasi-Newton method for optimization of the ionic
    positions. The approximated Hessian is updated using the
    Broyden-Fletcher-Goldfarb-Shano procedure~\cite{Fletcher80}.}

\keyword{BICANONICAL ENSEMBLE}{\{CHEMICALPOTENTIAL,\,XWEIGHT\} }{[INFO]}{}{\&CPMD}
    %
  \desc{Use the bicanonical ensemble AIMD for simulation of open
    systems~\cite{Frenzel2017}. 
    %
    \\
    %
    Note, requesting this runtype cpmd must be executed on more then one MPI
    thread.
    %
    \\
    Within the bicanonocal ensemble one out of two sampling modes must be 
    specified in the input:\\
    %
    \\[0.5em]
    %
    \textbf{CHEMICALPOTENTIAL} requests to sample the bicanonical ensemble at
    constant chemical potential ($\mu^\dagger_1$) or 
    %
    \\[0.25em]
    %
    \textbf{XWEIGHT} requests to sample the bicanonical ensemble at constant
    bicanonical weight ($x_1$).
    %
    \\[0.25em]
    %
    \textbf{Default} is to use no INFO printing at each AIMD step. 
    %
    \\[1em]
    %
    Two parameters read in from next line: 
    %
    \\[0.5em]
    %
    [\textbf{chemicalPotential}$~|~$\textbf{xWeight}]  [\textbf{temperature}]
    %
    \\[0.5em]
    %
    \textbf{chemicalPotential} 
    %
    ($\mu^\dagger_1$) if CHEMICALPOTENTIAL run type was requested. 
    %
    \\[0.25em]
    %
    A reasonable value can be obtained taking the ensemble average of $\langle
    \mu^\dagger_1 \rangle_{x_1=0.5}$ of a converged bicanonical AIMD simulation
    using the XWEIGHT run type (see Ref.~\cite{Frenzel2017}).
    %
    \\[0.5em]
    %
    \textbf{xWeight} ($x_1$) if XWEIGHT run type was requested. 
    %
    \\[0.25em]
    %
    The value of \textbf{xWeight} must be within the interval $[0,1]$. 
    %
    Setting the value of \textbf{xWeight} to 0 and 1 corresponds to the
    canonical limits of canonical system 2 and system 1, \textit{i.e.\@} small
    and the large system, respectively.
    %
    \\[0.5em]
    %
    \textbf{temperature} specifies the value of the temperature used to calculate 
    the $\gamma_1$ prefactor (see Ref.~\cite{Frenzel2017}).
    %
    \\[1em]
    %
    Preparation of the INPUT (see also CPMD regtests BICAN):
    %
    \\[0.25em]
    %
    The simulation setup requires two input files with the second one having
    the same name plus the extension "\textsc{\_2}".
    %
    The two files are for the large and the small canonical system, system 1
    and system 2, respectively. 
    %
    The two input files must differ in one atom. 
    %
    This atom must be declared as the \_last\_ atomic species in the input of
    system 1.
    %
    Thus the INPUT for system 2 is created from 1 by removing that species and
    then apply optional changes, \textit{e.g.\@} modify the number of
    electrons in that system via the CHARGE keyword. 
    %
    Note, positions and velocities of the atomic nuclei must be the same in
    both systems.
    %
    A restart from preoptimized electronic wavefunctions of the individual
    canonical systems is recommended (see Sec.~\ref{FILES} and CPMD regtests 
    BICAN/mdX0\_50, BICAN/c-cnf1, BICAN/c-cnf2).   
    %
    The output file of system 2 is written on OUTPUT\_CNF2 while that of system
    1 is stdout.
    %
    The suffixes "\_CNF1" and "\_CNF2" are used to specify the canonical
    systems for the other cpmd files.
    %
    For example the content of the file ENERGIES of a canoncal cpmd run is
    written to the files ENERGIES\_CNF1 and ENERGIES\_CNF2.
    %
    In the bicanonical AIMD the ENERGIES then stores: nfi, chemical potential,
    xweight, $\Delta U$, and $\mu_{x_1=0.5}$.
    %
    Where $\Delta U$ is the difference of the total energy of system 1 and
    system 2 at timestep nfi. 
    %
    \\
    %
    Note: For the moment bicanonical ensemble is only implemented for
    MOLECULAR DYNAMICS.  Other MD types BO, EH, PT, CLASSICAL or FILE 
    as well as runtypes QMMM, path integral or  surface hopping  
    are not supported.
    %
  }

\keyword{BLOCKSIZE STATES}{}{}{}{\&CPMD}
  \desc{Parameter read in from next line.\\
    {\sl NSTBLK} \\
    Defines the minimal number of states used per processor in the
    distributed linear algebra calculations.\\
    {\bf Default} is to equally distribute states over all processors.}

\keyword{BOGOLIUBOV CORRECTION}{}{[OFF]}{}{\&CPMD}
  \desc{Computes the Bogoliubov correction for the energy
    of the Trotter approximation or not.\\
%
    {\bf Default} is {\bf no Bogoliubov correction}.\\
%
    The keyword has to appear after \refkeyword{FREE ENERGY FUNCTIONAL}.}

\keyword{BOX WALLS}{}{}{}{\&CPMD}
  \desc{The thickness parameter for soft, reflecting QM-box walls
    is read from the next line. This keyword allows to reverse the
    momentum of the particles (${\bf p}_I \rightarrow -{\bf p}_I$)
    when they reach the walls of the simulation supercell in the
    case in which no periodic boundary conditions are applied.
    Specifically, in the unidimensional surface-like case, molecules
    departing from the surface are reflected back along the direction
    orthogonal to the surface, whereas in the bidimensional polymer-like
    case, they are reflected back in the two dimensions orthogonal to
    the "polymer" axis. Warning: This procedure, although keeping your
    particles inside the cell, affect the momentum conservation. \\
    This feature is {\bf disabled by default}}

\keyword{BROYDEN MIXING}{}{}{}{\&CPMD}
  \desc{Parameters read in from next line.\\
    {\sl BROYMIX, ECUTBROY, W02BROY, NFRBROY, IBRESET, KERMIX} \\
    These mean:\\
    \hfill\smallskip
    {\sl BROYMIX}: \hfill\begin{minipage}[t]{10cm}
        Initial mixing, e.g. $0.1$; \textbf{default} value is
        \defaultvalue{0.5}
        \end{minipage}

    {\sl ECUTBROY:} \hfill\begin{minipage}[t]{10cm}
        Cutoff for Broyden mixing. \defaultvalue{DUAL*ECUT} is the best choice
        and the \textbf{default}
        \end{minipage}

    {\sl W02BROY:} \hfill\begin{minipage}[t]{10cm}
        $w_0^2$ parameter of Johnson~\cite{Johnson88}. \textbf{Default}
        \defaultvalue{0.01}
        \end{minipage}

    {\sl NFRBROY:} \hfill\begin{minipage}[t]{10cm}
        Number of Anderson mixing steps done before Broyden mixing.
        \textbf{Default} is \defaultvalue{0}
        \end{minipage}

    {\sl IBRESET:} \hfill\begin{minipage}[t]{10cm}
        Number of Broyden vectors. $5$ is usually a good value and the default.
        \end{minipage}

    {\sl KERMIX:} \hfill\begin{minipage}[t]{10cm}
        Kerker mixing according to the original definition of Ref.~\cite{Kerker}.
        By default the mixing parameter is set to 0.
        \end{minipage}\\

    You can also specify some parameters with the following syntax:\\
        \textbf{[BROYMIX=}\textsl{BROYMIX}\textbf{]}
        \textbf{[ECUTBROY=}\textsl{ECUTBROY}\textbf{]}\\
        \textbf{[W02BROY=}\textsl{W02BROY}\textbf{]}
        \textbf{[NFRBROY=}\textsl{NFRBROY}\textbf{]}\\
        \textbf{[IBRESET=}\textsl{IBRESET}\textbf{]}\\
        \textbf{[KERMIX=}\textsl{KERMIX}\textbf{]}\\
    Finally, you can use the keyword {\bf DEFAULT} to use the default values.}

\keyword{CAYLEY}{}{}{}{\&CPMD}
  \desc{Used to propagate the Kohn-Sham orbitals in \refkeyword{MOLECULAR DYNAMICS} EH
        and \refkeyword{PROPAGATION SPECTRA}. At present is the only propagation scheme 
        available.}

\keyword{CDFT}{}{[NEWTON, DEKKER],[SPIN, ALL, PCGFI, RESWF, NOCCOR, HDA[AUTO, PHIOUT, PROJECT]]}{}{\&CPMD}
  \desc{The main switch for constrained DFT. Parameters $N_\text{c}$, $V_\text{init}$, 
        and MAXSTEP are read from the next line.\\
        {\bf NEWTON}, {\bf DEKKER} (\defaultvalue{off}) are switches to enable either the 
        Newton or the Dekker optimisation scheme for the constraint. If neither of those 
        are set a simple gradient scheme is used.\\
        {\bf SPIN} (\defaultvalue{off}) if activated the constraint will act on the spin 
        density instead of the charge density. This may help against excessive spin contamination.\\
        {\bf ALL} (\defaultvalue{off}) activates dual spin and charge constraint, all inputs for 
        $N_\text{c}$ and $V_\text{init}$ have to be given twice (first for charge then for spin)\\
        {\bf PCGFI} (\defaultvalue{off}) instructs CPMD to do PCG for the first V optimisation cycle
        regardless of the choice of optimiser.\\
        {\bf RESWF} (\defaultvalue{off}) if activated this switch re-initialises the wavefunction 
        after each $V$ optimisation step. This is useful if the wavefunction convergence between 
        the optimisation steps is slow. Usage in conjunction with \refkeyword{INITIALIZE WAVEFUNCTION}
        RANDOM may help.\\
        {\bf NOCCOR} (\defaultvalue{off}) if activated this switch turns off cutoff correction for 
        the forces.\\
        {\bf HDA} (\defaultvalue{off}) if activated this switch turns on the calculation of the 
        transition matrix element between the constrained states given by $N_\text{c}$ and 
        $\hat{N}_\text{c}$ which is then read from the second line. For this keyword to take 
        effect the \refkeyword{OPTIMIZE WAVEFUNCTION} option has to be activated.\\
        Sub-options of {\bf HDA}\\
        {\bf AUTO} (\defaultvalue{off}) if activated this switch lets CPMD choose the constraint 
        values for the transition matrix calculation. $N_\text{c}$ is chosen from the initial 
        charge distribution and $\hat{N}_\text{c}=-N_\text{c}$. It might be a good idea to use 
        \refkeyword{INITIALIZE WAVEFUNCTION} ATOMS and \refkeyword{ATOMIC CHARGES} (\&ATOM section) 
        so that CPMD initialises the wavefunction with the desired pseudo wavefunction.\\
        {\bf PHIOUT} (\defaultvalue{off}) if activated this switch tells CPMD to write out the 
        overlap matrices $\Phi_\text{AA},\Phi_\text{BB},\Phi_\text{AB},$ and $\Phi_\text{BA}$ 
        to the file PHI\_MAT.\\
        {\bf PROJECT} (\defaultvalue{off}) if activated this switch lets CPMD read in two reference 
        states from RESTART.REF1 and RESTART.REF2 after the actual HDA calculation in order to project 
        the two constrained states on them and thus calculate the diabatic transition matrix element 
        in an orthogonalised ``dressed'' basis.\\
        If CDFT is activated the program writes the current $V$ value to CDFT\_RESTART every time the 
        RESTART file is written.}


\keyword{CELL}{\{ABSOLUTE, DEGREE, VECTORS\}}{}{}{\&SYSTEM}
  \desc{The parameters specifying the super cell are read from the next
    line. Six numbers in the following order have to be provided: $a$, $b/a$,
    $c/a$, $\cos \alpha$, $\cos \beta$, $\cos \gamma$. For cubic phases, $a$ is
    the lattice parameter. CPMD will check those values, unless you turn off
    the test via \refkeyword{CHECK SYMMETRY}.
    With the keyword {\bf ABSOLUTE}, you give $a$, $b$ and $c$. With the
    keyword {\bf DEGREE}, you provide $\alpha$, $\beta$ and $\gamma$ in degrees
    instead of their cosine. With the keyword {\bf VECTORS}, the lattice
    vectors $a1$, $a2$, $a3$ are read from the next line instead of the 6
    numbers. In this case the {\bf SYMMETRY} keyword is not used.}

\keyword{CENTER MOLECULE}{}{[OFF]}{}{\&CPMD}
  \desc{The center of mass is moved/not moved to the center of the
    computational box in a calculation with the cluster option. This is only
    done when the coordinates are read from the input file.}

\keyword{CENTROID DYNAMICS}{}{}{}{\&PIMD}
  \desc{Adiabatic centroid molecular dynamics,
    see Ref.~\cite{Cao93,Martyna96,aicmd} for
    theory and details of our implementation, which yields
    quasiclassical dynamics of the nuclear centroids at a specified
    temperature of the non--centroid modes.
    This keyword makes only sense if used in conjunction with
    the normal mode propagator via the keyword
    NORMAL MODES {\em and} FACSTAGE~$>1.0$ {\em and} WMASS~$=1.0$.
    The centroid adiabaticity control parameter FACSTAGE, which makes the
    non-centroid modes artificially fast in order to sample adiabatically
    the quantum fluctuations, has to be chosen carefully;
    note that FACSTAGE~$= 1/\gamma$ as
    introduced in Ref.~\cite{aicmd} in eq.~(2.51).}

\keyword{CG-ANALYTIC}{}{}{}{\&RESP}
  \desc{The number of steps for which the step length in the conjugate
  gradient optimization is calculated assuming a quadratic functional
  E(2) (quadratic in the linear response vectors). No accuracy impact,
  pure convergence speed tuning.\\ \textbf{Default} value is
  \defaultvalue{3} for NMR and \defaultvalue{99} otherwise.}

\keyword{CG-FACTOR}{}{}{}{\&RESP}
  \desc{The analytic length calculation of the conjugate-gradient step
  lengths yields in general a result that is slightly too large. This
  factor is used to correct for that deficiency. No accuracy impact,
  pure convergence speed tuning.\\ \textbf{Default} is
  \defaultvalue{0.8}.}

\keyword{CHANGE BONDS}{}{}{}{\&ATOMS}
  \desc{The buildup of the empirical Hessian can be affected.\\
    You can either add or delete bonds. The number of changed bonds is read
    from the next line. This line is followed by the description of the bonds.
    The format is \\
      {\sl \{ ATOM1 \ \ ATOM2 \ \ FLAG\} }. \hfill \\
      {\sl ATOM1} and {\sl ATOM2} are the numbers of the atoms involved in the
          bond.
      A {\sl FLAG} of $-1$ causes a bond to be deleted and
      a {\sl FLAG} of $1$ a bond to be added. \hfill\\
      Example: \\
         {\tt
         \begin{tabular}{ccc}
         \multicolumn{3}{l}{\bf CHANGE BONDS}\\
         2 &   &         \\
         1 & 2 & +1      \\
         6 & 8 & -1
         \end{tabular}
         }
      }

\keyword{CHARGES}{}{}{}{\&PROP}
  \desc{Calculate atomic charges. Charges are calculated according to the
    method of Hirshfeld~\cite{Hirshfeld77} and charges derived from the
    electrostatic potential~\cite{Cox84}.}

\keyword{CHARGE}{}{}{}{\&SYSTEM}
  \desc{The total charge of the system is read from the next line.\\
    \textbf{Default} is \defaultvalue{0}.}

\keyword{CHECK MEMORY}{}{}{}{\&CPMD}
  \desc{Check sanity of all dynamically allocated arrays whenever a change in
    the allocation is done. By default memory is checked only at break points.}

\keyword{CHECK SYMMETRY}{[OFF]}{}{}{\&SYSTEM}
  \desc{The precision with which the conformance of the \refkeyword{CELL}
    parameters are checked against the (supercell) \refkeyword{SYMMETRY}
    is read from the next line. With older versions of CPMD, redundant
    variables could be set to arbitrary values; now \textbf{all} values have
    to conform. If you want the old behavior back, you can turn
    the check off by adding the keyword {\bf OFF} or by providing a
    negative precision. \textbf{Default} value is: \defaultvalue{1.0e-4}}

\keyword{CLASSICAL CELL}{}{[ABSOLUTE, DEGREE]}{}{\&SYSTEM}
  \desc{Not documented.}

\keyword{CLASSICAL TEST}{}{}{}{\&PIMD}
  \desc{Test option to reduce the path integral branch to the classical code for
    the special case $P=1$ in order to allow for a one-to-one comparison
    to a run using the standard branch of CPMD.
    It works only with primitive propagator, i.e.\ not
    together with NORMAL MODES, STAGING and/or \refkeyword{DEBROGLIE} CENTROID.}

\keyword{CLASSTRESS}{}{}{}{\&CPMD}
  \desc{Not documented.}

\keyword{CLUSTER}{}{}{}{\&SYSTEM}
  \desc{Isolated system such as a molecule or a cluster. Same effect as
  \refkeyword{SYMMETRY} 0, but allows a non-orthorhombic cell. Only rarely
  useful.}

\keyword{CMASS}{}{}{}{\&CPMD}
  \desc{The fictitious mass of the cell in atomic units is read from the next
    line. \\
    \textbf{Default} value is \defaultvalue{200}}

\keyword{COMBINE SYSTEMS}{}{[REFLECT,NONORTH,SAB]}{}{\&CPMD}
  \desc{Read in two wavefunctions from RESTART.R1 and RESTART.R2 and combine them into 
    RESTART.1 which can then be used in an FODFT calculations. The option NONORTH disables
    orthogonalisation of the combined WF's. Parameters NTOT1, NTOT2, NSUP1, NSUP2 are read
    from the next line.\\
    NTOT1/NTOT2 total number of electrons in state 1/2 (mandatory).\\
    NSUP1/NSUP2 number of alpha electrons in state 1/2 (only LSD).\\
    If the option REFLECT is given a fifth parameter (CM\_DIR) is read and the WF given 
    in RESTART.R2 will be either mirrored through the centre of the box (CM\_DIR=0), 
    mirrored through the central yz-plane of the box (CM\_DIR=1) or if CM\_DIR=4 
    mirrored through the central yz-plane and translated in x direction by CM\_DR 
    (sixth parameter to be read).\\
    If the option SAB is set, write out the overlap matrix element between orbitals 
    K and L. Parameters K and L are read from the next line.\\
    After combining the wavefunctions CPMD will exit. For this option to work the RESTART 
    option and \refkeyword{OPTIMIZE WAVEFUNCTION} have to be activated.}

\keyword{COMPRESS}{}{[WRITEnn]}{}{\&CPMD}
  \desc{Write the wavefunctions with nn bytes precision to the restart file. \\
    Possible choices are \texttt{WRITE32}, \texttt{WRITE16}, \texttt{WRITE8}
    and \texttt{WRITEAO}. \\
    \texttt{WRITE32} corresponds to the compress option in older versions.
    \texttt{WRITEAO} stores the wavefunction as a projection on atomic basis
    sets. The atomic basis set can be specified in the section \&BASIS \ldots
    \&END. If this input section is missing a default basis from Slater type
    orbitals is constructed. See section~\ref{input:basis} for more details.}

\keyword{CONDUCTIVITY}{}{}{}{\&PROP}
  \desc{Computes the optical conductivity according to the
     Kubo-Greenwod formula
     \begin{equation*}
     \sigma(\omega) =
     \frac{2 \pi e^2}{3m^2 V_{\rm cell}} \frac{1}{\omega }
     \sum_{i,j} (f_i-f_j)
     |\langle \psi _i| \hat{\bf p} |\psi _j \rangle |^2
     \delta(\epsilon _i -\epsilon_j - \hbar \omega)
    \label{condu}
    \end{equation*}
   where $\psi _i$ are the Kohn-Sham eigenstates, $\epsilon _i$ their
    corresponding eigenvalues, $f_i$ the occupation number and the
    difference $f_i-f_j$ takes care of the fermionic occupancy.
    This calculation is executed when the keyword PROPERTIES is
    used in the section \&CPMD ... \&END. In the section \&PROP ... \&END
    the keyword CONDUCTIVITY must be present and the interval
    interval $\Delta \omega$ for the calculation of the spectrum
    is read from the next line. Note that, since this is a "PROPERTIES"
    calculation, {\it you must have previously computed the electronic structure
    of your system and have a consistent \refkeyword{RESTART} file ready to use}.
    Further keyword: \texttt{STEP=0.14}, where (e.g.) 0.14 is the bin
    width in eV of the $\sigma(\omega)$ histogram if you want it
    to be different from $\Delta \omega$. A file MATRIX.DAT
    is written in your working directory, where all the non-zero transition
    amplitudes and related informations are reported
    (see the header of MATRIX.DAT). An example of application is given in
    Refs.~\cite{solve,solve2}.}

\keyword{CONFINEMENT POTENTIAL}{}{}{}{\&ATOMS}
  \desc{The use of this label activates a spherical Gaussian confinement 
    potential in the calculation of the form factor of pseudopotentials.
    In the next line(s) two parameters for each atomic species must
    be supplied: the amplitude $\alpha$ and the cut off radius $r_c$.
    The Gaussian spherical amplitude is computed as 
    $A=\pi ^{3/2}r_c^3\cdot \alpha$ and the Gaussian confinement
    potential reads
    \begin{equation*}
    V({\bf G}) = \sum_{\bf G} A \cdot |{\bf G}|\cdot e^{-G^2r_c^2/4}
    \label{pconf}
    \end{equation*}
    being {\bf G} the G-vectors, although in practice the loop runs only 
    on the G-shells $G=|{\bf G}|$.}

\keyword{CONJUGATE GRADIENTS}{}{[ELECTRONS, IONS, NOPRECONDITIONING]}{}{\&CPMD}
  \desc{For the electrons, the keyword is equivalent to \refkeyword{PCG}. The
    \texttt{NOPRECONDITIONING} parameter only applies for electrons. For the
    ions the conjugate gradients scheme is used to relax the atomic positions.}

\keyword{CONSTANT CUTOFF}{}{}{}{\&SYSTEM}
  \desc{Apply a cutoff function to the kinetic energy term~\cite{bernasconi95}
    in order to simulate
    constant cutoff dynamics. The parameters $A$, $\sigma$ and $E_o$ are read
    from the next line (all quantities have to be given in Rydberg).
    $$
      G^2 \to G^2 + A \left[ 1 + \mbox{erf}
      \left( {\frac{1}{2} G^2 -  \frac{E_o}{\sigma}} \right) \right]
    $$
    }

\keyword{CONSTRAINTS ... END CONSTRAINTS}{}{}{}{\&ATOMS}
  \desc{With this option you can specify several constraints and
   restraints on the atoms. (see section~\ref{sec:cnstr} for more
    information on the available options and the input format).}

\keyword{CONVERGENCE}{}{[ADAPT, ENERGY, CALFOR, RELAX, INITIAL]}{}{\&CPMD}
  \desc{The adaptive convergence criteria for the wavefunction during a
    geometry optimization are specified. For more informations,
    see~\cite{LSCAL}. The ratio {\sl TOLAD} between the smallest maximum
    component of the nuclear gradient reached so far and the maximum allowed
    component of the electronic {\bf gradient} is specified with {\bf
    CONVERGENCE ADAPT}. This criterion is switched off once the value {\sl
    TOLOG} given with {\bf CONVERGENCE ORBITALS} is reached. By default, the
    adaptive gradient criterion is not active. A reasonable value for the
    parameter {\sl TOLAD} is 0.02.\\
%
    If the parameter {\sl TOLENE} is given with {\bf CONVERGENCE ENERGY}, in
    addition to the gradient criterion for the wavefunction, the energy change
    between two wavefunction optimization cycles must be smaller than the
    energy change of the last accepted geometry change multiplied by {\sl
    TOLENE} for the wavefunction to be considered converged. By default, the
    adaptive energy criterion is not active. It is particularly useful for {\bf
    transition state search} with P-RFO, where the trust radius is based on the
    quality of energy prediction. A reasonable value for {\sl TOLENE} is
    0.05.\\
%
    To save CPU time, the gradient on the ions is only calculated if the
    wavefunction is almost converged. The parameter {\sl TOLFOR} given with
    {\bf CONVERGENCE CALFOR} is the ratio between the convergence criteria for
    the wavefunction and the criteria whether the gradient on the ions is to be
    calculated. \textbf{Default} value for {\sl TOLFOR} is
    \defaultvalue{3.0}.\\
%
    If the wavefunction is very slowly converging during a geometry
    optimization, a small nuclear displacement can help. The parameter {\sl
    NSTCNV} is given with {\bf CONVERGENCE RELAX}. Every {\sl NSTCNV}
    wavefunction optimization cycles, the convergence criteria for the
    wavefunction are relaxed by a factor of two. A geometry optimization step
    resets the criteria to the unrelaxed values. By default, the criteria for
    wavefunction convergence are never relaxed.\\
%
    When starting a geometry optimization from an unconverged wavefunction, the
    nuclear gradient and therefore the adaptive tolerance of the electronic
    gradient is not known. To avoid the {\bf full convergence} criterion to be
    applied at the beginning, a convergence criterion for the wavefunction of
    the initial geometry can be supplied with {\bf CONVERGENCE INITIAL}.
    By default, the initial convergence criterion is equal to the full
    convergence criterion.}

\spekeyword{CONVERGENCE}{}{[ORBITALS, GEOMETRY, CELL]}{}{\&CPMD}{CONVERGENCE 2}
  \desc{The convergence criteria for optimization runs is specified. \\
%
    The maximum value for the biggest element of the gradient of the
    wavefunction ({\bf ORBITALS}), of the ions ({\bf GEOMETRY}), or the cell
    ({\bf CELL}) is read from the next line. \\
%
    \textbf{Default} values are \defaultvalue{10$^{-5}$} for the wavefunction,
    \defaultvalue{5$\times$10$^{-4}$} for the ions and \defaultvalue{1.0} for
    the cell. For diagonalization schemes the first value is the biggest
    variation of a density component. \textbf{Defaults} are
    \defaultvalue{10$^{-3}$} and \defaultvalue{10$^{-3}$}.}

\spekeyword{CONVERGENCE}{CONSTRAINT}{}{}{\&CPMD}{CONVERGENCE 3}
  \desc{
    Set constraint convergence parameters. Parameters VCCON and VCCONU are read from the next line:\\
    VCCON $\in \mathbb{R}_+$ is the maximally allowed total deviation of
    the constraint from the desired value $\text{N}_\text{c}$.\\
    VCCONU $\in \mathbb{R}_+$ is the upper bound for the deviation in MD runs,
    excess of which triggers a new optimisation of $V$.\\
    \textbf{Defaults} are
    \defaultvalue{10$^{-5}$} and \defaultvalue{VCCON}.}
    
\spekeyword{CONVERGENCE}{}{}{}{\&LINRES}{CONVERGENCE LINRES}
  \desc{Convergence criterion for linear response calculations.\\
%
    \textbf{Default} value is \defaultvalue{10$^{-5}$}.}

\spekeyword{CONVERGENCE}{}{}{}{\&RESP}{CONVERGENCE RESP}
  \desc{
    Convergence criterion on the gradient $\delta E/\delta \psi^*$
%
    \textbf{Default} value is \defaultvalue{10$^{-5}$}.}

\keyword{CORE SPECTRA}{}{}{}{\&PROP}
  \desc{Computes the X-ray adsorption spectrum and related transition
    matrix elements according to Ref.~\cite{xray}.
    This calculation is executed when the keyword PROPERTIES is
    used in the section \&CPMD ... \&END. In the section \&PROP ... \&END
    the keyword CORE SPECTRA must be present and the core atom
    number (e.g. 10 if it is the 10$th$ atom in your list)
    and core level energy (in au) are read from the
    next line, while in the following line the $n$ and $l$ quantum
    numbers of the selected core level, along with the exponential
    factor $a$ of the STO orbital for the core level must be provided.
    In the case of $1s$ states, the core orbital is reconstructed as
    \begin{equation*}
     \psi _{1s}(r) = 2 a^{\frac{3}{2}} r \cdot \exp (-a\cdot r)
    \label{1s}
    \end{equation*}
    and it is this $a$ value in au that must be supplied in input.
    As a general rule, first-row elements in the neutral case have
    the following $a$ values:
    B (4.64), C (5.63), N (6.62), O (7.62). For an excited
    atom these values would be of course a bit larger; e.g. for
    O it is 7.74453, i.e. 1.6 \% larger.
    Since this is a "PROPERTIES" calculation,
    {\it you must have previously computed the electronic structure
    of your system and have a consistent \refkeyword{RESTART} file ready to use}.
    A file XRAYSPEC.DAT is written in your working directory,
    containing all the square transition amplitudes and related informations,
    part of which are also written in the standard output. Waring: in order
    to use this keyword you need special pseudopotentials. These are
    provided, at least for some elements, in the PP library of CPMD and
    are named as *\_HOLE.psp}

\keyword{COULOMB ATTENUATION}{}{}{}{\&DFT}
  \desc{Activates the long-range correction/Coulomb attenuation method (\textbf{LC,
    CAM}). With this keyword, all selected \emph{internal} LDA or GGA exchange
    functionals will be Coulomb-attenuated, and the corresponding exact exchange
    contribution is added by default. The three CAM parameters
    $\alpha$, $\beta$ and $\mu$ are read from the next line. Setting
    $\alpha=0$ and $\beta=1.0$ is equivalent to using the long-range correction (LC)
    method.
    This option is only available in combination with the new \refspekeyword{XC\_DRIVER}{XC DRIVER}.
    The keyword is aliased and may be abbreviated by \textbf{CAM}.}


\keyword{COUPLINGS}{\{FD=$\epsilon$,PROD=$\epsilon$\}}{[NAT]}{}{\&SYSTEM}
  \desc{Calculate non-adiabatic couplings~\cite{nonadiabatic} using finite
    differences (FD and PROD are two different finite-difference
    approximations). The displacement $\epsilon$ is expected in atomic units.
    If NAT=$n$ is given, the coupling vector acting on only a subset of $n$
    atoms is calculated. In this case, a line containing $n$ atom sequence
    numbers is expected.
    See \refkeyword{COUPLINGS NSURF}.}

\keyword{COUPLINGS LINRES}{\{BRUTE FORCE,NVECT=$n$\}}{[THR,TOL]}{}{\&SYSTEM}
  \desc{Calculate non-adiabatic couplings~\cite{nonadiabatic} using
    linear-response theory. With BRUTE FORCE, the linear response to the
    nuclear displacements along all Cartesian coordinates is calculated.
    With NVECT=$n$, at most $n$ cycles of the iterative scheme in
    \cite{nonadiabatic} are performed. However, the iterative calculation is
    also stopped earlier if its contribution to the non-adiabatic coupling
    vector is smaller a given tolerance (TOL=$C_{\mathrm{tol}}$).
    In the case of the iterative scheme, also the option THR can be given,
    followed by three lines each containing a pair of a threshold contribution
    to the non-adiabatic coupling vector and a tolerance for the
    linear-response wavefunction (see \cite{nonadiabatic}).
    Do not forget to include a \&LINRES section in the input, even if the
    defaults are used.
    See \refkeyword{COUPLINGS NSURF}.}

\keyword{COUPLINGS NSURF}{}{}{}{\&SYSTEM}
  \desc{Required for non-adiabatic couplings: the Kohn-Sham states involved in
    the transition. For the moment, only one pair of states makes sense,
    NSURF=1. On the following line, the orbital numbers of the two Kohn-Sham
    states and a weight of 1.0 are expected. For singlet-singlet transitions,
    the ROKS-based Slater transition-state density
    (\refkeyword{LOW SPIN EXCITATION LSETS}) should be used. For
    doublet-doublet transitions, the local spin-density approximation
    (\refkeyword{LSD}) with the occupation numbers (\refkeyword{OCCUPATION},
    \refkeyword{NSUP}, \refkeyword{STATES}) of the corresponding Slater
    transition-state density should be used.}

\spekeyword{CP\_GROUPS}{}{}{}{\&CPMD}{CP GROUPS}
 \desc{ Set the number of groups to be used in the calculation.
   Default is 1 group. The number of groups is
   read from the next line and shall be a
   divisor of the number of nodes in a parallel run.}

\spekeyword{CP\_LIBRARY\_ONLY}{}{}{}{\&DFT}{CP LIBRARY ONLY}
 \desc{Use the new xc driver, but use only functionals that are available internally
       (this excludes libxc).
       \emph{Cf.{}} \refspekeyword{XC\_DRIVER}{XC DRIVER}.}
 
\keyword{CUBECENTER}{}{}{}{\&PROP}
  \desc{Sets the center of the cubefiles produced by the
  \refkeyword{CUBEFILE} flag. The next line has to contain
   the coordinates of the center in Bohr or Angstrom, depending
   on whether the \refkeyword{ANGSTROM} keyword was given.
   \textbf{Default} is the geometric center of the system.}

\keyword{CUBEFILE}{ORBITALS,DENSITY}{HALFMESH}{}{\&PROP}
  \desc{Plots the requested objects in .CUBE file format. If ORBITALS
  are demanded, the total number as well as the indices have to be
  given on the next and second next line. HALFMESH reduces the number
  of grid points per direction by 2, thus reducing the file size by a factor of 8.}

\keyword{CUTOFF}{}{[SPHERICAL,NOSPHERICAL]}{}{\&SYSTEM}
  \desc{The {\bf cutoff} for the plane wave basis in {\bf Rydberg} is
      read from the next line.
      The keyword {\bf SPHERICAL} is used with k points in order to have
      $|g + k|^2 < E_{cut}$ instead of $|g|^2 < E_{cut}$.
      This is the default.}

\keyword{CZONES}{}{[SET]}{}{\&CPMD}
  \desc{Activates convergence zones for the wavefunction during the 
      \refkeyword{CDFT} constraint minimisation. If SET is set the 
      parameters CZONE1, CZONE2, and CZONE3 are read from the next 
      line and CZLIMIT1 and CZLIMIT2 from the line after.\\
      CZONE1 \defaultvalue{$10^{-3}$},CZONE2 \defaultvalue{$10^{-4}$},CZONE3 \defaultvalue{$10^{-5}$}
      $\in \mathbb{R}_+$ are the orbital convergences in zones 1-3, respectively.\\
      CZLIMIT1 \defaultvalue{0.3}, CZLIMIT2 \defaultvalue{0.1} $\in \mathbb{R}_+$ define
      the boundaries between zone 1-2 and 2-3, respectively.}
      
\keyword{DAMPING}{\{IONS,ELECTRONS,CELL\}}{}{}{\&CPMD}
  \desc{Add a damping factor $f_{damp}(x) = - \gamma \cdot v(x)$ to 
    the ionic, electronic, or cell forces in every time step. The 
    scaling factor $\gamma$ is read from the next line. Useful 
    values depend on the employed masses are generally in the 
    range $5.0 \to 50.0$.

    Damping can be used as a more efficient alternative to 
    \refkeyword{ANNEALING} for wavefunction, geometry or cell 
    optimization (and particularly combinations thereof) of 
    systems where the faster methods 
    (e.g. \refkeyword{ODIIS}, \refkeyword{PCG}, \refkeyword{LBFGS}, 
    \refkeyword{GDIIS}) fail to converge or may converge to the 
    wrong state. 
  }

\keyword{DAVIDSON DIAGONALIZATION}{}{}{}{\&CPMD}
  \desc{Use Davidson diagonalization scheme.\cite{davidson75}}

\keyword{DAVIDSON PARAMETER}{}{}{}{\&CPMD}
  \desc{This keyword controls the Davidson diagonalization routine used to
    determine the Kohn-Sham energies. \\
%
    The maximum number of additional vectors to construct the Davidson matrix,
    the convergence criterion and the maximum number of steps are read from the
    next line.\\
%
    \textbf{Defaults} are \defaultvalue{10$^{-5}$} and the same number as
    states to be optimized. If the system has 20 occupied states and you ask
    for 5 unoccupied states, the default number of additional vectors is 25. By
    using less than 25 some memory can be saved but convergence might be
    somewhat slower.}

\spekeyword{DAVIDSON PARAMETER}{}{}{}{\&TDDFT}{DAVIDSON PARAMETER TDDFT}
  \desc{The maximum number of Davidson iterations, the convergence criteria for
    the eigenvectors and the maximal size of the Davidson subspace are set. The
    three parameters {\sl ndavmax, epstdav, ndavspace} are read from the next
    line.\\
%
    \textbf{Default} values are \defaultvalue{100}, \defaultvalue{10$^{-10}$}
    and \defaultvalue{10}.}

\keyword{DAVIDSON RDIIS}{}{}{}{\&TDDFT}
  \desc{This keyword controls the residual DIIS method for TDDFT diagonalization.
        This method is used at the end of a DAVIDSON diagonalization for
        roots that are not yet converged. The first number gives the maximum
        iterations, the second the maximum allowed restarts, and the third
        the maximum residual allowed when the method is invoked.\\
%
    \textbf{Default} values are \defaultvalue{20}, \defaultvalue{3}
    and \defaultvalue{$10^{-3}$}.}

\keyword{DEBROGLIE}{}{[CENTROID]}{}{\&PIMD}
  \desc{An initial configuration assuming quantum free particle behavior is
    generated for each individual atom according to its physical mass at the
    temperature given in Kelvin on the following input line.
%
    Using DEBROGLIE each nuclear position obtained from the \&ATOMS
    \ldots\ \&END section serves as the starting point for a Gaussian
    L\'evy walk of length $P$ in three dimensions, see e.g.\
    Ref.~\cite{Fosdick66}.
%
    Using DEBROGLIE CENTROID each nuclear position obtained from the \&ATOMS
    \ldots\ \&END section serves as the centroid (center of geometry) for
    obtaining the centroid (center of geometry) for obtaining the $P$ normal
    modes in three dimensions, see e.g.\ Ref.~\cite{Tuckerman96}.
%
    This option does only specify the generation of the initial configuration
    if INITIALIZATION and GENERATE REPLICAS are active.
%
    Default is DEBROGLIE CENTROID and 500~Kelvin.}

\keyword{DCACP}{}{}{}{\&CPMD}
  \desc{Activates the DCACP scheme ~\cite{opt-ecp04} (dispersion-corrected atom-centred potentials)
        in order to account for van der Waals-interactions. Advanced settings can
        be specified in the \&VDW section.}

\keyword{DCACP Z=z}{}{}{}{\&VDW}
  \desc{Set custom DCACP parameters for the element with core charge $Z=$\texttt{z}.
        The values of the DCACP projectors $\sigma_1$ and $\sigma_2$ are read from the
        next line.\\
        The keyword can be repeated multiple times, once for every element.
        Useful if the desired parameters are not available in the internally coded
        library, \emph{e.g.{}} for an exotic xc functional.}

\keyword{DEBUG CODE}{}{}{}{\&CPMD}
  \desc{Very verbose output concerning subroutine calls for debugging purpose.}

\keyword{DEBUG FILEOPEN}{}{}{}{\&CPMD}
  \desc{Very verbose output concerning opening files for debugging purpose.}

\keyword{DEBUG FORCES}{}{}{}{\&CPMD}
  \desc{Very verbose output concerning the calculation of each contribution
        to the forces for debugging purpose.}

\keyword{DEBUG MEMORY}{}{}{}{\&CPMD}
  \desc{Very verbose output concerning memory for debugging purpose.}

\keyword{DEBUG NOACC}{}{}{}{\&CPMD}
  \desc{Do not read/write accumulator information from/to the 
        \refkeyword{RESTART} file.
        This avoids putting timing information to the restart and makes
        restart files identical for otherwise identical runs.}

\keyword{DENSITY CUTOFF}{}{[NUMBER]}{}{\&SYSTEM}
  \desc{Set the plane wave energy cutoff for the density. The value is read
    from the next line. The density cutoff is usally automatically determined 
    from the wavefunction \refkeyword{CUTOFF} via the \refkeyword{DUAL} factor.\\
    With the additional flag {\bf NUMBER} the number of plane waves can be specified
    directly. This is useful to calculate bulk modulus or properties depending on 
    the volume. The given energy cutoff has to be bigger than the one to have 
    the required plane wave density number.}

\keyword{DIAGONALIZER}{\{DAVIDSON,NONHERMIT,PCG\}}{[MINIMIZE]}{}{\&TDDFT}
  \desc{Specify the iterative diagonalizer to be used.\\
%
    \textbf{Defaults} are {\sl DAVIDSON} for the Tamm--Dancoff method, {\sl
    NONHERMIT} (a non-hermitian Davidson method) for TDDFT LR and {\sl PCG}
    (Conjugate gradients) for the optimized subspace method. The additional
    keyword {\sl MINIMIZE} applies to the PCG method only. It forces a line
    minimization with quadratic search.\\
%
    \textbf{Default} is \defaultvalue{not to use line minimization}.}

\keyword{DIAGONAL}{}{[OFF]}{}{\&HARDNESS}
  \desc{Not documented}

\keyword{DIFF FORMULA}{}{}{}{\&LINRES}
  \desc{Number of points used in finite difference formula for second
    derivatives of exchange--correlation functionals. Default is two point
    central differences.}

\keyword{DIIS MIXING}{}{}{}{\&CPMD}
  \desc{Use the direct inversion iterative scheme to mix density.\\
%
    Read in the next line the number of previous densities (NRDIIS) for the
    mixing (however not useful).}

\spekeyword{DIIS MIXING}{}{[$N=n$]}{}{\&CPMD}{DIIS MIXING N=n}
  \desc{Like DIIS MIXING, but number of previous densities for the
    mixing can be specified as a function of the density.\\
%
    $n$ different thresholds for the density can be set. The program reads $n$
    lines with a threshold density and a NRDIIS number (number of previous
    densities for the mixing). Numbers NRDIIS have to increase. If the NRDIIS
    is equal to 0, Anderson mixing is used. Very efficient is to use Anderson
    mixing and afterwards DIIS mixing.}

\keyword{DIPOLE DYNAMICS}{\{SAMPLE,WANNIER\}}{}{}{\&CPMD}
  \desc{Calculate the dipole moment~\cite{vdb_berry,resta} every {\sl NSTEP} iteration in MD.\\
%
    {\sl NSTEP} is read from the next line if the keyword SAMPLE is present.\\
%
    {\bf Default} is {\bf every} time step.\\
%
    The keyword {\bf Wannier} allows the calculation of optimally localized
    Wannier functions\cite{Marzari97,PSil99,berghold}. The procedure used is
    equivalent (for single k-point) to Boys localization.

    The produced output is IONS+CENTERS.xyz, IONS+CENTERS,
    DIPOLE, WANNIER\_CENTER and WANNIER\_DOS. The localization procedure is
    controlled by the following keywords.}

\keyword{DIPOLE MOMENT}{}{[BERRY, RS]}{}{\&PROP}
  \desc{Calculate the dipole moment.\\
%
    Without the additional keywords {\bf BERRY} or {\bf RS}
    this is only implemented for simple cubic and fcc supercells.
    The keyword {\bf RS} requests the use of the real-space algorithm.
    The keyword {\bf BERRY} requests the use of the Berry phase algorithm.\\
%    
    {\bf Default} is to use the real-space algorithm.}

\keyword{DISCARD}{}{[OFF, PARTIAL, TOTAL, LINEAR]}{}{\&RESP}
  \desc{Request to discard trivial modes in vibrational analysis from linear response (both
\refkeyword{PHONON} and \refkeyword{LANCZOS}).\\
  
  {\bf OFF} = argument for performing no projection.\\
  {\bf PARTIAL} = argument for projecting out only translations (this is the default).\\
  {\bf TOTAL} = argument for projecting both rotations and translations.\\
  {\bf LINEAR} = argument for projecting rotations around the $C - \infty$ axis in a linear molecule (not implemented yet).}

\keyword{DISTRIBUTED LINALG}{\{ON, OFF\}}{}{}{\&CPMD}
  \desc{Perform linear algebra calculations using distributed memory algorithms.
        Setting this option ON will also enable (distributed) initialization from atomic wavefunctions
        using a parallel Lanczos algorithm \cite{distrib.lanczos.07}. Note that distributed initialization
        is not available with {\bf KPOINTS} calculations. In this case, initialization from atomic wavefunctions
        will involve replicated calculations.

        When setting {\bf LINALG ON} the keyword  \refkeyword{BLOCKSIZE STATES} becomes relevant (see entry). The number of
         \refkeyword{BLOCKSIZE STATES} must be an {\bf exact divisor} of the number of  \refkeyword{STATES}.}

\keyword{DISTRIBUTE FNL}{}{}{}{\&CPMD}
  \desc{The array \texttt{FNL} is distributed in parallel runs.}

\keyword{DONOR}{}{}{}{\&SYSTEM}
  \desc{Set the \refkeyword{CDFT} donor atoms. Parameter NACCR must be 
    specified next to the keyword. NDON $\in \mathbb{R}_+$ is the number of Donor 
    Atoms ($N$ being the total number of atoms).  If NDON$>0$ the indices of NDON atoms 
    separated by whitespace are read from the next line else only use an
    Acceptor group in the CDFT weight.}

\keyword{DUAL}{}{}{}{\&SYSTEM}
  \desc{The ratio between the wavefunction energy \refkeyword{CUTOFF} 
    and the \refkeyword{DENSITY CUTOFF} is read from the next line. \\
%
    {\bf Default} is {\bf 4}.\\
%
    There is little need to change this parameter, except when
    using ultra-soft pseudopotentials, where the wavefunction
    cutoff is very low and the corresponding density cutoff
    is too low to represent the augmentation charges accurately.
    In order to maintain good energy conservation and have
    good convergence of wavefunctions and related parameters,
    {\bf DUAL} needs to be increased to values of 6--10. \\
    Warning: You can have some trouble if you use the {\bf DUAL} 
    option with the symmetrization of the electronic density.}

\keyword{DUMMY ATOMS}{}{}{}{\&ATOMS}
  \desc{The definition of dummy atoms follows this keyword.\\
      Three different kinds of dummy atoms are implemented.
      Type 1 is fixed in space, type 2 lies at the arithmetic
      mean, type 3 at the center of mass of the coordinates 
      of real atoms. \\
% For types 2, 3 and 4 you can also have a
%      negative weight (NOTE: works only for restraints).\\
      The first line contains the total number of dummy atoms.
      The following lines start with the type label {\bf TYPE1, TYPE2, TYPE3, TYPE4}.\\
      For type 1 dummy atoms the label is followed by the Cartesian
      coordinates. \\
      For type 2 and type 3 dummy atoms the first number
      specifies the total number of atoms involved in the
      definition of the dummy atom. Then the number of these atoms
      has to be specified on the same line. 
      A negative number of atoms stands for all atoms. 
      For type 4, the dummy atom is defined as a weighed average
      of coordinates of real atoms with user-supplied weights. This
      feature is useful e.~g.\@ in constrained dynamics, because
      allows to modify positions and weights of dummy atoms according 
      to some relevant quantity such as forces on selected atoms.
% A negative atom index means that a negative weight is assigned 
% to this atom (works only with restraints) \\
      Example: \\
      
      {\tt
      \begin{tabular}{llll}
      \multicolumn{4}{l}{\bf DUMMY ATOMS }\\
      3           &     &     &          \\
      {\bf TYPE1} & 0.0 & 0.0 & 0.0      \\
      {\bf TYPE2} & 2   & 1   & 4        \\
      {\bf TYPE3} & -1
      \end{tabular}
      }\\
      
      Note: Indices of dummy atoms always start with total-number-of-atoms plus 1. 
      In the case of a Gromos-QM/MM interface simulations with dummy hydrogen atoms 
      for capping, it is total-number-of-atoms plus number-of-dummy-hydrogens plus 1 }

\keyword{EIGENSYSTEM}{}{}{}{\&RESP}
  \desc{Not documented.}
  
\keyword{ELECTRONIC SPECTRA}{}{}{}{\&CPMD}
  \desc{Perform a TDDFT calculation~\cite{tddft_all,tddft_pw}
      to determine the electronic spectra. See below under
      \referto{sec:ElectronicSpectra}{Electronic Spectra} and
      under the other keywords for the input sections
      \referto{inputkw:linres}{\&LINRES} and
      \referto{inputkw:tddft}{\&TDDFT} for further options.}

\keyword{ELECTROSTATIC POTENTIAL}{}{[SAMPLE=nrhoout]}{}{\&CPMD}
  \desc{Store the electrostatic potential on file. The resulting file
  is written in platform specific binary format. You can use the
  cpmd2cube tool to convert it into a Gaussian cube file for 
  visualization. Note that this flag automatically activates
  the \refkeyword{RHOOUT} flag as well.

  With the optional keyword {\bf SAMPLE} the file will be written
  every {\em nrhoout} steps during an MD trajectory. The corresponding
  time step number will be appended to the filename.
}

\keyword{ELF}{}{[PARAMETER]}{}{\&CPMD}
  \desc{Store the total valence density and the valence electron localization
      function ELF~\cite{Becke90,silsav,marx-savin-97,homeofelf} on files.
      The default
      smoothing parameters for ELF can be changed optionally
      when specifying in addition the PARAMETER keyword.
      Then the two parameters ``elfcut'' and
      ``elfeps'' are read from the next line.
      The particular form of ELF that is implemented is defined
      in the header of the subroutine elf\_utils.mod.F90.
      
      Note 1: it is a {\em very} good idea to increase
      the planewave cutoff and then specify
      ``elfcut''~$=0.0$ and ``elfeps''~$=0.0$
      if you want to obtain a smooth ELF for a given nuclear configuration.
      In the case of a spin--polarized (i.e. spin unrestricted)
      DFT calculation (see keyword \refkeyword{LSD}) in addition
      the spin--polarized average of ELF as well as the
      separate $\alpha$-- and $\beta$--orbital parts are written to the files
      LSD\_ELF, ELF\_ALPHA and ELF\_BETA, respectively;
      see Ref.~\cite{Kohut96} for definitions and further infos.
      
      Note 2: ELF does not make much sense when using Vanderbilt's
      ultra-soft pseudopotentials!}

\keyword{EMASS}{}{}{}{\&CPMD}
  \desc{The fictitious electron mass in atomic units is
      read from the next line. \\
      {\bf Default} is {\bf 400 a.u.}.}

\keyword{ENERGY PROFILE}{}{}{}{\&SYSTEM}
  \desc{Perform an energy profile calculation at the end of a
       wavefunction optimization using the ROKS or ROSS methods.}

\keyword{ENERGYBANDS}{}{}{}{\&CPMD}
  \desc{Write the band energies (eigenvalues) for k points in the file
      ENERGYBANDS. }

% MPB: No more include files (has not eneded up in response_p_utils, though)
%\keyword{EPR}{}{}{options, see response\_p.inc}{\&RESP}
 \keyword{EPR}{}{}{options}{\&RESP}
  \desc{Calculate the EPR $g$ tensor for the system. This routine
  accepts most, if not all, of the options available in the NMR
  routine (RESTART, NOSMOOTH, NOVIRTUAL, PSI0, RHO0, OVERLAP and
  FULL). Most important new options are: \\
  {\bf FULL SMART}: does a calculation with improved accuracy.
  A threshold value (between 0 and 1) must be present on the next line.
  The higher the threshold value, the lower the computational cost, but this
  will also reduce the accuracy (a bit). Typically, a value of 0.05
  should be fine.\\
  {\bf OWNOPT}: for the calculation of the $g$ tensor, an effective
  potential is needed. By default, the EPR routine uses the local potential
  ($V_{LOC} = V_{PP,LOC} + V_{HARTREE} + V_{XC}$).
  This works well with Goedecker pseudopotentials, but rather poor with
  Troullier-Martins pseudopotentials. When using this option, the following
  potential is used instead:
      $$
      V_{EFF} = -\frac{Z}{r}\mathrm{erf}(r/r_c) + V_{HARTREE} + V_{XC}
      $$
  and $r_c$ (greater than 0) is read on the next line.\\
  {\bf HYP}: calculates the hyperfine tensors. See epr\_hyp\_utils.mod.F90 for details.\\

  Contact Reinout.Declerck@UGent.be should you require further information.}

\keyword{EXCHANGE CORRELATION TABLE}{}{[NO]}{}{\&DFT}
  \desc{Specifies the range and the  granularity of the lookup table
      for the local exchange-correlation energy and potential.
      The number of table entries and the maximum
      density have to be given on the next line. \\
      Note that this keyword is only relevant when using
      \refkeyword{OLDCODE} and
      even then it is set to \textbf{NO} be default.
      Previous default values were 30000 and 2.0.}

\keyword{EXCITED DIPOLE}{}{}{}{\&PROP}
  \desc{Calculate the difference of dipole moments
      between the ground state density and a density generated by
      differently occupied Kohn-Sham orbitals. \\
      On the next line
      the number of dipole moments to calculate and the total number
      orbitals has to be given. On the following lines the occupation
      of the states for each calculation has to be given. By default
      the dipoles are calculated by the method used for the
      {\bf DIPOLE MOMENT} option and the same restrictions apply.
      If the {\bf LOCAL DIPOLE} option is specified the dipole
      moment differences are calculated within the same boxes.}

\keyword{EXTERNAL POTENTIAL}{\{ADD\}}{}{}{\&CPMD}
  \desc{Read an external potential from file. With ADD specified, its effects
    is added to the forces acting on the ions.}

\spekeyword{EXT\_PULSE}{}{}{}{\&PTDDFT}{EXT-PULSE}
  \desc{A Dirac type pulse is applied to the KS orbitals before propagation.
This keyword is used with the principal keyword \refkeyword{PROPAGATION SPECTRA}. The intensity
of the perturbing field is read from the next line.}

\spekeyword{EXT\_POTENTIAL}{}{}{}{\&PTDDFT}{EXT-POTENTIAL}
  \desc{An external potential is applied to the KS orbitals before propagation.
This keyword is used with the principal keyword {\bf{MOLECULAR DYNAMICS EH}} or \refkeyword{PROPAGATION SPECTRA}. 
The type of the perturbation is specified with the keyword {\bf{PERT\_TYPE}} (in \&PTDDFT) and its intensity
with {\bf{PERT\_AMPLI}} (in \&PTDDFT).}

\keyword{EXTERNAL FIELD}{}{}{}{\&SYSTEM}
  \desc{Applies an external electric field to the system using the Berry phase. The electric field vector in AU is read from the next line.}

\keyword{EXTPOT}{}{}{}{\&TDDFT}
  \desc{Non adiabatic (nonadiabatic, non-adiabatic) Tully's trajectory surface hopping dynamics using TDDFT energies and forces,
        coupled with an external field~\cite{tavernelli2010}.
        To be used together with the keywords \refkeyword{MOLECULAR DYNAMICS} BO,
        \refkeyword{TDDFT} in the \&CPMD section, and \refkeyword{T-SHTDDFT} in the \&TDDFT section.
        Do NOT use the keyword \refkeyword{T-SHTDDFT} together with the keyword
        \refkeyword{SURFACE HOPPING} in \&CPMD, which invokes the SH scheme based on \refkeyword{ROKS}~\cite{surfhop}
        (see \refkeyword{SURFACE HOPPING}).\\
        This keyword follow the same principle as described for the keyword \refkeyword{T-SHTDDFT}, except that, 
        in the present dynamics, the trajectory starts on the ground state and is coupled with an external field
        through the equations of motion for the amplitudes of Tully's trajectory surface hopping.
        According to the evolution of the amplitudes of the different excited states, the running trajectory can
        jump on an excited state. From there, deactivation through nonradiative processes is possible, within the normal
        trajectory surface hopping scheme.\\
        Parameter \textit{aampl}, \textit{adir}, \textit{afreq}, and \textit{apara1} are read from the next line.
        The amplitude of the vector potential is provided in \textit{aampl} and its polarization is given in \textit{adir} (1 = x-polarized,
        2 = y-polarized, 3 = z-polarized, 4 = all components). The keyword \textit{afreq} gives the frequency of the field and \textit{apara1} is a
        free parameter for a specific user-specified pulse.\\
        Important points: the applied electromagnetic field needs to be hard coded in the subroutine sh\_tddft\_utils.mod.F90, in the subroutine SH\_EXTPOT.
        The vector potential is used for the coupling with the amplitudes equations. Be careful to use a time step small enough for a correct
         description of the pulse. The pulse is printed in the file SH\_EXTPT.dat (step, A(t), E(t)).
       }

\keyword{EXTRAPOLATE WFN}{\{STORE\}}{}{}{\&CPMD}
  \desc{Read the number of wavefunctions to retain from the next line. \\
    These wavefunctions are used to extrapolate the initial guess wavefunction in
    Born-Oppenheimer MD. This can help to speed up BO-MD runs significantly by
    reducing the number of wavefunction optimization steps needed through
    two effects: the initial guess wavefunction is much improved and also
    you do not need to converge as tightly to conserve energy. BO-MD without 
    needs CONVERGENCE ORBITALS to be set to 1.0e-7 or smaller to maintain 
    good energy conservation.\\
    With the additional keyword {\bf STORE} the wavefunction history is also
    written to restart files. See \refkeyword{RESTART} for how to read it back.
    }
    
\keyword{EXTRAPOLATE CONSTRAINT}{}{}{}{\&CPMD}
  \desc{In a CDFT MD run extrapolate the next value for $V$ using a Lagrange polynomial.
    The order $k$ of the polynomial is read from the next line. { \bf Default} is \defaultvalue{k=5}, but
    it pays off to use the orderfinder.py python script on the ENERGIES file of
    a former run to estimate the optimal extrapolation order $k_\text{opt}$.}
    
\keyword{FACMASS}{}{}{}{\&PIMD}
  \desc{Obtain the fictitious nuclear masses $M_I^\prime$
        within path integral molecular
        dynamics from the real physical atomic masses $M_I$
        (as tabulated in the DATA ATWT / \ldots /  statement in atoms\_utils.mod.F90)
        by {\em multiplying} them with the dimensionless factor WMASS
        that is read from the following line;
        if the NORMAL MODES or STAGING propagator is used
        obtain $M_I^{\prime (s)}= \mbox{WMASS} \cdot M_I^{(s)}$
        for {\em all} replicas $s=1, \dots , P$;
        see e.g. Ref.~\cite{aicmd} eq.~(2.37) for nomenclature. \\
        \textbf{Default} value of WMASS is \defaultvalue{1.0}}

\keyword{FACTOR}{}{}{}{\&PATH}
  \desc{Step for propagating string (see \cite{Eijnden06}).\\
    \textbf{Default} value is \defaultvalue{1.0}}

% MPB: No more FFTW
% \keyword{FFTW WISDOM}{}{[ON,OFF]}{}{\&CPMD}
%   \desc{ Controls the use of the ``wisdom'' facility when using 
%     FFTW or compatible libraries. When enabled, CPMD will switch 
%     to using the ``measure'' mode, which enables searching for 
%     additional runtime optimizations of the FFT. The resulting 
%     parameters will be written to a file called {\sl FFTW\_WISDOM} and 
%     re-read on subsequent runs. The parameters in the file are 
%     machine specific and when moving a job to a different machine, 
%     the file should be deleted. 
% 
%     The use of fftw wisdom incurs additional overhead and thus may
%     lead to slower execution. It is recommended to stick with the 
%     default settings unless you know what you are doing.
% }

\keyword{FILE FUSION}{}{}{}{\&CPMD}
  \desc{ Reads in two separate \refkeyword{RESTART} files for ground state and 
      \refkeyword{ROKS} excited state and writes them into a single restart file.\\
       Required to start \refkeyword{SURFACE HOPPING} calculations.}

\keyword{FILEPATH}{}{}{}{\&CPMD}
  \desc{The path to the files written by CPMD (RESTART.x,
      MOVIE, ENERGIES, DENSITY.x etc.) is read from the next line.
      This overwrites the value given in the environment variable
      {\bf CPMD\_FILEPATH}.
      {\bf Default} is the {\bf current directory}.}

\keyword{FINITE DIFFERENCES}{}{}{}{\&CPMD}
  \desc{The step length in a finite difference
      run for vibrational frequencies
      ({VIBRATIONAL ANALYSIS} keywords)
      is read from the next line.\\
      With the keywords {\bf COORD=}{\sl coord\_fdiff(1..3)}
      and {\bf RADIUS=}{\sl radius} put in the same line as
      the step length, you can specify a sphere in order to
      calculate the finite differences only for the atoms inside it.
      The sphere is centered on the position {\sl coord\_fdiff(1..3)}
      with a radius {\sl radius} (useful for a point defect).\\

      \textbf{NOTE:} The the step length for the finite difference
      is \textbf{always} in Bohr (default is 1.0d-2 a.u.).
      The (optional) coordinates of the center and the radius are
      read in either Angstrom or Bohr, depending on whether the
      \refkeyword{ANGSTROM} keyword is specified or not.
  }

\keyword{FIXRHO UPWFN}{}{[VECT LOOP WFTOL]}{}{\&CPMD}
  \desc{Wavefunctions optimization with the method of direct inversion
      of the iterative subspace (DIIS), performed without updating the
      charge density at each step.
      When the orbital energy gradients are below the given
      tolerance or when the maximum number of iterations is reached,
      the KS energies and the
      occupation numbers are calculated, the density is updated,
      and a new wavefunction optimization is started.
      The variations of the density coefficients are used as
      convergence criterion. The default electron temperature is
      1000 K and 4 unoccupied states are added.
      Implemented also for k-points.
      Only one sub-option is allowed per line and the respective parameter
      is read from the next line. The parameters mean:\\
      \hfill\smallskip{\sl VECT}:
                       \hfill\begin{minipage}[t]{10cm}
                       The number of DIIS vectors is read from the next line.
                       (ODIIS with 4 vectors is the default).
                       \end{minipage}\hfill

      {\sl LOOP:} \hfill\begin{minipage}[t]{10cm}
                       the minimum and maximum number of DIIS iterations
                       per each wfn optimization is read from the following
                       line. Default values are 4 and 20.
                      \end{minipage}\hfill

      {\sl WFTOL:} \hfill\begin{minipage}[t]{10cm}
                       The convergence tolerance for the wfn optimization
                       during the ODIIS is read from the following line.
                       The default value is $10^{-7}$. The program
                       adjusts this criterion automatically, depending
                       on the convergence status of the density.
                       As the density improves (when the density updates
                       become smaller), also the
                       wavefunction convergence criterion
                       is set to its final value.
                      \end{minipage}\hfill
      }

\keyword{FORCE FIELD ... END FORCE FIELD}{}{}{}{\&CLASSIC}
  \desc{}
  
%_FM[  
\keyword{FORCEMATCH}{}{}{}{\&CPMD}
  \desc{Activates the QM/MM force matching procedure. This keywords requires the presence 
  of a \&QMMM ... \&END section with a corresponding \refkeyword{FORCEMATCH ... END FORCEMATCH} block. See sections~\ref{sec:qmmm} and~\ref{sec:forcematch-desc} for more details.} 
%_FM]

\keyword{FORCE STATE}{}{}{}{\&TDDFT}
  \desc{The state for which the forces are calculated is read from the
       next line. Default is for state 1.}

\keyword{FREE ENERGY FUNCTIONAL}{}{}{}{\&CPMD}
  \desc{Calculates the electronic free energy using
      free energy density functional~\cite{Alavi94,PSil,mbaops}
      from DFT at finite temperature.\\
      This option needs additional keywords (free energy keywords).\\
      By {\bf default} we use {\bf Lanczos diagonalization} with
      {\bf Trotter factorization} and {\bf Bogoliubov correction}.
      If the number of states is not specified,
      use $N_{electrons}/2+4$.}

\keyword{FREEZE QUANTUM}{}{}{}{\&CLASSIC}
  \desc{Freeze the quantum atoms and performs a classical MD on the
        others (in QMMM mode only !).}

\keyword{FULL TRAJECTORY}{}{}{}{\&CLASSIC}
  \desc{Not documented}

\keyword{FUNCTIONAL}{}{}{functional\_1, functional\_2, \dots}{\&DFT}
  \desc{The xc functional(s) are specified on the same line, separated by a space.
      \\~\\
      \textbf{XC\_DRIVER}:\\
      Both functionals from the internal CP library and those linked from
      libxc are available within the same run and can be freely mixed when using the
      \refspekeyword{XC\_DRIVER}{XC DRIVER}.
      Please note that the internal CP functionals offer a considerable performance
      advantage over libxc.
      While the code will default to the internal implementations,
      use of libxc for single functionals can be requested by the keyword \refkeyword{LIBRARY}.
      % MPB: No space for this
      % or may be forced for all the functionals by using \refspekeyword{LIBXC\_ONLY}{LIBXC ONLY}.
      % (Similarly, the use of libxc can be specifically excluded by using
      % \refspekeyword{CP\_LIBRARY\_ONLY}{CP LIBRARY ONLY}.)
      The functional code names for the functionals available internally follow the nomenclature
      introduced by libxc: The functional name must be preceeded by a tag specifying its level
      (hybrid, MGGA, GGA...). B3LYP would therefore be specified as HYB\_GGA\_XC\_B3LYP. \\
      Fixed (hybrid) exchange-correlation functionals available in the internal CP library include:
      \\~\\
      \textbf{GGA\_XC} \_BLYP, \_BP86, \_HCTH\_93, \_HCTH\_120, \_HCTH\_147, \_HCTH\_407, \_OLYP, \_OPBE, \_PBE, \_PBE\_SOL, \_REVPBE
      \\
      \textbf{MGGA\_XC} \_TPSS, \_M06\_L, \_REVM06\_L, \_M11\_L, \_MN12\_L, \_MN15\_L
      \\
      \textbf{HYB\_GGA\_XC} \_B3LYP, \_CAM\_B3LYP,  \_O3LYP, \_PBE0, \_PBE0\_SOL, \_REVPBE0
      \\
      \textbf{HYB\_MGGA\_XC} \_M05, \_M05\_2X, \_M06, \_M06\_2X, \_M06\_HF, \_M08\_SO, \_M08\_HX, \_M11, \_MN12\_SX
      \\~\\
      \textbf{Important note on Minnesota functionals:}
      \\
      At the usual plane wave cutoff values, the Minnesota functionals require unusually dense integration grids.
      In order to avoid introducing noise, this should be achieved by increasing the value of the \refkeyword{DUAL}.
      For usage recommendations and more details: Bircher, Lopez-Tarifa and Rothlisberger\cite{bircher_rothlisberger19},
      DOI: 10.1021/acs.jctc.8b00897.
      \\~\\
      % MPB: No space for this on one page
      % Please note that, for your convenience, some of these combinations have been hard-coded in the internal library only,
      % they may not be available directly in libxc.
      % \\~\\
      The following exchange and correlation functionals are available as separate keywords
      within the internal CP library, both in spin-restricted and spin-unrestricted form,
      and can be freely combined:
      \\~\\
      \textbf{LDA\_X}
      \\
      \textbf{LDA\_C} \_PZ, \_PW, \_OB\_PW, \_VWN
      \\
      \textbf{GGA\_X} \_B88, \_OPTX, \_PBE, \_PBE\_SOL, \_REVPBE, \_PBE\_FLEX
      \\
      \textbf{GGA\_C} \_LYP, \_P86, \_PBE, \_PBE\_SOL, \_PBE\_FLEX
      \\
      \textbf{MGGA\_X} \_TPSS
      \\
      \textbf{MGGA\_C} \_TPSS
      \\~\\
      The parameters for VWN used within the CP library correspond to what is sometimes denoted `VWN5'.
      The version of HCTH available in prior releases corresponds to GGA\_XC\_HCTH\_120.
      The definitions for the LDA part of the P86, PBE and TPSS correlation have been updated to comply with the standard definitions of most
      codes, for certain functionals, the definitions used in CPMD versions prior to 4.3 may be accessed by using
      \refspekeyword{OLD\_DEFINITIONS}{OLD DEFINITIONS}.
      A custom value of the parameter $\beta$ used in GGA\_X\_B88 can be input using the keyword \refkeyword{BECKE BETA}. \\
      A completely customised parameterisation of PBE can be set up by using the \_PBE\_FLEX routines in combination with
      the PBE\_FLEX\_\dots keywords in the \&DFT section. Since many different but highly specific parametrisations
      of PBE exist, this allows access to virtually all declinations based on the PBE xc equations.
      \\~\\
      \textbf{NEWCODE} and \textbf{OLDCODE}:\\
      Single keyword for setting up XC-functionals. 
      Only \emph{one single} functional keyword can be specified in when using \refkeyword{OLDCODE} or \refkeyword{NEWCODE}. \\
      Available functionals are NONE, SONLY, LDA (in PADE form),
      \goodbreak BONLY, BP, BLYP, XLYP, GGA (=PW91), PBE, PBES, REVPBE,
      \goodbreak HCTH, OPTX, OLYP, TPSS, PBE0, B1LYP, B3LYP, X3LYP,PBES,
      \goodbreak HSE06}

\keyword{FUKUI}{}{[N=nf]}{}{\&RESP}
  \desc{Calculates the response to a change of occupation number of 
    chosen orbitals. The indices of these orbitals are read from the 
    following nf lines ({\bf default nf=1}). The orbitals themselves are 
    not read from any \refkeyword{RESTART} file but from WAVEFUNCTION.* 
    files generated with \refkeyword{RHOOUT} in the \&CPMD section; 
    to recall this the orbital numbers have to be negative, just like 
    for the \refkeyword{RHOOUT} keyword.\\
     
    A weight can be associated with each orbital if given just after 
    the orbital number, on the same line. It corresponds to saying how 
    many electrons are put in or taken from the orbital. For example;}
\begin{verbatim}
   FUKUI N=2
   -i 1.0
   -j -1.0
\end{verbatim}
  \desc{corresponds to the response to taking one electron from orbital i 
    and put it in orbital j.}

\keyword{GAUGE}{\{PARA,GEN,ALL\}}{}{}{\&LINRES}
  \desc{Gauge of the linear-response wavefunctions. Default is the
      parallel-transport gauge (PARA) for closed-shell calculations and
      a sensible combination of the parallel-transport gauge and the
      full-rotation gauge (GEN) for all other cases. The full-rotation gauge
      can be enforced for all states by selecting ALL. See \cite{lsets}.}

\keyword{GC-CUTOFF}{}{}{}{\&DFT}
  \desc{On the next line the density cutoff for the calculation of
      the gradient correction has to be specified. The default value
      is $10^{-8}$. Experience showed that for a small
      \refkeyword{CUTOFF} value (e.g. when using Vanderbilt
      pseudopotentials) a bigger values have to be used.
      A reasonable value for a 25~ryd cutoff calculation is $5 \cdot 10^{-6}$.\\
      {\bf Warning}: for the HCTH functional, since it includes
      both the $xc$ part and the gradient correction in a unique
      functional, a GC-CUTOFF too high (e.g. $\geq 5 \cdot 10^{-5}$)
      could result in not including any $xc$ part with uncontrolled
      related consequences.}

\keyword{GDIIS}{}{}{}{\&CPMD}
  \desc{Use the method of direct inversion in the
      iterative subspace combined with a quasi-Newton method
      (using BFGS) for optimization of
      the ionic positions~\cite{Csaszar84}.%\cite{Fischer}\\
      The number of DIIS vectors is read from the next line.\\
      GDIIS with {\bf 5 vectors} is the {\bf default} method in
      optimization runs.}

\keyword{GENERATE COORDINATES}{}{}{}{\&ATOMS}
  \desc{The number of generator atoms for
      each species are read from the next line. \\
      These atoms
      are used together with the point group information
      to generate all other atomic positions. The input still
      has to have entries for all atoms but their coordinates
      are overwritten. Also the total number of atoms per species
      has to be correct.}

\keyword{GENERATE REPLICAS}{}{}{}{\&PIMD}
  \desc{Generate quantum free particle replicas
        from scratch given a classical input
        configuration according to the keyword \refkeyword{DEBROGLIE} 
        specification.
        %
        This is the default if \refkeyword{INITIALIZATION} is active.}

\spekeyword{GLE\_LAMBDA}{}{}{}{\&PIMD}{GLE LAMBDA}
  \desc{Set the scaling factor $\lambda$ of the generalized Langevin thermostat  
        read from the next line for removing the resonances between the vibrations 
        of the system and the harmonic potential representing the quantum kinetic 
        energy term in the description of the path integral. 
        The default value is 0.5 as suggested in Ref.~\cite{Rossi2014}.}

\keyword{GRADIENT CORRECTION}{}{}{[functionals]}{\&DFT}
  \desc{Individual components of gradient corrected functionals can be
    selected. Rarely needed anymore, use the \refkeyword{FUNCTIONAL}
    keyword instead. Only available for \textbf{OLDCODE} and \textbf{NEWCODE}; but 
    the new \refspekeyword{XC\_DRIVER}{XC DRIVER} offers similar flexibility in combination
    with the \textbf{FUNCTIONAL} keyword.\\

      Functionals implemented are for the
      exchange energy:\\
      {\bf BECKE88}~\cite{Becke88}, {\bf GGAX}~\cite{Perdew92}
      {\bf PBEX}~\cite{Perdew96}, {\bf REVPBEX}~\cite{Zhang98},
      \goodbreak{\bf HCTH}~\cite{Handy98}, {\bf OPTX}~\cite{Optx},{\bf PBESX}~\cite{Perdew07} \\
      and for the correlation part:\\
      {\bf PERDEW86}~\cite{Perdew86}, {\bf LYP}~\cite{Lee88},
      {\bf GGAC}~\cite{Perdew92}, {\bf PBEC} \cite{Perdew96},
      {\bf REVPBEC} \cite{Zhang98}, {\bf HCTH} \cite{Handy98}
      {\bf OLYP}~\cite{Optx},{\bf PBESC}~\cite{Perdew07}. \\
      Note that for HCTH, exchange and correlation are treated as
      a unique functional.\\
      The keywords {\bf EXCHANGE} and {\bf CORRELATION}
      can be used for the default functionals (currently BECKE88
      and PERDEW86). If no functionals are specified the default
      functionals for exchange and correlation are used.}

\keyword{GSHELL}{}{}{}{\&CPMD}
  \desc{Write a file {\bf GSHELL} with the information
        on the plane waves for further use in S(q) calculations.}

\keyword{HAMILTONIAN CUTOFF}{}{}{}{\&CPMD}
  \desc{The lower cutoff for the diagonal
      approximation to the Kohn-Sham matrix~\cite{Tuckerman94} is read from the
      next line.\\
      {\bf Default} is {\bf 0.5} atomic units.\\
      For variable cell dynamics only the kinetic energy as
      calculated for the reference cell is used.}

\spekeyword{HAMILTONIAN CUTOFF}{}{}{}{\&RESP}{HAMILTONIAN CUTOFF RESP}
  \desc{
   The value where the preconditioner (the approximate
   Greens function $(V_{kin}+V_{coul}-\epsilon_{KS})^{-1})$ is truncated.
   HTHRS can also be used. Default value is 0.5.
   }

\keyword{HARDNESS}{}{}{}{\&RESP}
  \desc{Not documented.}

\keyword{HARMONIC REFERENCE SYSTEM}{}{[OFF]}{}{\&CPMD}
  \desc{Switches harmonic reference system integration~\cite{Tuckerman94} on/off. \\
      The number of shells included in the
      analytic integration is controlled with the keyword
      \refkeyword{HAMILTONIAN CUTOFF}. \\
      By {\bf default} this option is switched {\bf off}.}

\keyword{HARTREE-FOCK}{}{}{}{\&DFT}
  \desc{Do a Hartree--Fock calculation. This only works correctly for
       isolated systems. It should be used with care, it needs enormous
       amounts of CPU time.}

\keyword{HARTREE}{}{}{}{\&DFT}
  \desc{Do a Hartree calculation. Only of use for testing purposes.}

\keyword{HESSCORE}{}{}{}{\&CPMD}
  \desc{Calculates the partial Hessian after relaxation of the environment,
       equivalent to {\sl NSMAXP=0} ({\bf PRFO NSMAXP}).}

\keyword{HESSIAN}{}{[DISCO,SCHLEGEL,UNIT,PARTIAL]}{}{\&CPMD}
  \desc{The initial approximate {\bf Hessian}
      for a {\bf geometry optimization}
      is constructed using empirical rules with the DISCO~\cite{Fischer92}
      or Schlegel's~\cite{Schlegel84} parametrization
      or simply a unit matrix is used. \\
      If the option {\bf PARTIAL} is used the initial approximate Hessian
      for a geometry optimization
      is constructed from a block matrix formed
      of the parametrized Hessian and the partial Hessian (of the reaction
      core). If the reaction core spans the entire system, its Hessian is simply
      copied.  The keywords \refkeyword{RESTART} PHESS are required.}

\spekeyword{HFX\_BLOCK\_SIZE}{}{}{}{\&DFT}{HFX BLOCK SIZE}
 \desc{Set the block size for the HFX calculations. This keyword is active only when no thresholding of the integrals is used.
   Default is 2. The block size is read from the next line.
}

\keyword{HFX CUTOFF}{}{}{}{\&SYSTEM}
 \desc{
 Set an additional cutoff for wavefunction and density to be used in the calculation
 of exact exchange. Cutoffs for wavefunctions and densities are read from the next
 line in Rydberg units. Defaults are the same cutoffs as for the normal calculation.
 Only lower cutoffs than the defaults can be specified.
 }

\spekeyword{HFX\_DISTRIBUTION}{}{}{}{\&DFT}{HFX DISTRIBUTION}
 \desc{Set the block distribution for the parallel HFX calculations. This keyword is active only when no thresholding of the integrals is used.
   Default is BLOCK\_CYCLIC. The distribution is read from the next line.
}

\spekeyword{PHFX}{}{}{}{\&DFT}
 \desc{Set the amount of HFX in the next line. The numerical value must be between 0 (no HFX) and 1 (100\% of HFX. For HSE06 the standard valus is 0.25 if this keyword and related numerical value ios not is not selected. Warning: PHFX is incompatible with XC_DRIVER. Use NEWCODE instead.
}

\keyword{HFX SCREENING}{\{WFC,DIAG,EPS\_INT,RECOMPUTE\_TWO\_INT\_LIST\_EVERY\}}{}{}{\&DFT}
  \desc{Read value from the next line. \\
        Perform the calculation of exact exchange using Wannier functions.
        Orbital pairs are screened according to the distance of the Wannier
        centers {\sl WFC}, the value of the integrals {\sl EPS\_INT}, or only the diagonal terms
        are included ({\sl DIAG}). {\sl RECOMPUTE\_TWO\_INT\_LIST\_EVERY} 
        allows to set how often the integral list is recomputed. }

\spekeyword{HIGH\_LEVEL\_FORCES}{\{DFT, EXTERNAL\}}{}{}{\&MTS}{HIGH LEVEL FORCES}
 \desc{
    Set the computational model for the calculation of high level forces in the MTS scheme.
    For now two options exists: (i) The forces are obtained from the DFT code internal to CPMD. 
    In that case, the high level functional has to be set in the 
    \&DFT section with the \refspekeyword{MTS\_HIGH\_FUNC}{MTS HIGH FUNC} keyword.
    (ii) The forces are calculated by an external program which called via a script which should be named
    \texttt{EXT\_HIGH\_FORCES}. When the forces are needed, CPMD will write a file \texttt{geometry.xyz} containing
    the current geometry in xyz format (cartesian coordinates in Angstroms). Then the script will 
    be called and CPMD will wait until a file \texttt{forces.xyz} becomes available with the new forces
    in xyz format (atomic units are expected). See the \texttt{get\_external\_forces} routine in the
    \texttt{interface\_utils.mod.F90} file.

    The keyword \refspekeyword{LOW\_LEVEL\_FORCES}{LOW LEVEL FORCES} can be used in exactly the same way for the 
    selection of the computational model used to calculate the low level forces in the MTS scheme.
    Similarly, if the DFT code internal to CPMD is chosen as low level, 
    the keyword \refspekeyword{MTS\_LOW\_FUNC}{MTS LOW FUNC} 
    (\&DFT section), can be used to select the low level density functional. \\

    {\bf Default:} Forces are calculated with the DFT code internal to CPMD.
 }

\keyword{HTHRS}{}{}{}{\&LINRES}
 \desc{
 Threshold for Hessian in preconditioner for linear response optimizations.
 Default is 0.5.
 }

\keyword{HUBBARD}{[NORM,ORTHO,NUATM=$nuatm$,OCCMAT=$printfreq$,VERB]}{}{}{\&DFT}
\desc{Use the \textbf{HUBBARD-U} correction to partly correct the self interaction error of DFT.
Specifying NORM and/or ORTHO causes the orbitals for the projections to be normalized and/or orthogonalized, respectively.
VERB causes the occupation matrix to be printed in the output, otherwise it is printed to the file OCCMAT.
The keyword OCCMAT=$printfreq$ specifies the frequency in which the OCCMAT is written. Default is 1.
$nuatm$ is the number of atoms which are treated with the Hubbard U correction (U-atom). Default is 1.

For each U-atom the atom number, the U parameter [eV], the linear response parameter $\alpha$ [eV] and the number of different angular momenta for projectors is read from a new line. %In case of a broken symmetry calculation, two U parameters (U\_BS and U\_HS) are expected.
%For each of the angular momenta for projectors
For each of those, the shell number of the projector and angular momentum $l$ is read from the next lines.}

\keyword{IMPLICIT NEWTON RAPHSON}%
        {\{PREC, CONTINUE, VERBOSE, ALTERNATIVE, STEP\}}{[N=nreg]}{}{\&CPMD}
  \desc{Not documented.}

\spekeyword{INCLUDE\_METALS}{}{}{}{\&VDW}{INCLUDE METALS}
  \desc{When using the \refkeyword{DCACP} dispersion-correction scheme, 
        include contributions from metal centres, too (where available).
        This should only rarely be needed. In the vast majority of cases
       (metal complexes, metallo-organic compounds), contributions from the
        metal centres are vanishingly small.}

\keyword{INITIALIZATION}{}{}{}{\&PIMD}
  \desc{Provide an initial configuration for all replicas as specified either
    by \refkeyword{GENERATE REPLICAS} or by \refkeyword{READ REPLICAS}.
%
    This option is automatically activated if \refkeyword{RESTART} COORDINATES is not
    specified.
%
    It is defaulted to GENERATE REPLICAS together with \refkeyword{DEBROGLIE} CENTROID and a
    temperature of 500~Kelvin.}

\keyword{INITIALIZE WAVEFUNCTION}{}{[RANDOM, ATOMS]}{[PRIMITIVE]}{\&CPMD}
  \desc{The initial guess for wavefunction optimization are either random
    functions or functions derived from the atomic pseudo-wavefunctions.\\
    For INITIALIZE WAVEFUNCTION ATOMS PRIMITIVE, CPMD will use
    the occupation information given in the \&BASIS section in order to construct a
    minimum spin multiplicity (i.e. doublet or singlet) initial wavefunction from the
    pseudo atomic orbitals. This option may be helpful to avoid excessive spin contamination
    in CDFT calculations (together with an already good initial guess for $V$) as it allows
    a strict initial localisation of excess spins on any atom species.

%
    {\bf Default} is to use the {\bf atomic pseudo-wavefunctions}.}
    
\keyword{INTERACTION}{}{}{}{\&RESP}
  \desc{Not documented.}

\keyword{INTERFACE}{\{EGO,GMX\}}{\{[MULLIKEN, LOWDIN, ESP, HIRSHFELD],PCGFIRST\}}{}{\&CPMD}
  \desc{Use CPMD together with a classical molecular dynamics code.
    CPMD and the classical MD code are run simultaneously and
    communicate via a file based protocol. See the file egointer\_utils.mod.F90
    for more details.
    This needs a specially adapted version of the respective classical MD code.
    So far, there is an interface\cite{egoqmmm,gmxqmmm} to the MD programs
    ego\cite{ego1,ego2} and Gromacs\cite{gmx3}.

    When using the suboption PCGFIRST the code will use
    \refkeyword{PCG}~MINIMIZE on the very first wavefunction
    optimization and then switch back to DIIS.
}

\keyword{INTFILE}{[READ,WRITE,FILENAME]}{}{}{\&CPMD}
  \desc{This keyword means {\it Interface File} and allows to select a 
   special file name in the reading and writing stages.
   The file name (max 40 characters) must be supplied in the next line.} 

\keyword{ISOLATED MOLECULE}{}{}{}{\&CPMD}
  \desc{Calculate the ionic temperature assuming that the system consists of an
    isolated molecule or cluster.\\
%
    Note:
%
    This keyword affects exclusively the determination of the number of
    dynamical degrees of freedom.
%
    This keyword does \textbf{not} activate the 'cluster option'
    \refkeyword{SYMMETRY} 0, but it is activated if SYMMETRY 0 is
    used \textbf{unless} the keyword \refkeyword{QMMM} is set as well.
%
    It allows studying an isolated molecule or cluster within periodic boundary
    conditions.}

\keyword{ISOTOPE}{}{}{}{\&ATOMS}
  \desc{Changes the default masses of the atoms. \\
%
    This keyword has to be followed by {\sl NSP} lines (number of atom types).
    In each line the new mass (in a.m.u.) of the respective species has to be specified (in
    order of their definition).}

\keyword{ISOTROPIC CELL}{}{}{}{\&SYSTEM}
  \desc{Specifies a constraint on the super cell in constant pressure
    dynamics or geometry optimization.
    The shape of the cell is held fixed, only the volume changes.}

\keyword{KEEPREALSPACE}{}{}{}{\&RESP}
  \desc{Like the standard CPMD option, this keeps the C0 ground state
    wavefunctions in the direct space representation during the calculation.
    Can save a lot of time, but is incredibly memory intensive.}

\keyword{KOHN-SHAM ENERGIES}{}{[OFF,NOWAVEFUNCTION]}{}{\&CPMD}
  \desc{Calculation of the Kohn-Sham energies and the corresponding 
        orbitals~\cite{KS}. \\
%
    The number of empty states that have to be calculated in addition to the
    occupied states is read from the next line. \\
%
    The Kohn-Sham orbitals are stored on the file {\bf RESTART.x} except if the
    keyword {\bf NOWAVEFUNCTION} is used. In this case, the program does not
    allocate memory for wavefunctions for all k points. It computes eigenvalues
    k point per k point losing information about wavefunctions. This keyword is
    used for band structure calculation to compute the eigenvalues for many k
    points.\\
%
    \textbf{Default} is not to calculate Kohn-Sham energies
    (\defaultvalue{OFF}). \\
%
    {\bf Warning:} The usage of this keyword needs special care (especially
    restarts).}

\keyword{KSHAM}{}{[MATRIX,ROUT,STATE]}{}{\&CPMD}
  \desc{Write out the Kohn-Sham Hamiltonian Matrix in the orbital basis given
  in the RESTART file to KS\_HAM. For this option to work the \refkeyword{RESTART} option and
  \refkeyword{OPTIMIZE WAVEFUNCTION} have to be activated. This option is useful for fragment
  orbital DFT (FODFT) calculations. Orbitals for the output of the FO-DFT matrix element can
  be given with the option {\bf STATE}, then indics of the two orbitals are read from the next
  line. {\bf ROUT} controls printing of involved orbitals.\\ {\bf MATRIX} instructs CPMD to read
  a transformation matrix from the file LOWDIN\_A to transform the KS-Hamiltonian to the
  non-orthogonal orbital basis}.
    
\keyword{KPERT}{}{[MONKHORSTPACK,SCALE]}{}{\&RESP}
\label{sec:kpert}
  \desc{Calculation of total energy and electronic density of states with
    an arbitrary number of k-points (at almost no additional
    computational effort).  The method is based on a
    ${\bf k}\cdot {\bf p}-$like approximation developed in the framework
    of the density functional perturbation theory \cite{mimp}.
    For a sampling of the BZ determined by the Monkhorst-Pack algorithm,
    the option {\bf MONKHORSTPACK}
    has to be specified, followed by the
    dimension of the mesh along the 3 reciprocal space axis
    $(NK_{1} , NK_{2} , NK_{3})$.
    If omitted, the individual absolute coordinates of the k-points have
    to be given one by one in the following lines.
    The  \refkeyword{SCALE} option allows to specify them in units of
    the reciprocal cell vectors.

    The line after {\bf KPERT} has to contain the
    the total number of k-points $(NKPTS)$, which have then to be given by
    their coordinates and the associated weights $(RK,WK)$
    in the format: \\
    $NKPTS$ \\
    $RK_{x1} \; RK_{y1} \; RK_{z1} \; WK_{1}$ \\
    $\dots$\\
    $RK_{x NKPTS} \; RK_{y NKPTS} \; RK_{z NKPTS} \; WK_{NKPTS}$. \\
    Three response wavefunctions are calculated, corresponding to
    the three independent orientations of the
    k basis vectors in reciprocal space.
    Therefore, 3 independent optimization loops are started ($x,y$ and $z$),
    and the 3 sets of wfns are stored (you need 4 times the
    memory required for a standard wavefunction optimization).
    The second order correction to the $\Gamma$-point total energy
    is calculated for the requested k-point mesh.

    Further options are (each in a new line of the input file ):
    \begin{description}
    \item[WRITE\_C1]
     the 3 sets of response wfns are stored in three separate restart files.
    \item[HAMILTONIAN] the k-dependent Hamiltonian is constructed via
     the second order perturbation theory approximation, and the
    corresponding KS energies are calculated. Due to technical reasons, for
    each k-point $2*NSTATE$ KS energies are calculated, however only
    those corresponding to occupied orbitals are reliable.
    \item[READ\_C1] the response wfns are read from RESTART.P\_\{xyz\}.
    \item[BUILD\_C00] the set of k-dependent wfns (first order correction)
    is calculated from the
    unperturbed $\Gamma$-point wfns together with the
    response orbitals.
    They are then written in a standard \refkeyword{RESTART} file. 
    From this restart file one can perform a calculation of the 
    Hamiltonian matrix for each kpoint and calculate the KS energies 
    (use {\bf LANCZOS DIAGO} in \&CPMD and the {\bf KPOINT} option {\bf ONLYDIAG} in \&SYSTEM.
    The k-point mesh must be the same used in the linear response
     calculation. set also {\bf NOSPHERICAL CUTOFF} in \&SYSTEM).
    \item[NORESTART] no RESTART file is written.
    \end{description}
    }

\keyword{KPOINTS}{}{}{options}{\&SYSTEM}
  \desc{With no option, read in the next line with the number of k-points and for
    each k-point, read the components in the Cartesian coordinates
    (units~$2\pi/a$) and the weight.}

%%%%%%%%%%%%%%%%%%%%%%%%%
% special input
%%%%%%%%%%%%%%%%%%%%%%%%%
    \begin{description}

    \item[MONKHORST-PACK]
      Read in the next line three numbers for the Monkhorst-Pack mesh. The
        program calculates then the special k-points. With the keyword {\bf
        SHIFT=kx ky kz} in the same line, you can precise the constant vector
        shift.

    \item[SYMMETRIZED]
      Symmetrized special k-points mesh (useful if you use a constant vector
        shift).

    \item[FULL]
      Construct full Monkhorst-Pack mesh with only inversion symmetry. Useful
        for molecular dynamics simulation The keywords {\bf SYMMETRIZED FULL}
        preserves all symmetry of Bravais lattice so there is no need to
        symmetrize density and forces.
    \item[SCALED]
      You can give k-points in reciprocal space coordinates.
    \item[BANDS]
      This option is to calculate the band structure.\\
%
      For each line you have to specify the number of k-points for the band,
        the initial and the final k-point.
%
        To finish the input, put:\\
          0  0. 0. 0.  0. 0. 0.
    \item[BLOCK=n {[OPTIONS]}]
      The block option, specifies the number of k-points in the memory. The
        program uses a swap file to store the wavefunctions only by default.
        With the following options, you can change this behaviour:
    \begin{description}
      \item[ALL]
        Three swap files are used to store wavefunctions and others arrays
          related to k-points. Swap files are in the current directory or the
          temporary directory given by environment variable TMPDIR. The use of
          memory is smaller than with the above option.

      \item[CALCULATED]
        One swap file is used to store only wavefunctions. The other arrays
          related to k-points are calculated each time if needed.

      \item[NOSWAP]

        The wavefunctions are not swapped. This is useful to
          calculate eigenvalues for each k point with few memory used.\\
%
        {\bf Warning:} The wavefunctions calculated are irrelevant. You have to
          specify explicitly some other options to use it:\\
          MAXSTEP 1 and \\ STORE OFF WAVEFUNCTIONS DENSITY POTENTIAL.
      \end{description}
    \end{description}
%%%%%%%%%%%%%%%%%%%%%%%%%
% end of special input
%%%%%%%%%%%%%%%%%%%%%%%%%

\keyword{LANCZOS DIAGONALIZATION}{\{ALL\}}{}{}{\&CPMD}
  \desc{Use {\bf Lanczos diagonalization} scheme. \\
    \textbf{Default} with \textbf{free energy functional}.}
                
\spekeyword{LANCZOS DIAGONALIZATION}{\{OPT,RESET=n\}}{}{}{\&CPMD}{LANCZOS DIAGONALIZATION OPT}
  \desc{Use {\bf Lanczos diagonalization} scheme after (OPT) or periodically
    during (RESET=n) direct wavefunction optimization using \refkeyword{ODIIS}.
    The number n specifies the number of DIIS resets (ODIIS NO\_RESET=nreset)
    due to poor progress until the wavefunction is diagonalized. This can be
    helpful if the wavefunction is converging very slowly.}

\keyword{LANCZOS PARAMETER}{[N=n]}{[ALL]}{}{\&CPMD}
  \desc{Give four parameters for Lanczos diagonalization in the
      next line:
      \begin{itemize}
      \item Maximal number of Lanczos iterations (50 is enough),
      \item Maximal number for the Krylov sub-space (8 best value),
      \item Blocking dimension ( $\leq NSTATE$, best in range 20-100)
            If you put a negative or zero number, this parameter is
            fixed by the program in function of the number of states
            ($(n+1)/(int(n/100+1))$).
      \item Tolerance for the accuracy of wavefunctions\\
            ($10^{-8}$ otherwise $10^{-12}$ with Trotter approximation)
      \end{itemize}
      If n is specified, read $n-1$ lines after the first one,
      containing a threshold density and a tolerance.
      See the hints section \ref{hints:lanczos} for more information.}

\keyword{LANCZOS}{}{ [CONTINUE,DETAILS]}{}{\&RESP}
  \desc{
   lanczos\_dim  iterations   conv\_threshold
   lanczos\_dim= dimension of the vibrational d.o.f.
   iterations = no.\@ of iterations desired for this run
   conv\_threshold = threshold for convergence on eigenvectors
   CONTINUE = argument for continuing Lanczos diagonalization
              from a previous run
              (reads file LANCZOS\_CONTINUE)
   DETAILS  = argument for verbosity. prints a lot of stuff
   }

\keyword{LANGEVIN}{\{WHITE, CPMD, OPTIMAL, SMART, CUSTOM, CENTROIDOFF\}}{[MOVECM]}{[W0, NS]}{\&CPMD}
  \desc{Use a (generalized) Langevin equation to thermostat the simulation\cite{Ceriotti10}. 
   By default, the component of the noise parallel to the center of mass velocity is 
     removed at each step of the thermostat. Removal can be disabled by the option {\sl MOVECM}.
    \\\smallskip
    {\sl CUSTOM:} \hfill\begin{minipage}[t]{10cm}
                  The {\bf number of additional momenta} of the generalized Langevin equation
                  {\sl NS} is read from the next line. The drift matrix 
                  (dimension $(NS+1)\times(NS+1)$) is read from the file \texttt{GLE-A},
                  which must be in the same directory in which the program is run. 
                  Optionally, the static covariance for the GLE dynamics can be provided
                  in the file \texttt{GLE-C}, so as to generate {\bf non-canonical sampling}. 
                  A library of GLE parameters can be downloaded from 
                  \htref{http://gle4md.berlios.de/}{http://gle4md.berlios.de/}
                      \end{minipage}
   \smallskip\\ A few {\bf presets} are provided, and are activated by the keywords:\\
    {\sl WHITE:} \hfill\begin{minipage}[t]{10cm}
                  A simple {\bf white-noise} Langevin dynamics is used. The optimally-sampled
                  frequency {\sl W0} (in cm$^{-1}$) is read from the next line. Note that use of 
                  {\sl LANGEVIN WHITE} in conjunction with {\sl MOLECULAR DYNAMICS CPMD} 
                  will most likely cause a large drift of the electronic temperature.
                      \end{minipage}
    {\sl OPTIMAL:} \hfill\begin{minipage}[t]{10cm}
                  An {\bf optimal-sampling} generalized Langevin dynamics is used. 
                  The frequencies in the range from $10^{-4}${\sl W0} up to {\sl W0}
                  will be sampled efficiently. Note that use of 
                  {\sl LANGEVIN OPTIMAL} in conjunction with {\sl MOLECULAR DYNAMICS CPMD} 
                  will cause a large drift of the electronic temperature.
                  This option is suggested for use in Born-Oppenheimer MD.
                      \end{minipage}
    {\sl CPMD:} \hfill\begin{minipage}[t]{10cm}
                  A generalized Langevin dynamics is used which is designed to 
                  work in conjunction with Car-Parrinello MD. 
                  The highest ionic frequency {\sl W0} (in cm$^{-1}$) is read from the 
                  next line. Ionic frequencies down to $10^{-4}${\sl W0} will be sampled 
                  efficiently, but not as much as for the {\sl OPTIMAL} keyword. 
                      \end{minipage}
                      
    {\sl SMART:} \hfill\begin{minipage}[t]{10cm}
                  A generalized Langevin dynamics that aims to be as efficient as possible
                  on the slowest time scale accessible to a typical ab initio simulation. 
                  In practice, vibrations with a time scale which is about 10000 time steps
                  will be sampled optimally, and faster modes will be sampled as efficiently
                  as possible without disturbing slower modes. 
                  The highest ionic frequency {\sl W0} (in cm$^{-1}$) is read from the 
                  next line. Will be about 50\%{} more efficient than {\sl OPTIMAL} for slow
                  modes, but less efficient for fast vibrations. Use only with Born-Oppenheimer
                  dynamics.
                      \end{minipage}

   {\sl CENTROIDOFF:} \hfill\begin{minipage}[t]{10cm}
                 For centroid and ring-polymer dynamics the generalized Langevin thermostat
                 for the centroids can be switched off by using the optional keyword CENTROIDOFF 
                 otherwise the thermostat with the frequency read from the next line is attached 
                 to the centroids. For each non-centroid mode the frequency of the thermostat 
                 is automatically determined depending on the frequency of its associated harmonic 
                 potential representing the quantum kinetic energy term.  
                      \end{minipage}
  }

\keyword{LBFGS}{}{[NREM, NTRUST, NRESTT, TRUSTR]}{}{\&CPMD}
  \desc{Use the limited-memory BFGS method (L-BFGS) for linear scaling
      {\bf optimization} of the {\bf ionic positions}. For more informations,
      see~\cite{LSCAL}. The information about the Hessian for the quasi-Newton
      method employed is derived from the history of the
      optimization~\cite{LSCAL,Liu89}.\\
      Only one sub-option is allowed per line and the respective parameter
      is read from the next line. The parameters mean:\\
      \hfill\smallskip {\sl NREM}:
                       \hfill\begin{minipage}[t]{10cm}
                       {\bf Number} of {\bf ionic gradients} and
                       {\bf displacements remembered} to approximate the Hessian.
                       The default is either 40 or the number of ionic
                       degrees of freedom, whichever is smaller.
                       Values greater the number of degrees of freedom are not
                       advisable.
                       \end{minipage}
      {\sl NTRUST:} \hfill\begin{minipage}[t]{10cm}
                       {\sl NTRUST=1} switches from a trust radius algorithm
                       to a {\bf line search} algorithm.
                       The default value of 0 ({\bf trust radius}) is
                       recommended.
                      \end{minipage}
      {\sl NRESTT:} \hfill\begin{minipage}[t]{10cm}
                       {\sl NRESTT$>$0} demands a {\bf periodic reset} of the
                       optimizer every {\sl NRESTT} steps.
                       Default is 0 (no periodic reset).
                       This option makes only sense if the ionic gradient is
                       not accurate.
                      \end{minipage}
      {\sl TRUSTR:} \hfill\begin{minipage}[t]{10cm}
                       Maximum and initial {\bf trust radius}.
                       Default is 0.5 atomic units.
                      \end{minipage}\\
      It can be useful to combine these keywords with the keywords
      \refkeyword{PRFO}, \refkeyword{CONVERGENCE} ADAPT, 
      \refkeyword{RESTART} LSSTAT, \refkeyword{PRINT} LSCAL ON and others.}

\keyword{LDA CORRELATION}{}{}{[functional]}{\&DFT}
  \desc{The LDA correlation functional is specified. Only available for \textbf{OLDCODE} and \textbf{NEWCODE}.\\
      Possible functionals are {\bf NO}
      (no correlation functional),
      {\bf PZ}~\cite{Perdew81}, \penalty 1000 {\bf VWN}~\cite{Vosko80},
      {\bf LYP}~\cite{Lee88} and {\bf PW}~\cite{Perdew91}. \\
      Default is the {\bf PZ}, the Perdew and Zunger fit to the data of
      Ceperley and Alder~\cite{Ceperley80}.}

\keyword{LDOS}{}{}{}{\&PROP}
  \desc{
 Calculate the layer projected density of states.
 The number of layers is read from the next line.\\

 To use the LDOS keyword, the user must first have
 performed a wavefunction optimization and then
 restart with with the \refkeyword{PROPERTIES} and
 \refkeyword{LANCZOS DIAGONALIZATION} keywords in the
 \&CPMD section (and LDOS in the \&PROP section).\\

\textbf{Warning:} If you use special k-points for a special structure
 you need to symmetrize charge density for which you must
specify the \refkeyword{POINT GROUP}.
}

\keyword{LIBRARY}{}{}{library\_for\_functional\_1, library\_for\_functional\_2, \dots}{\&DFT}
 \desc{In combination with \refspekeyword{XC\_DRIVER}{XC DRIVER}, both
       functionals from a new, internal CP library and from libxc are available.
       On the same line, for every functional specified in \textbf{FUNCTIONAL},
       the library has to be indicated by \textbf{INTERNAL} or \textbf{CP} for the
       internal CP library, or by \textbf{LIBXC} for libxc. If the keyword is
       omitted, the functionals will default to the internal CP library.\\
       The use of LIBXC is usually associated to a considerable overhead
       with respect to the internal implementation. It is therefore discouraged
       for functionals that are also available in the internal library, \emph{cf.{}}
       \refkeyword{FUNCTIONAL}.}

\spekeyword{KERNEL\_LIBRARY}{}{}{library\_for\_kernel\_1, library\_for\_kernel\_2, \dots}{\&DFT}{KERNEL LIBRARY}
 \desc{Same functionalty as \refkeyword{LIBRARY}, but applies to the \refkeyword{FUNCTIONAL} rather than the
       \refkeyword{LR KERNEL}.}

\spekeyword{LIBXC\_ONLY}{}{}{}{\&DFT}{LIBXC ONLY}
 \desc{Use the new xc driver, but use only functionals from libxc
       (this excludes all internal functionals).
       \emph{Cf.{}} \refspekeyword{XC\_DRIVER}{XC DRIVER}.}

\keyword{LINEAR RESPONSE}{}{}{}{\&CPMD}
  \desc{A perturbation theory calculation is done, according to the
        (required) further input in the \&RESP section. In the latter,
        one of the possible perturbation types (PHONONS, LANCZOS,
        RAMAN, FUKUI, KPERT, NMR, EPR, see section \ref{sec:resp-section})
        can be chosen, accompanied by further options.}

\keyword{LOCAL DIPOLE}{}{}{}{\&PROP}
  \desc{Calculate $numloc$ local dipole moments. \\
      $numloc$ is read
      from the next line followed by two numloc lines with the format:
      \\  $xmin$ $ymin$ $zmin$
      \\  $xmax$ $ymax$ $zmax$
      }

\keyword{LOCALIZATION}{}{}{}{\&TDDFT}
  \desc{Use localized orbitals in the TDDFT calculation. Default
       is to use canonical orbitals.}

\spekeyword{LOCALIZE}{}{}{}{\&HARDNESS}{LOCALIZE HARDNESS}
  \desc{Use localized orbitals in an orbital hardness calculation}

\keyword{LOCALIZE}{}{}{}{\&PROP}
  \desc{Localize the molecular orbitals \cite{Hutter94b} as defined through
      the atomic basis set. \\
      The same localization transformation is
      then applied also to the wavefunctions in the plane wave basis.
      These wavefunction can be printed with the keyword {\bf RHOOUT}
      specified in the section \&CPMD section.}

\keyword{LR KERNEL}{}{}{functional}{\&DFT}
  \desc{Use another functional for the linear response kernel. To be used like
        \refkeyword{FUNCTIONAL}.}

\spekeyword{XC\_KERNEL}{}{}{functional}{\&DFT}{XC KERNEL}
  \desc{Alias for \refkeyword{LR KERNEL}.}

\keyword{LR-TDDFT}{}{}{}{\&TDDFT}
  \desc{Use full linear response version of TDDFT. Default is to use
        \refkeyword{TAMM-DANCOFF} approximation.}

\keyword{LSD}{}{}{}{\&CPMD}
  \desc{Use the local spin density approximation.\\
      {\bf Warning:} Not all functionals in \textbf{OLDCODE} or \textbf{NEWCODE}
       are implemented for this option. Only in the \textbf{XC\_DRIVER}, all functionals
       are available both for spin-restricted and -unrestricted calculations.}

\keyword{LOCAL SPIN DENSITY}{}{}{}{\&CPMD}
  \desc{Alias for \refkeyword{LSD}.}

\spekeyword{LOW\_LEVEL\_FORCES}{\{DFT, EXTERNAL\}}{}{}{\&MTS}{LOW LEVEL FORCES}
 \desc{Analogue to \refspekeyword{HIGH\_LEVEL\_FORCES}{HIGH LEVEL FORCES}.}

\keyword{LOW SPIN EXCITATION}{}{[ROKS,ROSS,ROOTHAAN,CAS22]}{}{\&SYSTEM}
  \desc{Use the low spin excited state functional~\cite{Frank98}. 
For ROKS calculations, see also the \refkeyword{ROKS} keyword in the \&CPMD-section.}

\keyword{LOW SPIN EXCITATION LSETS}{}{}{}{\&SYSTEM}
  \desc{Slater transition-state density with restricted open-shell Kohn-Sham
  (low spin excited state). Currently works only with ROKS but not with ROSS,
  ROOTHAAN, or CAS22. See Ref.~\cite{lsets}.}

\keyword{LSE PARAMETERS}{}{}{}{\&SYSTEM}
  \desc{Determines the energy expression used in LSE calculations. The two 
    parameters LSEA and LSEB are read from the next line.\\
    \[E = \mbox{LSEA} \cdot E(Mixed) + \mbox{LSEB} \cdot E(Triplet)\]\\
    The default (LSEA $= 2$ and LSEB $= 1$) corresponds to singlet symmetry. 
    For the lowest triplet state, the \refkeyword{LSE PARAMETERS} must be set 
    to 0 and 1 (zero times mixed state plus triplet). See ref \cite{Frank98} 
    for a description of the method.}

\keyword{LZ-SHTDDFT}{}{}{}{\&TDDFT}
  \desc{Non adiabatic (nonadiabatic, non-adiabatic) trajectory surface hopping 
        scheme based on Landau-Zener transition probabilities. 
        Excited state dynamics is performed on TDDFT potential energy surfaces.
        To be used together with the keywords \refkeyword{MOLECULAR DYNAMICS} BO
        and \refkeyword{TDDFT} in the \&CPMD section (see section~\ref{sec:TDDFTdynamics}).
        Do NOT use this keyword together with 
        \refkeyword{SURFACE HOPPING} in \&CPMD, which invokes the surface hopping scheme based on
        \refkeyword{ROKS} (see \refkeyword{SURFACE HOPPING}).
        See also the related approach based on Tully's hopping probabilities~\refkeyword{T-SHTDDFT}.}

\keyword{MAXRUNTIME}{}{}{}{\&CPMD}
  \desc{The maximum RUN TIME (ELAPSED TIME) in seconds to be used is read from the 
        next line. The calculation will stop after the given amount of time.\\
      {\bf Default} is no limit.}

\keyword{MAXITER}{}{}{}{\&CPMD}
  \desc{The maximum number of iteration steps for the
      self-consistency of wavefunctions. Recommended use instead
      of \refkeyword{MAXSTEP} for pure wavefunction optimisation.
      The value is read from the next line. \\
      {\bf Default} is {\bf 10000} steps.}

\keyword{MAXSTEP}{}{}{}{\&CPMD}
  \desc{The maximum number of steps for geometry optimization
      or molecular dynamics to be performed. In the case of pure
      wavefunction optimisation, this keyword may be used instead
      of \refkeyword{MAXITER}. The value is read from the next line. \\
      {\bf Default} is {\bf 10000} steps.}

\spekeyword{MAXSTEP}{}{}{}{\&LINRES}{MAXSTEP LINRES}
 \desc{
 Maximum number of optimization steps for linear response optimizations.
 Default is 1000.
 }

\keyword{MEMORY}{\{SMALL, BIG\}}{}{}{\&CPMD}
  \desc{Using {\bf BIG}, the structure factors for the density cutoff
      are only calculated once and stored for reuse. \\
      This option allows for considerable time savings in connection
      with Vanderbilt pseudopotentials.\\
      {\bf Default} is ({\bf SMALL}) to {\bf recalculate} them
      whenever needed.}

\keyword{MESH}{}{}{}{\&SYSTEM}
  \desc{The number of {\bf real space mesh} points in $x-$, $y-$ and
      $z-$direction is read from the next line. \\
      If the values provided
      by the user are not compatible with the plane-wave cutoff or the
      requirements of the FFT routines the program chooses the next
      bigger valid numbers. \\
      {\bf Default} are the {\bf minimal values} compatible
      with the energy cutoff and the {\bf FFT} requirements.}

\keyword{METADYNAMICS ... END METADYNAMICS}{}{}{}{\&ATOMS}
  \desc{Initiate Metadynamics (see section~\ref{sec:meta} for more
    information on the available options and the input format).}

\keyword{MIRROR}{}{}{}{\&CPMD}
  \desc{Write the input file to the output.}

\keyword{MIXDIIS}{}{}{}{\&CPMD}
  \desc{Not documented}

\keyword{MIXSD}{}{}{}{\&CPMD}
  \desc{Not documented}

\keyword{MODIFIED GOEDECKER}{}{[PARAMETERS]}{}{\&CPMD}
  \desc{To be used in combination with \refkeyword{LOW SPIN EXCITATION}~\textbf{ROKS}. \\
  Calculation of the off-diagonal Kohn-Sham matrix elements $F_{AB}$ and
  $F_{BA}$ (with A, B: ROKS-SOMOs) is performed according to a modified
  Goedecker-Umrigar scheme
( $F_{AB} := (1-\lambda _{AB})F_{AB} + \lambda _{AB} F_{BA}$ and
  $F_{BA} := (1-\lambda _{BA})F_{BA} + \lambda _{BA} F_{AB}$ ).
 Default values are $\lambda _{AB}=-0.5$ and $\lambda _{BA}=0.5$.
 see Ref.~\cite{GrimmJCP2003}.\\

  With the optional keyword \textbf{PARAMETERS}: $\lambda _{AB}$ and $\lambda _{BA}$
  are read from the next line. Can be used to avoid unphysical rotation of the SOMOs.
  Always check the orbitals!\\
  
  See also \ref{hints:roks}.
   }

\keyword{MOLECULAR DYNAMICS}{}{[CP, BO, ET, PT, CLASSICAL, FILE [XYZ, NSKIP=N, NSAMPLE=M]]}{}{\&CPMD}
  \desc{Perform a molecular dynamics (MD) run.
      {\bf CP} stands for a Car-Parrinello type MD.
      With the option {\bf BO} a Born-Oppenheimer MD is performed
      where the wavefunction is re-converged after each MD-step. 
      {\bf EH} specifies Ehrenfest type dynamics according to which the Kohn-Sham
      orbitals are propagated in time (real electronic dynamics coupled to the nuclear dynamics). 
      In this case the time step has to be decreased accordingly due to the small mass of the
      electrons (typical values between 0.01 and 0.1 au). If you use EH dynamics an additional input
      section {\&PTDDFT} has to be specified. You need to start the dynamics with well converged
      KS orbitals from the RESTART file (before starting the EH dynamics do an optimization of the
      wavefunction with a convergence of {1.D-8} or {1.D-9}, if possible. An additional file
      called "wavefunctions" is produced, which contains the complex KS orbitals needed for
      the restart of the EH dynamics (see restart options in {\&PTDDFT}). Typical (minimal) input
      \&CPMD and \&PTDDFT sections to be used with EH dynmiacs\\
      \&CPMD \\
        MOLECULAR DYNAMICS EH \\
        RESTART WAVEFUNCTION COORDINATES LATEST \\
        CAYLEY \\
        RUNGE-KUTTA \\
        TIMESTEP \\ 
        0.01 \\
        MAXSTEP \\ 
        10000 \\
     \&END \\
     \&PTDDFT \\
       ACCURACY \\ 
       1.0D-8 \\ 
       RESTART \\ 
       2 \\
     \&END \\
      The keywords CAYLEY and RUNGE-KUTTA specifies the algorithms used for the propagation of the KS
      orbitals (are the default and recommended options).\\
      {\bf CLASSICAL }
      means that a MD that includes classical atoms is performed.\\
      {\bf BD} stands for Bohminan Dynamics. In the next line the number of trajectories
      ($N_{traj}$) used to describe the nuclear wavepacket are specified.
      The initial spread of the Gaussians associated to each element is guessed from the De Broglie wavelength.
      A special tuning of this value for the different elements may be necessary (not yet automatized).
      During the dynamics the $N_{traj}$ trajectories (each corresponding to a configuration space point or
      fluid element) are interacting within each other through the action of the quantum potential.
      For the more details about the implementation see Ref.~\cite{ABDY}.

      If {\bf FILE} is set, then the trajectory is reread from a file instead of being calculated. This
      is useful for performing analysis on a previous trajectory. Can be used in conjunction with the 
      standard MD options like DIPOLE DYNAMICS and WANNIER; some other features like LINEAR RESPONSE are 
      also enabled. The trajectory is read from a file named TRAJSAVED (usually a copy of a previous TRAJECTORY file), 
      or TRAJSAVED.xyz if {\bf XYZ} is set. {\bf NSKIP} and {\bf NSAMPLE} control the selection of 
      frames read: the frame read at step ISTEP is NSKIP+ISTEP*NSAMPLE.

      {\bf Default} is {\bf CP}.}

\keyword{MOLECULAR STATES}{}{}{}{\&TDDFT}
  \desc{Calculate and group Kohn--Sham orbitals into
        molecular states for a TDDFT calculation.}

\keyword{MOVERHO}{}{}{}{\&CPMD}
  \desc{Mixing used during optimization of geometry or molecular dynamics.
      Use atomic or pseudowavefunctions to project wavefunctions
      in order to calculate the new ones with movement of atoms.
      Read in the next line the parameter (typically 0.2). }

\keyword{MOVIE TYPE}{}{}{}{\&ATOMS}
  \desc{Assign special movie atom types to the species.\\
      The types are read from the next line. Values from 0 to 5
      were allowed in the original MOVIE format.}

\keyword{MOVIE}{}{[OFF, SAMPLE]}{}{\&CPMD}
  \desc{Write the atomic coordinates without
      applying periodic boundary conditions in MOVIE
      format every {\sl IMOVIE} time steps on file {\em MOVIE}.
      {\sl  IMOVIE} is read from the next line. \\
      {\bf Default} is {\bf not} to
      write a movie file.}

\keyword{MULTIPLICITY}{}{}{}{\&SYSTEM}
  \desc{This keyword only applies to LSD calculations.\\
      The multiplicity (2$S$+1) is read from the next line.\\
      {\bf Default} is the {\bf smallest possible} multiplicity.}

\spekeyword{MTS\_HIGH\_FUNC}{}{}{functionals}{\&DFT}{MTS HIGH FUNC}
  \desc{Alias for \refkeyword{FUNCTIONAL} that makes sense when used with the MTS scheme since it
  corresponds to the high level functional in that context (see \&MTS section).
  {\bf Note:} The functionals in combination with the MTS scheme have to be set with 
      \refspekeyword{XC\_DRIVER}{XC DRIVER}.}

\spekeyword{MTS\_LOW\_FUNC}{}{}{functionals}{\&DFT}{MTS LOW FUNC}
  \desc{Select the low level functional in the MTS scheme. To be used like \refkeyword{FUNCTIONAL}.
  {\bf Note:} The functionals in combination with the MTS scheme have to be set with 
      \refspekeyword{XC\_DRIVER}{XC DRIVER}.}

\spekeyword{N\_CYCLES}{}{}{}{\&PTDDFT}{N-CYCLES}
  \desc{Defines the number of cycles (time steps) used in the propagation
        of the perturbed KS orbitals in a \refkeyword{PROPAGATION SPECTRA} calculation.
        The number of cycles is read from the next line.}

\keyword{NEQUI}{}{}{}{\&PATH}
  \desc{Number of equilibration steps discarded to calculate the mean force.}

\keyword{NEWCODE}{}{}{}{\&DFT}%
  \desc{This keyword will be deprecated in a future release. Both \textbf{NEWCODE}
      and \textbf{OLDCODE} have been replaced by the new 
      \refspekeyword{XC\_DRIVER}{XC DRIVER}. \\~\\
      The following description applies up to CPMD version prior to 4.1:\\
      \textbf{NEWCODE} is a switch to select one out of two versions of
      code to calculate exchange-correlation functionals. \\
      NEWCODE is the default, but not all functionals are available with
      NEWCODE, if you select one of these, \refkeyword{OLDCODE} is used automatically.
      NEWCODE is highly recommended for all new projects and
      especially for vector computers, also some of the newer
      functionality is untested or non-functional with OLDCODE.
      }

\keyword{NLOOP}{}{}{}{\&PATH}
  \desc{Maximum number of string searches for Mean Free Energy Path searches.}

% MPB: response_p.inc is gone
%\keyword{NMR}{}{}{options, see response\_p.inc}{\&RESP}
\keyword{NMR}{}{}{options}{\&RESP}
  \desc{Calculate the NMR chemical shielding tensors for the
  system. Most important option: FULL, does a calculation with
  improved accuracy for periodic systems but takes a lot of
  time. Isolated systems: Use OVERLAP and 0.1 (on next line) for the
  same effect. \textit{Be careful for non-hydrogen nuclei.} The
  shielding is calculated without contribution from the core
  electrons. Contact sebastia@mpip-mainz.mpg.de for further details.}

\spekeyword{NO\_CONTRIBUTION}{}{}{}{\&VDW}{NO CONTRIBUTION}
  \desc{If the \refkeyword{DCACP} van der Waals-correction scheme is adpoted, 
        the indeces of atomic species (in the same order as specified in the
        \&ATOMS section) that shall not contribute to the dispersion forces
        (capping atoms, metals...) are read from the next line.}

\keyword{NOGEOCHECK}{}{}{}{\&CPMD}
  \desc{Default is to check all atomic distances and stop the program
        if the smallest disctance is below 0.5 Bohr. This keyword requests
        not to perform the check.}

\keyword{NONORTHOGONAL ORBITALS}{}{[OFF]}{}{\&CPMD}
  \desc{Use the norm constraint
      method~\cite{HutterIP} for molecular dynamics or non\-orthogonal
      orbitals in an optimization run.\\
      On the next line the limit of the off diagonal elements of the
      overlap matrix is defined.
      {\bf Warning:} Adding or deleting this option during a MD run
      needs special care.}

\keyword{NOOPT}{}{}{}{\&RESP}
  \desc{
   Do not perform a ground state wfn optimization. Be sure
   the restarted wfn is at the BO-surface.
   }

\keyword{NOPRINT ORBITALS}{}{}{}{\&PROP}
  \desc{Do not print the wavefunctions in the atomic basis set.}

\keyword{NORMAL MODES}{}{}{}{\&PIMD}
  \desc{Use the normal mode representation~\cite{Tuckerman96}
        of the path integral propagator. It is possible to impose a
        mass disparity between centroid and non--centroid coordinates by
        dividing the fictitious masses of only the {\em non}--centroid
        $s=2, \dots ,P$ replicas by
        the adiabaticity control factor FACSTAGE. This dimensionless
        factor {\em must always} be specified in the following line.
        Note: the eigen--{\em frequencies} of the $s>1$ replicas are changed
        by only $\sqrt{\mbox{FACSTAGE}}$, see Ref.~\cite{Martyna96}(b).
        Using FACSTAGE~$\not= 1.0$ makes only sense in conjunction
        with CENTROID DYNAMICS where WMASS=1.0 has to be used as well.}

\keyword{NOSE PARAMETERS}{}{}{}{\&CPMD}
  \desc{The {\bf parameters} controlling the {\bf Nos\'e
      thermostats}~\cite{Nose84,Hoover85} are read in the following order 
      from the next line:\\
      The {\bf length} of the Nos\'e-Hoover chain for the {\bf ions},\\
      the {\bf length} of the Nos\'e-Hoover chain for the {\bf electrons},\\
      the {\bf length} of the Nos\'e-Hoover chain for the {\bf cell
      parameters}.\\
      (The respective {\bf default} values are {\bf 4}.)\\
      The {\bf multiplication factor} (NEDOF0, a real number) for the number of
      {\bf electronic} degrees of freedom.
      The used degrees of freedom (NEDOF) are defined as $NEDOF=NEDOF0*X$
      If NEDOF0 is a negative number X is the true number of DOFs, if
      it's a positive number, X is the number of electronic states
      ({\bf default} for NEDOF0 is {\bf 6}).
      \\
      The order of the {\bf Suzuki/Yoshida integrator}
      ({\bf default} is {\bf 7}, choices are 3, 5, 7, 9, 15, 25, 125 and 625),\\
      and the {\bf decomposition ratio} of the time step
      ({\bf default} is {\bf 1}).\\
      If this keyword is omitted, the defaults are used. \\
      {\bf If the keyword is used \underline{all} parameters have to be
      specified.}}

\keyword{NOSE}{\{IONS, ELECTRONS, CELL\}}{[ULTRA,MASSIVE,CAFES,LOCAL,CENTROIDOFF]}{[T0]}{\&CPMD}
  \desc{{\bf Nos\'e-Hoover chains}~\cite{Nose84,Hoover85} for the {\bf
      ions}, {\bf electrons}, or {\bf cell parameters} are used.\\
      The {\bf target temperature} in Kelvin and
      the {\bf thermostat frequency}
      in $cm^{-1}$, respectively the {\bf fictitious kinetic
      energy} in atomic units and the {\bf thermostat frequency}
      in $cm^{-1}$ are read from the next line.
      Two files NOSE\_ENERGY and NOSE\_TRAJEC are written at each step containing 
      the Nos\'e-Hoover kinetic, potential and total energies along the dynamics
      (NOSE\_ENERGY) and the Nos\'e-Hoover variables and their velocities (NOSE\_TRAJEC); 
      these are useful in a wealth of post-processing calculations such as, e.~g. 
      heat transfer problems\cite{heat1,heat2}.
      For the ionic case the additional
      keyword {\bf ULTRA} selects a thermostat for each species,
      the keyword {\bf MASSIVE} selects a thermostat for each degree of
      freedom, and the keyword {\bf CAFES} can be used to give different
      temperatures to different groups of atoms\cite{cafes02}.
      The syntax in the {\bf CAFES} case is:\\[2ex]
      \texttt{NOSE IONS CAFES}\\
      ~~~~\textsl{ncafesgrp}\\
      ~~\textsl{cpnumber\_a\_1}~~\textsl{cpnumber\_a\_2}~~Temperature Frequency\\
      \dots\\
      ~~\textsl{cpnumber\_n\_1}~~\textsl{cpnumber\_n\_2}~~Temperature Frequency\\[2ex]
      There are \textsl{ncafesgrp} groups, specified by giving their
      first CPMD atom number (\textsl{cpnumber\_X\_1}) and last CPMD atom
      number (\textsl{cpnumber\_X\_2}). In the case of hybrid QM/MM
      simulations, you have to consult the QMMM\_ORDER file to find those
      numbers. The temperature and frequency can be different for each
      group. All atoms of the system have to be in a CAFES group.
      A new file, \texttt{CAFES} is created containing the temperature
      of each group (cols. 2 \dots \textsl{ncafesgrp+1}) and the energy
      of the Nose-Hoover chains of that group (last columns).\\
      Using CAFES with different temperatures only makes sense if the
      different groups are decoupled from each other by increasing the
      masses of the involved atoms. The
      mass can be specified in the topology / or with the \refkeyword{ISOTOPE}
      keyword. However, you can only change the mass of a complete CPMD
      species at a time. Hence, the topology and/or the input should be
      such that atoms of different CAFES group are in different species.\\
      {\bf NOTE:} CAFES is currently not restartable.\\[2ex]
      The keyword {\bf LOCAL} collects groups of atoms to separate
      thermostats, each having its own Nos\'e-Hoover chain. Specify
      the local thermostats as follows:\\[1ex]
      \begin{tabular}{lll}
        \multicolumn{3}{l}{\tt NOSE IONS LOCAL}\\
        \multicolumn{3}{l}{$n_l$ \em (number of local thermostats)}\\
        \em temperature 1 & \em frequency 1&\\
        \vdots\\
        \em temperature $n_l$ & \em frequency $n_l$ &\\[1ex]
       \multicolumn{3}{l}{$n_r$ \em (number of atom ranges)}\\ 
        \em thermostat number & \em start atom & \em end atom\\
        \vdots &\em ($n_r$ entries)&\\
      \end{tabular}

      The parser for the atom ranges uses either the CPMD ordering or
      the GROMOS ordering in case of classical or QM/MM runs. Multiple
      ranges may be specified for the same thermostat. Atoms belonging
      to the same CPMD constraint or the same solvent molecule in
      QM/MM runs must belong to the same local thermostat.
      
      If {\bf T0} option is present, the initial temperature for the
      Nos{\'e}-Hoover chains are read soon after the thermostat frequencies in the same line (also for the
      LOCAL thermostat). By default it is same as the target temperature of the thermostat. 
      Note: This is not implemented for the CAFES thermostat.}

      The keyword {\bf CENTROIDOFF} is intended to be used in connection with 
      the centroid and ring-polymer dynamics. In this type of path integral MD,
      you can request not to attach the Nos{\' e}-Hoover chains for ions to the centroids 
      by explicit use of the keyword CENTROIDOFF, otherwise the thermostat with the 
      frequency read from the next line is attached to the centroids.
      For each non-centroid mode the frequency with the Nos{\' e}-Hoover chains is 
      automatically determined depending on the frequency of its associated harmonic potential 
      representing the {\it quantum} kinetic energy term.  

\keyword{NPREVIOUS}{}{}{}{\&PATH}
  \desc{String index to restart from. Note that this is just for numbering files, the initial path in collective variables for the search is always {\em string.inp}.}

\spekeyword{NUMERICAL\_DIV}{}{}{}{\&DFT}{NUMERICAL DIV}
  \desc{Calculate the $\mathbf{G}=0$ term for the Coulomb-attenuated exact
        exchange using a numerical description of the integral term.
        This is the default only when CAM-B3LYP is used with LIBXC.
        In all other situations, \refspekeyword{ANALYTICAL\_DIV}{ANALYTICAL DIV} is used.}

\keyword{OACP}{}{[DENSITY, REF\_DENSITY, FORCE]}{}{\&RESP}
  \desc{Not documented.}
  
\keyword{OCCUPATION}{}{[FIXED]}{}{\&SYSTEM}
  \desc{The occupation numbers are read from the next line.\\
      This keyword must be preceded by  \refkeyword{STATES}.
      The FIXED option fixes the occupation numbers for the
      diagonalization scheme, otherwise this option is meaningless.}

\keyword{ODIIS}{}{[NOPRECONDITIONING,NO\_RESET=nreset]}{}{\&CPMD}
  \desc{Use the method of {\bf direct inversion} in the iterative
      subspace for {\bf optimization} of
      the {\bf wavefunction}~\cite{Hutter94a}.\\
      The number of DIIS vectors is read from the next line. \\
      (ODIIS with {\bf 10 vectors} is the {\bf default} method in
      optimization runs.)\\
      The preconditioning is controlled by the keyword
      \refkeyword{HAMILTONIAN CUTOFF}.
      Optionally preconditioning can be disabled.\\
      By default, the number of wavefunction optimization cycles until DIIS is
      {\bf reset} on poor progress, is the number of DIIS vectors. With
      {\bf ODIIS NO\_RESET}, this number can be changed, or DIIS resets can
      be {\bf disabled} altogether with a value of -1.}

\spekeyword{OLD\_DEFINITIONS}{}{}{}{\&DFT}{OLD DEFINITIONS}
   \desc{For certain functionals, older versions of CPMD did not use standard
         definitions. When using the new \refspekeyword{XC\_DRIVER}{XC DRIVER},
         those definitions can be invoked by using \textbf{OLD\_DEFINITIONS}.
         The option is available in combination with any derivative of PBE
         and TPSS correlation.}

\keyword{OLDCODE}{}{}{}{\&DFT}
  \desc{This keyword will be deprecated in a future release. Please use the new
      \refspekeyword{XC\_DRIVER}{XC DRIVER}. \emph{Cf.{}} \refkeyword{NEWCODE}.}

\keyword{OPTIMIZE GEOMETRY}{[XYZ, SAMPLE]}{}{}{\&CPMD}
  \desc{This option causes the program to optimize the geometry of the 
      system through a sequence of wavefunction optimizations and position
      updates. The additional keyword XYZ requests writing the ``trajectory''
      of the geometry additionally in xmol/xyz-format in a file {\em GEO\_OPT.xyz}. 
      If the keyword SAMPLE is given, {\em NGXYZ} is read from the next line, and
      then only every {\em NGXTZ} step is written to the xmol/xyz file.
      The {\bf default} is to write every step ({\em NGXYZ} = $1$).\\
      By default the a BFGS/DIIS algorithm is used (see \refkeyword{GDIIS})
      to updated the ionic positions. Other options are: \refkeyword{LBFGS},
      \refkeyword{PRFO}, and \refkeyword{STEEPEST DESCENT} IONS. See 
      \refkeyword{OPTIMIZE WAVEFUNCTION} for details on the corresponding 
      options for wavefunction optimizations.
    }

\keyword{OPTIMIZE SLATER EXPONENTS}{}{}{}{\&PROP}
  \desc{Not documented}

\keyword{OPTIMIZE WAVEFUNCTION}{}{}{}{\&CPMD}
  \desc{Request a single point energy calculation through a wavefunction 
      optimization. The resulting total energy is printed (for more output
      options see, e.g.,: \refkeyword{PRINT}, \refkeyword{RHOOUT}, 
      \refkeyword{ELF}) and a \refkeyword{RESTART} file is written. 
      This restart file is a prerequisite for many other subsequent calculation
      types in CPMD, e.g. \refkeyword{MOLECULAR DYNAMICS} CP or \refkeyword{PROPERTIES}.
      By default a DIIS optimizer is used (see \refkeyword{ODIIS}), but other
      options are: \refkeyword{PCG} (optionally with MINIMIZE), 
      \refkeyword{LANCZOS DIAGONALIZATION}, \refkeyword{DAVIDSON DIAGONALIZATION},
      and \refkeyword{STEEPEST DESCENT} ELECTRONS.}

\keyword{OPTIMIZER}{}{[SD,DIIS,PCG,AUTO]}{}{\&LINRES}
 \desc{
 Optimizer to be used for linear response equations. Default is ``AUTO''
 which will first use PCG, then switch to DIIS and finally switch to
 DIIS with full storage and state dependent preconditioner.
 \refkeyword{THAUTO} sets the two tolerances for when to do the switch.
 }

\keyword{ORBITAL HARDNESS}{}{[LR,FD]}{}{\&CPMD}
  \desc{Perform an orbital hardness calculation. See section \&Hardness
       for further input options.}

\keyword{ORBITALS}{}{}{}{\&HARDNESS}
  \desc{Specify the number of orbitals to be used in a hardness calculation on the next line.}

\keyword{ORTHOGONALIZATION}{\{LOWDIN, GRAM-SCHMIDT\}}{[MATRIX]}{}{\&CPMD}
  \desc{Orthogonalization in optimization runs is done either by
      a L\"owdin (symmetric) or Gram-Schmidt procedure.\\
      {\bf Default} is Gram-Schmidt except for parallel runs where L\"owdin
      orthogonalization is used with the conjugate-gradient scheme.\\
      With the additional keyword {\bf MATRIX} the L\"owdin transformation matrix 
      is written to a file named LOWDIN\_A.}

\keyword{OUTPUT}{}{[ALL, GROUPS, PARENT]}{}{\&PIMD}
  \desc{Output files for each processor, processor group, or only
        grandparent.
        %
        Default is PARENT to standard output file (Note:
        some information such as messages for correct reading~/ writing of
        restart files is lost);
        GROUPS and ALL write to the files OUTPUT\_$n$ where $n$ is
        the group and bead number, respectively.}

 \spekeyword{OUTPUT}{}{[ALL, GROUPS, PARENT]}{}{\&PATH}{OUTPUT PATH}
   \desc{Idem as above, here for Mean Free Energy Path runs.}


\spekeyword{PARA\_BUFF\_SIZE}{}{}{}{\&CPMD}{PARA BUFF SIZE}
\desc{ Set the buffer size for parallel operation (sum, ...).
   Default is 2**16. The size is read from the next line.
}

\spekeyword{PARA\_STACK\_BUFF\_SIZE}{}{}{}{\&CPMD}{PARA STACK BUFF SIZE}
\desc{ Set the stack buffer size for parallel operation (sum, ...).
   Default is 2**8. The size is read from the next line.
}

\spekeyword{PARA\_USE\_MPI\_IN\_PLACE}{}{}{}{\&CPMD}{PARA USE MPI IN PLACE}
\desc{ Use MPI\_IN\_PLACE for parallel operation (sum, ...).
   Default is FALSE.
}


\keyword{PARRINELLO-RAHMAN}{\{NPT,SHOCK\}}{}{}{\&CPMD}
  \desc{To be used together with \refkeyword{MOLECULAR DYNAMICS}.\\
      A {\bf variable cell MD} with
      the {\bf Parrinello-Rahman Lagrangian}
      is performed~\cite{parrah,parrah2}. 
      With the additional keyword a {\bf constant NPT
      MD} using the method of Martyna, Tobias,
      and Klein~\cite{Martyna94}. \\
      If this keyword is used together with other run options
      like OPTIMIZE WAVEFUNCTIONS, calculations with different reference
      cells can be performed.\\
      With the additional keywork { \bf SHOCK} } a MD simulation using the multiscale
      shock method \cite{shock} is performed.

\keyword{PATH INTEGRAL}{}{}{}{\&CPMD}
  \desc{Perform a {\bf path integral molecular dynamics}
      calculation~\cite{Marx94,Marx96}.\\
      This keyword requires further input in
      the section \&PIMD ... \&END.}

\keyword{PATH MINIMIZATION}{}{}{}{\&CPMD}
  \desc{Perform a {\bf mean free energy path}
      search~\cite{Eijnden06}.\\
      This keyword requires further input in
      the section \&PATH ... \&END.}

\keyword{PATH SAMPLING}{}{}{}{\&CPMD}
  \desc{Use CPMD together with a reaction path sampling~\cite{tps} program.
      This needs special software.
      Note: this keyword has {\em nothing} to do with path integral
      MD as activated by the keyword PATH INTEGRAL and as specified in the
      section \&PIMD ... \&END.}


\spekeyword{PBE\_FLEX\_BETA}{}{}{}{\&DFT}{PBE FLEX BETA}
  \desc{In combination with the PBE\_FLEX correlation functional, the
        parameter $\beta$ can be specified on the next line.}

\spekeyword{PBE\_FLEX\_GAMMA}{}{}{}{\&DFT}{PBE FLEX GAMMA}
  \desc{In combination with the PBE\_FLEX correlation functional, the
        parameter $\gamma$ can be specified on the next line.}

\spekeyword{PBE\_FLEX\_KAPPA}{}{}{}{\&DFT}{PBE FLEX KAPPA}
  \desc{In combination with the PBE\_FLEX exchange or correlation functionals, the
        parameter $\kappa$ can be specified on the next line.}

\spekeyword{PBE\_FLEX\_MU}{}{}{}{\&DFT}{PBE FLEX MU}
  \desc{In combination with the PBE\_FLEX exchange or correlation functionals, the
        parameter $\mu$ can be specified on the next line.}

\spekeyword{PBE\_FLEX\_UEG\_CORRELATION}{}{}{functional}{\&DFT}{PBE FLEX UEG CORRELATION}
  \desc{In combination with the PBE\_FLEX correlation functional, the
        LDA correlation functional can be specified on the same line. Available options
        are LDA\_C\_PZ, LDA\_C\_PW (the default), LDA\_C\_OBPW and LDA\_C\_PZ.}

\keyword{PCG PARAMETER}{}{}{}{\&TDDFT}
  \desc{The parameters for the PCG diagonalization are read from the
       next line. If {\sl MINIMIZE} was used in the \refkeyword{DIAGONALIZER}
       then the total number of steps (default 100) and the convergence
       criteria (default $10^{-8}$) are read from the next line.
       Without minimization in addition the step length (default 0.5) has also
       to be given.}

\keyword{PCG}{}{[MINIMIZE,NOPRECONDITIONING]}{}{\&CPMD}
  \desc{Use the method of {\bf preconditioned conjugate gradients} for
      {\bf optimization} of the {\bf wavefunction}.\\
      The fixed step length is controlled by the keywords
      \refkeyword{TIMESTEP ELECTRONS} and \refkeyword{EMASS}.\\
      If the additional option {\bf MINIMIZE} is chosen, then additionally
      line searches are performed to improve the preconditioning.\\
      The preconditioning is controlled by the keyword
      \refkeyword{HAMILTONIAN CUTOFF}.
      Optionally preconditioning can be disabled.}

 \spekeyword{PERT\_TYPE}{}{}{}{\&PTDDFT}{PERT-TYPE}
  \desc{The type of the static perturbation used with \refkeyword{MOLECULAR DYNAMICS} EH
or \refkeyword{PROPAGATION SPECTRA} is read from the next line. 1: Dirac pulse, 2: Heaviside step
function (constant value after $t=0$).}

 \spekeyword{PERT\_AMPLI}{}{}{}{\&PTDDFT}{PERT-AMPLI}
  \desc{The amplitude of the perturbation used with \refkeyword{MOLECULAR DYNAMICS} EH
or \refkeyword{PROPAGATION SPECTRA} is read from the next line. }

 \spekeyword{PERT\_DIRECTION}{}{}{}{\&PTDDFT}{PERT-DIRECTION}
   \desc{Use with MOLECULAR DYNAMICS EH or PROPAGATION SPECTRA to specify the
direction (x=1,y=2,z=3) of the perturbing field. The direction code (1,2,3) is
read from the next line.}

\keyword{PHONON}{}{}{}{\&RESP}
  \desc{Calculate the harmonic frequencies from perturbation theory.}

\keyword{PIPULSE}{}{}{}{\&PTDDFT}
  \desc{Specifies a time dependent pi-pulse to be used with MOLECULAR DYNAMICS EH.
        Use PIPULSE together with TD\_POTENTIAL. The pulse strength is read from the
        next line (see subroutine gaugepot\_laser in td\_utils.mod.F90 for further details).}

\keyword{POINT GROUP}{}{[MOLECULE], [AUTO], [DELTA=delta]}{}{\&SYSTEM}
  \desc{The point group symmetry of
      the system can be specified in the next line.
      With the keyword {\sl AUTO} in the next line,
      the space group is determined automatically.
      This affects the calculation
      of nuclear forces and ionic positions.
      The electronic density and nuclear forces are symmetrized
      in function of point group symmetry.
      The group number is read from the next line.\\
      Crystal symmetry groups:}
\begin{verbatim}
               1  1 (c1)     9   3m (c3v)   17 4/mmm (d4h)   25  222 (d2)
               2 <1>(ci)    10  <3>m(d3d)   18   6   (c6)    26  mm2 (c2v)
               3  2 (c2)    11   4  (c4)    19  <6>  (c3h)   27  mmm (d2h)
               4  m (c1h)   12  <4> (s4)    20   6/m (c6h)   28  23  (t)
               5 2/m(c2h)   13  4/m (c4h)   21   622 (d6)    29  m3  (th)
               6  3 (c3)    14  422 (d4)    22   6mm (c6v)   30  432 (o)
               7 <3>(c3i)   15  4mm (c4v)   23  <6>m2(d3h)   31 <4>3m(td)
               8 32 (d3)    16 <4>2m(d2d)   24  6/mmm(d6h)   32  m3m (oh)
\end{verbatim}
\desc{You can specify the point group by its name using
      the keyword {\sl NAME=} followed by the name of the point group
      (one of both notations).\hfill\\
      For molecular point groups the additional keyword {\sl MO\-LECULE}
      has to be specified. The Sch\"onflies symbol of the group
      is read in the following format from the next line:  \\
      {\em Group symbol; order of principle axis}  \smallskip \\
      Possible group symbols are any Sch\"onflies symbol with the
      axis number replaced by $n$ (e.g. DNH). For molecular
      point groups a special orientation is assumed. The principle
      axis is along $z$ and vertical symmetry planes are orthogonal
      to $x$.\\
      {\sl DELTA=} specifies the required accuracy (default=$10^{-6}$).\\
      With the keyword {\bf AUTO}, the point group is
      determined automatically.}


\keyword{POISSON SOLVER}{ \{HOCKNEY, TUCKERMAN, MORTENSEN\}}{[PARAMETER]}{}{\&SYSTEM}
  \desc{This keyword determines the method for the solution of the
      Poisson equation for isolated systems. Either Hockney's
      method~\cite{Hockney70} or Martyna and Tuckerman's
      method~\cite{Martyna99} is used. The smoothing parameter (for Hockney's
      method) or $L \times \alpha$ for Tuckerman's method can be read from the
      next line using the {\bf PARAMETER} keyword.

      For more information about the usage of this parameter see also
      section \ref{hints:symm0}.}

\keyword{POLAK}{}{}{}{\&RESP}
  \desc{
   Uses the Polak-Ribiere formula for the conjugate
   gradient algorithm. Can be safer in the convergence.
   }

\keyword{POLARIZABILITY}{}{}{}{\&PROP}
  \desc{Computes the polarisability of a system, intended as dipole
        moment per unit volume.}

\keyword{POLYMER}{}{}{}{\&SYSTEM}
  \desc{Assume {\bf periodic boundary} condition in {\bf $x$-direction}.\\
%       You also need to set the 'cluster option' (i.e. \refkeyword{SYMMETRY} 0).
  }

\keyword{POPULATION ANALYSIS}{}{[MULLIKEN, DAVIDSON],[n-CENTER]}{}{\&PROP}
  \desc{The type of population analysis that is performed with the
      projected wavefunctions. \\
      L\"owdin charges are given with both
      options. For the Davidson analysis~\cite{Davidson67} the maximum
      complexity can be specified with the keyword {\bf n-CENTER}.\\
      Default for n is 2, terms up to 4 are programmed. For the
      Davidson option one has to specify the number of atomic orbitals
      that are used in the analysis. For each species one has to give
      this number in a separate line.
      An input example for a water molecule is given in the hints
      section \ref{hints:pop}.}

\keyword{PRESSURE}{}{}{}{\&SYSTEM}
  \desc{The {\bf external pressure} on the system is read from
      the next line (in {\bf kbar}).}

\keyword{PRFO}{}{[MODE, MDLOCK, TRUSTP, OMIN, PRJHES, DISPLACEMENT, HESSTYPE]}{}{\&CPMD}
  \desc{Use the partitioned rational function optimizer (P-RFO) with a quasi-Newton
      method  for {\bf optimization} of the {\bf ionic positions}. For more
      informations, see~\cite{LSCAL}. The approximated Hessian is updated
      using the Powell method~\cite{Powell71}.
      This method is used to find {\bf transition states} by
      {\bf following eigenmodes} of the approximated
      Hessian~\cite{Banerjee85,LSCAL}.\\
      Only one suboption is allowed per line and the respective parameter
      is read from the next line. The suboption {\bf PRJHES} does not take any
      parameter. If it is present, the translational and rotational modes are
      removed from the Hessian. This is only meaningful for conventional (not
      microiterative) transition state search. The parameters mean:\\
      \hfill\smallskip {\sl MODE}:
                       \hfill\begin{minipage}[t]{9.6cm}
                       Number of the initial Hessian {\bf eigenmode} to be
                       followed. Default is 1 (lowest eigenvalue).
                       \end{minipage}\\
      {\sl MDLOCK:} \hfill\begin{minipage}[t]{9.6cm}
                       {\sl MDLOCK=1} switches from a mode following algorithm
                       to a {\bf fixed eigenvector} to be maximized.
                       The default value of 0 ({\bf mode following}) is
                       recommended.
                      \end{minipage}\\
      {\sl TRUSTP:} \hfill\begin{minipage}[t]{9.6cm}
                       Maximum and initial {\bf trust radius}.
                       Default is 0.2 atomic units.
                                      \end{minipage}\\
      {\sl OMIN:} \hfill\begin{minipage}[t]{9.6cm}
                       This parameter is the minimum {\bf overlap} between the
                       maximized mode of the previous step and the most
                       overlapping eigenvector of the current Hessian.
                       The trust radius is reduced until this requirement is
                       fulfilled. The default is 0.5.
                      \end{minipage}\\
      {\sl DISPLACEMENT:} \hfill\begin{minipage}[t]{9.6cm}
                       Finite-difference {\bf displacement} for initial partial
                       Hessian. The default is 0.02.
                      \end{minipage}\\
      {\sl HESSTYPE:} \hfill\begin{minipage}[t]{9.6cm}
                      {\bf Type} of initial partial Hessian.
                      0: Finite-difference.
                      1: Taken from the full Hessian assuming a block-diagonal
                      form. See keyword \refkeyword{HESSIAN}.
                      The default is 0.
                      \end{minipage}\\
      It can be useful to combine these keywords with the keywords
      \refkeyword{CONVERGENCE} ENERGY, \refkeyword{RESTART} LSSTAT, 
      \refkeyword{RESTART} PHESS, \refkeyword{PRFO} NSVIB,
      \refkeyword{PRINT} LSCAL ON and others.}

\spekeyword{PRFO}{}{[NVAR, CORE, TOLENV, NSMAXP]}{}{\&CPMD}{PRFO NVAR}
  \desc{If any of these suboptions is present, the {\bf microiterative transition
      state search} scheme for {\bf optimization} of the {\bf ionic positions}
      is used. For more informations, see~\cite{LSCAL}.
      A combination of the {\bf L-BFGS} and {\bf P-RFO} methods is employed for
      linear scaling search for transition states~\cite{LSCAL,Turner99}.
      Before each P-RFO step in the reaction core towards the transition
      state, the {\bf environment} is fully {\bf relaxed} using L-BFGS.\\
      Only one suboption is allowed per line.
      The {\bf reaction core} can be selected using the {\bf NVAR} or
      {\bf CORE=ncore} suboptions. The value in the line after {\bf PRFO NVAR}
      sets the number of ionic {\bf degrees of freedom} in the reaction core.
      The {\sl ncore} values following the line {\bf PRFO CORE=ncore} select
      the {\bf member atoms} of the reaction core.
      If unspecified, the {\sl NVAR/3} first atoms form the reaction core.
      The parameters read with the two remaining suboptions are:\\
      \hfill\smallskip {\sl TOLENV}:
                       \hfill\begin{minipage}[t]{10cm}
                       {\bf Convergence criterion} for the maximum component of
                       the gradient acting on the ions of the {\bf environment}
                       until a P-RFO step within the reaction core is performed.
                       Default is one third of the convergence criterion for
                       the gradient of the ions ({\bf CONVERGENCE GEOMETRY}).
                       \end{minipage}
      {\sl NSMAXP:} \hfill\begin{minipage}[t]{10cm}
                       Maximum number of P-RFO {\bf steps} to be performed in the
                       reaction core. The keyword {\bf HESSCORE} corresponds
                       to {\bf PRFO NSMAXP} with {\sl NSMAXP=0}.
                      \end{minipage}
      It can be useful to combine these keywords with the keywords
      \refkeyword{LBFGS}, \refkeyword{CONVERGENCE} ADAPT, 
      \refkeyword{CONVERGENCE} ENERGY, \refkeyword{RESTART} LSSTAT,
      \refkeyword{RESTART} PHESS, \refkeyword{PRFO NSVIB}, 
      \refkeyword{PRINT} LSCAL ON, the other suboptions of PRFO,
      and others.}

\keyword{PRFO NSVIB}{}{}{}{\&CPMD}
  \desc{Perform a {\bf vibrational analysis} every NSVIB P-RFO steps {\bf on the
      fly}.
      This option only works with the P-RFO and microiterative transition state
      search algorithms. In case of microiterative TS search, only the reaction
      core is analyzed.}

\keyword{PRINT COORDINATES}{}{}{}{\&CLASSIC}
  \desc{Not documented}

\keyword{PRINT ENERGY}{ \{ON, OFF\}} {[EKIN, ELECTROSTATIC,
       ESR, ESELF, EFREE, EBAND, ENTROPY, EPSEU, EHEP,
       EHEE, EHII, ENL, EXC, VXC, EGC, EBOGO]}{}{\&CPMD}
  \desc{Display or not information about energies.}

\keyword{PRINT FF}{}{}{}{\&CLASSIC}
  \desc{Not documented}

\spekeyword{PRINT\_FORCES}{[OFF]}{}{}{\&MTS}{PRINT FORCES}
  \desc{
     Turn on/off the printing of the high and low level forces along the MTS-MD trajectory.
     The forces are printed to two distinct trajectory files in the \texttt{.xyz} format,
     \texttt{MTS\_LOW\_FORCES.xyz} and \texttt{MTS\_HIGH\_FORCES.xyz}. \\

     {\bf Default}: OFF.
  }

\keyword{PRINT LEVEL}{}{}{}{\&PIMD}
  \desc{The detail of printing information is read as an integer number
        from the next line.
        %
        Currently there is only minimal output
        for $<5$ and maximal output for $\geq 5$.}

\spekeyword{PRINT LEVEL}{}{}{}{\&PATH}{PRINT PATH}
  \desc{Idem as above, here for Mean Free Energy Path searches.}

\keyword{PRINT}{\{ON,OFF\}}{[INFO, EIGENVALUES, COORDINATES, LSCAL, FORCES, WANNIER]}{}{\&CPMD}
  \desc{A {\bf detailed output} is printed every {\sl IPRINT} iterations.
      Either only different contribution to the energy or
      in addition the atomic coordinates and the forces are printed.
      {\sl IPRINT} is read from the next line if the keywords {\bf ON}
      or {\bf OFF} are not specified. \\
      {\bf Default} is {\bf only energies} after the first step and
      at the end of the run. OFF switches the output off.}

\keyword{PRNGSEED}{}{}{}{\&CPMD}
  \desc{The seed for the random number generator is read as an integer number
        from the next line.}

\keyword{PROCESSOR GROUPS}{}{}{}{\&PIMD}
  \desc{%
        This is only needed for {\em fine}--tuning load balancing in case of
        path integral runs {\em iff} two level parallelization is used.
        The default optimizes the combined load balancing
        of the parallelization over replicas and g--vectors.
        The default load distribution is usually optimal.
        Separate the total number of processors into
        a certain number of processor groups that is read from the
        following line; only 2$^N$ = 2, 4, 8, 16, $\dots$ groups
        are allowed and the maximum number of groups is
        the number of replicas.
        Every processor group is headed by one PARENT and has
        several CHILDREN
        that work together on a single replica at one time; the processor
        groups work sequentially on replicas if there is more
        than one
        replica assigned to one processor group.
        %
        Note: if the resulting number of processor groups is much smaller than
        the number of replicas (which occurs in ``odd'' cases)
        specifying the number of
        processor groups to be equal to the number of replicas might be
        more efficient.
        %
        This keyword is only active in parallel mode.}

\spekeyword{PROCESSOR GROUPS}{}{}{}{\&PATH}{PROCESSOR GROUPS PATH}
  \desc{%
       Idem as above, here for mean free energy path search.}

\keyword{PROJECT WAVEFUNCTION}{}{}{}{\&PROP}
  \desc{The wavefunctions are projected on atomic orbitals. \\
      The projected wavefunctions are then used
      to calculate atomic populations and bond orders.
     The atomic orbitals to project on are taken from the
    \&BASIS section. If there is no \&BASIS section in the input
    a minimal Slater basis is used. See section~\ref{input:basis}
    for more details.
     }

\keyword{PROJECT}{}{[NONE, DIAGONAL, FULL]}{}{\&CPMD}
  \desc{This keyword is controlling
      the calculation of the constraint force in optimization runs.}

\spekeyword{PROP\_TSTEP}{}{}{}{\&PTDDFT}{PROP-TSTEP}
  \desc{Propagation timestep used in Ehrenfest dynamics. It is used in the
spectra calculation ({\bf PROPAGATION SPECTRA} ) to specify the time step
for the propagation of the KS orbitals.}

\keyword{PROPAGATION SPECTRA}{}{}{}{\&CPMD}
  \desc{Calculates the electronic absorption spectra using the
        TDDFT propagation of the Kohn-Sham orbitals. Use the section
        \&PTDDFT to define the parameters. Use this principal keyword 
        always with CAYLEY (in \&CPMD). The program produces a file 
        "dipole.dat" with the time series of the variation of the dipole
        in x, y, and z directions. After Fourier transform of this file 
        one gets the desired absorption spectra.\\
        Typical (minimal) input file (for the sections \&CPMD and \&PTDDFT)
        \&CPMD \\
          PROPAGATION SPECTRA
          RESTART WAVEFUNCTION COORDINATES LATEST
          CAYLEY
        \&END \\
        \&PTDDFT \\ 
         ACCURACY \\
         1.0D-8 \\
         N\_CYCLES \\ 
         100000 \\
         PROP\_TSTEP \\
         0.01 \\ 
         EXT\_PULSE \\ 
         1.D-5 \\ 
         PERT\_DIRECTION \\ 
         1 \\
         RESTART \\
         2 \\
       \&END \\
       The time step is specified by setting \refkeyword{PROP-TSTEP}. The
       total number of iteration is controlled by \refkeyword{N-CYCLES}.}

\keyword{PROPERTIES}{}{}{}{\&CPMD}
  \desc{Calculate some properties.\\
      This keyword requires
      further input in the section \&PROP \dots \&END.}

\keyword{PROPERTY}{ \{ STATE \}}{}{}{\&TDDFT}
  \desc{Calculate properties of excited states at the end of an
        \refkeyword{ELECTRONIC SPECTRA} calculations. default is to calculate
        properties for all states. Adding the keyword {\bf STATE} allows
        to restrict the calculation to only one state. The number of the
        state is read from the next line.}

\keyword{RING-POLYMER DYNAMICS}{}{}{}{\&PIMD}
  \desc{Perform ring-polymer molecular dynamics (RPMD)~\cite{Craig2004}.  
        In the present implementation the fictitious mass of the  nuclei for 
        all the replicas is set to the real physical mass divided by the number 
        of replicas (P) in the primitive representation and the physical mass 
        in the normal mode representation for compatibility with the formulation 
        of the conventional path integral. According to our choice of the fictitious 
        masses, the target temperature you should set is not TP in the original
        formulation of Ref.~\cite{Craig2004}  but just T as in the standard path 
        integral and centroid MD. Note that RPMD in the staging mode representation 
        is not supported and the generalized Langevin thermostat is available in the 
        normal mode representation only.}

\keyword{QMMM}{}{[QMMMEASY]}{}{\&CPMD}
\desc{Activate the hybrid QM/MM code. This keyword requires
      further input in the section \&QMMM \dots \&END.

      The QM driver is the standard CPMD.
      An interface program ({\bf MM\_Interface}) and a classic force field
      (Gromos\cite{gromos96}/Amber\cite{amber7}-like) are needed to run the
      code in hybrid mode\cite{qmmm02,qmmm03,qmmm04,qmmm05,qmmm06}. 
      This code requires a {\it special licence} and
      is {\bf not} included in the standard CPMD code.
% FIXME: AK 2005/07/10
% we should put a contact address or web page here.
      (see section~\ref{sec:qmmm} for more
      information on the available options and the input format).}

\spekeyword{QS\_LIMIT}{}{}{}{\&LINRES}{QS-LIMIT}
 \desc{
 Tolerance above which we use quadratic search algorithm in linear response calculations.
 }

\keyword{QUENCH}{}{[IONS, ELECTRONS, CELL, BO]}{}{\&CPMD}
  \desc{The {\bf velocities} of the {\bf ions}, {\bf wavefunctions} or
      the {\bf cell} are set to zero at the beginning of a run.\\
      With the option {\bf BO} the wavefunctions are converged
      at the beginning of the MD run.}

\keyword{RAMAN}{}{}{}{\&RESP}
  \desc{Calculate the polarizability (also in periodic systems) as
  well as Born-charges and dipole moment.}

\keyword{RANDOMIZE}{}{[COORDINATES, WAVEFUNCTION, DENSITY, CELL]}{}{\&CPMD}
  \desc{The {\bf ionic positions} or the {\bf wavefunction} or the
      {\bf cell parameters} are {\bf randomly displaced} at the
      beginning of a run.\\
      The maximal amplitude of the displacement is read from the
      next line.}

\spekeyword{RANDOMIZE}{}{}{}{\&TDDFT}{RANDOMIZE TDDFT}
  \desc{Randomize the initial vectors for the diagonalization in a TDDFT
       calculation. The amplitude is read from the next line. Default is not
       to randomize the vectors.}

\keyword{RATTLE}{}{}{}{\&CPMD}
  \desc{This option can be used to set
      the maximum number of iterations and the tolerance for the
      {\bf iterative orthogonalization}. These two numbers are read from
      the next line. \\
      {\bf Defaults} are 30 and $10^{-6}$.}

\keyword{READ REPLICAS}{}{}{}{\&PIMD}
  \desc{Read all $P$ replicas from a file with a name to be specified
        in the following line, for the input format see subroutine
        rreadf\_utils.mod.F90.}

\keyword{REAL SPACE WFN KEEP}{}{[SIZE]}{}{\&CPMD}
  \desc{The real space wavefunctions are kept in memory for later reuse.
      This minimizes the number of Fourier transforms and can result in
      a significant speedup at the expense of a larger memory use.
      With the option {\bf SIZE} the maximum available memory for the
      storage of wavefunctions is read from the next line (in MBytes).
      The program stores as many wavefunctions as possible within the
      given memory allocation.}

\keyword{REFATOM}{}{}{}{\&HARDNESS}
  \desc{Specify the reference atom to be used in a hardness calculation on the next line.
       This option is to be used together with the \refkeyword{ORBITALS} and
       \refkeyword{LOCALIZE}.}

\keyword{REFERENCE CELL}{}{[ABSOLUTE, DEGREE, VECTORS]}{}{\&SYSTEM}
  \desc{This cell is used to calculate the Miller indices in a
      constant pressure simulation.
      This keyword is only active together with the
      option {\bf PARRINELLO-RAHMAN}.\\
      The parameters specifying the reference (super) cell are
      read from the next line. \\
      Six numbers in the following order
      have to be provided: $a$, $b/a$, $c/a$, $\cos \alpha$,
      $\cos \beta$, $\cos \gamma$.\\
      The keywords {\bf ABSOLUTE} and {\bf DEGREE } are
      described in {\bf CELL} option.}

\keyword{REFUNCT}{}{}{functionals}{\&DFT}
  \desc{Use a special reference functional in a calculation.
        This option is not active.}

\keyword{REORDER LOCAL}{}{}{}{\&TDDFT}
  \desc{Reorder the localized states according to a distance criteria.
       The number of reference atoms is read from the next line.
       On the following line the position of the reference atoms
       within the set of all atoms has to be given.
       The keyword \refkeyword{LOCALIZE} is automatically set.
       The minimum distance of the center of charge of each
       state to the reference atoms is calculated and the states
       are ordered with respect to decreasing distance.
       Together with the {\sl SUBSPACE} option in
       a \refkeyword{TAMM-DANCOFF} calculation this can be used
       to select specific states for a calculation.}

\keyword{REORDER}{}{}{}{\&TDDFT}
  \desc{Reorder the canonical Kohn--Sham orbitals prior to a TDDFT calculation.
       The number of states to be reordered is read from the next line.
       On the following line the final rank of each states has to be given.
       The first number given corresponds to the HOMO, the next to the HOMO - 1
       and so on. All states down to the last one changed have to be specified,
       no holes are allowed.
       This keyword can be used together with the {\sl SUBSPACE} option in
       a \refkeyword{TAMM-DANCOFF} calculation to select arbitrary states.
       Default is to use the ordering of states according to the Kohn--Sham
       eigenvalues.}

\keyword{REPLICA NUMBER}{}{}{}{\&PATH}
  \desc{Number of replicas along the string.}

\keyword{RESCALE OLD VELOCITIES}{}{}{}{\&CPMD}
  \desc{Rescale {\bf ionic} velocities after \refkeyword{RESTART} to the 
    temperature specified by either \refkeyword{TEMPERATURE}, 
    \refkeyword{TEMPCONTROL} {\bf IONS}, or \refkeyword{NOSE} {\bf IONS}.
    Useful if the type of ionic thermostatting is changed,
    (do not use RESTART NOSEP in this case). \\
    Note only for path integral runs: the scaling is only applied
    to the first (centroid) replica.}

\keyword{RESTART}{}{}{[{\it OPTIONS}]}{\&CPMD}
  \desc{This keyword controls what data is read (at the beginning)
      from the file RESTART.x.\\
      {\bf Warning:} You can only read data that has been previously
      written into the RESTART-file.\\
      A list of different {\it OPTIONS}\ can be specified.
      List of valid options:}
%%%%%%%%%%%%%%%%%%%
% special input
%%%%%%%%%%%%%%%%%%%
      \begin{description}
         \item[WAVEFUNCTION]
               Read old {\bf wavefunction} from restart file.
         \item[OCCUPATION]
               Read old {\bf occupation numbers} (useful for free energy
               functional.
         \item[COORDINATES]
               Read old {\bf coordinates} from restart file.
         \item[VELOCITIES]
               Read old {\bf ionic, wavefunction and (cell) velocities}
               from restart file.
         \item[CELL]
               Read old {\bf cell parameters} from restart file.
         \item[GEOFILE]
               Read old {\bf ionic positions and velocities} from
               file {\bf GEOMETRY}. This file is updated
               every time step. It has higher priority than
               the COORDINATES option.
         \item[ACCUMULATORS]
               Read old {\bf accumulator values}, for example the time step number, from restart file.
         \item[HESSIAN]
               Read old {\bf approximate Hessian} from file {\em HESSIAN}.
         \item[NOSEE]
               Restart {\bf Nos\'e thermostats} for {\bf electrons} with
               values stored on restart file.
         \item[NOSEP]
               Restart {\bf Nos\'e thermostats} for {\bf ions} with
               values stored on \\ restart file.
         \item[NOSEC]
               Restart {\bf Nos\'e thermostats} for {\bf cell}
               parameters with values stored on restart file.
         \item[LATEST]
               Restart from the {\bf latest restart} file as indicated
               in file {\em LATEST}.
         \item[PHESS]
               Read partial Hessian (Hessian of the reaction core) for
               {\bf transition state search} or {\bf vibrational analysis}
               from restart file.
               Useful with the keywords \refkeyword{PRFO} or
               \refkeyword{HESSIAN} [DISCO,SCHLEGEL,UNIT] PARTIAL.
         \item[LSSTAT]
               Read all {\bf status information} of the {\bf linear scaling
               optimizers} (L-BFGS and P-RFO) including L-BFGS history but
               excluding partial Hessian for P-RFO from restart file.
               The {\bf partial Hessian} is read separately using
               {\bf RESTART PHESS}.
               Useful with the keywords \refkeyword{LBFGS} and/or \refkeyword{PRFO}.
         \item[ADPTTL]
               Read {\bf wavefunction convergence criteria} at the current point
               of geometry optimization from restart file.
               Useful with the keywords \refkeyword{CONVERGENCE} [ADAPT, ENERGY, CALFOR].
         \item[VIBANALYSIS]
               Use the information on finite differences
               stored in the file {\em FINDIF}. This
               option requires a valid restart file for
               the wavefunctions, even when wavefunctions
               and coordinates are recalculated or read
               from the input file.
         \item[POTENTIAL]
               Read an old potential from the restart file.
               This applies to restarts
               for Kohn-Sham energy calculations.
         \item[KPOINTS]
               Restart with k points.
         \item[DENSITY]
              Restart with electronic density.
         \item[GLE]
              Restart the extended variables of the GLE dynamics
         \item[PRNG]
              Restart the internal state of the Marsaglia random number generator
         \item[CONSTRAINTS]
              Restart with old values for constraints. This option
              is mainly for restraints with GROWTH option.
         \item[EXTRAP]
              Restart from a previously saved wavefunction history.
              See \refkeyword{EXTRAPOLATE WFN} for details.
         \item[CVDFT]
              Read CDFT multiplier $V$ from \verb|CDFT_RESTART|, for HDA run read 
              $V$ and wavefunction of the appropriate state.
         \item[ALL]
              Restart with all fields of RESTART file
      \end{description}
%%%%%%%%%%%%%%%%%%%%%%%
% end of special input
%%%%%%%%%%%%%%%%%%%%%%%

\keyword{RESTFILE}{}{}{}{\&CPMD}
  \desc{The number of distinct \refkeyword{RESTART} files generated
    during CPMD runs is read from the next line.\\
    The restart files are written in turn.
    {\bf Default is 1}. If you specify e.g.~3, then the files
    RESTART.1, RESTART.2, RESTART.3 are used in rotation.}

\keyword{RESTFILE}{}{}{}{\&PTDDFT}
   \desc{Defines a restart code for the restart of the Ehrenfest dynamics
         (\refkeyword{MOLECULAR DYNAMICS} EH) and the propagation spectra 
         (\refkeyword{PROPAGATION SPECTRA}). The restart option is read from the 
         next line: 0(=default) restart from the (complex)wavefunctions in the file
         wavefunctions. This option is used in case of a continuation run; 
         1. restart from the orbital files WAVEFUNCTION.n, where $n$ is the index 
         of the KS orbital and runs from $1$ to the number of states (This states 
         a prepare in a previous run using the KOHN-SHAM ENERGIES
         principal keyword), 2; restart from the orbitals stored in RESTART (obtained
         from a optimization run with tight convergence (at least 1.D-7)).}

\keyword{REVERSE VELOCITIES}{}{}{}{\&CPMD}
  \desc{Reverse the ionic and electronic (if applicable) velocities
      after the initial setup of an MD run. This way one can, e.g., 
      go ``backwards'' from a given \refkeyword{RESTART} to improve
      sampling of a given MD ``path''. }

\keyword{RFO ORDER=nsorder}{}{}{}{\&CPMD}
  \desc{Rational function approximation combined with
      a quasi-Newton me\-thod (using BFGS) for {\bf optimization} of the
      {\bf ionic positions} is used~\cite{Banerjee85}.
      A saddle point of order nsorder is searched for.}

\keyword{RHOOUT}{}{[BANDS,SAMPLE=nrhoout]}{}{\&CPMD}
  \desc{{\bf Store} the {\bf density} at the end of the run
      on file {\em DENSITY}. \\
      If the keyword BANDS is defined then on
      the following lines the number of bands (or orbitals) to be 
      plotted and their index (starting from 1) have to be given. 
      If the position specification is a negative number, then the 
      wavefunction instead of the density is written. Each band is 
      stored on its own file {\em DENSITY.num}. For spin polarized 
      calculations besides the total density also the spin density 
      is stored on the file {\em SPINDEN}. The following example will
      request output of the orbitals or bands number 5, 7, and 8 as
      wavefunctions:}
\begin{verbatim}
            RHOOUT BANDS
              3 
              -5 -7 -8

\end{verbatim}
    \desc{
      With the optional keyword {\bf SAMPLE} the requested file(s) will 
      be written every {\em nrhoout} steps during an MD trajectory. 
      The corresponding time step number will be appended to the filename.
}

\keyword{ROKS}{}{\{SINGLET, TRIPLET\},\{DELOCALIZED, LOCALIZED, GOEDECKER\}}{}{\&CPMD}
  \desc{Calculates the first excited state using Restricted Open-shell Kohn-Sham 
      theory~\cite{Frank98}. By default, the singlet state is calculated using the 
      delocalized variant of the modified Goedecker-Umrigar scheme, which is supposed 
      to work in most cases. That is, for doing a ROKS simulation, it is usually 
      sufficient to just include this keyword in the CPMD section (instead of using 
      the \refspekeyword{LSE}{LOW SPIN EXCITATION} input). 
      See \ref{hints:roks} for further information.}

\keyword{ROTATION PARAMETER}{}{}{}{\&TDDFT}
  \desc{The parameters for the orbital rotations in an optimized
       subspace calculation (see \refkeyword{TAMM-DANCOFF})
       are read from the next line. The total number of iterations
       (default 50), the convergence criteria (default $10^{-6}$) and
       the step size (default 0.5) have to be given.}

\spekeyword{RUNGE\_KUTTA}{}{}{}{\&CPMD}{RUNGE-KUTTA}
  \desc{Defines the integration schemes used in the Ehrenfest \refkeyword{MOLECULAR DYNAMICS}.
        Always used this option.}

\keyword{SCALED MASSES}{}{[OFF]}{}{\&CPMD}
  \desc{Switches the usage of g-vector dependent masses on/off. \\
      The number of shells included in the analytic integration
      is controlled with the keyword {\bf HAMILTONIAN CUTOFF}.\\
      By {\bf default} this option is switched {\bf off}.}

\keyword{SCALE}{}{[CARTESIAN]}{[S=sascale] [SX=sxscale] [SY=syscale] [SZ=szscale]}{\&SYSTEM}
  \desc{{\bf Scale atomic coordinates} of the system with the
      lattice constants (see {\bf CELL}).
      You can indicate an additional scale for each axis with the options
      {\bf SX}, {\bf SY} and {\bf SZ}.
      For instance, if you indicate SX=sxscale,
      you give your x-coordinates
      between $0.$ and sxscale (by default $1.$).
      This is useful when you use many primitive cells.
      With the keyword {\bf CARTESIAN}, you specify that the given
      coordinates are in Cartesian basis, otherwise the default with
      the {\bf SCALE} option is in direct lattice basis.
      In all cases, the coordinates are multiplied by the lattice
      constants. If this keyword is present an output file
      GEOMETRY.scale is written. This file contains the lattice vectors
      in \AA and atomic units together with the atomic coordinates in
      the direct lattice basis.}

\keyword{SCALES}{}{}{scaling\_for\_functional\_1, scaling\_for\_functional\_2, \dots}{\&DFT}
  \desc{Only available in combination with \refspekeyword{XC\_DRIVER}{XC DRIVER}.
        On the same line, rescaling values ($\in [0,1]$) for the different exchange-correlation
        functionals specified in \refkeyword{FUNCTIONAL} are read in the same
        order in which the functionals were specified.}

\spekeyword{HFX\_SCALE}{}{}{scaling\_for\_exact\_exchange}{\&DFT}{HFX SCALE}
  \desc{Only available in combination with \refspekeyword{XC\_DRIVER}{XC DRIVER}.
        On the same line, a scaling value ($\in [0,1]$) for the exact exchange contribution to the
        overall xc functional is specified.
        If used with a built-in hybrid functional, the default value is changed; if used with any other functional,
        the keyword activates the inclusion of exact exchange. Note that its value should always be set to one
        if combined with a Coulomb-attenuated functional (in which case the global scaling can be modified by changing
        the CAM parameters $\alpha$ and $\beta$, instead).}

\spekeyword{KERNEL\_SCALES}{}{}{scaling\_for\_kernel\_1, scaling\_for\_kernel\_2, \dots}{\&DFT}{KERNEL SCALES}
  \desc{Identical functionality as described in \refkeyword{SCALES}, but applies to the \refkeyword{LR KERNEL}
        rather than the \refkeyword{FUNCTIONAL}.}

\spekeyword{KERNEL\_HFX\_SCALE}{}{}{scaling\_for\_exact\_exchange\_in\_kernel}{\&DFT}{KERNEL HFX SCALE}
  \desc{Identical functionality as described in \refspekeyword{HFX\_SCALE}{HFX SCALE}, but applies to the \refkeyword{LR KERNEL}
        rather than the \refkeyword{FUNCTIONAL}.}

\keyword{SCEX}{}{}{}{\&DFT}
  \desc{Activate the use of the coordinate-scaled exact exchange scheme by Bircher and
        Rothlisberger\cite{bircher_rothlisberger18b} for \emph{isolated systems}.
        Speedups of up to one order of magnitude can be achieved. Please note that
        the simulation supercell must imperatively span twice the size of the charge density,
        and that the molecule has to remain centred in the box.}

\keyword{SCALED EXCHANGE}{}{}{}{\&DFT}
  \desc{Alias for \refkeyword{SCEX}.}

\keyword{SCREENED EXCHANGE}{\{ASHCROFT,CAM,ERFC,EXP\}}{}{}{\&DFT}
  \desc{Screening / range separation applied to the Hartree-Fock exchange term \emph{only}.
        Possible options are: 

     \begin{description}
        \item[ASHCROFT]\hfill \\
          {Ashcroft exchange. $r_\text{cut}$ is read from the next line.}
        \item[CAM]\hfill \\
          {Coulomb-attenuation method. $\alpha$, $\beta$ and $\mu$ are read from the next line.
           Please note that in case attenuation of the (semi-)local part is desired, too, the
           keyword \refkeyword{COULOMB ATTENUATION} has to be used.}
        \item[ERFC]\hfill \\
          {Screened exchange using the complementary error function.
           The switching parameter $\gamma$ is read from the next line.}
        \item[EXP]\hfill \\
          {Screened exchange using an exponential function.
           The switching parameter $\gamma$ is read from the next line.}
     \end{description}

        }

\keyword{SHIFT POTENTIAL}{}{}{}{\&CPMD}
  \desc{After this keyword, useful in Hamiltonian diagonalization, the 
      shift value $V_{\rm shift}$ must be provided in the next line.
      This option is used in the Davidson diagonalization 
      subroutine and shifts rigidly the total electronic potential as 
      $V_{\rm pot}({\bf r}) \to V_{\rm pot}({\bf r})+V_{\rm shift}$
      then it is subtracted again at the end of the main loop, restoring
      back the original $V_{\rm pot}({\bf r})$ that remains basically
      unaffected once that the calculation is completed.}

\keyword{SLATER}{}{[NO]}{}{\&DFT}
  \desc{The $\alpha$ value for the Slater exchange
      functional~\cite{Slater51} is read from the next line.
      With NO the exchange functional is switched off.\\
      Default is a value of 2/3.\\
      This option together with no correlation functional, allows for
      $X\alpha$ theory.}

\keyword{SMOOTH}{}{}{}{\&DFT}
  \desc{A smoothening function is applied to the density~\cite{Laasonen93}.\\
      The function is of the Fermi type.
      \[  f(G) = \frac{1}{%
    \displaystyle{1 + e^{\frac{\scriptstyle{G - G_{\scriptstyle cut}}}
                              {\scriptstyle\Delta}}}} \]
      G is the wavevector, $G_{cut} = \alpha\,G_{max}$ and
      $\Delta = \beta\,G_{max}$. Values for $\alpha$ and $\beta$
      have to be given on the next line.}

\keyword{SOC}{}{}{}{\&CPMD}
 \desc{Compute the spin-orbit coupling (SOC) elements between the two states specified in the next line.
       To be used together with the principal keyword SPECTRA.
       The calculation is based on the Auxiliary Many-Electron Wavefunction Approach for TDDFT developed
       in Ref.~\cite{taver}. For more information see Ref.~\cite{SOC}.}

\keyword{SPLINE}{}{[POINTS, QFUNCTION, INIT, RANGE]}{}{\&CPMD}
  \desc{This option controls the generation of the pseudopotential
      functions in g-space. \\
      All pseudopotential functions are first initialized on a evenly
      spaced grid in g-space and then calculated at the needed
      positions with a spline interpolation.
      The number of spline points is read from the next
      line when {\bf POINTS} is specified. \\
      ( The {\bf default} number is {\bf 5000}.)
      For calculations with the small cutoffs typically used together with
      Vanderbilt PP a much smaller value, like 1500 or 2000, is sufficient. \\
      In addition it is possible to keep the Q-functions of
      the Vanderbilt pseudopotentials on the spline grid during
      the whole calculation and do the interpolation whenever needed.
      This option may be useful to save
      time during the initialization phase and memory in the case of
      Vanderbilt pseudopotentials when the number of
      shells is not much smaller
      than the total number of plane waves, i.e. for all cell
      symmetries except simple cubic and fcc.}

\keyword{SSIC}{}{}{}{\&CPMD}
  \desc{Apply an {\it ad hoc} Self Interaction Correction (SIC) to the 
        ordinary DFT calculation expressed in terms of total energy as
        \begin{equation*}
        E^{\rm tot}-a\cdot  E_H[m]- b\cdot E_{xc}[m, 0]
        \end{equation*}
        where $m({\bf x}) = \rho_\alpha({\bf x})-\rho_\beta({\bf x})$.
        The value of $a$ must be supplied in the next line, while
        in the present implementation $b$ is not required, being
        the optimal values $a=0.2$ and $b=0.0$ according to
        Ref.~\cite{SSIC}. These are assumed as default values
        although it is not always the case \cite{dna_sic}.
        Note that if you select negative $\{a, b \}$ parameters,
        the signs in the equation above will be reversed.
        The Hartree electronic potential is changed accordingly
        as $V_H[\rho] \to V_H[\rho] \pm a\cdot V_{\rm SIC}[m]$,
        being
        \begin{equation*}
        V_{\rm SIC}[m]=\frac{\delta E_H[m]}{\delta m({\bf x})}
        \end{equation*}
        where the sign is $+$ for $\alpha$ spin and $-$ for
        $\beta$ spin components, respectively.
        Be aware that this keyword should be used together with
        $LSD$ (set by default).}

\keyword{STAGING}{}{}{}{\&PIMD}
  \desc{Use the staging representation~\cite{Tuckerman96}
        of the path integral propagator. It is possible to impose a
        mass disparity between centroid and non--centroid coordinates by
        dividing the fictitious masses of only the {\em non}--centroid
        $s=2, \dots ,P$ replicas by
        the adiabaticity control factor FACSTAGE. This dimensionless
        factor {\em must always} be specified in the following line.
        Note: the eigen--{\em frequencies} of the $s>1$ replicas are changed
        by only $\sqrt{\mbox{FACSTAGE}}$, see Ref.~\cite{Martyna96}(b).
        Note: using FACSTAGE~$\not= 1.0$ essentially makes no sense
        within the STAGING scheme, but see its use within
        CENTROID DYNAMICS and NORMAL MODES.}

\keyword{STATES}{}{}{}{\&SYSTEM}
  \desc{The number of states used in the calculation is read
      from the next line. \\
      This keyword has to precede the keyword
      {\bf OCCUPATION}.}

\keyword{NSUP}{}{}{}{\&SYSTEM}
  \desc{The number of states of the same spin as the first state is read
      from the next line. \\
      This keyword makes only sense in spin-polarized calculations (keyword
      \refkeyword{LSD}).}

\spekeyword{STATES}{\{MIXED,SINGLET,TRIPLET\}}{}{}{\&TDDFT}{STATES TDDFT}
  \desc{The number of states to be calculated is read from the next line.
       The type of state {\sl SINGLET, TRIPLET} can be given for
       non-spinpolarized calculations. Default is to calculate
       one singlet state for LDA and 1 mixed state for LSD calculations.}

\keyword{STEEPEST DESCENT}{}{[ELECTRONS, IONS, CELL],[NOPRECONDITION\-ING],[LINE]}{}{\&CPMD}
  \desc{NOPRECONDITIONING works only for electrons and LINE only for ions.
      Use the method of {\bf steepest descent} for the {\bf optimization}
      of wavefunction and/or atomic positions and/or cell.\\
      If both options are specified in a geometry optimization
      run, a simultaneous optimization is performed. \\
      Preconditioning of electron masses (scaled masses)
      is used by default. The preconditioning is controlled by the
      keyword {\bf HAMILTONIAN CUTOFF}.
      Optionally preconditioning can be disabled.\\
      For ions optimization, the step length is controlled by the
      keywords {\bf TIMESTEP} and {\bf EMASS}.}

\keyword{STEPLENGTH}{}{}{}{\&LINRES}
 \desc{
 Step length for steepest descent and preconditioned conjugate gradient
 methods used in linear response calculations. Default is 0.1.
 }

\keyword{STORE}{ \{OFF\}} {[WAVEFUNCTIONS, DENSITY, POTENTIAL]}{}{\&CPMD}
  \desc{The \refkeyword{RESTART} file is {\bf updated} every {\sl ISTORE} steps.
      {\sl ISTORE} is read from the next line.
      {\bf Default} is at the {\bf end of the run}.\\
      Moreover, in the same line of the number ISTORE, you can specify
      the number of self-consistent iterations (with SC=number) between
      two updates of restart file.
      If OFF is specified , do not store wavefunctions and/or density
      ({\sl ISTORE} is not necessary).}

\spekeyword{STRESS TENSOR}{}{}{}{\&CPMD}{STRESS TENSOR CPMD}
  \desc{Calculate the {\bf stress tensor} every {\sl NSTEP}
      iteration in a constant volume MD.\\
      {\sl NSTEP} is read from the next line.
      Works also for wavefunction or geometry optimisation.
      In this case NSTEP is meaningless.}

\spekeyword{STRESS TENSOR}{}{}{}{\&SYSTEM}{STRESS TENSOR SYSTEM}
  \desc{In extension to the keyword PRESSURE the complete
      {\bf stress tensor} in kbar can be specified.
      The {\bf stress} on the system is read in the form:
      \begin{center}
       $t_{11}\ t_{12}\ t_{13}$\\
       $t_{21}\ t_{22}\ t_{23}$\\
       $t_{31}\ t_{32}\ t_{33}$\\
       \end{center}
       }

\keyword{STRUCTURE}{}{[BONDS, ANGLES, DIHEDRALS, SELECT]}{}{\&CPMD}
  \desc{Print {\bf structure information} at the end of the run. \\
      Bonds, angles and dihedral angles can be printed.
      Dihedral angles are defined between
      0 and 180 degrees. This might change in the future.\\
      If the option {\bf SELECT} is used the output is restricted
      to a set of atoms. The number of atoms and a list of the selected
      atoms has to be given on the next lines.}

\keyword{SUBTRACT}{}{[COMVEL, ROTVEL]}{}{\&CPMD}
  \desc{If COMVEL is selected, the total momentum of the system is
    removed, if ROTVEL is selected the global angular momentum of the
    system is removed. Both options can be used separately and
    simultaneously. The subtraction is done each {\bf ncomv} or
    {\bf nrotv} steps, where the value is read in the next line.\\

    If this key is activated but no number provided,
    the {\bf default} is $10000$ steps. \\

    {\bf Note}: The use of these keywords is strongly recommended
    for long runs (e.g. $t>10$ ps) and/or low density systems (e.g.
    isolated molecules, gas phase \& Co.). Otherwise the whole system
    will start to translate and/or rotate toward a (random) direction
    with increasing speed and spinning. The ``relative'' translation
    within the system slows down correspondingly and thus the system
    effectively cools down. As a consequence dynamic properties, like
    self-diffusion coefficients will be wrong.\\

    This option should not be used for systems, where some atoms are
    kept at fixed positions, e.g. slab configurations. Here the center
    of mass may (or should) move. Due to the interactions with the fixed
    atoms, a drift of the whole system should be much reduced,
    anyways.\\

    {\bf Note}: since the subtracted kinetic energy is put back into
    the system by simple rescaling of the ionic velocities, these
    options is not fully compatible with \refkeyword{NOSE} thermostats.
}

\keyword{SURFACE HOPPING}{}{}{}{\&CPMD}
  \desc{Nonadiabatic dynamics involving the ground state and a \refkeyword{ROKS}
      excited state\cite{surfhop}.
      Do NOT use this keyword together with \refkeyword{T-SHTDDFT}, which invokes
      the surface hopping MD scheme based on TDDFT~\cite{TDDFT-SH} (see \refkeyword{T-SHTDDFT}).
      }

\keyword{SURFACE}{}{[XY, YZ, ZX]}{}{\&SYSTEM}
  \desc{By default, if nothing is specified, assume {\bf periodic boundary} condition in
      {\bf $x$- and $y$-direction}. With the extra keywords {\sl XY}, {\sl YZ} or 
       {\sl ZX}, the periodicity of the systems is assumed to be along $(x,y)$, $(y,z)$
        or $(z,x)$, respectively.\\
%        You also need to set the 'cluster option' (i.e. \refkeyword{SYMMETRY} 0).
      }

\keyword{SYMMETRIZE COORDINATES}{}{}{}{\&SYSTEM}
  \desc{{\bf Input coordinates} are {\bf symmetrized} according
      to the {\bf point group}
      specified. \\
      This only makes sense when the structure already
      is close to the symmetric one.}

\keyword{SYMMETRY}{}{}{}{\&SYSTEM}
  \desc{The {\bf supercell symmetry type} is read from the next line.\\
      You can put a number or a keyword.
      {\small
      \begin{description}
      \renewcommand{\makelabel}[1]{\hbox to 2em {\hfill#1}}
               \item[0]  {\bf ISOLATED}
                         system in a cubic/orthorhombic
                         box~\cite{Hockney70,Landman}
                  with ISOLATED MOLECULE option activated.
                  By default the Hockney method (see \refkeyword{POISSON SOLVER}) is used for
                  solving the Poisson equations.
                  You can use this option in combination with \refkeyword{POLYMER} or
                  \refkeyword{SURFACE} for systems that are periodic in only
                  1 or 2 dimensions. The default Poisson solver is MORTENSEN in
                  this case. See the Hints and Tricks section
                  for some additional requirements when calculating
                  isolated system.
               \item[1]  Simple {\bf CUBIC}
               \item[2]  {\bf FACE CENTERED CUBIC}
                                         ({\bf FCC})
               \item[3]  {\bf BODY CENTERED CUBIC}
                         ({\bf BCC})
               \item[4]  {\bf HEXAGONAL}
               \item[5]  {\bf TRIGONAL} or {\bf RHOMBOHEDRAL}
               \item[6]  {\bf TETRAGONAL}
               \item[7]  {\bf BODY CENTRED TETRAGONAL}
                         ({\bf BCT})
               \item[8]  {\bf ORTHORHOMBIC}
               \item[12] {\bf MONOCLINIC}
               \item[14] {\bf TRICLINIC}
       \end{description}
       }
       Warning: This keyword should not be used with the keyword
       {\bf CELL VECTORS}.}

\keyword{T-SHTDDFT}{}{}{}{\&TDDFT}
  \desc{Non adiabatic (nonadiabatic, non-adiabatic) Tully's trajectory surface hopping dynamics using TDDFT energies and forces.
        To be used together with the keywords \refkeyword{MOLECULAR DYNAMICS} BO
        and \refkeyword{TDDFT} in the \&CPMD section (see section~\ref{sec:TDDFTdynamics}).
        Do NOT use the keyword \refkeyword{T-SHTDDFT} together with the keyword
        \refkeyword{SURFACE HOPPING} in \&CPMD, which invokes the SH scheme based on \refkeyword{ROKS}~\cite{surfhop}
        (see \refkeyword{SURFACE HOPPING}).\\
        For a given initial configuration, the run produces a trajectory
        that undergoes surface hopping according to the algorithm
        by Tully adapted to TDDFT~\cite{TDDFT-SH}.
        The forces on the excited state surfaces are computed using TDDFT as
        for the adiabatic case.
        A sufficiently large number of excited states must be declared using the
        keyword \refkeyword{STATES} in the section \&TDDFT. The initial running surface is
        specified with the keyword \refkeyword{FORCE STATE} in the section \&TDDFT.
        This can change during the dynamics when a surface hop occurs. After a
        restart the value of the running state is taken from the file SH\_STATE.dat (see below).
        The run produces a series of additional files: SH\_COEFA.dat (absolute value of the state
        amplitudes), SH\_COEFC.dat (their complex values), SH\_COUPL.dat
        (the coupling strength per state), SH\_ENERG.dat (the energy of the different
         states: setp number, ground state energy, first excited state energy, $\dots$, highest
         excited state energy, energy of the running state), SH\_PROBS.dat (transition
         probabilities between running state and all other states), SH\_STATE.dat
         (the running state at each step). All these files (in addition to SH\_WAVEFUNCTIONS and
        SH\_LRWAVEFUNCTIONS) are needed to restart the SH dynamics.
        Note that each run produces a single SH trajectory. Several subsequent runs starting from
        different initial coordinates and velocities are required to collect statistics.
       }

\keyword{TAMM-DANCOFF}{}{[SUBSPACE,OPTIMIZE]}{}{\&TDDFT}
  \desc{Use the Tamm--Dancoff approximation. This is the default for
       TDDFT calculations.
       Optionally, only a {\sl SUBSPACE}
       of the occupied orbitals can be included in the calculation.
       The subspace can be optimized at each step (not recommended).
       Default is to use all states.}

%\keyword{TASKGROUPS}{}{[MINIMAL,MAXIMAL,CARTESIAN]}{}{\&CPMD}
%  \desc{The number of taskgroups is
%      read from the next line. The number of taskgroups has to be a
%      divisor of the number of nodes in a parallel run;
%      Cartesian Taskgroups use cartesian communicators.}

\spekeyword{TD\_METHOD\_A}{}{[{\em functionals}]}{}{\&TDDFT}{TD METHOD A}
  \desc{Use a different potential for the eigenvalue differnce part
        of the response equations than was used to generate the ground state
        orbitals. The potential generating functional has to be given after the
        keyword. For possible functionals see the code. Most likely you want
        to use the {\bf SAOP} functional. \\
        This functional does not affect the choice of functional used in the
        TDDFT kernel. The kernel functional is set in the \&DFT section.
        It is either the standard functional or the functional defined by
        the keyword \refkeyword{LR KERNEL}.}

\keyword{TDDFT}{}{}{}{\&CPMD}
  \desc{Calculate the energy according to TDDFT. This keyword can be used
       together with \refkeyword{OPTIMIZE GEOMETRY} or \refkeyword{MOLECULAR DYNAMICS} BO.
       Use the \&TDDFT section to set parameters for the calculation.
       This keyword requires \refkeyword{RESTART} LINRES.}

\spekeyword{TD\_POTENTIAL}{}{}{}{\&PTDDFT}{TD-POTENTIAL}
  \desc{Defines a time dependent external potential to be used in Ehrenfest dynamics
        (\refkeyword{MOLECULAR DYNAMICS} EH). Can be used with the keyword \refkeyword{PIPULSE}.
        The frequency of the external field is read from the next line (in atomic units).
        For more information see subroutine gaugepot\_laser in td\_utils.mod.F90 .}

\keyword{TEMPCONTROL}{}{[IONS, ELECTRONS, CELL]}{}{\&CPMD}
  \desc{The {\bf temperature} of the {\bf ions} in Kelvin
      or the {\bf fictitious kinetic energy} of the {\bf electrons}
      in atomic units
      or the {\bf kinetic energy} of the {\bf cell} in atomic units (?)
      is controlled by scaling. \\
      The {\bf target} temperature and the {\bf tolerance} for the
      ions or the target kinetic energy and the tolerance for the
      electrons or the cell are read from the next line.

      As a gentler alternative you may want to try 
      the \refkeyword{BERENDSEN} scheme instead.}

\keyword{TEMPERATURE ELECTRON}{}{}{}{\&CPMD}
  \desc{The {\bf electronic temperature} is read from the next line.\\
      {\bf Default} is $1000$K.}

\keyword{TEMPERATURE}{}{[RAMP]}{}{\&CPMD}
  \desc{The {\bf initial temperature} in Kelvin of the {\bf system} is read from the next line. 
        With the additional keyword {\bf RAMP} the temperature can be linearly ramped to a target 
	value and two more numbers are read, the ramping target temperature in Kelvin and the
        ramping speed in Kelvin per atomic time unit (to get the change per timestep you have to multiply it
        with the value of \refkeyword{TIMESTEP}). Note that this ramping affects the target temperatures for
        \refkeyword{TEMPCONTROL}, \refkeyword{BERENDSEN} and the global \refkeyword{NOSE} thermostats.}

\keyword{TESR}{}{}{}{\&SYSTEM}
  \desc{The number of additional supercells included in the real space sum for
        the Ewald term is read from the next line. Default is 0, for small unit
        cells larger values (up to 8) have to be used.}

\keyword{THAUTO}{}{}{}{\&LINRES}
  \desc{The two values read from the next line control the switch to different
        optimizers for an automatic selection of optimizers during a linear response
        calculation. This also applies to the Z-vector optimization for
        TDDFT forces. The first value is the threshold for switching from
        conjugate gradients to DIIS (with compressed storage and averged preconditioner,
        subspace size defined with \refkeyword{ODIIS}).
        The second value is the threshold for switching to DIIS with
        full storage and state dependent preconditioner. See also
        \refkeyword{ZDIIS} for specification of the subspace size.}

\keyword{TIGHTPREC}{}{}{}{\&RESP}
  \desc{
   Uses a harder preconditioner. For experts: The
   Hamiltonian is approximated by the kinetic energy,
   the G-diagonal Coulomb potential and the KS-energies.
   The number obtained this way must not be close to zero.
   This is achieved by smoothing it with
   This is achieved by smoothing it with
     $$x \to f(x) = \sqrt{x^2 + \epsilon^2} \; \; [{\rm default}] $$
   or
     $$x \to f(x) = (x^2 + \epsilon ^2)/x \; \; [{\rm this \; option}] $$
   The HARD option conserves the sign of the approximate
   Hamiltonian whereas the default formula does never
   diverge.
   }

\keyword{TIMESTEP ELECTRONS}{}{}{}{\&CPMD}
  \desc{The time step for electron dynamics in atomic units is read from the next line. This is can
   be used to tweak the convergence behavior of the wavefunction optimization in Born-Oppenheimer
   dynamics, where the default time step may be too large. see, e.g. \refkeyword{PCG}}

\keyword{TIMESTEP IONS}{}{}{}{\&CPMD}
  \desc{The time step in atomic units is read from the next line.}

\keyword{TIMESTEP}{}{}{}{\&CPMD}
  \desc{The time step in atomic units is read from the next line. \\
      {\bf Default} is a time step of {\bf 5 a.u.}
      ($1\, a.u. = 0.0241888428$ fs).}

\spekeyword{TIMESTEP\_FACTOR}{}{}{}{\&MTS}{TIMESTEP FACTOR}
  \desc{
     The time step factor ($n$) used in the MTS scheme is read from the next line. \\

  For instance, if \refkeyword{TIMESTEP IONS} is set to {\bf 15 a.u.}, and ($n=4$) then the high level correction
  to the forces in the MTS scheme will be calculated every 4 steps, i.e. every {\bf 60 a.u.}
  {\bf Default} is a factor of 1, i.e. the MD will be identical to a Velocity-Verlet MD with the high level forces.
  }

\keyword{TRACE}{}{[ALL,MASTER]}{}{\&CPMD}
\desc{ Activate the tracing of the procedures. {\sl ALL} specifies that all the mpi tasks are traced.
  {\sl ALL} specifies that only the master is traced.
}

\spekeyword{TRACE\_PROCEDURE}{}{}{}{\&CPMD}{TRACE PROCEDURE}
\desc{Select a procedure to be traced. The procedure is read from the next line.}

\spekeyword{TRACE\_MAX\_DEPTH}{}{}{}{\&CPMD}{TRACE MAX DEPTH}
\desc{Set the maximal depth for tracing. The depth is read from the next line.}

\spekeyword{TRACE\_MAX\_CALLS}{}{}{}{\&CPMD}{TRACE MAX CALLS}
\desc{Set the maximal number of calls for tracing. The number is read from the next line.}

\keyword{TRAJECTORY}{}{[OFF, XYZ, DCD, SAMPLE, BINARY, RANGE, FORCES]}{}{\&CPMD}
  \desc{Store the atomic positions, velocities and optionally forces
      at every {\em NTRAJ} time step on file {\em TRAJECTORY}.
      This is the {\bf default for MD runs}. With the additional keyword
      XYZ the trajectory is also writthen in xyz-format on the file {\em
      TRAJEC.xyz}, similarly with the additional keyword DCD a trajectory
      in dcd-format (binary and single precision, as used by CHARMM, X-PLOR
      and other programs) is written on the file {\rm TRAJEC.dcd}.
      If the keyword SAMPLE is given
      {\em NTRAJ} is read from the next line, otherwise the default value
      for {\em NTRAJ} is $1$. A negative value of {\em NTRAJ} will
      disable output of the {\em TRAJECTORY} file, but e.g. {TRAJEC.xyz}
      will still be written every {\em -NTRAJ} steps. A value of 0 for
      {\em NTRAJ} will disable writing of the trajectory files altogether.

      The TRAJECTORY file is written in binary format
      if the keyword BINARY is present. If FORCES is specified also the forces
      are written together with the positions and velocities into the file
      FTRAJECTORY. It is possible to store the data of a subset of atoms by
      specifying the suboption RANGE, the smallest and largest index of atoms
      is read from the next line.
      If both, SAMPLE and RANGE are given, the RANGE parameters have to
      come before the SAMPLE parameter.}

\keyword{TRANSITION MOMENT}{}{}{}{\&PROP}
  \desc{Calculate the dipole transition matrix element.\\
%
    On the following lines, the number of transitions and the involved orbitals
    are given.
      Example: \\
         {\tt
         \begin{tabular}{ccc}
         \multicolumn{2}{l}{\bf TRANSITION MOMENT}\\
         2 &  \\
         6 & 7\\
         6 & 8\\
         \end{tabular}
         }
%
    This calculates the dipole transition matrix elements between KS states 6
    and 7, and between 6 and 8.}

\keyword{TROTTER DIMENSION}{}{}{}{\&PIMD}
  \desc{The Trotter number $P$, i.e. the number of ``replicas'',
        ``beads'', or ``imaginary time slices'' which are used in order
        to discretize the Feynman--Kac path integral of the nuclei,
        is read from the next line. If NORMAL MODES or STAGING is
        not activated
        the path integral is discretized in cartesian coordinates
        in real space (so--called ``primitive coordinates'').
        A discussion about controlling discretization errors
        and on estimating $P$ in advance is given in
        Ref.~\cite{knoll-marx-00}.}

\keyword{TROTTER FACTORIZATION OFF}{}{}{}{\&CPMD}
  \desc{Do not use Trotter factorization to calculate free energy
      functional.\\
      Remark: Place this keywords only after FREE ENERGY FUNCTIO\-NAL;
      before it has no effect.
      Note: this keyword has {\em nothing} to do with path integral
      MD as activated by the keyword PATH INTEGRAL and as specified in the
      section \&PIMD ... \&END.}

\keyword{TROTTER FACTOR}{}{}{}{\&CPMD}
  \desc{Solve $e^{-H/k_BT}$ directly using {\bf Trotter approximation}\\
      $\left( e^{-pH} \simeq  e^{-pK/2}e^{-pV}e^{-pK/2}\right)$.\\
      The Trotter approximation is twice as fast.\\
      The Trotter factor is read from the next line
      (typically 0.001 is very accurate).}


\spekeyword{USE\_IN\_STREAM}{}{}{}{\&CPMD}{USE IN STREAM}
 \desc{ Specify that the RESTART file shall be read in the stream mode.
   This allows for parallel restarting (to be used with {\sl USE\_MPI\_IO}).
 }

\spekeyword{USE\_OUT\_STREAM}{}{}{}{\&CPMD}{USE OUT STREAM}
 \desc{ Specify that the RESTART file shall be written in the stream mode.
   This allows for parallel restarting (to be used with {\sl USE\_MPI\_IO}).
 }

\spekeyword{USE\_MPI\_IO}{}{}{}{\&CPMD}{USE MPI IO}
 \desc{ Specify that MPI shall be used for parallel reading/writing of the RESTART file.
   This shall be used with {\sl USE\_IN\_STREAM} and /or {\sl USE\_OUT\_STREAM}.
 }

\spekeyword{USE\_MTS}{}{}{}{\&CPMD}{USE MTS}
 \desc{Switch on the Multiple Time-Step scheme for molecular dynamics (see \&MTS section).\cite{Liberatore2018}}

\keyword{VDW CORRECTION}{[ON,OFF]}{}{}{\&CPMD}

\desc{An empirical van der Waals correction scheme is applied to
  pairs of atom types specified with this keyword. This activates
  reading the corresponding parameters from the \&VDW ... \& END 
  in which you have to specify all the VDW parameters between the opening
  and closing section keywords EMPIRICAL CORRECTION and END EMPIRICAL CORRECTION. 
  Note that the two possible vdW options, EMPIRICAL CORRECTION  and
  WANNIER CORRECTION are mutually exclusive.
  See \refkeyword{VDW PARAMETERS} for more details.}

\keyword{VDW DCACP}{}{}{}{\&VDW}
\desc{Synonymous to \refkeyword{DCACP}.}

\keyword{VDW PARAMETERS}{}{}{}{\&VDW}

\desc{ Parameters for empirical van der Waals correction schemes are set
  with the keyword. This requires the \refkeyword{VDW CORRECTION} keyword
  to be set in the \&CPMD section.  For Grimme's {\bf DFT-D2} and {\bf DFT-D3} type (see below) 
  an automatic assignment of the parameters can be requested by putting 
  {\bf ALL DFT-D2} or {\bf ALL DFT-D3} on the next line. Otherwise the number of pairs
  {\itshape NVDW} is read from the next line and followed by {\itshape NVDW} 
  lines of parameters: {\itshape TYPE}, $\alpha$, $\beta$, $C_6^{\alpha\beta}$,
  $R_0^{\alpha\beta}$, and $d$ for each pair of atom types $\alpha$ and
  $\beta$, where $\alpha$ and $\beta$ are the indexes of
  pseudopotentials (and their associated groups of atoms) in the order
  they are listed in the \&ATOMS section. For type {\bf DFT-D2} and {\bf DFT-D3} only $\alpha$
  and $\beta$ are required. If the other parameters are omitted the internal
  table of parameters is used.
% Note:  References to two papers by R. LeSar have
% been removed from this entry because Elstner's
% damping function is quite different from LeSars,
% Elstner does not reference LeSar's work,
% and LeSar's damping function was adopted from
% earlier work (ie., LeSar was not the first to
% use such corrections.)  It does appear that
% LeSar's function may be in the CPMD source code,
% but it is commented out.

  A presently implemented damped dispersion model, described by M.
  Elstner {\itshape et al.}\cite{Elstner}, having the same form as that
  constructed by Mooij {\itshape et al.}\cite{mooij:99}, is activated by
  specifying {\bf C6} as {\itshape TYPE}.  This model is expressed as
% Elstner's Damping function:
\begin{equation}
\label{elstner-damping-function} %\ref{elstner-damping-function}
E_{vdW} = \sum_{ij}
    \frac{C_6^{\alpha\beta}}{{R^{\alpha\beta}_{ij}}^6}
\left(1 - \exp{
\left[-d
\left(\frac{R^{\alpha\beta}_{ij}}{R^{\alpha\beta}_0}
\right)^7
\right]}
\right)^4.
\end{equation}
A table of parameters appropriate for this particular model, using the
PBE and BLYP functionals, is available \cite{williams-vdw:06}.

Alternatively Van der Waals correction according to Grimme can be used \cite{Grimme06}
by selecting {\itshape TYPE} {\bf DFT-D2} and its D3 version \cite{GrimmeD3} with {\bf DFT-D3}. 
 \begin{equation}
  E_{disp} = - s_6 \sum_{i=1}^{N_{at} -1} \sum_{j=i+1}^{N_{at}} 
               \frac{C_6^{ij}}{R_{ij}^6} f_{dmp} (R_{ij})
 \end{equation}       
The values of $C_6$ and $R_0$ are not specific that are used by this method are taken 
from \cite{Grimme06} 
and stored internally (see above for details). Namely, all elements from H ($Z=1$) to
Rn ($Z=86$) are available, whereas elements beyond Rn give by default a zero contribution.
Note that the parameter $s_6$ depends on the functional used and has to be provided consistently
with the DFT one chosen for the calculation. The following line has to be added {S6GRIMME}
and the type of functional is read from the next line. One of the following labels has to be provided:
{BP86, BLYP, B3LYP, PBE, TPSS, REVPBE, PBE0}. Note that Grimme vdW does not support other functionals.
In the {\bf DFT-D3} version \cite{GrimmeD3} there is no need to specify manually $s_6$
}


\keyword{VDW-CUTOFF}{}{}{}{\&VDW}
   \desc{On the next line the short range cutoff of van der Waals correction
    has to be specified. The default value is $10^{-2}$.}

\keyword{VDW-CELL}{}{}{}{\&VDW}
   \desc{The number of additional supercells to be included in the sum of
    van der Waals correction.}

\keyword{VDW WANNIER}{}{}{}{\&CPMD}
\desc{A first-principle van der Waals correction scheme \cite{psil1,psil2,psil3}
  is applied to selected groups of atoms on which maximally localized Wannier
  functions (WF) and centers (WFC) have been previously computed.
  The file WANNIER-CENTER generated upon WFC calculation must be present.
  This activates the reading procedure of the corresponding parameters from
  the \&VDW ... \&END section.}

\keyword{WANNIER CORRECTION ... END WANNIER CORRECTION}{}{}{}{\&VDW}
\desc{Between these opening and ending keywords, the partitioning of the system
 and the calculation procedure must be selected. Three implementations are available
 for partitioning the system: 
 (1) choosing a {\it zlevel}, namely a z coordinate separating
  the first fragment form the second (this is appropriate for cases where
  there are only two fragments such as, for instance two graphene layers or
  adsorption of molecules on surfaces); in this case the keyword FRAGMENT ZLEVEL
  must be used. 
 (2) give reference ion and a cut-off radius around which WFCs  are supposed to 
  belong to the given atom or fragment; in this case the keyword FRAGMENT RADIUS
  must be used.
 (3) the system is subdivided into fragments automatically detected by using 
  predefined covalent bond radii. in this case the keyword FRAGMENT BOND
  must be used. This is also the default in case no specification is done.
 
  The syntax for the different options is:

    VERSION \\
     iswitchvdw (method 1 \cite{psil1} or 2 \cite{psil2}) \\
    FRAGMENT ZLEVEL \\
     zlevel (in a.u.) \\
   FRAGMENT RADIUS \\
     multifrag \\
     i radius(i) \\
     ... \\
   FRAGMENT BOND \\
     tollength  \\
   DAMPING [DIPOLE] \\
     a6 \\
   RESTART WANNIER \\
   ENERGY MONOMER \\
     enmonomer \\
   TOLERANCE WANNIER \\
     tolwann  \\
   TOLERANCE REFERENCE \\
     tolref \\
   CHANGE BONDS \\
    nboadwf \\
    i  j $\pm$ 1  \\
   CELL \\
    nxvdw nyvdw nzvdw \\
   PRINT $[$INFO,FRAGMENT,C6,FORCES$]$

  Note that the total number of WFCs in your system depends on the
  spin description you use (1 for LSD, 2 for LDA).
  The coefficient a6 is the smoothing parameter and the
  reference total energy intended as a sum of all the total energies
  of your fragments (e.g. the ETOT you get by a standard calculation
  not including vdW corrections). For a6 the suggested parameter is
  20.0 \cite{molphy}.
  An alternative method avoiding the use of the empirical parameter a6
  is the one exploiting the replacement of the short-range damping function
  by an estimation of the Pauli exchange repulsion given in terms of solely
  Wannier function centers and spreads \cite{psil3}.
  To activate this option the keyword DAMPING has to be given along with DIPOLE
  on the same line. 
  The following line (a6) has to be obviously removed, since anyhow is neither
  read nor used in this (DAMPING DIPOLE) case.
  Note that the two possible vdW options, EMPIRICAL CORRECTION  and
  WANNIER CORRECTION are mutually exclusive.}

\keyword{VELOCITIES ... END VELOCITIES}{}{}{}{\&ATOMS}
  \desc{Sets an {\bf initial velocity} for specified atoms.\\
      The first line contains first the total number of specified atomic velocities
      followed \textbf{on the same line} by the list of atomic numbers
      for which the velocities will be read.
      On each of the following lines the x, y and z coordinates of the
      velocities of an atom have to be specified. These values will
      ignored in case of starting with \refkeyword{RESTART} VELOCITIES..

      NOTE: these velocities are rescaled to produce the initial
      temperature as specified by \refkeyword{TEMPERATURE}.
      The default temperature, however, is 0K, so you \textbf{have}
      to set the matching temperature or your initial velocities
      will be useless.
}

\keyword{VGFACTOR}{}{}{}{\&CPMD}
  \desc{For \refkeyword{CDFT} runs read the inverse of the gradient optimiser step size
  ($1/dx$) from the next line. The standard value of \defaultvalue{10.0} should be fine
  in most situations.}

\keyword{VIBRATIONAL ANALYSIS}{}{[FD, LR, IN], [GAUSS, SAMPLE, ACLIMAX]}{}{\&CPMD}
  \desc{Calculate harmonic frequencies by finite differences of first
      derivatives {\bf (FD)} (see also keyword \refkeyword{FINITE DIFFERENCES}),
      by {\bf linear response} to ionic
      displacements {\bf (LR)} or from a {\bf pre-calculated} Hessian
      {\bf (IN)}.
      K-point sampling is currently possible using finite differences.
      If the option GAUSS is specified, additional output is written
      on the file {\em VIB1.log} which contains the modes in a style similar
      to GAUSSIAN 98 output. This file can be read in and visualized
      with programs like MOLDEN or MOLEKEL. The option SAMPLE reads an
      integer from the next line. If this number is 2 an additional file
      {\em VIB2.log} containing the lowest modes is written. The {\bf
      default} value is 1.
      If the option ACLIMAX is specified, additional output is written on
      the file VIB.aclimax which contains the modes in a style readable by
      aClimax (\htref{http://www.isis.rl.ac.uk/molecularspectroscopy/aclimax/}{http://www.isis.rl.ac.uk/molecularspectroscopy/aclimax/}).
      If a section {\bf \&PROP} is present with the keyword
      \refkeyword{DIPOLE MOMENT}[BERRY] or
      \refkeyword{DIPOLE MOMENT}[RS],
      the Born charge tensor is calculated on the fly.
      See also the block \&LINRES ... \&END and the keywords
      \refkeyword{RESTART} PHESS and \refkeyword{HESSIAN} \{DISCO,SCHLEGEL,UNIT\} PARTIAL.
      }

\keyword{VMIRROR}{}{}{}{\&CPMD}
  \desc{For \refkeyword{CDFT} HDA runs initialise $V$ for the second state as the negative
     final $V$ value of the first state. Useful in symmetric systems.}


\keyword{WANNIER DOS}{}{}{}{\&CPMD}
  \desc{Outputs the projected density of states of the Wannier orbitals 
    (file WANNIER\_DOS) and the KS Hamiltonian in the Wannier states 
    representation (file WANNIER\_HAM). 

    When running \refkeyword{MOLECULAR DYNAMICS} CP the files WANNIER\_DOS and 
    WANNIER\_HAM solely written at the last step.}

\keyword{WANNIER MOLECULAR}{}{}{}{\&CPMD}
  \desc{Generates effective molecular orbitals from the Wannier representation. 
    It first attributes Wannier orbitals to molecules and then diagonalizes by 
    molecular blocks the KS Hamiltonian.

    Does not work with \refkeyword{MOLECULAR DYNAMICS} CP.}

\keyword{WANNIER NPROC}{}{}{}{\&CPMD}
\desc{ Set the number of mpi tasks to be used for localization. 
  Default is to use all the tasks available. The number of tasks is
   read from the next line and shall be a
   divisor of the number of tasks in a parallel run.
}

\keyword{WANNIER OPTIMIZATION}{\{SD,JACOBI,SVD\}}{}{}{\&CPMD}
  \desc{Use steepest descent or Jacobi rotation method for the orbital
      localization.\\
      Default are Jacobi rotations.}

\keyword{WANNIER PARAMETER}{}{}{}{\&CPMD}
  \desc{{\sl W\_STEP, W\_EPS, W\_RAN, W\_MAXS} are read from the next line.
      {\sl W\_STEP} is the step size of the steepest descent algorithm
      used in the optimization procedure (default value 0.1).
      {\sl W\_EPS} the convergence criteria for the gradient
      (default value $1.e-7$).
      {\sl W\_RAN} the amplitude for the initial random rotation
      of the states (default value 0.0).
      {\sl W\_MAXS} is the maximum steps allowed in the optimization
      (default value 200).}

\keyword{WANNIER REFERENCE}{}{}{}{\&CPMD}
  \desc{The vector {\sl W\_REF} is read from the next line, which consists
      of 3 coordinates $x, y, z$. These are assumed as the origin for
      the WFCs positions and related ionic coordinates (i.e.
      ${\bf R}_I \to {\bf R}_I-(x, y, z)$).
      The default value is the center of the supercell, if
      \refkeyword{CENTER MOLECULE} keyword is active (Note, that
      this is implicitly turned on, for calculations with
      \refkeyword{SYMMETRY} 0). Otherwise it is set to (0,0,0), which
      is usually not the center of the box.
      In order to get the best results displaying the
      IONS+CENTERS.xyz file this parameter should be set explicitly.}

\spekeyword{WANNIER RELOCALIZE\_IN\_SCF}{}{}{}{\&CPMD}{WANNIER RELOCALIZE IN SCF}
   \desc{ If present, relocalize/project the wavefunction at every SCF step.}

\spekeyword{WANNIER RELOCALIZE\_EVERY}{}{}{}{\&CPMD}{WANNIER RELOCALIZE EVERY}
   \desc{ If present, relocalize the wavefunction at every SCF step with the Jacobi method.
     The stride is read from the next line.}

\keyword{WANNIER SERIAL}{}{}{}{\&CPMD}
  \desc{Requests that the calculation of Wannier functions is performed using
        the serial code, even in parallel runs.}

\keyword{WANNIER TYPE}{\{VANDERBILT,RESTA\}}{}{}{\&CPMD}
  \desc{Indicates the type of Wannier functions.
      Vanderbilt type is the default.}

\keyword{WANNIER WFNOUT}{}{[ALL,PARTIAL,LIST,DENSITY]}{}{\&CPMD}
  \desc{Controls the printing of Wannier functions. Either all
      or only some of the functions can be printed. This will be
      done at the end of each calculation of Wannier functions.
      For {\bf PARTIAL} output you have to give the indices of the 
      first and the last Wannier function to print; the {\em LIST}
      directive follows the syntax of \refkeyword{RHOOUT} {\em BANDS}.
    }
\begin{verbatim}
            WANNIER WFNOUT PARTIAL
               5  8

\end{verbatim}

\keyword{WCUT}{}{}{CUT}{\&SYSTEM}
  \desc{Set the radial \refkeyword{CDFT} weight cutoff for all 
     atom species to CUT, which is specified next to the keyword. 
     Default is a species specific cutoff at the distance where 
     the magnitude of the respective promolecular density is 
     smaller than $10^{-6}$.}

\keyword{WGAUSS}{}{}{NWG}{\&SYSTEM}
  \desc{Use Gaussian weight functions instead of Hirshfeld promolecular orbitals
     in the \refkeyword{CDFT} weight. Parameter NWG is specified next to the keyword
     and has to be equal to the number of different atom species in the calculation. 
     The Gaussian widths $\sigma_i$ of the species $i$ are read from subsequent lines.}

\keyword{WOUT}{}{[FULL]}{}{\&CPMD}
  \desc{Controls the printing of the CDFT weight(s). If the keyword FULL is set
     the full weight is written out in the form of a density to WEIGHT-(suff),
     where (suff) is defined by the kind of the CDFT job. (suff)=WFOPT for single point
     calculations, while for geometry optimisations and MD two weights are written,
     (suff)=INIT at the beginning and (suff)=FINAL for the last step.
     If FULL is not set write out a slice of the weight in gnuplot readable form to
     WEIGHT-(suff).dat. Parameters WSLICE and WSTEP are read from the next line.\\
     WSLICE \defaultvalue{0.5} is if larger than zero the z coordinate of
     the x-y weight plane to write out divided by the total box height. If WSLICE$<0$ the weight at
     the z coordinate of the first acceptor atom will be used.\\
     WSTEP \defaultvalue{1} is the grid point step size for the output.}

\spekeyword{XC\_ANALYTIC}{}{}{}{\&LINRES}{XC-ANALYTIC}
 \desc{
 Use analytic second derivatives of
 the XC functional (only available for some LDA functionals)
 }

\spekeyword{XC\_EPS}{}{}{}{\&LINRES}{XC-EPS}
 \desc{
 Finite difference parameter for XC derivative. Default is $5 \cdot 10^{-4}$.
 }

\spekeyword{XC\_DD\_ANALYTIC}{}{}{}{\&LINRES}{XC-DD-ANALYTIC}
 \desc{
 Use analytic second derivatives of the XC functional, see Ref.~\cite{xcder}
 (only available for some LDA and gradient-corrected functionals).
 For the analytic third derivatives of some LDA XC functionals,
 \refspekeyword{XC\_ANALYTIC}{XC-ANALYTIC} can be combined with this keyword
 }

\spekeyword{XC\_DRIVER}{}{}{}{\&DFT}{XC DRIVER}
 \desc{Use the new CP xc driver. This keyword replaces both \refkeyword{OLDCODE}
       and \refkeyword{NEWCODE} and offers more ample customisation options (\emph{cf.{}}
       \refkeyword{COULOMB ATTENUATION}, \refkeyword{LIBRARY} and \refkeyword{SCALES}).
       All functionals are now consistently available for both spin-restricted and open-shell systems.
       Additionally, functionals from libxc can be used and freely mixed with the internal functionals,
       \emph{cf.{}} \refkeyword{LIBRARY}.
       Please note that the new \refkeyword{FUNCTIONAL} abbreviations differ from \textbf{OLDCODE}
       and \textbf{NEWCODE}, \emph{cf.{}} the corresponding entry for a list of available functionals.
       The new driver uses the standard definitions for all functionals; where those definitions
       differ from older versions of CPMD, the old definitions (where implemented)
       can be accessed by using the keyword \refspekeyword{OLD\_DEFINITIONS}{OLD DEFINITIONS}.}

\keyword{ZDIIS}{}{}{}{\&LINRES}
  \desc{The subspace size for the optimizer is read from the next line.}


\keyword{ZFLEXIBLE CELL}{}{}{}{\&SYSTEM}
  \desc{Specifies a constraint on the super cell in constant
   pressure dynamics or geometry optimizations.
   The supercell may only shrink or grow in z-direction.
   Should be very useful for ``dense slab'' configurations,
   e.g. a water layer between solid slabs.\\
   \textbf{Please note:} this is by no means intended to give a
   statistically meaningful ensemble, but merely to provide a
   tool for efficient equilibration of a specific class of system.
}

\spekeyword{{[TSDE, TSDP, TSDC]}}{}{[NOPRECONDITIONING]}{NOPRECONDITIONING only
  electrons}{\&CPMD}{TSDE, TSDP, TSDC}
  \desc{Short forms for the different \refkeyword{STEEPEST DESCENT} options.}

\keyword{n-CENTER CUTOFF}{}{}{}{\&PROP}
  \desc{The cutoff for printing the n-center shared electron numbers is read
    from the next line. All one and two center terms are printed.}

%
\clearpage
%
%---------------------------------------------------------------------
\subsection{Further details of the input}\label{further_input}
%---------------------------------------------------------------------
%
\subsubsection{Pseudopotentials}
%
\label{S_Pseudopotentials}
%

  The general format for entering the pseudo potentials in the input file is:
\begin{itemize}
\item The input for a {\bf new atom type} is started
      with a ``{\bf *}'' in the first column. This line further contains:
      \begin{itemize}
      \item the {\bf file name} ({\sl ECPNAME}) where to find the
            {\bf pseudopotential}
            information starting in column 2
      \item and several {\bf \sl labels}:
            \begin{itemize} %labels
            \item[.] The {\bf first label} \\
            {\bf [GAUSS-HERMITE, KLEINMAN-BYLANDER]}\\
                  specifies the method to be used for the calculation of
                  the {\bf nonlocal parts} of the {\bf pseudopotential} \cite{KB}.
                  It can be omitted for Vanderbilt pseudopotentials
                  and Stefan Goedecker's pseudopotentials. For semi-local
                  pseudopotentials the default is Gauss-Hermite integration
                  with 20 special points. The number of integration points
                  can be changed using {\bf GAUSS-HERMITE=xx}.
            \item[.] It is further possible to specify {\bf nonlinear core
                  correction \cite{NLCC} [NLCC]} and the width of the 
                  {\bf ionic charge distribution [RAGGIO]}.
                  ({\bf Default} is {\bf no NLCC} and the default value for
                  {\bf RAGGIO} is {\bf 1.2}.)
            \item[.] The  label {\bf UPF} indicates that the pseudopotential
                  was stored using the Universal Pseudopotential Format.
            \item[.] For {\bf Vanderbilt ultrasoft pseudopotentials}
                one of the following options has to be specified:
                {\bf BINARY} indicates the binary version of the output
                file from Vanderbilt's atomic code. \\
                {\bf FORMATTED} indicates the formatted version of the
                Vanderbilt pseudopotential files after a conversion with
                the program `reform.f' from the
                Vanderbilt atomic code package (see section \ref{intro:further}) \\
                For Vanderbilt pseudopotentials the option NLCC is ignored.
                The nonlinear core correction will always be used if the
                pseudopotential was generated with a partial core. \\
                It is strongly recommended to use only Vanderbilt pseudopotentials that
                were generated with a new version of Vanderbilt's atomic code
                (version 7.3.0 or newer).
            \item[.] The  label {\bf CLASSIC} indicates that the following atoms are to be
                treated with classical force fields only. See section \&CLASSIC for more information.
            \item[.] The  label {\bf \_EAM\_} indicates that the following atoms are treated using
                the EAM approach.
            \item[.] The  label {\bf FRAC} indicates that the core charge of
                a pseudopotential should not be rounded for the calculation of the
                number of electrons (for pseudopotentials with fractional core charge).
            \item[.] The  label {\bf ADD\_H} indicates that the
              potential should used to saturate dangling bonds or
              ``hydrogenize'' united atom potentials in a CPMD/Gromos-QM/MM
              calculation (see section \ref{sec:qmmm} for more details).
            \end{itemize} % labels
      \end{itemize}
\item The next line contains information on the {\bf nonlocality} of
      the {\bf pseudopotential} ({\sl LMAX, LOC, SKIP}) \cite{BHS,SGS,TM}.
\item On the following lines the {\bf coordinates} for this
      {\bf atomic species} have to be given.\\
      The first line gives the number of atoms ({\sl NATOMS}) of the
      current type.\\
      Afterwards the coordinates of the atoms are listed
      (in Cartesian coordinates by default). For CPMD/Gromos-QM/MM
      calculation, however, the Gromos atom numbers have to be given
      instead of coordinates (see section \ref{sec:qmmm} for more details).
\end{itemize}


\noindent
The information on the {\bf nonlocal part} of the pseudopotential \cite{BHS,SGS,TM}
can be given in two different styles:
\begin{itemize}
\item[-] You can specify the maximum $l$ - quantum number with
         ``{\bf LMAX}=$l$'' where $l$ is S, P or D. \\
         If this is the only input, the program assumes that LMAX
         is the local potential (LOC). \\
         You can use another local function by specifying ``{\bf LOC}=$l$''.\\
         In addition it is possible to assign the local potential to a
         further potential with ``{\bf SKIP}=$l$''.
\item[-] Alternatively you can specify these three angular quantum
         numbers by their numerical values (S=0, P=1, D=2) in the
         order ``LMAX LOC SKIP''.\\
         If values for LOC and SKIP are provided outside the range
         0 - LMAX the program uses the default.
\end{itemize}
\begin{flushleft}
{\bf Examples:} The following lines are {\bf equivalent} \medskip \\
{\bf
  LMAX=P \smallskip \\
  LMAX=P LOC=P \smallskip\\
    1   1   2 \smallskip\\
    1   2   2 \smallskip\\
}
\end{flushleft}


\noindent
{\bf Note:}

Also for Vanderbilt and Goedecker pseudopotentials \cite{GTH}
this line has to be in a {\bf valid format},
but the actual values are {\bf not} used.

\subsubsection{Constraints and Restraints}
\label{sec:cnstr}

\refkeyword{CONSTRAINTS ... END CONSTRAINTS}\hfill \smallskip \\
     Within this input block you can specify several
     \textbf{constraints} and \textbf{restraints} on the atoms.\\
    Please note, that for calculations using the Gromos
    QM/MM-interface (see section \ref{sec:qmmm}) the atom indices
    refer to the ordering of the atoms
    as it appears in the respective Gromos coordinate file.
    In all cases the indices of dummy atoms start sequentially from total-number-of-atoms plus one.

    The following suboptions are possible:

     \begin{description}
        \item[FIX ALL]\hfill \\
          {\bf All coordinates} of {\bf all atoms} are kept fixed.\\
           For wavefunction optimization via simulated annealing.

        \item[FIX QM]\hfill \\
          {\bf All coordinates} of {\bf all QM atoms} are kept fixed.\\
          This is the same as above unless you are running a QM/MM
          calculation with the Gromos interface code.

        \item[FIX MM]\hfill \\
          {\bf All coordinates} of {\bf all MM atoms} are kept fixed.\\
           This is ignored unless you are running a QM/MM
           calculation with the Gromos interface code.

        \item[FIX SOLUTE]\hfill \\
          {\bf All coordinates} of {\bf all solute atoms} are kept fixed.\\
           This is ignored unless you are running a QM/MM
           calculation with the Gromos interface code.
           The definition of what is a solute is taken from the
           respective GROMOS topology file.

        \item[FIX SEQUENCE]\hfill \\
          {\bf All coordinates} of a series of atoms are kept fixed.\\
          This keyword is followed by the index numbers of the first and
          the last atoms to be fixed in the next line.
          Example: \\
          {\tt
            {\bf FIX SEQUENCE}\\
            5~25~~~~~~{\small \sl all coordinates of atoms no. 5 to 25 are kept fixed}
            }

        \item[FIX ELEMENT {[SEQUENCE]}] \hfill \\
          \textbf{All coordinates} of all atoms belonging to the same
          \textbf{element} are kept fixed. This works across
          pseudopotential types or QM and MM atoms in case of a QM/MM
          calculation. The keyword is followed by the core charge of
          the respective element. With the optional SEQUENCE modifier
          two more numbers are read in, specifying the first and the
          last index of a sequence of atoms to which this keyword will
          be applied.
          Example: \\
          {\tt
            {\bf FIX ELEMENT}\\
            8~~~~~~~~~{\small \sl all coordinates of oxygen atoms are kept fixed}
            }


        \item[FIX PPTYPE {[SEQUENCE]}] \hfill \\
          \textbf{All coordinates} of all atoms belonging to the same
          potential type are kept fixed. The keyword is followed by the
          atom type index number on the next line, corresponds to the
          sequence of how the atom types are specified in the \&ATOMS
          section of the CPMD input. In case of a QM/MM calculation
          this is expanded to respective classical atom types. In this
          case the QM atom types come first followed by the GROMOS atom
          types.
% FIXME: AK: 2005/05/25
          With the optional SEQUENCE modifier
          two more numbers are read in, specifying the first and the
          last index of a sequence of atoms to which this keyword will
          be applied.
          Example: \\
          {\tt
            {\bf FIX PPTYPE SEQUENCE}\\
            ~~2~~~5~25~~~~~~{\small \sl atoms corresponding to the
              second atom type with an index between 5 and 25 are kept fixed}
            }

        \item[FIX ATOMS]\hfill \\
          {\bf All coordinates} of certain atoms can be fixed.\\
          This keyword is followed by the number of atoms to be fixed
          and a list of these atoms specifying them by the number of
          their position in the input file (NOTE: in the file
          GEOMETRY.xyz the atoms have the same ordering).
          Example: \\
          {\tt
            {\bf FIX ATOMS  }\\
            5~~~~2~~5~20~21~23~~~~~{\small \sl all coordinates of atoms
              2, 5, 20, 21, and 23 are kept fixed}
            }

        \item[FIX COORDINATES]\hfill \\
          {\bf Certain coordinates} of atoms are fixed.\\
          This keyword is followed by the number of
          atoms with fixed coordinates and a list of these atoms
          together with flags indicating which coordinates are fixed.
          A zero indicates a fixed coordinate.
          Example: \\
          {\tt
            \begin{tabular}{lllll}
              \multicolumn{5}{l}{\bf FIX COORDINATES }\\
              2 &   &   &        &{\small \sl Two atoms have fixed coordinates}\\
              1 & 1 & 1 & 0      &{\small \sl for atom \#1 $z$ is fixed}\\
              4 & 0 & 1 & 0      &{\small \sl for atom \#4 $x$ and $z$ are fixed}
            \end{tabular}
            }
        \item[FIX COM] \hfill \\
          Fix the center of mass.\\
          \textbf{NOTE:} This currently works only for \refkeyword{OPTIMIZE GEOMETRY}
          and not for the \refkeyword{LBFGS} optimizer.

        \item[FIX STRUCTURE {[SHOVE]} ] \hfill \\
          This keyword starts a group of individual constraints where
          whole {\bf structural units} can be fixed. The keyword is
          followed by the number of individual constraints on the next line.
          \begin{description}
          \item[DIST]  $n1$ $n2$ $R$ \hfill\\
            Fixes the distance $R$ between the atoms $n1$ and $n2$.
          \item[STRETCH]  $n1$ $n2$ $R$ \hfill\\
            Fixes $R^2$ defined by the atoms $n1$ and $n2$.
                          \item[DIFFER]  $n1$ $n2$ $n3$ $R$ \hfill\\
            Fixes $R_{12}-R_{23}$ defined by the atoms $n1$, $n2$, and $n3$,
            where $R_{ab}$ is the distance between atoms a and b.
          \item[DISAXIS] $n1$ $n2$ $k$ $R_0$  [+1,-1,0] GROWTH aa DEST Rd \hfill\\
            Distance between two atoms $n1$ and $n2$ along $x$ or $y$ or $z$ direction. 
            $n1$ $n2$ $k$ and the value $R_0$ are read next on the same line. 
            Here $k=1$ means $x$, $k=2$ means $y$ and $k=3$ means $z$ coordinate.
          \item[BEND]  $n1$ $n2$ $n3$ $\theta$ \hfill\\
            Fixes the bending angle defined by the atoms
            $n1$, $n2$ and $n3$.
          \item[TORSION]  $n1$ $n2$ $n3$ $n4$ $\Theta$ \hfill\\
            Fixes the torsion angle defined by the atoms
            $n1$, $n2$, $n3$ and $n4$.
          \item[OUTP]  $n1$ $n2$ $n3$ $n4$ $\Theta$ \hfill\\
            ``Out of Plane''; Angle between plane
            ($n1$, $n2$, $n3$) and atom $n4$ is fixed.

          \item[RIGID]  $nr$ $n1$ $n2$ ... $nx$  \hfill\\
            Keeps the structure formed by the $nr$ atoms
            $n1$, $n2$, \dots\\
            You can put your atom index in several lines.
            The number of constraints {\bf nfix} is equal to
            $3nr-6$ for $nr>2$ ($nfix=1$ for $nr=2$).

          \item[COORD]  $n1$ $\kappa$ $Rc$ $d^0$ \hfill\\
            Constraint on the coordination (number of atoms around a selected
            one within a specific spherical range of radius $\sim Rc$) for atom $n1$.
            The parameters $\kappa$ and $Rc$ for the Fermi function
            are given in Bohr ($Rc$) and 1/Bohr ($\kappa$), or
            in \AA   ($Rc$) and 1/\AA   ($\kappa$), if the
            keyword \refkeyword{ANGSTROM} is set.
            See Ref.~\cite{Sprik98a}.

          \item[COORSP] $n1$ $jsp$ $\kappa$ $Rc$ $d^0$\hfill\\
            Fixes the coordination number (CN) of one
            selected atom $i$ with respect to only one selected
            species $jsp$. The CN is defined by a Fermi-like function as for $COORD$,
            but in this case $j$ runs only over the atoms belonging to the selected
            species $jsp$.

          \item[COOR\_RF] $n1$ $jsp$ $p$ $q$ $Rc$ $d^0$\hfill\\
            CN of one selected atom $i$ with respect to one selected
            species, $jsp$.
            The CN value is calculated as the sum of rational functions
            \begin{equation}
              CN_i = \sum_{j \neq i}^{n_{list}} \frac{1-\left(\frac{d_{ij}}{d^0}\right)^{p}}
              {1-\left(\frac{d_{ij}}{d^0}\right)^{p+q}},
            \end{equation}
            where j runs over the indexes of the atoms belonging to $jsp$ or over the
            indexes given in the list $j1 \cdots jn_{list}$.

          \item[BNSWT] $n1$ $n2$  $p$ $q$ $Rc$ $d^0$\hfill\\
            Reciprocal CN between 2 selected atoms, defined with the same
            functional form as the one described for $COOR\_RF$.
            This coordinate states the presence of the bond between the
            two atoms $i$ and $j$.

          \item[TOT\_COOR] $isp$ $jsp$ $p$ $q$ $Rc$ $d^0$\hfill\\
            Average CN of the atoms belonging to a selected species $isp$
            with respect to a second selected species, $jsp$, or with respect to a given
            list of atoms, $j1 \cdots jn_{list}$.
            The same functional forms and input options are used, as those
            described for $COOR\_RF$, but the index of one selected species $isp$
            is read in place of the index of one atom.

          \end{description}
          $n1$, ... are the atom numbers, $R$ distances and $\Theta$ angles.
             A function value of -999. for $R$ or $\Theta$ refers to the
             current value to be fixed. The constraint is linearly added
             to the CP Lagrangian according to the {\it Blue Moon} ensemble
             prescription\cite{Sprik98b}. The values of the Lagrange
             multipliers and of the actual constraint are printed in the file
             CONSTRAINT.\\
             The options {\bf DIST, STRETCH, BEND, TORSION, OUTP, DIFFER, COORD,
             COORSP, COOR\_RF, TOT\_COOR} can have
             an optional additional keyword at the end of the line of the form \\
             {\tt DIST  1 2 -999.  GROWTH  0.001 }\\
             The keyword {\bf GROWTH} indicates that the constraint value should be
             changed at each time step. The rate of change is given after the keyword
             in units per atomic time unit, i.e. {\bf independent} from
             the current length of a time step.\\
             {\bf Note:} In MD runs {\bf only} the actual initial value
             (-999.) can be fixed.\\
             The {\bf SHOVE} option requires an additional entry at the end
             of each constraint line. This entry has to be either
             $-1$, $0$, or $1$. The constraint is then either fixed ($0$)
             or allowed to shrink ($-1$) or grow ($1$).

           \item[RESTRAINTS [HYPERPLANE [K=scal]]] \hfill \\
             Defines restraints. 
             \begin{description}
             \item[nres] \hfill\\
               Number of restraints.
             \item[DIST]  $n1$ $n2$ $R$ $kval$ \hfill\\
               Restrains the distance $R$ between the atoms $n1$ and $n2$ by
               a harmonic potential having spring constant $kval$.
             \item[STRETCH]  $n1$ $n2$ $R$ $kval$ \hfill\\
               Restrains $R^2$ defined by the atoms $n1$ and $n2$ by
               a harmonic potential having spring constant $kval$.
             \item[DIFFER]  $n1$ $n2$ $n3$ $R$ $kval$ \hfill\\
               Restrains $R_{12}-R_{23}$ defined by the atoms $n1$, $n2$, and $n3$,
               where $R_{ab}$ is the distance between atoms a and b by
               a harmonic potential having spring constant $kval$.
              \item[DISAXIS] $n1$ $n2$ $k$ $R_0$ $kval$ \hfill\\ 
               Restraints the distance between two atoms $n1$ and $n2$ along $x$ or $y$ or 
               $z$ direction. $n1$ $n2$ $k$ and the value $R_0$ are read next on the same line.
               Here $k=1$ means $x$, $k=2$ means $y$ and $k=3$ means $z$ coordinate. 
             \item[BEND]  $n1$ $n2$ $n3$ $\theta$ $kval$ \hfill\\
               Restrains the bending angle defined by the atoms
               $n1$, $n2$ and $n3$ by a harmonic potential having spring constant $kval$.
             \item[TORSION]  $n1$ $n2$ $n3$ $n4$ $\Theta$ $kval$ \hfill\\
               Restrains the torsion angle defined by the atoms
               $n1$, $n2$, $n3$ and $n4$ by a harmonic potential having spring constant $kval$.
             \item[OUTP]  $n1$ $n2$ $n3$ $n4$ $\Theta$ $kval$ \hfill\\
               ``Out of Plane''; Angle between plane
               ($n1$, $n2$, $n3$) and atom $n4$ is restrained by
               a harmonic potential having spring constant $kval$.
             \item[COORD]  $n1$ $K$ $RC$ $C0$ $kval$ \hfill\\
              Coordination restraint for atom $n1$. The parameters $K$ and $RC$ for the
              Fermi-like function are given in Bohr ($RC$) and 1/Bohr ($K$), or in \AA  ($RC$) 
              and 1/\AA  ($K$), if the keyword \refkeyword{ANGSTROM} is set.
              See Ref.~\cite{Sprik98a}. The harmonic potential spring constant is $kval$.
             \item[COORSP] $n1$ $is$ $K$ $RC$ $C0$ $kval$ \hfill\\
              Retraint on the coordination number (CN) of one selected atom $n1$ with respect
              to a single selected species $is$. The CN is defined by a Fermi-like function 
              as for $COORD$. 
              As in all the above cases, the harmonic potential spring constant is $kval$.
             \item[COOR\_RF] $n1$ $is$ $N$ $M$ $RC$ $C0$ $kval$ \hfill\\
              Restraint on CN of one selected atom $n1$ with respect to one selected
              species, $is$. The CN value is calculated as the sum of rational functions
              \begin{equation}
              CN_i = \sum_{j \neq i}^{n_{list}} \frac{1-\left(\frac{d_{ij}}{d^0}\right)^{p}}
              {1-\left(\frac{d_{ij}}{d^0}\right)^{p+q}},
              \end{equation}
              where j runs over the indexes of the atoms belonging to $is$ or over the
              indexes given in the list $j1 \cdots jn_{list}$.
              As in all the above cases, the harmonic potential spring constant is $kval$.
             \item[BNSWT] $n1$ $n2$ $N$ $M$ $ RC$ $C0$ $kval$ \hfill\\
              Reciprocal CN between 2 selected atoms, defined with the same
              functional form as the one described for $COOR\_RF$.
              This coordinate states the presence of the bond between the
              two atoms $n1$ and $n2$.
              As in all the above cases, the harmonic potential spring constant is $kval$.
             \item[TOT\_COOR] $is1$ $is2$ $N$ $M$ $RC$ $C0$ $kval$ \hfill\\
              Average CN of the atoms belonging to a selected species $is1$
              with respect to a second selected species, $is2$, or with respect to a given
              list of atoms, $j1 \cdots jn_{list}$.
              The same functional forms and input options are used, as those
              described for $COOR\_RF$, but the index of one selected species $isp$
              is read in place of the index of one atom.
              As in all the above cases, the harmonic potential spring constant is $kval$.
             \item[RESPOS] $n1$, $x_0$, $y_0$, $z_0$ $d_0$ $kval$ \hfill\\
               Restrains the position ${\bf R}=(x,y,z)$ of atom $n1$ to oscillate
               around ${\bf R}_0=(x_0,y_0,z_0)$ with a constraint harmonic
               potential $V_c=(kval/2)(|{\bf R}-{\bf R}_0|-d_0)^2$ \cite{cco}.
               The limits $kval=0$ and $kval \to \infty$ correspond
               to free and fixed atomic positions, respectively.
               The keyword GROWTH is not supposed to be used for this restraint.
               For the sake of clarity and consistency with the atomic
               units used through the code, coordinates and distances are 
               expected to be in atomic units (not $\rm \AA$).
             \end{description}

             $n1$, ... are the atom numbers, $R$ distances and $\Theta$ angles. A
             function value of -999. for $R$ or $\Theta$ refers to the current value.
             The restraining potential is harmonic with the force constant $kval$.
             The options can have an optional additional keyword at the
             end of the line of the form \\
             {\tt DIST  1 2 -999. 0.1  GROWTH  0.001 }\\
             The keyword {\bf GROWTH} indicates that the constraint value should be
             changed at each time step. The rate of change is given after the keyword
             in units per atomic time unit.\\
             If the keyword {\bf HYPEPLANE} is set, the system is not restrained around 
             a point in the collective variable space but in an hyperplane. This 
             hyperplane is defined as going through a point in the collective variable 
             space, defined from the $R$ and $\Theta$ above, and by a vector defined 
             from the $kval$ values. {\bf K=scal} applies a scaling to the vector 
             defining the hyperplane so as to modulate the strength of the restraint.\\
             The energy formula for an hyperplane restraint is then:\\
             $E_r=\frac{1}{2}\left((\vec{c}-\vec{c}_0)\cdot \vec{n}\right)^2$,\\
             where the vectors are vectors in the collective variable space.\\
             If a file {\bf RESVAL} is found after parsing the input, the current 
             restraint target values will be replaced by the values found in this file.
        \item[PENALTY]\hfill \\
             The weight factors for the penalty function
             for stretches, bends and torsions are read from the
             next line.
           \end{description}


\subsubsection{Atomic Basis Set}
\label{input:basis}

The \&BASIS section is used in CPMD only to provide an atomic basis set
for generating the initial guess and for analyzing orbitals. If the
input file contains no \&BASIS section, a minimal Slater basis is used.\\

There have to be {\em number of species} different entries in this
section. \\

The order of the basis sets has to correspond with the order
of the atom types in the section {\bf \&ATOMS \ldots \&END}.\\
With the keyword {\bf SKIP} the species is skipped and the
default minimal Slater function basis is used.\\
Basis sets are either specified as Slater functions or given
on an additional input file.

\medskip

\noindent
The respective input formats are given below:

\medskip

\noindent
Slater type basis
\begin{verbatim}
SLATER   nshell  [OCCUPATION]
  n1   l1    exp1
  ..   ..    ....
  nx   lx    expx
  [f1 f2 ... ]
\end{verbatim}
\bigskip

\noindent
Pseudo atomic orbitals
\begin{verbatim}
PSEUDO AO nshell  [OCCUPATION]
  l1   l2   ..   lx     !a function with l=-1 is skipped
  [f1 f2 ... ]
\end{verbatim}
\bigskip

\noindent
Numerical functions
\begin{verbatim}
*filename  nshell  FORMAT=n  [OCCUPATION]
  l1   l2   ..   lx
  [f1 f2 ... ]
\end{verbatim}
\bigskip

\noindent
Gaussian basis functions
\begin{verbatim}
*filename  nshell  GAUSSIAN  [OCCUPATION]
  l1   l2   ..   lx
  [f1 f2 ... ]
\end{verbatim}
\bigskip

\noindent
Skip atom type and use default minimal slater function
\begin{verbatim}
SKIP
\end{verbatim}

\medskip

\noindent
{\tt nshell} is the number L-values\ \ {\tt l1 l2  ..  lx}\ \
to be used.\\
{\tt [f1 f2 ... ]} is their occupation.

\medskip

\noindent
The format {\bf PSEUDO AO} refers to the \&WAVEFUNCTION section
on the corresponding pseudopotential file. \\
With a L-value of -1 a specific function can be skipped.

\medskip

\noindent
The * for the numerical basis has to be in the first column. The
default format is 1, other possible formats are 2 and 3. The numbers
correspond to the format numbers in the old pseudopotential definitions
for the atomic wavefunctions.

\medskip

\noindent
The format {\bf GAUSSIAN} allows to use any linear combination of
Gaussian functions. The format of the file is as follows:
\begin{verbatim}
Comment line
Lmax
(for each l value)
  Comment line
  # of functions; # of exponents
  exp1 exp2 ... expn
  c11  c21      cn1
  c12  c22      cn2
  ...  ...      ...
  c1m  c2m      cnm
\end{verbatim}

\subsubsection{Van der Waals potential}
This section ( \&VDW \ldots \&END) contains information about the
van der Waals correction scheme. Currently, three schemes are
available:
\begin{itemize}
\item A Wannier-center based approach (\refkeyword{VDW WANNIER})
\item A non-local potential-based approach (\refkeyword{DCACP}~\cite{opt-ecp04})
\item A parametrized force-field-like potential (\refkeyword{VDW CORRECTION})
\end{itemize}
The first two approaches both depend on the actual electron density of the system, the third is
a post-SCF correction that is solely dependent on nuclear coordinates.\\~\\
\textbf{DCACP:} When using DCACP, the electron density is updated self-consistently according to the vdW potential
that the system experiences. See the description of the keywords \refkeyword{DCACP Z=z},
\refspekeyword{NO\_CONTRIBUTION}{NO CONTRIBUTION} and \refspekeyword{INCLUDE\_METALS}{INCLUDE METALS}
for customisation options. \\~\\
\textbf{VDW CORRECTION:} Two major types of correction types are implemented in
connection with the \refkeyword{VDW CORRECTION} keyword:
The one described by M.~Elstner~{\itshape et al.}\cite{Elstner} requires parameter sets 
designed to use only in conjunction with a specific corresponding 
density functional, the alternate parametrization by S.~Grimme\cite{Grimme06}
is less specific and therefore easier to use and independent
of the chose functional. Parameters for elements up to Xe have
been directly coded into CPMD. See the description of the keyword
\refkeyword{VDW PARAMETERS} for more details regarding custom input.
\vfill
\clearpage

%---------------------------------------------------------------------
\part{Miscellaneous}
\section{Postprocessing}\label{post}

  The given output from a calculation with the CPMD code can be used for
postprocessing. There are several types of output. The most typical types of
output are density-like files, trajectories and/or xyz-files. These can
be visualized or analyzed with a number of different programs. Some of them
are (in no specific order):\\
Molden: (homepage: 
\htref{http://www.cmbi.kun.nl/~schaft/molden/molden.html}{http://www.cmbi.kun.nl/~schaft/molden/molden.html}\\
gOpenMol: (homepage: \htref{http://www.csc.fi/gopenmol/}{http://www.csc.fi/gopenmol/})\\
Molekel: (homepage: \htref{http://www.cscs.ch/molekel/}{http://www.cscs.ch/molekel/})\\
VMD: (homepage:
\htref{http://www.ks.uiuc.edu/Research/vmd/}{http://www.ks.uiuc.edu/Research/vmd/})\\
Starting with version 1.8.2 VMD does fully support the CPMD trajectory format,
xyz-movie format and Gaussian Cube files. Since version 1.8.3 VMD
supports periodic display of non-orthogonal supercells.
A tutorial on Visualization and Analysis of CPMD data with VMD can be
found at
\htref{http://www.theochem.ruhr-uni-bochum.de/go/cpmd-vmd.html}{http://www.theochem.ruhr-uni-bochum.de/go/cpmd-vmd.html}.



\subsection{Density files}

\subsubsection{List}

DENSITY.x, ELF, LSD\_ELF, SPINDEN.x, WANNIER\_1.x ...

\subsubsection{Postprocessing}

These files are created in a binary format, they have to be transformed
to a Gaussian cube-File format to be readable by visualization programs.
The {\tt cpmd2cube.x} to convert the output
can be download at www.cpmd.org and is used in the following way:
\begin{verbatim}
cpmd2cube: Convert CPMD's Wannier-function files to cube
usage is: cpmd2cube [options] Wannier_file [Wannier_file...]
   If you specify more than one Wannier file, they MUST have the
   same g-vectors and (for the moment) atom positions
   The program will create one cube file for each Wannier file
   and one pdb file with the atom positions
Example:
   cpmd2cube.x WANNIER_1.*
possible options are
   -v <verbosity>:
      <verbosity> is 0-2 (default is 1)
   -double:
      Read the density in double precision (default is single)
   -halfmesh:
      leave out half the grid points in each direction.
      Reduces the file size by 1/8th (on by default).
   -fullmesh:
      use the full real space grid.
   -n <n1> <n2> <n3>:
      change the REAL-space mesh. Default is to take the same mesh as CPMD
   -o <prefix>:
      specify the prefix of the name used for the cube and pdb-files
   -rep <n1> <n2> <n3>:
      replicate the cell n<j> times along the <j>-th direction by periodicity
   -shift <r1> <r2> <r3>:
      shift cube density by r1*a1+r2*a2+r3*a3
   -centre:
   -center:
      centre density around centre of mass of system.
   -inbox:
      put atoms inside unit cell centred around origin
   -rho:
   -dens:
      store the density instead of the wavefunction into the cube file.
   -psi:
   -wave:
      store the wavefunction instead of the density into the cube file.
   --:
      last option. Useful if you have a file with the same name as an option
   -h  or  -?  or  -help  or  --help  or no files:
      write this help
\end{verbatim}

\subsection{xyz-files}

\subsubsection{List}

GEOMETRY.xyz, ION+CENTERS.xyz

\subsubsection{Postprocessing}

  These files can be directly read by a visualisation program. Note, that at
the current status it is useful to include a reference point (WANNIER
REFERENCE) for the ION+CENTERS.xyz which has to be put at the middle of the
box.

\subsection{TRAJECTORY-File}

\subsubsection{List}

TRAJECTORY

\subsubsection{Postprocessing}

1. Looking at a movie\hfill\\
The TRAJECTORY files contains the coordinates and the
velocities. To create a movie file in the xyz-format, you have to transfer the
coordinates from Bohr to {\AA} and you have to add the symbols of the atoms in
the first position. Two lines have to be at the beginning of each time step,
from which the first line gives the number of the total atoms. An .xyz
file can also be recorded directly during the simulation Using the
\refkeyword{TRAJECTORY} keyword with the option {\bf XYZ}.\\
Please note, that CPMD does not apply the minimum image convention
to these trajectory files, i.e. atoms are \textbf{not} replaced by their
images if they leave the supercell.

\medskip

 2. Calculating radial pair distribution functions\hfill\\
The simplest analysis of the structure is
given by the radial pair distribution function g(r). This quantity is a to
unity normalized function and describes the probability of finding two atoms
separated by a distance r relative to the probability expected for a completely
random distribution at the same density. It is formally defined as:
\begin{eqnarray*}
g(r)=\rho^{-2}\left < \sum_i\sum_{i\neq j} \delta({\bf r}_i)
\delta({\bf r}_j-{\bf r}) \right > \nonumber\\
= \frac{V}{N^2}\left < \sum_i\sum_{i\neq j} \delta({\bf r}-{\bf r}_{ij})
 \right >
\end{eqnarray*}
with r being the atomic separation, $\rho$ the number density, N the number of
atoms, V the volume, {\textbf r} the atomic position, and {\textbf r$_{ij}$}
the position of the atom i relative to the atom j. The average $<>$ is taken
over particles and time. Examples of code can be found in the following
references \cite{Allen87,Frenkel02}.

\subsection{The MOVIE format}

Besides the TRAJECTORY file CPMD also produces a specially
formatted trajectory in the MOVIE file. This file contains the position
of all atoms of the system in Angstrom units at a rather low
precision ($10^{-4}$). The sampling of the positions can be done
independently from the trajectory file. The sets of coordinates are
following each other contiguously. The format of each line is \\[10pt]
x-coordinate, y-coordinate, z-coordinate, atomic number, type .


\clearpage
%---------------------------------------------------------------------
\section{Hints and Tricks}\label{hints}

\subsection{Pseudopotentials and Plane Wave Cutoff}\label{hints:cutoff}
The selection of proper pseudopotentials and the corresponding
plane wave cutoff are crucial for obtaining good results with
a CPMD calculation. The cutoff required is mainly determined by
the type and the ``softness'' of a pseudopotential.\\
Ideally a pseudopotential for a specific atom type should be
usable for all kinds of calculations (Transferability), but in
practice one frequently has to make compromises between accuracy
and impact on the computational effort when creating a pseudopotential.
Therefore one always has to test pseudopotentials before using them
on new systems. There are quite a large number of CPMD calculations
published (see
\htref{http://www.cpmd.org/cpmd_publications.html}{http://www.cpmd.org/cpmd\_publications.html})
which can serve as a guideline.\\

Since CPMD uses a plane wave basis, a concept of several different,
formalized basis sets of different quality, like with gaussian basis set
based quantum chemical software, does not apply here.
Since plane waves are 'delocalized' in space, they provide the same
'quality' everywhere in space and one can increase the basis set almost
arbitrarily by increasing the number of plane waves via the
\refkeyword{CUTOFF} keyword.  The cutoff has to be chosen
in such a way, that all required quantities are reasonably converged
with respect to the plane wave cutoff. For a molecular dynamics
run this refers primarily to the atomic forces. For the calculation
of other properties like the stress tensor a different, usually a much
higher cutoff is required. It's always a good idea to make checks at
some critical points of the calculations by increasing the cutoff.\\

Typical cutoff values range from 20--40~ry for Vanderbilt ultra-soft
pseudopotentials, 60--100~ry for Troullier-Martins norm-conserving
pseudopotentials to 80--200~ry for Goedecker
pseudopotentials. Pseudopotentials of different types can be freely
mixed, but the required plane wave cutoff is determined by the
``hardest'' pseudopotential. Support for Vanderbilt ultra-soft
pseudopotentials is (mostly) limited to the basic functionality
like molecular dynamics and geometry optimization.\\

\subsection{Wavefunction Initialization}

  The default initial guess for the wavefunctions is calculated from the atomic
pseudo-wavefunctions and gives usually excellent results. Good results can also
be obtained by using wavefunctions from other calculations with a different
cutoff or slightly different geometry. The other initialization available,
starts from random coefficients and should only be used as a last resort. Cases
where the default method does not work are when the molecule has less occupied
states than one of the atoms (in this case add some empty states for the
molecule) or when the additional memory required for the atomic calculation is
not available.

\subsubsection{Using Vanderbilt Ultrasoft Pseudopotentials}
When using Vanderbilt ultrasoft pseudopotentials (USPPs) \cite{Vanderbilt} and 
starting from atomic pseudo-wave\-func\-tions, the calculations often do not
converge or converge to a wrong state, especially if 3d-elements are
involved. Convergence is generally much better when assigning (partial)
charges via the \refkeyword{ATOMIC CHARGES} keyword in the \&SYSTEM
\ldots \&END section. Values from a classical MD forcefield or an NBO
calculation are usually good values. Alternatively a random initialization
of the wave functions (via \refkeyword{INITIALIZE WAVEFUNCTION} RANDOM)
can be used.

Also, due to the comparatively small plane wave cutoffs, you will have small
but significant modulations of the density in especially in regions with
little electron density. These lead to "strange" effects with
gradient corrected functionals, causing the optimization to fail.
To avoid this, you can skip the calculation of the gradient correction
for low electron density areas using \refkeyword{GC-CUTOFF} with a value
between 1.D-6 and 1.D-5 in the \&DFT section.

In case of geometry optimizations, also the accurate calculation of the
forces due to the augmentation charges may need a higher density cutoff
and/or a tighter real space grid.
This can be achieved by either using a higher plane wave cutoff or via
increasing \refkeyword{DUAL} to 5.0 or even 6.0 and/or setting
the real space grid explicitly via the \refkeyword{MESH}
keyword in the \&SYSTEM section. For the same reason, these options
may be needed to increase energy conservation during molecular dynamics runs.
Use these options with care, as they will increase the cpu time
and memory requirements significantly and thus can easily take away
one of the major advantages of ultra-soft pseudopotentials.

\subsection{Wavefunction Convergence}\label{hints:wfconv}
Some general comments on wavefunction optimizations:\\[2ex]
%
Any optimization that takes more than 100 steps
should be considered slow.\\[1ex]
%
Optimizations using ODIIS that have repeated resets
(more than a few) will probably never converge.
%
Convergence for LSD is normally slower and more
difficult than for unpolarized cases.\\[1ex]
%
If the ODIIS convergence gets stuck (more than one reset)
stop the run and restart using
\begin{verbatim}
  PCG MINIMIZE
  TIMESTEP
   20
\end{verbatim}
The conjugate gradient minimizer with line search is
much more robust. For LSD and larger systems it should be used from
the start.\\[1ex]
%
A typical behavior will be that after the restart the energy goes down
and the gradient increases. This means that we are in a region where
there are negative curvatures. In such regions the DIIS minimizer moves
in the wrong direction.  After some iterations we will be back to normal
behavior, energy and gradient get smaller. At this point it may be save
to switch back to ODIIS.\\[1ex]
%
Sometimes, it can also be helpful to wait longer for a DIIS reset and to
diagonalize after repeated resets to get out of this region. This can be
accomplished using
\begin{verbatim}
  ODIIS NO_RESET=20
    5
  LANCZOS DIAGONALIZATION RESET=2
\end{verbatim}
%
Starting a Car-Parrinello MD from a random wavefunction with all atom
positions fixed, a comparatively high electron mass and using
\refkeyword{ANNEALING} ELECTRONS is another alternative to get
to a reasonably converged wavefunction. Due to the exponential
convergence of the annealing procedure, one should switch to
a different optimizer as soon as the fictitious kinetic energy of
the electrons drops below the typical range for a (normal) MD run.\\[1ex]

Wavefunction optimizations for geometries that are far from equilibrium
are often difficult. If you are not really interested in this geometry
(e.g. at the beginning of a geometry optimization or this is just the
start of a MD) you can relax the convergence criteria to $10^{-3}$ or
$10^{-4}$ and do some geometry steps.  After that optimization will be
easier.\\[1ex]
%
Some general remarks on comparing the final energies:\\[2ex]
Converge the wavefunction very well, i.e. set \refkeyword{CONVERGENCE} ORBITALS to
$10^{-6}$ or better.\\[2ex]
Make sure that all parameters are the same:\\
- same geometry,\\
- same functional,\\
- same number of grid points (this may differ
     if you use different FFT libraries)\\
- same number of spline points for the PP\\
(IMPORTANT: the default for \refkeyword{SPLINE} \textbf{POINTS} has
changed between different CPMD versions, $500 \to 3000 \to 5000$).
A very good test is to start always from the same
RESTART file and only do one single step. This
way ALL energies have to be exactly the same and
no problems with different convergence rates occur.



\subsection{Cell Size for Calculations with \textbf{SYMMETRY~0}}\label{hints:symm0}
Calculations of isolated systems (i.e. decoupling of the electrostatic
images in the Poisson solver) are initialized with:\\
\texttt{
\refkeyword{SYMMETRY}\\
\hspace*{4ex}0
}

The box is assumed to be orthorhombic. With the additional options
\refkeyword{SURFACE} or \refkeyword{POLYMER} periodic boundary
conditions in two or one dimensions, respectively, are assumed. 
\\
Three different kinds of \refkeyword{POISSON SOLVER} \{HOCKNEY,TUCKERMAN,MORTENSEN\}
are available.
All methods require that the charge density is zero at
the border of the box. For normal systems and the Hockney solver,
this means that about 3 Angstrom space between the outermost atoms and
the box should be enough. For large molecules, Tuckerman will require
a considerably higher margin, see below.
However, for some systems and for high accuracy these rules of thumb may not be enough.
Some methods have additional requirements (see below).
\\
Note that the \refkeyword{ISOLATED MOLECULE} keyword has only an effect on the
calculation of the degrees of freedom (3N-6 vs. 3N-3 for periodic
systems). The main purpose of \refkeyword{CENTER MOLECULE} ON/OFF is to
center the molecule (center of mass) in the box. This is needed for the
HOCKNEY Poisson solver. This solver gives wrong results if the charge
density is not centered in the computational box. All other solvers
behave like the periodic counterpart, i.e. the relative position of the
charge density and the box are not important.
\\
Further requirements on the Poisson solvers:\\

HOCKNEY Method:
\begin{itemize}
 \item  molecule has to be in the center of the box
 \item  box size molecule + 3 {\AA} border
 \item  expensive for very small systems
 \item  not available for some response calculations
 \item  \refkeyword{POLYMER} is available but
    gives (currently, Version 3.9) wrong results.
 \item  \refkeyword{SURFACE} is available and works.
\end{itemize}

TUCKERMAN-MARTYNA Method:
\begin{itemize}
 \item box size : molecule + 3 {\AA} border \textbf{and} - imperatively! - \emph{twice} the size of the electron charge distribution
       (check the density after a single-point wavefunction optimization).
 \item expensive for large systems, smaller boxes might be used without
 loosing too much accuracy
 \item \refkeyword{SURFACE} or \refkeyword{POLYMER} are not available
\end{itemize}

MORTENSEN Method:
\begin{itemize}
 \item same as TUCKERMAN, but using analytic formula
    made possible by using special boundary conditions
    (sphere, rod)
 \item \refkeyword{SURFACE} and \refkeyword{POLYMER} are available and
   should be safe to use (MORTENSEN is default for SURFACE and POLYMER)
 \item If you do an isolated system calculation, your cell has
   to be cubic, if you use \refkeyword{POLYMER} cell dimensions b and c
   have to be equal.
\end{itemize}


Finally, for many systems using a large enough cell and
periodic boundary conditions is also an option.
In general, the computed properties of molecules
should be independent of the scheme used (either pbc or isolated
box) except in difficult cases such as charged molecules,
where the calculation in an isolated box is recommended.
The PBC calculation is always cheaper for a box of the same size,
so for a neutral molecule such as water molecule you would
save time and memory by not using \refkeyword{SYMMETRY} 0.


\subsection{Geometry Optimization}

  Any combination of methods for geometry optimization and wavefunction
optimization is allowed. Possible options for geometry optimization are GDIIS,
LBFGS, PRFO, RFO, BFGS and steepest descent. If you choose steepest descent for
both, geometry variables and the wavefunction, a combined method is used. For
all other combinations a full wavefunction optimization is performed between
changes of the ionic coordinates. The convergence criteria for the wavefunction
optimization can be adapted to the requirements of the geometry optimization
(CONVERGENCE ADAPT and CONVERGENCE ENERGY). The default options are GDIIS and
ODIIS. Some quasi-Newton methods (GDIIS, RFO and BFGS) are using the BFGS
method to update an approximate Hessian. At the beginning of a run the Hessian
can either be initialized as a unit matrix {\bf HESSIAN UNIT} or with an
empirical force field. Two force fields are implemented: The {\bf DISCO} and
the {\bf SCHLEGEL} force field. The algorithm for the empirical force fields
has to identify bonds in the system. For unusual geometries this may fail and
the Hessian becomes singular. To prevent this you can add or delete bonds with
the keyword {\bf CHANGE BONDS}.

The linear-scaling geometry optimizers (options LBFGS and PRFO) do not require
an approximate Hessian. To achieve linear scaling with the system size, the
L-BFGS optimizer starts from a unit Hessian and applies the BFGS update on the
fly using the history of the optimization. The P-RFO method can find transition
states by following eigenmodes of the Hessian. The mode to be followed does not
necessarily have to be the lowest eigenvalue initially ({\bf PRFO MODE}). For
larger systems, only the reaction core should be handled by the P-RFO optimizer
({\bf PRFO NVAR} and {\bf PRFO CORE}), and the environment is variationally
decoupled using the L-BFGS optimizer. The normal LBFGS options can be used for
the environment. The Hessian used for transition-state search therefore spans
only a subset of all degrees of freedom, is separate from the Hessian for the
other optimizers and the vibrational analysis, but it can be transferred into
the appropriate degrees of freedom of the regular Hessian
({\bf HESSIAN PARTIAL}). In order to allow negative eigenvalues, the Powell
update instead of BFGS is used for transition-state search.

Although tailored for large systems, the linear-scaling geometry optimizers
are suitable for smaller systems as well.

\subsection{Molecular Dynamics}

\subsubsection{Choosing the Nos\'{e}-Hoover chain thermostat parameters}
\label{hints:nose}

  The Nos\'{e}-Hoover chain thermostat is defined by specifying three
parameters: A target kinetic energy, a frequency and a chain length. For the
ions, given the target temperature $T_W$, the target kinetic energy is just
$gkT_W$, where $g$ is the number of degrees of freedom involved in a common
thermostat. For example, if there is one thermostat on the entire ionic system,
then $g=3N_{AT}-N_{const}$, where $N_{const}$ is the number of constraints to
which the atoms are subject. The frequency for the ionic thermostat should be
chosen to be some characteristic frequency of the ionic system for which one
wishes to insure equilibration. In water, for example, one could choose the O-H
bond vibrational frequency. (Having a precise value for this frequency is not
important, as one only wishes to insure that the thermostat will couple to the
mode of interest.) The choice of chain length is not terribly important as it
only determines how many extra thermostats there will be to absorb energy from
the system. Usually a chain length of 4 is sufficient to insure effective
equilibration. Longer chains may be used in situations where heating or cooling
effects are more dramatic.

  For the electrons, the target kinetic energy is not usually known {\it a
priori} as it is for the ions. However, by performing a short run without
thermostats, one can determine a value about which the electron kinetic energy
`naturally' fluctuates and take this as the target value. While the precise
value is not important, a little experience goes a long way, as a choice that
is either too small or too large can cause spurious damping of the ions or
departures from the Born-Oppenheimer surface, respectively. A good choice for
the frequency of the electron thermostat can be made based on $\Omega_I^{\rm
max}$, the maximum frequency in the phonon spectrum. The frequency of the
electron thermostat should be at least 2-3 times this value to avoid coupling
between the ions and the electron thermostats. As an example, for silicon, the
highest frequency in the phonon spectrum is 0.003 a.u., so a good choice for
the electron thermostat frequency is 0.01 a.u. The chain length of the electron
thermostat can be chosen in the same way as for the ions. 4 is usually
sufficient, however longer chains may be used if serious heating is expected.
In addition, the electron thermostats have an extra parameter that scales the
number of dynamical degrees of freedom for the electrons. ($1/\beta_e =
2E_e/N_e$, where $E_e$ is the desired electron kinetic energy and $N_e$ is the
number of dynamical degrees of freedom for the electrons -- see Eq.~(3.4) in
Ref.\cite{Tuckerman94}). The default value is the true number of dynamical
degrees of freedom $N_e = (2*N_{GW}-1)*N_{ST} - N_{ST}^p$, where $p=2$ for
orthonormality constraints and $p=1$ for norm constraints. When this number is
very large, it may not be possible to integrate the electron chain thermostats
stably using a frequency above that top of the phonon spectrum. Should this be
the case in your problem, then the number of dynamical degrees of freedom
should be scaled to some smaller number such that the system can once again be
integrated stably. This parameter has no other effect that to change the
relative time scales between the first element of the electron thermostat chain
and the other elements of the chain.

  In addition to the basic parameters defining the chains themselves, one needs
to specify two more parameters related to the integration of the thermostated
equations of motion. The first is the order $M_{SUZ}$ of the Suzuki integrator.
Experience shows that the choice $M_{SUZ}=3$ is sufficient for most
applications. Finally, one must specify the number of times the Suzuki
integrator will be applied in a given update. This is the parameter $N_{SUZ}$
which determines the basic Suzuki time step $\delta t$ = $\Delta t/N_{SUZ}$,
where $\Delta t$ is the time step being used in the MD run. $N_{SUZ} =2$ or 3
is usually large enough to give stable integration. If more stable integration
is required, try $M_{SUZ}=4$ or make $N_{SUZ}$ larger.


\subsection{Restarts}

\subsubsection{General information}

  All restart information for CPMD simulations are stored within one binary
file. There are very few exceptions we will discuss later. The name of the
restart files is \texttt{RESTART} or \texttt{RESTART.n}, where $n$ stands for an
integer number. If the keyword \refkeyword{RESTART} is found the program
processes the file with the name \texttt{RESTART}. Using suboptions to the
\refkeyword{RESTART} option, the information retained from the file can be
specified. For example the suboptions \texttt{COORDINATES WAVEFUNCTION} will
force the program to use the geometry and orbitals from the \texttt{RESTART}
file.

  At the end of a simulation or at regular intervals (using the keyword
\refkeyword{STORE}) a restart file with the default name \texttt{RESTART.1} is
written. If this happens more than once (e.g. during a molecular dynamics run)
the restart file is being overwritten. Using the keyword \refkeyword{RESTFILE}
it can be specified that more than one restart file should be used for writing.
If the \refkeyword{RESTFILE} parameter was set to 4, then 4 restart files with
the names \texttt{RESTART.1}, $\ldots$, \texttt{RESTART.4} will be written. If
more than 4 restart files are needed the first one will be overwritten. This
option is useful if there is the possibility that restart files get corrupted
(e.g. on instable systems), or if simulations are performed that might lead to
unphysical results. In this case it might be possible to go back to a restart
file which contains still intact information.

  The name of the last restart file written is stored in the file {\tt LATEST}.
Using the suboption LATEST to the keyword RESTART changes the default name of
the file to be read from RESTART to the name found in the file {\tt LATEST}.
The danger of using this option is that the file from which the simulation is
started gets overwritten during the simulation. Using the default (starting
from RESTART) ensures that the original file stays intact. However, it requires
the renaming of the final file of a simulation from \texttt{RESTART.n} to {\tt
RESTART}.

\subsubsection{Typical restart scenarios}

\paragraph{Wavefunction optimizations}

  The restart options used in wavefunction optimizations are \\ {\bf RESTART
WAVEFUNCTION COORDINATES}. The suboption COORDINATES is not really necessary
but it is advised to use it anyway, as in this way the correspondence of
wavefunction and ionic geometry is assured.

\paragraph{Geometry optimizations}

  Typical suboptions used in a geometry optimizations are \\ {\bf RESTART
WAVEFUNCTION COORDINATES HESSIAN}. With the suboption HESSIAN the information
from previous runs stored in the updated approximate HESSIAN can be reused.

\paragraph{Molecular dynamics}

  Molecular dynamics simulations use restart options of the kind \\ {\bf
RESTART WAVEFUNCTION COORDINATES VELOCITIES}. These are the minimal options
needed for a smooth continuation of a Car--Parrinello molecular dynamics
simulation. Use of the suboption ACCUMULATORS ensures that the calculated means
(e.g. temperature) are correct for the whole simulation, not just the current
run. If Nos\'e thermostats are used it is important also the restart the
thermostat variables. This is achieved by adding the corresponding keywords to
the RESTART (NOSEE, NOSEP, NOSEC).

\paragraph{Kohn--Sham energies}
  The calculation of canonical Kohn--Sham orbitals requires a restart. In
general, this will be a restart from converged orbitals from a wavefunction
optimization. There is no way that the program can check this. However, if the
same convergence criteria are used, the number of occupied states orbitals
should converge in the first iteration of the diagonalization.

\subsubsection{Some special cases}

  The suboption VELOCITIES will result in a restart from both, ionic and
wavefunction velocities. In special cases, this is not the desired behavior.
Using the additional keyword \refkeyword{QUENCH} the read velocities can be set
back to zero. This will be most likely used for wavefunctions with QUENCH
ELECTRONS. Another possibility is to reoptimize the wavefunction at the start
of a molecular dynamics simulation. This is achieved with the keywords QUENCH
BO.
% This also sets the velocities of the wavefunctions to zero.
% AK: FIXME: 02/2004
% QUENCH BO does not do QUENCH ELECTRONS automatically anymore, is this intentional?

  For performance reasons the writing of the restart file should be done only
occasionally. This might cause problems if the simulation was terminated
incorrectly. Several hundreds or thousands of simulation steps might be lost.
For this reason CPMD writes a special output file {\tt GEOMETRY} after each
molecular dynamics step. Together with a normal restart file this allows to
start the simulation form the last ionic configuration and velocities. To
achieve this another suboption GEOFILE has to be added to the RESTART keyword.
After reading the positions and velocities of the ions from the restart file,
they are also read from the GEOMETRY file and overwritten.

  Special restarts to be used with the keywords \refkeyword{TDDFT} and
\refkeyword{VIBRATIONAL ANALYSIS} are discussed in the sections covering that
type of simulations.

\subsection{TDDFT}

  The TDDFT part of CPMD is rather new. Therefore it hasn't yet reached the
stability of other parts of the code. It has to be used with special care.

  There are four different type of calculations that can be performed using
the TDDFT module; calculation of the electronic spectra, geometry optimization and
vibrational analysis, and molecular dynamics in excited states.

  All options (spectra and forces, spin polarized and unpolarized) are
implemented for the Tamm--Dancoff approximation to TDDFT. Only part of these
options are available for the full TDDFT response calculation.

\subsubsection{Electronic spectra}
\reflabel{sec:ElectronicSpectra}{}

  Electronic excitation energies can be calculated using the keyword
\refkeyword{ELECTRONIC SPECTRA} in the \&CPMD section. This calculation is
performed in three parts. First, the ground state wavefunctions are optimized,
then a limited set of unoccupied orbitals is determined and finally the TDDFT
response equations are solved. A typical input for such a calculation would
look like

\begin{verbatim}
&CPMD
  ELECTRONIC SPECTRA
  DIAGONALIZATION LANCZOS
  COMPRESS WRITE32
&END

&TDDFT
  STATES SINGLET
    5
  TAMM-DANCOFF
  DAVIDSON PARAMETER
  150 1.D-7 50
&END
\end{verbatim}

  For this calculation of the electronic spectra defaults are used for the
ground state optimization (ODIIS and $10^{-5}$ convergence). The calculation of
the empty states is performed using the Lanczos diagonalizer with default
settings. The final wavefunction will be stored in the restart file using 32
bit precision.

  Five single states with the Tamm--Dancoff approximation have to be
calculated. The parameters for the Davidson diagonalization have been changed
to 50 for the Davidson subspace and a convergence criteria of $10^{-7}$ is
used.

  Restarting this type of calculation has to be done with care.
At the end of each phase of the calculation a new restart file
is written. If the defaults are used, each time the file \texttt{RESTART.1}
is overwritten. For a restart from converged ground state wavefunctions
and canonical Kohn--Sham orbitals a restart with \\
\texttt{RESTART WAVEFUNCTION COORDINATES} \\
will be used. A restart also including the linear response orbitals
will use \\
\texttt{RESTART WAVEFUNCTION COORDINATES LINRES}. \\
In this case only restarts from the file \texttt{RESTART} are possible
as after phase one and two the file \texttt{RESTART.1} would be
overwritten and the information on the linear response orbitals,
read only in phase three, would be lost.

\subsubsection{Geometry optimizations and molecular dynamics}
\label{sec:TDDFTdynamics}{}

  Geometry optimizations and molecular dynamics simulations can only be
performed after an electronic spectra calculation. A typical input file would
contain the sections

\begin{verbatim}
 &CPMD
   OPTIMIZE GEOMETRY
   TDDFT
   RESTART WAVEFUNCTION COORDINATES LINRES
 &END

 &TDDFT
  STATES SINGLET
    1
  TAMM-DANCOFF
  DAVIDSON PARAMETER
  150 1.D-7 50
  FORCE STATE
   1
 &END
\end{verbatim}

  The keywords in section \&CPMD are all mandatory. The section \&TDDFT
specifies that the optimization should be performed for the first excited
singlet state. Replacing \refkeyword{OPTIMIZE GEOMETRY} by 
\refkeyword{MOLECULAR DYNAMICS} BO would result in a molecular dynamics simulation. 
In this case further input specifying the time step, maximal number of steps, 
thermostats, etc. would also be supplied.




%%%   BEGIN/Daniel   %%%%%%%%%%%%%%%%%%%%%%%%%%%%%%%%%%%%%%%%%%%%%%%%%
\subsection{Perturbation Theory / Linear Response}

\newcommand{\beq}{\begin{eqnarray}}
\newcommand{\eeq}{\end{eqnarray}}
\newcommand{\zero}{^{(0)}}
\newcommand{\one}{^{(1)}}
\newcommand{\PSI}[2]{\vert\varphi_{#1}^{(#2)}\rangle}
\newcommand{\rv}{\mathbf{r}}
\newcommand{\Rv}{\mathbf{R}}
\newcommand{\Gv}{\mathbf{G}}
\newcommand{\Bv}{\mathbf{B}}
\newcommand{\pv}{\mathbf{p}}
\newcommand{\respkeyword}[1]{{\bfseries #1}}

\newcommand{\verbatimsize}{\footnotesize}
\newcommand{\eofverbsize}{\normalsize}

\newcommand{\ddpart}[3]{\ensuremath{\frac{\partial^2{{#1}}} {\partial{{#2}}
\partial{{#3}}} }}

\newcommand{\bA}{\ensuremath{\mathbf{A}}}
\newcommand{\bI}{\ensuremath{\mathbf{I}}}
\newcommand{\bP}{\ensuremath{\mathbf{P}}}
\newcommand{\bR}{\ensuremath{\mathbf{R}}}
\newcommand{\bQ}{\ensuremath{\mathbf{Q}}}
\newcommand{\bT}{\ensuremath{\mathbf{T}}}
\newcommand{\be}{\ensuremath{\mathbf{e}}}
\newcommand{\bq}{\ensuremath{\mathbf{q}}}
\newcommand{\br}{\ensuremath{\mathbf{r}}}
\newcommand{\bs}{\ensuremath{\mathbf{s}}}
\newcommand{\bt}{\ensuremath{\mathbf{t}}}
\newcommand{\bw}{\ensuremath{\mathbf{w}}}

\newcommand{\dr}{d^3r}


\subsubsection{General}

Ref: \cite{apdsmp}.

Perturbation theory describes the reaction of a system onto an
external perturbation. At the time when the perturbation occurs, the
system is in its ground state (or unperturbed state). The perturbation
then changes slightly the potential energy surface and therefore also
the state where the system's energy is minimum. As a consequence, the
system tries to move towards that state of minimum energy. This
movement or the new state often have properties which can be accessed
experimentally. Example: An external electric field will slightly
deform the electronic cloud, creating a dipole. That dipole can then
be measured.


Assume that the magnitude of the perturbation is small compared to the
strength of the forces acting in the unperturbed system. Then, the
change in the minimum energy state will be small as well and
perturbation theory can be applied to compute how the system reacts
onto the perturbation. Generally, the Schr\"odinger equation is
expanded in powers of the perturbation parameter (ex: the strength of
the electric field), and the equations obtained for those powers are
solved individually. At power zero, one finds the equation of the
unperturbed system:

\beq
  (H\zero - \varepsilon_k) \PSI k 0 &=& 0.
\eeq

For the part which is linear in the perturbation,
the general format of the resulting equation is

\beq
  (H\zero - \varepsilon_k) \PSI k 1  &=& - H\one \PSI k 0.
\label{eqn:sternheimer}
\eeq

Grosso modo, this equation is solved during a linear response
calculation through a wavefunction optimization process for {$\PSI k
1$}.


The presence of a first order perturbation correction for the
wavefunctions, $\PSI k {\text{\sf tot}} = \PSI k 0 + \PSI k 1$ implies
that the total density of the perturbed system is no longer equal to
the unperturbed one, $n\zero$, but also contains a first order
perturbation correction, $n\one$.  That density is given by
\beq
n\one (\rv) &=& \sum_k
           \langle \varphi\one_k \vert \rv \rangle \;
           \langle \rv \vert \varphi\zero_k \rangle
           + \text{c.c.}
\eeq

The Hamiltonian depends on the electronic density. Therefore, the
first order density correction implies automatically an additional
indirect perturbation Hamiltonian coming from the expansion of the
unperturbed Hamiltonian in the density. It has to be added to the
explicit perturbation Hamiltonian determined by the type of the
(external) perturbation. The contribution is

\beq
H\one_{\text{\sf indirect}}(\rv) &=& \int \dr^\prime\;
   \frac{\partial H\zero(\rv)}{\partial n\zero(\rv^\prime)}\;
   n\one(\rv^\prime)
\label{eqn:implicit-hamiltonian}
\eeq

The calculation of this indirect Hamiltonian represents almost 50\% of
the computational cost of the response calculation, especially in
connection with xc-functionals. After several unsuccessful trials with
analytic expressions for the derivative of the xc-potential with
respect to the density, this is done numerically. That means that at
each step, the xc-potential is calculated for the density
$n\zero+\epsilon n\one$ and for $n\zero-\epsilon n\one$ (with an
$\epsilon$ empirically set to 0.005), and the derivative needed in
(\ref{eqn:implicit-hamiltonian}) is calculated as

\beq
  \int \dr^\prime\; n\one(\rv^\prime)\;
  \frac{\partial v_{\text{xc}}}{\partial n\zero(\rv^\prime)}
  &=& \frac{v_{\text{xc}}[n\zero+\epsilon n\one]
      - v_{\text{xc}}[n\zero-\epsilon n\one]}
      {2\epsilon}.
\eeq

In the case of the local density approximation, the derivative can be
done analytically, in which case it only needs to be done once. This
improves the performance of the optimization.

\subsubsection{\&RESP section input}
\label{sec:resp-section}

Generally, the keyword \refkeyword{LINEAR RESPONSE} in the
\&CPMD input section initiates the calculation. In the
section \&RESP, the type of the perturbation needs to be
specified. Either one of the following keywords must appear:

\begin{quote}
  \refkeyword{PHONON} \hfill \newline
  \refkeyword{LANCZOS} \hfill \newline
  \refkeyword{RAMAN} \hfill \newline
  \refkeyword{FUKUI} \hfill \newline
  \refkeyword{KPERT} \hfill \newline
  \refkeyword{NMR} \hfill \newline
  \refkeyword{EPR} \hfill \newline
  \refkeyword{HARDNESS} \hfill \newline
  \refkeyword{EIGENSYSTEM} \hfill \newline
  \refkeyword{INTERACTION} \hfill \newline
  \refkeyword{OACP}
\end{quote}

The first six types are discussed in detail in the following. An overview is also
contained in the file respin\_p\_utils.mod.F90. In addition to the
specific keywords of every option, there are several keywords which
are common to all perturbation types. They determine fine-tuning
parameters of the wavefunction optimization process, and usually you
do not need to change them. A \# indicates that a command takes an
argument which is read from the next line. All other keywords toggle
between two states and do not require any argument. Those keywords can
be put together, and they can also figure in the main keyword line
(example: \respkeyword{NMR NOOPT FAST WANNIERCENTERS})


NB: The linear response code works with all cell symmetries\footnote
{except for isolated systems, \respkeyword{SYMMETRY=0}, where only the NMR
part is adapted to.}, but it is \textbf{\textsl{not implemented}} for
$k$-points.


\begin{itemize}

\item[\#] \respkeyword{CG-ANALYTIC:}
The wavefunction optimization uses a preconditioned conjugate gradient
technique. The optimum length of the ``time step'' can be calculated
analytically assuming a purely linear equation, according to the
Numerical Recipes Eq.\ 10.6.4. However, this is somewhat expensive,
and experience shows that the time step is almost constant except at
the very beginning. Therefore, it is only calculated a few times, and
later on, the last calculated value is used. This option controls the
number of times the step length is calculated analytically. Default is
3 for NMR and 99 for all other perturbations.



\item[\#] \respkeyword{CG-FACTOR:}
The analytic formula for the time step assumes that the equation to be
solved is purely linear. However, this is not the case, since the
right hand side can still depend on the first order wavefunctions
through the dependence of the perturbation Hamiltonian $H\one$ on the
perturbation density $n\one$. Therefore, the analytic formula has a
tendency to overshoot. This is corrected by an empirical prefactor
which is controlled by this option. Default is 0.7.


\item[\#] \respkeyword{CONVERGENCE:} The criterion which determines
when convergence is reached is that the maximum element of the
gradient of the energy with respect to the wavefunction coefficients
be below a certain threshold. This value is read from the next
line. Default is 0.00001. Experience shows that often, it is more
crucial to use a strict convergence criterion on the ground state
wavefunctions than for the response. A rule of thumb is that good
results are obtained with a 10 times stricter convergence on the
ground state orbitals compared to that of the response orbitals.


\item[\#] \respkeyword{HTHRS or HAMILTONIAN CUTOFF:}
The preconditioning calculates the diagonal $(\Gv,\Gv)$ matrix
elements of $\eta=H\zero-\frac 1N \sum_k\varepsilon_k$ to do an
approximate inversion of Eq.~(\ref{eqn:sternheimer}). However, these
diagonal values can become very small, yielding numerical
instabilities. Therefore, a smoothing is applied instead of simply
taking the reciprocal values:
\beq
  \eta^{-1} &\mapsto& \left(\eta^2 + \delta^2\right)^{-1/2}
  \label{eqn:precond-1}\\
  \eta^{-1} &\mapsto& \frac {\eta} {\eta^2 + \delta^2}
  \label{eqn:precond-2}
\eeq
The value of the parameter $\delta$ in a.u. is read from the line
after {\respkeyword{HTHRS}}, default is 0.5. By default,
Eq.~(\ref{eqn:precond-1}) is used. \respkeyword{TIGHTPREC} switches to
Eq.~(\ref{eqn:precond-2}).


\item \respkeyword{NOOPT:} In order for the wavefunction optimization to
work properly, the ground state wavefunction must be converged. For
this reason, a ground state optimization is performed by default prior
to computing the response. When restarting from an already converged
wavefunction, this step can be skipped through this keyword and the
computer time for initializing the ground state optimization routine
is saved. However, the use of this option is strongly discouraged.


\item \respkeyword{POLAK:} There are several variants of the conjugate
gradient algorithm. This keyword switches to the Polak-Ribiere
formulation (see the Numerical Recipes, Eq.~10.6.7) which is usually
significantly slower but safer in the convergence.  By default, the
Fletcher-Reeves formula is used.


\item \respkeyword{TIGHTPREC:} Switches to another preconditioning
formula. See \respkeyword{HTHRS}.

\end{itemize}




\subsubsection{Response output}

While the calculations are being done, the program prints the progress
of the optimization process:

\begin{itemize}
\item
The ``scaling input'' prints the number by which the right-hand-side
of Eq.~(\ref{eqn:sternheimer}) is multiplied in order to deal with
reasonable numbers during the optimization. The number is determined
through the condition

\beq
  \vert\vert\;H\one \PSI k 0\;\vert\vert_2 &=& 1.
\eeq

When leaving the optimization routine, the inverse scaling is applied
to the $\PSI k 1$, of course.

\item
The standard output shows (A) the maximum gradient of the second order
energy with respect to the first order perturbation wavefunction, (B)
the norm of that gradient, (C) the second order energy, (D) its
difference with respect to the last value, and finally (E) the CPU
time needed for one step. The last value decreases by somewhat after
the number of steps determined by the {\respkeyword{CG-ANALYTIC}} keyword,
because the analytic line search is no longer performed, as discussed
                above.

\item
A line full of tildes ($\tilde{\ }$) indicates that the energy has
increased. In that case, the conjugate direction is erased and the
conjugate gradient routine is restarted. Also the time step is
calculated again using the analytic quadratic line search formula.

\item
The warning ``line search instable'' indicates that the length which
has been calculated using the analytic quadratic line search
approximation (Numerical Recipes Eq.~10.6.4) has given a numerical
values larger than 3. This does not happen in normal cases, and it can
yield to program crashes due to floating point overflows ($\mapsto$
{\tt NaN}, not a number). Thus, a safe value, 1, is used instead.

\item
The warning ``gradient norm increased by more than 200\%'' indicates
that the quadratic approximation is not valid in the momentary
position of the optimization. If it was, the gradient (of the energy
with respect to the first order perturbation wavefunctions) would only
decrease and finally reach zero. If this situation occurs, the
conjugate direction is erased and the conjugate gradient algorithm is
restarted.
\end{itemize}



%%%%%%%%%%%%%%%%%%%%%%%%%%%%%%%%%%%%%%%%%%%%%%%%%%%%%%%%%%%%%%%%%%%%%%%%
%%%%%%%%%%%%%%%%%%%%%%%%%%%%%%%%%%%%%%%%%%%%%%%%%%%%%%%%%%%%%%%%%%%%%%%%
\subsubsection{Phonons}
%%%%%%%%%%%%%%%%%%%%%%%%%%%%%%%%%%%%%%%%%%%%%%%%%%%%%%%%%%%%%%%%%%%%%%%%
%%%%%%%%%%%%%%%%%%%%%%%%%%%%%%%%%%%%%%%%%%%%%%%%%%%%%%%%%%%%%%%%%%%%%%%%

\textbf{Theory}

A phonon corresponds to small displacements of the ionic positions
with respect to their equilibrium positions. The electrons principally
follow them, in order to minimize again the energy of the system.

The expansion of the Hamiltonian in powers of the displacement
$u^R_\alpha$ of the ion (labeled by its position $R$) in the Cartesian
direction $\alpha=1,2,3$ consists of two parts\footnote
%
  {The reader might wonder whether Einstein summations over repeated
  indices like $\alpha$ and $R$ are done or not. The answer is that
  {\textsl{it depends}}. If only one atom is displaced along one of
  the axes, there is no summation needed. The generalization to a
  simultaneous displacement of all atoms in arbitrary directions is
  trivially done by assuming the summation over $\alpha$ and $R$.}:

\beq
H\one &=& H\one_C + H\one_{PP}
\eeq

The contribution $H\one_C$ comes from the Coulomb term, the
electrostatic potential:

\beq
H\one_C &=& u^R_\alpha
    \frac \partial {\partial R_\alpha} \; \frac{Z_R}{|\rv - \Rv|}.
\eeq

The second is due to the pseudopotential which is rigidly attached to
the ions and which must be moved simultaneously. In particular, the
nonlocal pseudopotential projectors must be taken into account as
well:

\beq
H\one_{PP} &=& u^R_\alpha
    \frac \partial {\partial R_\alpha} \;
    \left[\sum_i
       \vert\text{\tt P}^R_i\rangle \langle\text{\tt P}^R_i\vert
    \right]\\
   &=& u^R_\alpha\sum_i\;\left[
    \left[\frac\partial{\partial R_\alpha}\;
          \vert\text{\tt P}^R_i\rangle\right]
          \langle\text{\tt P}^R_i\vert
    +
    \vert\text{\tt P}^R_i\rangle
          \left[\frac\partial{\partial R_\alpha}\;
          \langle\text{\tt P}^R_i\vert\right]
    \right]
\label{eq.phon}
\eeq

where $\vert\text{\tt P}_i^R\rangle$ designates the projectors,
whatever type they are. The index $i$ comprises the $l$ and $m$
quantum numbers, for example. The superscript $R$ just says that of
course only the projectors of the pseudopotential of the displaced
atom at $R$ are considered in this equation.

In \cpmd, these projectors are stored in G-space, and only one copy is
stored (that is the one for a fictitious ion at the origin,
$R=0$). The projectors for an ion at its true coordinates is then
obtained as

\beq
  \langle \Gv \vert\text{\tt P}_i^R\rangle &=&
  \text{e}^{i\Gv\cdot\Rv}\;\langle \Gv \vert\text{\tt P}_i^{R=0}\rangle.
\label{eqn:translation-formula}
\eeq

This makes the derivative $\frac{\partial}{\partial R_\alpha}$
particularly simple, as only the $i\Gv$ comes down, and the
translation formula (\ref{eqn:translation-formula}) remains valid for
$\vert\text{\tt P}_i^R\rangle$. Thus, there is only an additional
nonlocal term appearing which can be treated almost in the same way as
the unperturbed pseudopotential projectors.



A perturbative displacement in Cartesian coordinates can have
components of trivial eigenmodes, that is translations and rotations.
They can be written {\em a priori} (in mass weighted coordinates in this case)
and thus projected out from the Hessian matrix;
\beq
\bt_j=\left(\begin{array}{c}
            \sqrt{m_1}(\be_j) \\
            \sqrt{m_2}(\be_j) \\
            \vdots          \\
            \sqrt{m_N}(\be_j) \\
            \end{array} \right) \qquad
\bs_j=\left(\begin{array}{c}
            \sqrt{m_1}(\be_j\times\bR_1) \\
            \sqrt{m_2}(\be_j\times\bR_2) \\
            \vdots          \\
            \sqrt{m_N}(\be_j\times\bR_N) \\
            \end{array} \right)
\eeq
where j=x,y,z, $\be_j$ = unit vectors in the Cartesian directions and
$\bR_k$ = positions of the atoms (referred to the COM). The rotations
$\bs_j$ constructed in this way are not orthogonal; before being used
they are orthogonalized with respect to the translations and with
each other.

The projection on the internal mode subspace is done this way:
first one constructs a projector of the form,
\beq
\bP=\sum_j(\bt_j\cdot\bt_j^T+\bs_j\cdot\bs_j^T)
\eeq
and then applies the projector to the Hessian matrix,
\beq
\bA=(\bI-\bP)\cdot\bA\cdot(\bI-\bP)
\eeq
with \bI\ being the unit matrix.
The projection is controlled by the keyword \refkeyword{DISCARD}, {\em vide
infra}.

\textbf{Phonon input}

The input for the phonon section is particularly simple due to
the absence of any special keywords. Only the word \refkeyword{PHONON}
should appear in the \&RESP section.


\textbf{Phonon output}

In total analogy to \cpmd's \refkeyword{VIBRATIONAL ANALYSIS}, the
displacement of all atoms in all Cartesian directions are performed
successively. The difference is, of course, that there is no real
displacement but a differential one, calculated in perturbation
theory. Thus, only one run is necessary per ion/direction\footnote{In
contrast to this, the \refkeyword{VIBRATIONAL ANALYSIS} displaces each
atom in each direction first by +0.01 and then by -0.01.}.

At the end, the harmonic frequencies are printed like in the
\refkeyword{VIBRATIONAL ANALYSIS}. They should coincide to a few percent.


%%%%%%%%%%%%%%%%%%%%%%%%%%%%%%%%%%%%%%%%%%%%%%%%%%%%%%%%%%%%%%%%%%%%%%%%
%%%%%%%%%%%%%%%%%%%%%%%%%%%%%%%%%%%%%%%%%%%%%%%%%%%%%%%%%%%%%%%%%%%%%%%%
\subsubsection{Lanczos}
%%%%%%%%%%%%%%%%%%%%%%%%%%%%%%%%%%%%%%%%%%%%%%%%%%%%%%%%%%%%%%%%%%%%%%%%
%%%%%%%%%%%%%%%%%%%%%%%%%%%%%%%%%%%%%%%%%%%%%%%%%%%%%%%%%%%%%%%%%%%%%%%%

\textbf{Theory}

Ref: \cite{ffmp}

A different way of diagonalizing the Hessian matrix comes from the
Lanczos procedure. It is easy to generalize eq.\ref{eq.phon} to
a collective displacement of atoms,
\beq
H\one_{PP} &=& \sum_{R,\alpha}u^R_\alpha
    \frac \partial {\partial R_\alpha} \;
    \left[\sum_i
       \vert\text{\tt P}^R_i\rangle \langle\text{\tt P}^R_i\vert
    \right]\\
   &=&  \sum_{R,\alpha} u^R_\alpha\sum_i\;\left[
    \left[\frac\partial{\partial R_\alpha}\;
          \vert\text{\tt P}^R_i\rangle\right]
          \langle\text{\tt P}^R_i\vert
    +
    \vert\text{\tt P}^R_i\rangle
          \left[\frac\partial{\partial R_\alpha}\;
          \langle\text{\tt P}^R_i\vert\right]
    \right]
\eeq
In this way the information contained in the Hessian matrix, {\bf A}, can
be used to compute terms of the type
\beq
\bA\cdot\bw =
\left( \begin{array}{c}
       \ddpart{E}{R_{1x}}{\bw}\\
       \ddpart{E}{R_{1y}}{\bw}\\
       \vdots                 \\
       \ddpart{E}{R_{Nz}}{\bw}
       \end{array}\right)
\label{eq.lr.12}
\eeq
where {\bf w} is the collective displacement. This is the building block
for Lanczos diagonalization, that is performed by iterative application
of the symmetric matrix to be diagonalized, over a set of vectors,
known as Lanczos vectors, according to the scheme
\begin{itemize}
\item $\br_0=\bq_1; \beta_0=1; \bq_0=0; k=0$
\item[WHILE] {$\beta_k\neq 0$}
\item  $k=k+1$
\item  $\alpha_k=\bq_k^T\bA\bq_k$
\item  $\br_k=\bA\bq_k-\alpha_k\bq_k-\beta_{k-1}\bq_{k-1}$
\item  $\beta_k=\parallel\br_k\parallel_2$
\item  $\bq_{k+1}=\br_k/\beta_k$
\item  Orthogonalization, $\bq_{k+1}\perp\left\{\bq_1,\cdots,\bq_k\right\}$
\item  Diagonalization of $\bT_k$
\item[END WHILE]
\end{itemize}

The matrix \bA\ is thus projected onto a k$\times$k subspace, in a
tridiagonal form, $\bT_k$. \bT's eigenvalues are approximations to \bA's,
while the eigenvectors are brought in the n$\times$n space by means of
the orthogonal matrix $\bQ=[\bq_1,\bq_2,\dots,\bq_k]$.
The advantage with respect to the usual diagonalization schemes is that
the highest and the lowest end of the spectrum tend to converge before
the ionic degrees of freedom are fully explored.

The projection of the trivial eigenmodes is performed by eliminating
their contribution from every Lanczos vector at the orthogonalization step
reported in the algorithm above,
\beq
\bq_k=\bq_k-\sum_j\left((\bq_k^T\cdot\bt_j)\bt_j +
(\bq_k^T\cdot\bs_j)\bs_j\right)
\eeq
This procedure seems slightly different from that of the \refkeyword{PHONON} case,
but they are perfectly equivalent. It is controlled by the same keyword
\refkeyword{DISCARD} with the same arguments.

\textbf{Lanczos input}

The input for the Lanczos routine is given by the keyword
\refkeyword{LANCZOS} plus some optional arguments and a mandatory following
line with numerical data. Please note that this keyword
is analogous to that for the  Lanczos diagonalization of the electronic
degrees of freedom. Given its presence in the \&RESP section
there should be no ambiguity.

\begin{itemize}

\item \respkeyword{LANCZOS [CONTINUE,DETAILS];} the keyword \refkeyword{LANCZOS}
simply activates the diagonalization procedure. At every cycle the program
writes a file, {\tt LANCZOS\_CONTINUE.1} which contains the information
about the dimension of the calculation, the number of iteration, the
elements of the Lanczos vectors and the diagonal and subdiagonal
elements of the \bT\ matrix.
\begin{itemize}
\item With the option \respkeyword{CONTINUE} one
restart a previous job, asking the program to read the needed information
from the file {\tt LANCZOS\_CONTINUE}; if the program does not find this file,
it stops. This option is to be used when the calculation that one wants
to perform does not fit in the time of a queue slot.
\item The argument \respkeyword{DETAILS} prints a lot of information about
the procedure at the end of the output. It is intended for debugging purposes.
\end{itemize}
The subsequent line is mandatory, and contains three numerical
parameters;
\begin{itemize}
\item {\tt Lanczos\_dimension}; it is the dimension of the vibrational
degrees of freedom to be explored. Normally it is 3$\times$N$_{\rm at}$,
-3 if you are eliminating the translations or -6 if you are eliminating
also the rotations({\em vide infra}). It is possible to use lower values.
Higher are non sense.
\item {\tt no. of iterations}; it is the number of cycles that are to be
performed during the actual calculation. It must be $\leq$ to
{\tt Lanczos\_dimension}; the program checks and stops if it is higher.
In case of a continued job, the program checks if the given number of
iterations + the index of the iteration from which it restarts is
within the limit of {\tt Lanczos\_dimension}; if not it resets the
number to fit the dimension, printing a warning on std. output.
\item {\tt conv\_threshold}; it is the threshold under which an eigenvector
is to be considered as converged. It is the component of the eigenvector
over the last Lanczos vector. The lower it is, the lower is the information
about this mode which has still to be explored.
\end{itemize}
\item \respkeyword{DISCARD \{PARTIAL,TOTAL,OFF,LINEAR\};} this keyword controls
the elimination of the trivial eigenmodes ({\em i.e.} translations and
rotations) from the eigenvector calculations. It works both for
\refkeyword{PHONON} and \refkeyword{LANCZOS}. Omitting it is equivalent
to \respkeyword{DISCARD PARTIAL}.
When using it, the choice of one of the arguments is mandatory; the
program stops otherwise. They are;
\begin{itemize}
\item \respkeyword{PARTIAL}; only the translational degrees of freedom are
eliminated (useful for crystals). This is the default.
\item \respkeyword{TOTAL}; both translational and rotational degrees of
freedom are eliminated.
\item \respkeyword{OFF}; the option is disactivated and no projection is
performed.
%\item \respkeyword{LINEAR}; it projects out all the translatios, plus
%only 2 rotational degrees of freedom. Valid for linear molecules.
%{\em Not implemented yet!}
\end{itemize}

\end{itemize}


\textbf{Lanczos output}
In the output, all the informations about the perturbed wavefunction
optimization are reported, just like in the \refkeyword{PHONON} case. The
differences are in the form of the perturbation and in the eigenmodes
information coming from the diagonalization of $\bT_k$.


The report of the perturbation for a water dimer reads like;
\verbatimsize\begin{verbatim}
 **********************   perturbations    **********************

 **** atom=    1     O    .04171821      .07086763      .07833475
 **** atom=    2     O    .03615521      .08499767      .07082773
 **** atom=    3     H    .29222111      .13357862      .09032954
 **** atom=    4     H    .33764012      .15750912      .25491923
 **** atom=    5     H    .06426954      .26020430      .01822161
 **** atom=    6     H    .15765937      .27370013      .29183999
 cpu time for wavefunction initialization:          89.22 seconds
\end{verbatim}\eofverbsize
where at every atom corresponds the x,y,z displacements applied to calculate
the perturbed wavefunctions.

At every iteration information about the elements of the $\bT_k$ matrix,
alpha's and beta's, are reported. Here we are at the end of the
calculation. Note the numerical zero in the final +1 beta value.
Then the spectrum in a.u. is reported,
together with the convergence information. At the last iteration there
are vectors which are not ``converged''. But this comes only from
the definition, since some of the eigenvectors {\em must} have a
component over the last Lanczos vectors. Following there are the
familiar eigenvalues in cm$^{-1}$.
\verbatimsize\begin{verbatim}

 ****************************************************************
  *+* L2 norm[n[i+1]]       =  0.100262974102936016E-02
  *=*   overlap: alpha[ 12 ] =  0.982356847531824437E-03
  *=*   off-diag: beta[ 13 ] =  0.300011777832112837E-19
  *=*   norm:               =  0.300011777832112837E-19
 ****************************************************************
 *** SPECTRUM, run 12 :
 *** eigenvalue  12 =     .4855525       (converged: .000000).
 *** eigenvalue  11 =     .4785309       (converged: .000000).
 *** eigenvalue  10 =     .4605248       (converged: .000000).
 *** eigenvalue   9 =     .4303958       (converged: .000000).
 *** eigenvalue   8 =     .0973733       (converged: .000000).
 *** eigenvalue   7 =     .0943757       (converged: .000000).
 *** eigenvalue   6 =     .0141633       (converged: .000004).
 *** eigenvalue   5 =     .0046807   (NOT converged: .004079).
 *** eigenvalue   4 =     .0020023   (NOT converged: .010100).
 *** eigenvalue   3 =     .0010310   (NOT converged: .313417).
 *** eigenvalue   2 =     .0009886   (NOT converged: .933851).
 *** eigenvalue   1 =     .0006303   (NOT converged: .171971).

 ****************************************************************
  harmonic frequencies [cm**-1]:

        129.0591        161.6295        165.0552        230.0241
        351.6915        611.7668       1579.1898       1604.0732
       3372.3938       3488.4362       3555.9794       3581.9733

 ****************************************************************

\end{verbatim}\eofverbsize

%%%%%%%%%%%%%%%%%%%%%%%%%%%%%%%%%%%%%%%%%%%%%%%%%%%%%%%%%%%%%%%%%%%%%%%%
%%%%%%%%%%%%%%%%%%%%%%%%%%%%%%%%%%%%%%%%%%%%%%%%%%%%%%%%%%%%%%%%%%%%%%%%
\subsubsection{Raman}
%%%%%%%%%%%%%%%%%%%%%%%%%%%%%%%%%%%%%%%%%%%%%%%%%%%%%%%%%%%%%%%%%%%%%%%%
%%%%%%%%%%%%%%%%%%%%%%%%%%%%%%%%%%%%%%%%%%%%%%%%%%%%%%%%%%%%%%%%%%%%%%%%

Ref: \cite{apmp}

Upon the specification of the RAMAN keyword, the polarizabilities of
the system are calculated by evaluating the response to an applied
electrical field. The calculation is done by means of the Berry phase
approach, which is also suited for periodic systems.




%%%%%%%%%%%%%%%%%%%%%%%%%%%%%%%%%%%%%%%%%%%%%%%%%%%%%%%%%%%%%%%%%%%%%%%%
%%%%%%%%%%%%%%%%%%%%%%%%%%%%%%%%%%%%%%%%%%%%%%%%%%%%%%%%%%%%%%%%%%%%%%%%
\subsubsection{Nuclear Magnetic Resonance}
%%%%%%%%%%%%%%%%%%%%%%%%%%%%%%%%%%%%%%%%%%%%%%%%%%%%%%%%%%%%%%%%%%%%%%%%
%%%%%%%%%%%%%%%%%%%%%%%%%%%%%%%%%%%%%%%%%%%%%%%%%%%%%%%%%%%%%%%%%%%%%%%%
\textbf{Theory}

Ref: \cite{dsmp}

A magnetic field $\Bv$ is applied to the system, which reacts by
induced electronic ring currents. These currents produce an additional
magnetic field by themselves, which is not homogeneous in
space. Therefore, the actual magnetic field at the ionic positions is
different for all atoms in the cell. This field determines the
resonance frequency of the nuclear spin, and this resonance can be
measured with a very high accuracy.

The perturbation Hamiltonian is given by

\beq
  H\one &=& \frac 12 \frac em \pv \times \rv \cdot \Bv.
\label{eqn:nmr}
\eeq

The difficult part of this Hamiltonian lies in the position operator
which is ill defined in a periodic system. To get around this, the
wavefunctions are localized and for each localized orbital,
Eq.~(\ref{eqn:nmr}) is applied individually assuming the orbital being
isolated in space. Around each orbital, a \textsl{virtual cell} is
placed such that the wavefunction vanishes at the borders of that
virtual cell.


The perturbation and therefore also the response are purely imaginary,
so that there is no first order response density. This simplifies the
equations and speeds up convergence.


\textbf{NMR input}

The options which control the specific NMR features are discussed
below. None of them requires an argument, they can all be put in the
same line.


\begin{itemize}

\item \respkeyword{RESTART:} The run is restarted from a previous
  stop. The user has to take care that the required files
  RESTART.\textit{xxx} (where \textit{xxx} are NMR, p\_x, p\_y, p\_z)
  exist.


\item \respkeyword{CURRENT:} Three density files containing current
  density values are written to disk. Further, the nucleus-independent
  chemical shift fields are written to disk. All in cube format.

\item \respkeyword{NOSMOOTH:} At the border of each virtual cell, the
  position operator is smoothened through an $\exp -r^2$. This option
  turns smoothing off. Can be useful if your cell is too small so that
  this smoothing would already occur in a region where the orbital
  density has not yet vanished.


\item \respkeyword{NOVIRTUAL:} If this keyword is specified, no individual
  virtual cells are used at all. All orbitals will have the same
  virtual cell, equal to the standard unit cell.


\item \respkeyword{PSI0:} With this option, the Wannier orbitals are
  plotted as cube files.

\item \respkeyword{RHO0:} With this option, the orbital densities are
  plotted as cube files.

\item[\#] \respkeyword{OVERLAP:}
  Overlapping orbitals (overlap criterion read from next line) are
  assigned the same virtual cells. Useful for isolated systems.

\item \respkeyword{FULL}
  A full calculation of the $\Delta j$ term in the NMR scheme is
  done. Relatively expensive, but quite important for accurate results
  in periodic systems.


\end{itemize}




\textbf{NMR output}

At the end of the six perturbation runs, the results are printed. This
includes the magnetic susceptibility and the chemical shieldings. For
the chemical shieldings, there are two values: the raw and the net
ones. The raw shieldings correspond to a molecule in vacuum, without
the susceptibility correction, whereas the net shieldings contains
that correction for a spherical sample. It consists of an additive
constant to all eigenvalues, which is printed out at the end of the
net shieldings.


In more detail, the results are:


\begin{itemize}
\item
The magnetic susceptibility tensor in both SI and cgs units. This
quantity is an extensive quantity, i.e. two molecules in the
simulation box will give twice the susceptibility of one.


\item
The raw shielding matrices of all atoms and its eigenvalues. Generally,
the shielding tensor is not symmetric. To obtain unique eigenvalues, it is
symmetrized ($A_{ij} \mapsto (A_{ij}+A_{ji})/2$) before the
diagonalization.

All values are given in ppm, parts per million. Chemical shieldings are
dimensionless quantities.


\item
The principal values (eigenvalues) of the raw and net shielding
tensors. As mentioned above, they only differ by an additive constant,
the susceptibility correction for a spherical sample, which is printed
out at the end of the list. The numbers are the shielding eigenvalues from
most negative to most positive, the isotropic shielding (the average of
the eigenvalues), and the anisotropy (the difference between the most
positive and the average of the other two).

\end{itemize}



\textbf{What to look at?}
If you search for the values which are peaked in a spectrum, you have
to take the isotropic shieldings (the {\respkeyword{iso}} column of the
output). If the system is a gas, take the raw shielding, if it is
condensed matter, take the net shielding. If your system is a molecule in
vacuo, but the experimentalists measure it in a solvent, add the
susceptibility correction to the raw shieldings by yourself.

\textbf{Why are my numbers so strange in absolute value?}
One more point shall be mentioned: For all nuclei except hydrogen,
pseudopotentials screen the core electrons. The chemical shielding,
however, is very sensitive to core and semi-core electrons. This can
be corrected through a semi-empirical additive constant in many
cases. This constant still needs to be added to the values given from
the program. It depends on the nucleus, on the pseudopotential, and on
the xc-functional.

In other cases, the calculated chemical shieldings are completely
meaningless due to this deficiency. Then, you have to use a
pseudopotential which leaves out the semicore states such that they
are correctly taken into account. Example: carbon shieldings can be
corrected very well through a constant number, silicon shieldings
cannot. For Si, you have to take the $n=3$ shell completely into the
valence band, requiring a cutoff larger than 300Ry.


\textbf{How to compare to experiment?}
Usually, experimentalists measure the {\textit{difference}} between
the resonance frequency of the desired system and that of a reference
system, and they call it $\delta$ (the {\bfseries shift}) instead of
$\sigma$ (the {\bfseries shielding}). To make life more complicated,
they usually define the shift of nucleus {\sffamily A} of molecule
{\sffamily X} with respect to reference molecule {\sffamily ref} as
%
$\delta^{\text{\sffamily A}}_{\text{\sffamily ref}}({\text{\sffamily X}}) =
 \sigma^{\text{\sffamily A}}({\text{\sffamily ref}})-
 \sigma^{\text{\sffamily A}}({\text{\sffamily X}})$.
%
Example: To calculate
%
$\delta^{\text{\sffamily H}}_{\text{\sffamily TMS}}({\text{\sffamily
CH}}_4)$,
%
where TMS=tetramethylsilane, the standard reference molecule for
H-shifts, one would have to calculate the H-shielding of TMS and of
CH$_4$ and subtract them. Unfortunately, TMS is nontrivial to
calculate, because it is a large molecule and the geometry is
complicated (and the shieldings probably must be calculated taking
into account vibrational and rotational averaging). Thus, in most
cases it is better to take for instance the CH$_4$ shielding as a
(computational) reference, and transform the shieldings relative to
CH$_4$ to those relative to TMS through the experimental shielding of
CH$_4$ with respect to TMS.

While doing so, you should not forget that the shielding is a property
which is usually not yet fully converged when energies and bonding
are. Therefore, the reference molecule should be calculated with the
same computational parameters as the desired system (to reproduce the
same convergence error). In particular, computational parameters
include the type of the pseudopotential and its projectors, the
xc-functional and the cutoff.



\textbf{What accuracy can I expect?}
This is a difficult question, and there is no overall answer. First,
one has to consider that on the DFT-pseudopotential level of theory,
one will never reach the accuracy of the quantum chemistry community.
                However, for ``normal'' systems, the absolute accuracy is typically
0.5-1 ppm for hydrogen and about 5-20ppm for Carbon. The spread
between extreme regions of the carbon spectrum is not reached: instead
of 200ppm, one only reaches 150ppm between aliphatic and aromatic
atoms, for instance. The anisotropy and the principal values of the
shielding tensor can be expected to be about 10-20\% too small.  For
hydrogen shieldings, these values are usually better, the error
remains in the region of a single ppm.


%%%%%%%%%%%%%%%%%%%%%%%%%%%%%%%%%%%%%%%%%%%%%%%%%%%%%%%%%%%%%%%%%%%%%%%%
\subsubsection{FUKUI}
Compute, within the linear response perturbation theory, the nuclear Fukui
function $\phi _I$~\cite{fukui} formally identified as a reactivity index of the 
density functional theory~\cite{fukui2,fukui3}, according to the postulated
criterion for $\delta \mu$.
The quantity $\delta \mu$ is the change in the electronic chemical
potential $\mu$ and is given by

\begin{equation}
\delta \mu =\int f({\bf r}) \delta V^{ext}({\bf r}) d^3r
= -\sum_I \phi_I \delta {\bf R}_I
\end{equation}

where $\phi _I =(\partial {\bf F}_I/\partial N)_{V^{ext}}=
-(\partial \mu /\partial R_I)_N$, $N$ is the number of electrons,
$f({\bf r})$ the electronic Fukui function \cite{fukui,fukui2},
$V^{ext}({\bf r})$ the external
potential at {\bf r}, ${\bf R}_I$ the Cartesian coordinate of the
$I^{th}$ nucleus and ${\bf F}_I$ the force on the $I^{th}$ nucleus.

\subsubsection{KPERT: kdp k-point calculations}
Described in section \ref{sec:kpert}, page{\pageref{sec:kpert}}.
Ref: \cite{mimp}

%%%   END/Daniel   %%%%%%%%%%%%%%%%%%%%%%%%%%%%%%%%%%%%%%%%%%%%%%%%%%%

%%%%%%%%%%%%%%%%%%%%%%%%%%%%%%%%%%%%%%%%%%%%%%%%%%%%%%%%%%%%%%%%%%%%%%%%
% METADYNAMICS  %%%%%%%%%%%%%%%%%%%%%%%%%%%%%%%%%%%%%%%%%%%%%%%%%%%%%%%%
%%%%%%%%%%%%%%%%%%%%%%%%%%%%%%%%%%%%%%%%%%%%%%%%%%%%%%%%%%%%%%%%%%%%%%%%
\subsection{Metadynamics}\label{sec:meta}

These are some notes about the use of the metadynamics (MTD) machinery within CPMD.
It is just a first version of a manual that I hope will be improved by the comment
and possibly the contributions of the users of this method.

The metadynamics can run in a standard NVE/NVT MD run or in a NPE/NPT run (variable cell).
In order to apply the MTD algorithms in CPMD some (not few) lines have to be added in
the input file.
These lines are to be in the \&ATOMS .. \&END section and they provide information about
the kind of MTD to be performed, the choice of collective variables (CV),
some parameters required to determine the time dependent potential
and some other options.
All the MTD input must be between an initial and a final line which are: \\
{\tt
METADYNAMICS \\
$\cdots$ \\
END METADYNAMICS }\\
If the initial line contains also the keyword $COLLECTIVE $ $ VARIABLES$,
the standard MTD, with one single set of CV, is initialized.
If, instead, the keyword  $MULTI$ is found, more than one MTD are performed
simultaneously on the same system; therefore, the whole set of CV is constituted
by $NSUBSYS$ subsets, which are independent one from each other.
The number of subsystems is given on the same line by writing $NS=$ followed
by an integer number  (default: 1). Alternatively, if $MULTIPLE$ $WALKERS$ is present in 
the same line multiple walker metadynamics is preformed using the extended Lagrangian
metadynamics; number of walkers is read in the same line immediately after $NW=$ (see \ref{sect:mw}).
Instead, if  $CELL  FULL$ is the
keyword, the CV are the 6 cell parameters (3 side lengths and 3 angles),
and the MTD is performed without extended Lagrangian, i.e.
the contribution coming from $V(t)$ is directly added into the stress
tensor (see below in {\bf MTD Algorithm}). 

For almost all the input parameters there is a reasonable
default value, but, since the range of applications of MTD is quite wide, it is likely that
the default values do not fit your problem. Therefore some effort is required to choose the
optimal conditions for your run.
Of course, it is important to know something about MTD before using it. There are some references
about the method \cite{alaio, iannuzzi, micheletti, gerlaio, michele}, and about some successful
applications, as e.g. \cite{andras, gervasio, iannuzzi2, sergey, scwo, ikeda, rna, Nair-jacs-08, mb11}.
It can be of great help to read about the choices and results obtained by other users.
But I remark that there are very few general rules that can be held valid for different
problems and systems.

The method is based on the definition of a manifold of CV as functions of the
degrees of freedom characterizing your system,
${\bf S} = \{S_{\alpha}({\bf R},{\bf \phi},{\bf h})\}$, where ${\bf R}$ are the ionic
degrees of freedom, ${\bf \phi}$ are the electronic wavefunctions, and
${\bf h}$ defines the cell box.
The CV which are implemented in the code, have been chosen according to the needs of those
who used the method up to now. Of course they do not exhaust all the problems, and many more
CV might be needed in the future. To implement them, once the analytical
formula and its derivatives are available,
is not complicated at all. In principle, the implementation should be easy for
anybody who knows a bit the CPMD code.

\subsubsection{MTD Algorithm}
Once the CV have bee chosen, the MTD method can be applied in two different fashions. \\
{\bf Direct MTD:} The simplest approach is to  define the time dependent
potential as function of ${\bf S}$, $V(t,{\bf S})$, and apply it directly onto
the involved degrees of freedom. In this case, the equations of motion of the
dynamic variables of the system, ${\bf R},{\bf \phi},{\bf h}$, will include
an additional term  in the total forces, due to the contribution of  $V(t,{\bf S})$.
The disadvantage of this simplified version is that there is scarce control
on the dynamics in the space defined by the CV (CV-space),
which is a projection of the space of all the possible configurations.
In general, we would like to span thoroughly the CV-space, and to acquire
information about the underlying potential.
Often, this means that we need a slow dynamics in this space,
where, for each set of values of the CV, we allow the system to equilibrate
and to choose the configuration with the highest probability.
Only in this way we will be able to construct a reasonable probability
distribution in the configurational space that has been explored and
consequently we will be able to reproduce the Free Energy surface.    \\
{\bf Lagrangian MTD:} This formulation is based on the method of the
extended Lagrangian. In addition to the dynamic variables that
characterize your system, a new set of variables
${\bf s}=\{s_{\alpha}\}$ is introduced.
Each $s_{\alpha}$ is associated to one of the selected $S_{\alpha}$,
it has a fictitious mass $M_{\alpha}$ and velocity $\dot{s}_{\alpha}$.
The equations of motion for the $s_{\alpha}$ variables are derived by
a properly extended Lagrangian, where we add the fictitious kinetic
energy and the potential energy as a function of ${\bf s}$.
Therefore the total potential energy includes two new terms,
a sum of harmonic potentials, which couple the $s_\alpha$
to the respective $S_{\alpha}({\bf R},{\bf \phi},{\bf h})$,
$\sum_{\alpha}k_{\alpha}(S_{\alpha}(\cdots)-s_{\alpha})^{2}$,
and the time dependent potential, which now is a function of ${\bf s}$, $V(t,{\bf s})$.
The coupling constants $\{k_{\alpha}\}$ and the fictitious masses $\{M_{\alpha}\}$
are the parameters that determine the dynamics of the $\{s_{\alpha}\}$
in the CV-space. Please notice that the units of $k$ are Hartree
divided by the square power of u.s., the characteristic units of the
specific CV (if CV is a distance it will be $a.u.^2$,
if an angle $radiants^2$, etc.). In analogy, the units of the
fictitious mass are $Hartree ((t)/(u.s.))^2$, where $t$ indicates the unit of time.
Some guide lines on the choice of these parameters will be given in the following paragraphs.
By choosing the temperature $T_{\bf s}$, the velocities of the components
of $\bf s$ can be initialized giving via a Boltzmann distribution.
Moreover, the velocities can be kept in a desired range by the
activation of a temperature control algorithm (at the moment only the
rescaling of velocity is  implemented).

\subsubsection{The Shape of $V(t)$}
Several shapes have been tested (and more might be proposed in the future).
The default choice is the construction of $V(t)$ by the accumulation of
Gaussian-like hills, i.e. (within the Lagrangian formulation, but the
expressions are the same for the direct MTD approach,
providing to exchange $\bf s$ with ${\bf S(\cdots)}$)
\begin{eqnarray}
V(t,{\bf s}) & = & \sum_{t_{i} < t } \Bigg[W_{i}\exp \left \{-
\frac{({\bf s}-{\bf s}^{i})^{2}}{2 (\Delta s^{\perp})^{2}} \right \} \nonumber \\
 & & \exp \left
\{-\frac{\left( ({\bf s}^{i+1}-{\bf s}^{i}) \cdot ({\bf s}-{\bf s}^{i})\right)^{2}}
{2 (\Delta s^{||}_{i})^4} \right\}\Bigg],
\label{eq: hills}
\end{eqnarray}
Here, $t$ indicates the actual simulation time, $i$ counts the metadynamics steps,
the first exponential gives the hill's shape in the direction perpendicular
to the trajectory, whereas the second exponential tunes the shape along the trajectory.
In this form, the width of the hill along the trajectory is determined by
the displacement in the CV-space, walked between two consecutive metadynamics
steps, $\Delta s^{||}_{i} = f_{b}\sqrt{\Big[\sum_{\alpha}(s_{\alpha}^{i+1}-(s_{\alpha}^{i})^2\Big]}$.
$f_b$ is a factor, which is read in input and can be used to change the
size of the hills along the trajectory, by default it is 1.
The height $W$ and the width $ \Delta s^{\perp}$ are input parameters that can
also be tuned during the MTD, in order to better fit the hill shape to the
curvature of the underlying energy surface (in the CV-space).
As a rule of thumb,  $ \Delta s^{\perp}$  should have roughly the size of the
fluctuations of CV at equilibrium (half the amplitude of the well) and $W$
should not exceed few percents of the barrier's height.
These information can be obtained by some short MD runs at equilibrium
(without MTD) and from some insight in the chemical/physical problem at hand.
Since, in general, different CV fluctuate in wells of different size,
it is important to define one scaling factor  $scf_{\alpha}$ for each component
$s_{\alpha}$, so that
$\langle \delta s_{\alpha}\rangle /scf_{\alpha} = \Delta s^{\perp} \, \forall \alpha$.

Other implemented shapes of $V(t)$ are:\\
{\bf Shift:} the tails of the Gaussians are cutoff, by
setting to zero the Gaussian at a distance $R_{cutoff}\Delta s^{\perp}$
from its center. In this way the problem of the overlap of the
tails in regions far from the actual trajectory is reduced. \\
{\bf Rational:} instead of Gaussian-like hills, some kind of rational functions are used,
\begin{eqnarray}
V(t,{\bf s}) & = & \sum_{t_{i} < t } \Bigg[W_{i}
\frac{1- \left(\frac{\sqrt{({\bf s}-{\bf s}^{i})^{2}}}{\Delta s^{\perp}}\right)^{n}}
{1- \left(\frac{\sqrt{({\bf s}-{\bf s}^{i})^{2}}}{\Delta s^{\perp}}\right)^{m}}
 \nonumber \\
 & & \exp \left
\{-\frac{\left( ({\bf s}^{i+1}-{\bf s}^{i}) \cdot ({\bf s}-{\bf s}^{i})\right)^{2}}
{2 (\Delta s^{||}_{i})^4} \right\}\Bigg],
\label{eq: hills2}
\end{eqnarray}
where the exponents $n$ and $m$ determine the decay. \\
{\bf Lorentzian:} Lorentzian functions are used in place of Gaussians.

In all the cases, a new hill is added at each step of MTD, where
$\Delta t _{meta}=t_{i+1}-t_{i}$ is usually chosen equal to $10\div500$
steps of CP-MD (it depends on the relaxation time of the system and the
size of the hills).
The center of the new hill at time $t_{i+1}$ is positioned along the vector ${\bf s} - {\bf s^{i}}$.


\subsubsection{Metadynamics Keywords}
Now let's start with the explanation of the keywords. First, the definition of the CV is required.
The selected CV are read from the input subsection enclosed between the initial and final lines: \\
$DEFINE $ $ VARIABLES$ \\
$\cdots$ \\
$END $ $DEFINE$ \\
Between these two lines the first input line states how many CV are used, $NCOLVAR$.
In the following, each variable is described by one or more lines, according to its type.
In general, each line must start with the name of the CV, $type-name$,
followed by some indexes or parameters that are needed to
specify its analytical function and the kind of atoms or species that are involved.
At the end, always on the same line of the $type-name$,
the scaling factor $scf$ and, if the extended Lagrangian is used, $k$ and $M$ can be given.
If not specified $scf$, $k$ and $M$ take some default values. \\
{\bf scf:} by default, $scf=1$ and it is fixed during the whole run.
Otherwise, you can write ${\bf SCF}$ followed by the value, or
${\bf SCA}$ followed by the value, a lower bound and an upper bound.
In the latter case, the $scf$ is tuned along the MTD run.
In practice, the average amplitude of the CV fluctuation is checked every
time to time, and, if $scf_{\alpha}\cdot\delta s_{\alpha}$ is far from
$\Delta s^{\perp}$, the $scf_{\alpha}$ is changed accordingly.  \\
{\bf M:} it determines how fast the $s$ variable spans the entire well.
Given the width of the well $scf\cdot\Delta s^{\perp}$  and the temperature
$T_s$, it is possible to choose $M$ by stating the number of complete
fluctuations per ps. The default value is taken for 10 fluctuation per ps.
Otherwise, you can write ${\bf MCV}$  followed by the desired value for
$M$ in $Hartree ((t)/(u.s.))^2/1822 = a.m.u. (a.u./a.s.)^2$. \\
{\bf k:} it determines the dynamics of $s$ with respect to the dynamics of the
physical CV. If $S(\cdots)$ is dominated by fast modes,
it is recommended that $s$ be slower and its fluctuations span the entire well.
Given the characteristic frequencies of the normal modes $\omega_{0}$, $k$ can
be chosen such that $\sqrt{k/M}< \omega_{0}$. On the other hand,
we want $k$ big enough, so that $s$ and $S$ stay close, and $S$ fluctuates
many times at each position in the configuration space.
By satisfying the latter condition, the average forces due to the
underlying potential can be accurately estimated, and the trajectory
lays on the minimum energy path. Therefore, also for $k$, the default
value is chosen in terms of  $T_s$ and $scf_{\alpha}\cdot\delta s_{\alpha}$
Otherwise,  you can write ${\bf KCV}$  followed by the desired value. \\

On the same line, by writing {\bf WALL+ } or {\bf WALL-}, some fixed upper
and lower boundaries
for the CV can be determined. After the keyword the position of the boundary
and the value of the constant repulsive force have to be specified.

{\it Warning}: if even only one $k$ or one $M$ is read from input,
the Lagrangian formulation of the MTD is initialized.

\subsubsection{The Implemented Types of CV}
Please note, that for calculations using the Gromos
QM/MM-interface (see section \ref{sec:qmmm}) the atom indices refer
to the ordering of the atoms as it appears in the respective GROMOS
coordinate file.

\begin{itemize}
\item{STRETCH:} Bond stretch: give the indexes of the 2 atoms  $i1$ $ i2$, $s = (d_{i1,i2})^{2}$
\item{BEND:} Bond angle: give the indexes of the 3 atoms defining the angle, $i1$ $ i2$ $i3$.
\item{TORSION:} Torsion angle: give the indexes of the 4 atoms
defining the torsion angle, $i1$ $ i2 $ $i3$ $ i4$.
\item{DIST:} Distance between two atoms: give the indexes of the 2 atoms $i1$ $ i2$, $s = d_{i1,i2}$.
\item{DISAXIS:} Distance between two atoms $i1$ and $i2$ along $x$ or $y$ or $z$ direction. $i1$ $i2$ $n$ are
read next on the same line. Here $n=1$ means $x$, $n=2$ means $y$ and $n=3$ means $z$ coordinate.
\item{OUTP:} Angle out of plane: give the indexes of the 3 atoms
defining the plane and a fourth index of the atom for which the
angle out of plane is computed, $i1$ $ i2 $ $i3$ $ i4$.
\item{COORD:} Coordination number (CN) of one atom with respect to all the
other atoms in the system. The CN is defined by a Fermi-like function
\begin{equation}
 CN_i = \sum_{j \neq i}^{NATOM}\frac{1}{1+e^{k(d_{ij}-d^0)}}
\end{equation}
where $i$ is the index of the selected atom, $j$ runs over all the other
atoms in the system, $k$ is the parameter which determines the steepness
of the decay and $d^0$ is the reference distance. After the type-name,
in the same line, give $i$ $k$ $d^0$.
\item{DIFFER:} Difference between two distances, give the indexes of the 3 atoms
defining the 2 vectors, $i1$ $ i2 $ $i3$, $s = d_{i1i2}-d_{i2i3}$.
\item{COORSP:} CN of one selected atom $i$ with respect to only one selected
species $jsp$. The CN is defined by a Fermi like function as for $COORD$,
but in this case $j$ runs only over the atoms belonging to the selected
species $jsp$. After the type-name, in the same line, give $i$ $jsp$ $k$ $d^0$.
\item{COORGROUP:} Sum of the CN of a group of atoms $A$ with respect to
individual group of atoms ($B$). CN is estimated using the Fermi function.
Different cutoff distances are allowed for each type of $A$ atoms.
\begin{equation}
 CN = \sum_{i}^{N_A}  \sum_{j}^{N_B(i)} \frac{1}{1+e^{k\left[d_{ij}-d^0(i)\right]}}
\end{equation}
After the keyword {\tt COORGROUP}, $N_A$ and $k$ should be specified.
In the next lines should be:\\
 $i$, $d^0(i)$, $N_B(i)$ \\
 \hspace*{1cm}$j(1) \cdots j(N_B(i))$ \\
This has to be done for all $i$ in list of $A$ type atoms.
\item{COOR\_RF:} CN of one selected atom $i$ with respect to one selected
species, $jsp$, or a list of atoms, $j1 \cdots jn_{list}$.
The CN value is calculated as the sum of rational functions
\begin{equation}
CN_i = \sum_{j \neq i}^{n_{list}} \frac{1-\left(\frac{d_{ij}}{d^0}\right)^{p}}
{1-\left(\frac{d_{ij}}{d^0}\right)^{p+q}},
\end{equation}
where j runs over the indexes of the atoms belonging to $jsp$ or over the
indexes given in the list $j1 \cdots jn_{list}$.
For the option of the species, you should provide, after the type-name,
the indexes $i$ and $jsp$, the exponents $p$ and $q$, and the reference
distance $d^0$ are read. If, instead, the list option is your choice,
write immediately after the type-name the keyword $INDAT$, and next
the values of  $i$, $n_{list}$, $p$, $q$, and $d^0$. The indexes of the
atoms belonging to the list are read from the next line. \\
If the keyword $2SHELL$ is found, in the same line as $COOR\_RF$, the
first real number after this keyword is a second reference distance $d_{2sh}$.
In this case, the functional form of CN is modified, in order to take into
account only the neighbors belonging to one farther shell, and $d_{2sh}$ is
the average distance of these atoms from $i$:
\begin{equation}
CN_i^{2sh} = \sum_{j \neq i}^{n_{list}} \frac{1-\left(\frac{(d_{ij}-d_{2s})}{d^0}\right)^{p}}
{1-\left(\frac{(d_{ij}-d_{2s})}{d^0}\right)^{p+q}}.
\end{equation}
For the modified CN the exponents must be even.

\item{BNSWT:} Reciprocal CN between 2 selected atoms, defined with the same
functional form as the one described for $COOR\_RF$.
This coordinate states the presence of the bond between the two atoms $i$ and $j$.
After the type-name, give $i$, $j$, $p$, $q$, and $d^0$.
\item{TOT\_COOR:} Average CN of the atoms belonging to a selected species $isp$
with respect to a second selected species, $jsp$, or with respect to a given
list of atoms, $j1 \cdots jn_{list}$.
The same functional forms and input options are used, as those
described for $COOR\_RF$, but the index of one selected species $isp$
is read in place of the index of one atom.
\item{DISPL1:} Average displacement of one group of species with respect
to a second group of species, computed along one specified direction in
space (lattice axis in crystals).
This  CV is useful to study diffusion processes in condensed matter.
If the keyword $MIL$ is found, the 3 Miller indexes, which define the direction
in space, are read immediately after (default: ${\bf v}=(hkl)=(100)$).
\item{COORDIS:}
\item{PLNANG:} Angle between two planes. Each plane is defined by the
coordinates of 3 atoms; after the type-name, give the indexes of the
3 atoms defining the first plane, $i1$ $i2$ $i3$, and the indexes
of the atoms defining the second plane,  $j1$ $j2$ $j3$.
\item{HBONDCH:}
\item{DIFCOOR:} Difference between the CN of two different atoms,
$i1$ and $i2$, with respect to the same species $jsp$, or the same list of
atoms,$j1 \cdots jn_{list}$.
The same functional forms and input options are used, as those described for
$COOR\_RF$, but the index of two selected atoms are read, $i1$ and $i2$,  rather than one.
\item{RMSD\_AB:} Given two atomic configurations $A$ and $B$, the root mean square
displacements (RMSD) of the actual configuration from $A$, $rmsdA$, and from $B$,
$rmsdB$, are calculated (global translation and rotation are subtracted by
the method of quaternions). The RMSD can be calculated on selected group of species:
after the type name give the number of species ($NUMSPEC$) and the indexes
of the selected species ($IS_{1} \cdots IS_{NUMSPEC}$).
If $NUMSPEC = 0$ all the species are included.
If in the same line the keyword ${\bf FILEAB}$ is found, next the file name
is read, where the atomic positions of the configurations $A$ and $B$ are given.
Otherwise the file name is by default ${\bf STRUCTURE\_AB}$.
File format: 2 consecutive blocks of $1+NATOM $ lines.
In each block, the first line is a title (Character) and it is
followed by the list of atomic coordinates in a.u. ($element\_name \, x \, y \, z$).
\item{COOR\_CHAIN:} Conditioned CN. Given three species
$isp1$, $isp2$, and $isp3$, the following average CN is calculated
\begin{equation}
CN = \frac{1}{N_{sp1}}\sum_{i1=1}^{N_{sp1}}\left[ \sum_{i2=1}^{N_{sp2}}\left(
\frac{1-\left(\frac{d_{i1i2}}{d_{12}^0}\right)^{p}}{1-\left(
\frac{d_{i1i2}}{d_{12}^0}\right)^{p+q}}\times \sum_{i3=1}^{N_{sp3}}
\frac{1-\left(\frac{d_{i2i3}}{d_{23}^0}\right)^{p}}{1-\left(
\frac{d_{i2i3}}{d_{23}^{0}}\right)^{p+q}}\right)\right].
\end{equation}
After the type-name, the parameters $isp1$, $isp2$, $isp3$, $p$, $q$, $d_{12}^0$,
and $d_{23}^0$ are read.
\item{HYDRONIUM:}
\item{DIS\_HYD:}
\item{SPIN:} Distance between a selected atom and the center of the spin
polarization $(\rho_{\uparrow} - \rho_{\downarrow})$, where $\rho_{}$
indicate the polarized density. The center is located where the difference
is maximum, and this kind of variable is useful only when some spin
polarization is present. The position of the center in systems with PBC
can be calculated by the definition proposed by Resta \cite{resta,berghold}.
Obviously, this CV can be used only together with $LSD$. After the type-name,
give the index of the selected atom.
\item{VOLVAR:} Volume of the cell. It can be used only with NPE/NPT MD.
\item{CELLSIDE:} Length of one cell's side: give the cell-side's index $i$
($i_a=1$,$i_b=2$,$i_c=3$).It can be used only with NPE/NPT MD.
\item{CELLANGLE:} Cell-angle: give the cell-angle's index $i$
($i_{\alpha}=1$,$i_{\beta}=2$,$i_{\gamma}=3$)). It can be used only 
with NPE/NPT MD.
\item{VCOORS}
This CV represents the coordination of one point (V) with respect to 
  a selected species of atoms $jspec$ in the system:
\begin{eqnarray}
CN^{jspec}_{V}=\sum^{N_{jspec}}_{i\in jspec}{1\over 1 + e^{k(d_{iV}-d_0)}}
\end{eqnarray}
After the keyword the parameters $jspec,k,d_0$ are read. In the next line 
the coordinates of the point $V$ are read in a.u.
\item{DIPOLE}
The dipole of the atoms $i_1,...,i_N$ with respect to the atom $j$ is defined as:
\begin{eqnarray}
\vec D_j={1\over Q}\sum_{i=i_1,..,i_N}q_i(\vec r_i -\vec r_j);\;\;\; Q=\sum_{i=i_1,..,i_N} q_i
\end{eqnarray}
The three spherical coordinates of $\vec D_j $, that is $(\rho_j,\theta_j,\phi_j)$, can be used
independently as CV. The keywords are {\bf DIPOLERHO}, {\bf DIPOLETHA}, {\bf DIPOLEPHI}. In the same
line after the keywords are read the index of the atom $j$ and the number $N$ of atoms which
constitute the dipole.  In the next two lines are read the indexes of the $N$ atoms and the
corresponding charge $q_i$.    
\end{itemize}

If $CELL $ $FULL$ is defined in the first line of the MTD input, none of the CV defined above is used.
The CV are the 6 cell parameters. In the section $DEFINE $ $VARIABLE$,
the number of CV is 6 and in the following 6 lines the scaling factors are given:
for each line write the index $i$ of the corresponding CV
($i_a=1$,$i_b=2$,$i_c=3$,$i_{\alpha}=4$,$i_{\beta}=5$,$i_{\gamma}=6$)
followed by $SCF$ or $SCA$ and the desired values
(see the description at the beginning of this subsection).

\subsubsection{Other Keywords}
\begin{itemize}
\item{ANALYSIS:} A standard MD run is performed, where the equations of motions are not
affected by the hills-potential or the coupling potential.
The selected CV are monitored and the values are reported in the
output file, after every 10 MD steps.
This option is useful in order to observe the behavior of the selected CV in equilibrium conditions.
With this option only two output files are written: {\it istvar\_mtd}, and {\it enevar\_mtd} (see section 
\ref{sub:outmtd}). The former file contains the values of the $S_{\alpha}$ and their averages in time.

\item{METASTEPNUM:} The maximum number of MTD steps is read from the
next line,\\$I\_META\_MAX$ (default: 100).

\item{META\_RESTART:}
To restart a metadynamic's run from where the former run has stopprd, one can use this keyword, and write in the 
following line, the number of meta-steps completed already $I\_META\_RES$ (default: 0).
 Beware that for restarting the MTD in this way, the output
files of the previous run are to be available in the run's directory and must contain a number
of lines at least equal to the $I\_META\_RES$.
From this files the previous history of the MTD is read and the MTD is initialized accordingly. \\
Otherwise, it is possible to restart from the restart file of the MTD, $MTD\_RESTART$.
 This is an unformatted file, which is written whenever the standard CPMD restart file, $RESTART.1$, is also written.
It contains the number of meta-steps already performed, the number of CV used in the previous run and the information 
about the position and the size of the hills which have been already located.
To restart the MTD from this file,  the same keyword is used, and the keyword $RFILE$ is added in the same line, 
providing that the unformatted restart file is available in the run's directory.
In this second case the number of performed meta-step is not read from the input file but from the restart file \\.
Obviously, when a run is restarted, the same number and the same kinds of CV must be used. However masses, force 
constants, scaling factors, and the width and height of the hills can be changed.

\item{MINSTEPNUM INTERMETA:} The minimum number of MD steps between two
MTD steps (in general, the MTD step is characterized by the positioning
of a new hill in the CV-space) is read from the next line,
$INTER\_HILL$ (default: 100). This is a lower bound because,
before the construction of a new hill, the displacement in the
CV-space is checked, and the new step is accepted only if the
calculated displacement is above a given tolerance.

\item{MOVEMENT CHECK:} The tolerance for the acceptance of a
new MTD step is read from the next line (default:  0.001D0)

\item{CHECK DELAY:} The number of MD steps to be run, before a
new check of the displacement is done, is read from the following line (default: 20).

\item{MAXSTEPNUM INTERMETA:} The maximum number of MD steps that can be run,
before a new MTD step is accepted anyway, is read from the following line (default: 300).

\item{METASTORE [NO TRAJECTORY]:}
In the following line, three integer numbers are given,
which indicate respectively how often (in terms of MTD steps)
the $RESTART.1$ and the $MTD\_RESTART$ files are over-written (default: 50),
how often the trajectory files are appended (default: 1),
and how often a quench of the electronic wavefunctions onto the
BO surface is performed (default: 10000).
With the additional flag NO TRAJECTORY, the trajectories are
still written according to the settings as indicated by the
\refkeyword{TRAJECTORY} keyword in the \&CPMD section. The selection
of the files (e.g. turning on TRAJEC.xyz via the XYZ flag and turning
TRAJECOTORY off via a negative value of NTRAJ in combination with
the SAMPLE flag) is always honored.

\item{MAXKINEN:} From the following line, the maximum electronic
kinetic energy is given, above which a quench of the
electronic wavefunctions onto the BO surface is performed
anyway (by default no quench is done whatever is the kinetic energy).

\item{LAGRANGE TEMPERATURE:} The temperature $T_s$ used to initialize
the velocities of the ${\bf s}$ CV is read from the next line.
By default $T_s$ is chosen equal to the temperature of the ions,
the units are Kelvin. Notice that this keyword causes the
initialization of the Lagrangian formulation of MTD.

\item{LAGRANGE TEMPCONTROL:} The control of the $T_s$ is activated,
and the rescaling velocities algorithm is used.
The average temperature and the permitted range of variation are
read from the next line. By default $T_s$ is not controlled.
Notice that this keyword causes the initialization of the extended degrees of freedom of the Lagrangian formulation of MTD. 

\item{LAGRANGE LANGEVIN:} Performs Langevin dynamics for the Lagrangian 
formulation of MTD. In the next line the Temperature (Kelvin) and the
friction $\gamma$ (a.u.) are read. The Langevin equation in its standard 
form writes ($z$ is a CV): 
\begin{eqnarray}
M\ddot z=F(z)-\gamma M \dot z+\sqrt{2k_BT\gamma M}\eta(t) 
\end{eqnarray}
where $M$ is the CV mass, $F$ a generic force field, $\gamma$ the friction
coefficient, $T$ the temperature and $\eta$ is a white noise. The integration
algorithm is a Velocity-Verlet which can be written as:
\begin{eqnarray}
& & \dot z_{n+1/2}=\dot z_n+{1\over 2}dt\Big[{F(z_n)\over M}-\gamma\dot z_n\Big]+{1\over2}\sqrt{2k_BT\gamma dt\over M}\xi_n\nonumber \\
& & z_{n+1}=z_n+\dot z_{n+1}dt\nonumber \\
& & \dot z_{n+1}=\dot z_{n+1/2}+{1\over 2}dt\Big[{F(z_{n+1})\over M}-\gamma\dot z_{n+1/2}\Big]+{1\over2}\sqrt{2k_BT\gamma dt\over M}\xi_n
\end{eqnarray}
the $\xi_n$ are independent Gaussian random numbers with mean zero and 
variance one.

\item{HILLS:} With this keyword it is defined the shape of $V(t)$.
If $OFF$ is read in the same line, no hill potential is used.
The default hills' shape is the Gaussian-like one described above.
If $SPHERE$ is read from the same line, the second exponential term in equation \ref{eq: hills}
is not applied. i.e., a normal Gaussian function rather than a Gaussian tube formalism is used.
If, instead, $LORENZIAN$ is read in the same line as $HILLS$,
Lorentzian functions are used in place of Gaussians. If, instead,
$RATIONAL$ is read in the same line as $HILLS$, the rational function
described in the previous section is used; in this case, if $POWER$ is read,
the exponents $n$ and $m$, and the boosting factor $f_b$ are also
read immediately after (defaults: $n=2$, $m=16$, $f_b=1$).
If, instead, $SHIFT$ is read on the same line as $HILLS$,
the shifted Gaussians are used, where the tails after a given
cutoff are set equal to zero; in this case, if $RCUT$ is read,
the cutoff $rshift$ and the boosting factor $f_b$ are also read
immediately after (defaults: $rshift = 2$,$f_b=1$). \\
In all this cases, if the symbol $=$ is read in the same
line as $HILLS$, the perpendicular width $\Delta s^{\perp}$
and the height $W$ are read immediately after (defaults:
$\Delta s^{\perp} = 0.1 \, u.s.$, $W = 0.001 \, Hartree$).

\item{TUNING HHEIGHT:} With this keyword the height of the hills is
tuned according to the curvature of the underlying potential.
If the symbol $=$ is read in the same line as $TUNING$ $ HHEIGHT$,
the lower and upper bounds of $W$ are read immediately after
(defaults: $W_{min}=0.0001 \, Hartree$, $W_{max}=0.016 \, Hartree$).

\item{HILLVOLUME:} With this keyword the volume of the hills,
$\sim W\cdot (\Delta s^{\perp})^{NCOLVAR} $, is kept constant
during the MTD run, i.e. when the height changes due to the
tuning (see the previous keyword), the width is changed accordingly.
This option is can be used only if the tuning of the hills' height is active.

\item{CVSPACE BOUNDARIES:} With this keyword the confinement of the
CV-space is required, in the direction of a selected group of CV.
The number of dimensions of the CV-space, in which the confinement
is applied, is read from the next line, $NUMB$ (default: 0).
The following $NUMB$ lines describe the type of confinement.
For each line the following parameters are required: index of the
CV (as from the list given in the definition of the CV),
the strength of the confinement potential $V_{0}$ in Hartree, two real numbers,
$C1$ and $C2$, that determine from which value of the CV the
confining potential is active. Finally, if the keyword $EPS$
is read in the same line, the real number, which is read
immediately after, determines how smoothly the confining
potential is switched on (default $\varepsilon = 0.01$).
The confining potential can be used with the following CV: \\
DIST: $V_{conf} = V_{0}\left(\frac{s_{\alpha}}{C1}\right)^{4}$
and it becomes active only if $s_{\alpha} > C2$. \\
DIFFER:  $V_{conf} = V_{0}\left(\frac{\vert s_{\alpha}\vert}{C1}\right)^{4}$
and it becomes active only if $s_{\alpha} > C2$ or $s_{\alpha} < -C2$ . \\
Coordination numbers: if $(CN-\varepsilon) < C1$, $V_{conf} =
V_{0} (\frac{C1}{CN})^{10}$, if $(CN+\varepsilon) > C2$,
$V_{conf} = V_{0} (\frac{CN}{C2})^{10}$.

\item{RESTRAIN VOLUME:} With this keyword a confining potential is
applied to the volume variations. The option can be used only in
combination with the NPE/NPT MD. From the next line, the following parameters are
required: $f_{min}$, $f_{max}$, and $V_0$. The factors $f_{min}$ and $f_{max}$
multiplied by the initial volume of the cell, give, respectively,
the lower and upper bounds for the volume, whereas $V_0$ gives the
strength of the confining potential.

\item{MULTI NUM:} his keyword should be used when a multiple set of
MTD runs are performed simultaneously on the same system. Here the
number of separated sets CV for each subsystem has to be given in
the following line, $NCVSYS_1 \cdots NCVSTS_{NSUBSYS}$.
This means that the first $NCVSYS_1$ CV, in the list of those
defined in the $DEFINE $ $ VARIABLES$ section, will belong to
the first subset, the next  $NCVSYS_2$ to the second,  and so on.
This option is implemented only together with the extended
Lagrangian formulation.

\item{MONITOR: } This keyword requires that an additional monitoring of the values
of the CV is performed along the MD trajectory. This means that the values are written
on an output file every {\it WCV\_FREQ} MD steps, even if no hill is added at that step.
The frequency for updating the file is read from the following line, the file name is
{\it cvmdck\_mtd}, and it is not created if this option is not activated.

\item{RANDWALK: } In the case of multiple walker metadynamics, collective variables of
all walkers are initialized with different random velocities. 

\end{itemize}

\subsubsection{Output files}
\label{sub:outmtd}
During a run of MTD, several output file are updated at each MTD step,
which are characterized by the extension $\_mtd$. These files contain
the history of the MTD, the parameters giving the additional potential,
and other information that can be useful during the analysis of results.
The first column for all these files is the CPMD step number (NFI) that
corresponds to this MTD step.
In the case of $MULTI$ MTD some of the files have a further extension
$\_s\#$, which indicates the related subsystem.

Extended Lagrangian MTD: \\
{\it istvar\_mtd}: the first $NCOLVAR$ columns are the $S_{\alpha}(\cdots)$,
the next $NCOLVAR$ columns are differences $S_{\alpha}(\cdots) - s_{\alpha}$. \\
{\it colvar\_mtd}: the first $NCOLVAR$ columns are the $s_{\alpha}$,
the next $NCOLVAR$ columns are the corresponding scaling factors
$scf_{\alpha}$. \\
{\it parvar\_mtd}: norm of the total displacement in the CV-space,
$\Delta s^{||}$, hill's width, $\Delta s^{\perp}$, hill's height, $W$. \\
{\it disvar\_mtd}: the first $NCOLVAR$ columns are the displacements
of the $s_{\alpha}$, the next $NCOLVAR$ columns are their diffusivities,
the final $NCOLVAR$ columns are the coupling constants, $k_\alpha$. \\
{\it velvar\_mtd}: the velocities of the $s_{\alpha}$. \\
{\it forfac\_mtd}: the first $NCOLVAR$ columns are the forces
coming from the coupling potential (sum of harmonic terms, $V_{harm}$),
the next $NCOLVAR$ columns are the forces coming from $V(t)$,
the last $NCOLVAR$  are the forces coming from the confining potential. \\
{\it enevar\_mtd}: ions temperature, electrons kinetic energy,
${\bf s}$ CV temperature $2 K_{\bf s}/(NCOLVAR k_{B})$,  $V_{harm}$,
$V(t)$ at the actual position in CV-space, KS energy
$E_{KS}$, $E_{tot}+ K_{s} + V_{harm}$, $E_{tot}+ K_{s} + V_{harm} + V(t)$.
{\it cvmdck\_mtd}:  monitoring o the CV along the MD trajectory. This file
is updated every {it WCV\_FREQ} MD steps (see previous section {\it MONITOR})

Direct MTD: \\
{\it colvar\_mtd}: the first $NCOLVAR$ columns are the $s(\cdots)_{\alpha}$,
the next $NCOLVAR$ columns are the corresponding scaling factors
$scf_{\alpha}$. \\
{\it sclvar\_mtd}: the first $NCOLVAR$ columns are the scaled
$s(\cdots)_{\alpha}$, the next $NCOLVAR$ columns are the corresponding
scaled diffusivities. \\
{\it parvar\_mtd}: norm of the total displacement in the CV-space,
$\Delta s^{||}$, hill's width, $\Delta s^{\perp}$, hill's height, $W$. \\
{\it disvar\_mtd}: the first $NCOLVAR$ columns are the displacements of
the $s_{\alpha}$, the next $NCOLVAR$ columns are their diffusivities,
the final $NCOLVAR$ columns are the coupling constants, $k_\alpha$. \\
{\it forfac\_mtd}: the first $NCOLVAR$ columns are the forces coming
from $V(t)$, the last $NCOLVAR$  are the forces coming from the
confining potential. \\
{\it enevar\_mtd}: ion temperature, electrons kinetic energy,
$V(t)$ at the actual position in CV-space, KS energy $E_{KS}$, $E_{tot}+ V(t)$.

\subsubsection{Using multiple walker metadynamics}
\label{sect:mw}

Multiple walker metadynamics is activated using the $MULTIPLE$ $WALKER$ keyword
in the initial input line of metadynamics (i.~e., after $METADYNAMICS$ keyword).
%
Multiple walker using the extended Lagrangian metadynamics in combination with
the Car--Parrinello type of dynamics is only implemented at the moment.
%
From the same line of $MULTIPLE$ $WALKER$ keyword, the number of walkers (NWALK)
is read as $NW=NWALK$ (without any space in between).

$NWALK$ replicas are created, and for each DFT forces and energy calculations are done
independently.
%
However, all the replicas fill the same free energy surface encompassed
by the set of reaction coordinates specified~\cite{Raiteri05}.
%
Implementation is done in such a way that each replica belongs to a different processor
group,and each processor group is able to perform independent DFT calculations in parallel;
if $NPROC$ number of processors are used, each replica is using $NPROC/NWALK$
number of processors~\cite{Nair-inside-08} for computations.
%
See the output file for the details on the division of processors in to corresponding processor groups.
%
Note that output of all the walkers are currently dumped in to the (same) standard output file.
%
Trajectory, geometry and energy files of each walkers are separately written out
in files having their usual names augmented with \_$IWALK$, where $IWALK$ is the walker ID.


If a multiple walker run has to be started from a single restart file, copy or link it
$NWALK$ times as RESTART\_1, $\cdots$ RESTART\_$NWALK$ (similarly the MTD\_RESTART file,
if is also restarted).
%
In the procedure of creating new walkers, like above from one restart or increasing walker numbers
during the run, it is advised to initially run with zero hill height, still keeping all the biasing
potentials accumulated upto then (i.\,e.  restarting from the MTD\_RESTART file if any previous
metadynamics runs have been done), and use $RANDWALK$ keyword until the (all) walkers are far apart
(at least 1.5 x hill width) from each other.
%
In the following (restart) run, use the required hill height and remove the $RANDWALK$ keyword.
%
Note that the frequency of adding hills will be nearly $NMTD/NWALK$ if $NMTD$ is the number
of MD steps required to add a hill using 1 walker.
%
Thus consider decreasing the $MINSTEPNUM$ $INTERMETA$ appropriately.
%
However, it is highly recommended to use the adaptive metadynamics time step using $MOVEMENT$ $CHECK$
keyword, and the tolerance is typically 1.5 times the hill width parameter~\cite{Nair-jacs-08}; better
set the $CHECK$ $DELAY$ to 1.
%
Displacement tolerance are also forced to satisfy between the walkers when $MOVEMENT$ $CHECK$ is used.
%
Note that it is possible to decrease the number of walkers, in a straight forward manner during a
restart run.
%

\subsubsection{Shooting from a Saddle}
Once one reactive trajectory has been found, one may want to
determine more precisely the position of the transition state region.
A standard way to do this is to select some points along the trajectory,
and, by shooting with random velocities a new MD from this point,
measure the probability to reach the surrounding basins of
attraction \cite{chandler}.
The different basins of attraction can be identified by different
values of a selected  set of CV. One can say that the trajectory
has fallen in one of the known basins when all the actual CV values
satisfy the values characterizing that basin, within a certain tolerance.
Given a set of coordinates, one can start a CPMD run where this check
is iterated as many times as you like, in order to establish the
commitor distribution.The search for the saddle point region is
                initialized when in the section \&ATOMS \&END of the input file,
the keyword $SADDLE$ $POINT$ is found.
In what follows, a subsection for the description of the selected CV
is required. It has the same format as the one used for the MTD run.
\subsubsection{Keywords}
The list of keyword regarding the shooting needs to be ended by the line \\
$END$ $SADDLE$ \\
Other keywords are:
\begin{itemize}
\item{KNOWN\_MINIMA} The values of the CV, which characterize the
known basin of attraction, are read from the following lines. The
first line after the keyword contains the number of the known
minima $NCVMIN$. The next $NCVMIN$ lines contain the set of
values for each of these minima. The list of the values must
keep the same order used in the definition of the CV.
If on the same line as $KNOWN\_MINIMA$, the keyword $EXTENDED$ is
also found, each line contains $NCOLVAR$ more entries, which are
the tolerances for the acceptance  of the corresponding minimum
configuration  (the order of the tolerances must be the same as
the one for the CV values). By using the $EXTENDED$ keyword,
each minimum configuration can be accepted with different tolerances.
\item{SADDLE TOLERANCES} If the $EXTENDED$ keyword is not used, one
single set of tolerances (one for each CV) can be given by using
this keyword. The tolerances are read from the next line in the
same order used for the CV definition.
Otherwise, default values are assigned at each tolerance.
\item{MAXSEARCH} The maximum number of trials, where a new MD trajectory
is generated, is read from the next line. At each trial, the MD starts
from the same initial coordinates, whereas the initial velocities are
randomly generated at every new restart.
During one trials, every $NSTEP$ the actual values of the CV are checked
and compared to the values given for the known minima. If all the
values of one of these minima are satisfied within the given
tolerances, the MD is stopped and restarted for the next trial.
\item{STEPCHECK}  The number of MD steps between two consecutive
checks is read from the next.
\item{MAXCHECKS}  The maximum number of checks for each trial is
read from the next line.
\end{itemize}
%---------------------------------------------------------------------
%%%%%%%%%%%%%%%%%%%%%%%%%%%%%%%%%%%%%%%%%%%%%%%%%%%%%%%%%%%%%%%%%%%%%%%%

\subsection{Restricted Open-Shell Calculations}\label{hints:roks}

Molecular dynamics simulations in the first excited state can be performed using Restricted Open-Shell Kohn-Sham (ROKS) 
theory \cite{Frank98}. The keyword \refkeyword{ROKS} in the \&CPMD section defaults to the first excited singlet state.
Solving open-shell equations is not simple unless
\begin{enumerate}
\item a high-spin state is computed.
\item the two singly occupied molecular orbitals (SOMOs) have different spatial symmetry. 
\end{enumerate}
In these two cases the Goedecker-Umrigar-Algorithm (GOEDECKER) may be used
which shows the best convergence properties and is applicable in connection
with Car-Parrinello molecular dynamics.
Otherwise it is necessary to use a modified variant of the Goedecker-Umrigar-Algorithm and to do Born-Oppenheimer molecular dynamics (unless you know what you are doing).
In almost all cases,
the default algorithm (DELOCALIZED) is applicable, whereas
for example some dissociation reactions require the localized variant to enable localization of the orbitals on the fragments.
\begin{verbatim}
      ROKS LOCALIZED
\end{verbatim}
In order to make sure that the chosen algorithm works for a certain system, the conservation of energy
during a molecular dynamics simulation and the shape of the orbitals should always be checked.
One of the SOMOs should have the same nodal structure
as the HOMO obtained by a ground state calculation. If using the unmodified Goedecker-Umrigar scheme
(GOEDECKER), the energy of the singlet may collapse to approximately the triplet energy if the two
SOMOs do not have different symmetries. The triplet energy can be calculated by specifiying
\begin{verbatim}
       ROKS TRIPLET
\end{verbatim}
or also
\begin{verbatim}
       ROKS TRIPLET GOEDECKER
\end{verbatim}

See the description of the keywords \refkeyword{LOW SPIN EXCITATION}, \refkeyword{LSE PARAMETERS} and \refkeyword{MODIFIED GOEDECKER} for a description of how to do ROKS calculations using the older input LOW SPIN EXCITATION ROKS.
ROKS GOEDECKER corresponds to LOW SPIN EXCITATION ROKS whereas ROKS DELOCALIZED corresponds to LOW SPIN EXCITATION ROKS with MODIFIED GOEDECKER. Do not use LOW SPIN EXCITATION in the \&SYSTEM section and ROKS in the \&CPMD section at the same time.\\

ROKS is not implemented with Vanderbilt pseudopotentials.\\

A Slater transition-state density between a singlet ground state and the
first excited singlet state (or any pair of states described with ROKS) can
be useful whenever one set of Kohn-Sham states is required which is equally
well suited for each of the states involved in a transition, e.g., to calculate
the couplings between the electronic transition and an external influence.
This method is analogous to state-averaged multiconfigurational SCF
methods and shares many of their benefits with them.
In CPMD, it can be used to calculate non-adiabatic couplings between singlet
states~\cite{nonadiabatic,lsets}, see options \refkeyword{COUPLINGS}.

\subsection{Hints on using FEMD}

  There are several parameters which crucially affect the speed, accuracy and
robustness of the FEMD method. These are related to: LANCZOS PARAMETERS, STATES
and ANDERSON MIXING. Less crucially, the ELECTRON TEMPERATURE.


\subsubsection{Lanczos Parameters}
\label{hints:lanczos}

Several parameters related to the Lanczos ( Friesner-Pollard)
method are given. Generically:
\begin{verbatim}
             LANCZOS PARAMETER [N=n]
                ncycle  nkrylov  nblock  tolerance
                drhomax(2)    tolerance(n)
                  .....
                  .....
                drhomax(n)    tolerance(n)
\end{verbatim}
$Ncycle$ can always be safely set to 50. Similarly, $Nkrylov=8$ is almost
always a good choice. Exceptionally, for certain d-metallic systems, increasing
$nkrylov=16$ may be more efficient. $Nblock$ is the dimension of the blocking
in the evaluation of $H[\psi_1,...,\psi_{nblock}]$. $Nblock$ should be a
divisor of $NSTATE$ and recommended values lie in the range of 20-100. The
$tolerance$ specifies the accuracy to be achieved in the Lanczos method. States
are considered converged if
\begin{equation}
   |H\psi-\epsilon\psi|^2 < tolerance
\end{equation}
For efficient calculations, the tolerance should vary according to closeness to
self-consistency ( as measured by DRHOMAX). During initial stages of the SC
cycle, the tolerance can be loose, gradually tightening until close to SC it is
high. An example of this might be:
\begin{verbatim}
             LANCZOS PARAMETER N=5
                50  8  20 1.D-9
                0.05      1.D-11
                0.01      1.D-13
                0.0025    1.D-16
                0.001     1.D-18
\end{verbatim}
For accurate forces, a final tolerance of at least 1.D-16 is recommended,
although accurate energies can be got using a lower tolerance. It is worth
experimenting how best to tighten the tolerance - it could be system dependent.

\subsubsection{Other important FEMD parameters}

  The keyword STATES defines the dimension of the subspace used in the
diagonalization. STATES must be greater than or equal to $N_{el}/2$, but it is
generally good to allow for a number of more or less empty bands (usually 10\%
or so). Finally, ANDERSON MIXING determines the rate of convergence to
self-consistency. Properly chosen the convergence can be very fast. Typically
for bulk systems we use values between 0.2-0.5, smaller values being necessary
for large systems. For metallic surfaces, small values are necessary (
typically 0.03-0.05).

  If using k-points, then it is usually a good idea ( and this is done by
default if using MONKHORST PACK k-points) to exploit symmetries. In this case,
however, beware of including the POINT GROUP keyword to symmetrise properly the
density. Finally, if starting from a high-symmetry structure, you may
nevertheless want to use the full k-point mesh ( apart from time-inversion
symmetry related k-points), and in this case specify the keyword FULL.


\subsection{The Davidson analysis and the shared electron number}\label{hints:pop}
  The calculation of the shared electron number can have the following input
section:
\begin{verbatim}
 &PROPERTIES
   PROJECT WAVEFUNCTION
   POPULATION ANALYSIS MULLIKEN DAVIDSON 2-CENT 3-CENT
     4
     1
   WAVEFUNCTION LATEST
 &END
\end{verbatim}
Note, that for the hydrogen it is enough to specify one atomic orbital to
project on, for the elements Li to Ne it is sufficient to specify
4 atomic orbitals.

%%%%%%%%%%%%%%%%%%%%%%%%%%%%%%%%%%%%%%%%%%%%%%%%%%%%%%%%%%%%%%%%%%%%%%%%
% CDFT and FO-DFT documentation %%%%%%%%%%%%%%%%%%%%%%%%%%%%%%%%%%%%%%%%
%%%%%%%%%%%%%%%%%%%%%%%%%%%%%%%%%%%%%%%%%%%%%%%%%%%%%%%%%%%%%%%%%%%%%%%%
\subsection{CDFT Theory}\label{sec:CDFT_FO-DFT}
We implemented the constrained DFT (CDFT) method as developed by Wu and van Voorhis \cite{Wu05,Wu06jcp,Wu06jctc,Wu06jpca,Wu07}.

\subsubsection{The density constraint}
Here, one imposes a scleronomic constraint on the electron density to reproduce a certain charge distribution on the atoms,
\begin{equation}
\int w(\mathbf{r}) \rho(\mathbf{r})\; d\mathbf{r}-N_\text{c}=0,
\end{equation}
with $\rho(r)$ being the electron density $w(r)$ the weight and $\text{N}_\text{c}$ the constraint value.
The new energy functional:
\begin{equation}
W[\rho,V]=E[\rho]+V (\int w(\mathbf{r}) \rho(\mathbf{r})\; d\mathbf{r}-N_\text{c}),
\end{equation}
with $E[\rho]$ being the normal Kohn-Sham energy functional.

\subsubsection{The weight}
We chose the Hirshfeld partitioning scheme \cite{Hirshfeld77} in order to define charges. The weight for imposing a charge difference between a donor $D$ and an acceptor $A$ is then given by:
\begin{equation}
w(\mathbf{r})=\frac{\sum_{i \in D} \rho_i(\mathbf{r}-\mathbf{R}_i)-\sum_{i \in A} \rho_i(\mathbf{r}-\mathbf{R}_i)}{\sum_{i=1}^{N} \rho_i(\mathbf{r}-\mathbf{R}_i)},
\end{equation}
where the sums in the numerator range over all donor and acceptor atoms, respectively, whereas the sum in the denominator ranges over all $N$ atoms. $\rho_i(r)$ is the unperturbed electron density of atom $i$ which is given by
\begin{equation}
\rho_i(r)=\sum_j n_j \frac{\vert \psi_i^j(r)\vert^2}{r^2},
\end{equation}
where the sum ranges over all orbitals and $\psi_i^j(r)$ and $n_j$ denote the reference orbitals and $n_j$ their occupation number, respectively.

\subsubsection{Constraint forces}
Using the Hellmann-Feynman theorem we can express forces on atom $k$ at position $R_k$
\begin{equation}
\frac{\partial W}{\partial \mathbf{R}_k}=\langle \psi\vert \frac{\partial \hat{H}}{\partial \mathbf{R}_k}\vert \psi\rangle.
\end{equation}
Applying this relation we can calculate the additional forces due to the bias potential:
\begin{eqnarray}
\mathbf{F}_{\text{Bias},k}= - V \int \rho(\mathbf{r})\frac{\partial w(\mathbf{r}-\mathbf{R}_k)}{\partial \mathbf{R}_k}\;d\mathbf{r},
\end{eqnarray}
with 
\begin{equation}
\frac{\partial w(\mathbf{r}-\mathbf{R}_k)}{\partial \mathbf{R}_k}=-\frac{\rho_k^\prime(\vert\mathbf{r}-\mathbf{R}_k\vert)}{\sum \rho_i(\vert\mathbf{r}-\mathbf{R}_k\vert)} G_k(\mathbf{r}-\mathbf{R}_k)
\end{equation}

\begin{eqnarray}
G_k(\mathbf{r}-\mathbf{R}_k) = \left\lbrace
\begin{array}{l c r}
w(\mathbf{r}-\mathbf{R}_k)-1 & & k \in D\\
w(\mathbf{r}-\mathbf{R}_k)+1 & & k \in A\\
w(\mathbf{r}-\mathbf{R}_k) & & k \not\in D\cup A\\
\end{array}\right.
\end{eqnarray}
Finally the derivative of $\rho_i$ is given by
\begin{eqnarray}
\rho_k^\prime(\vert\mathbf{r}-\mathbf{R}_k\vert)&=&\frac{\partial\rho_k(\vert\mathbf{r}-\mathbf{R}_k\vert)}{\partial\mathbf{R}_k}= \frac{\partial\rho_k(\vert\mathbf{r}-\mathbf{R}_k\vert)}{\partial \vert\mathbf{r}-\mathbf{R}_k\vert}\frac{\partial\vert\mathbf{r}-\mathbf{R}_k\vert}{\partial\mathbf{R}_k},\nonumber\\
&=& \frac{\partial\rho_k(\vert\mathbf{r}-\mathbf{R}_k\vert)}{\partial \vert\mathbf{r}-\mathbf{R}_k\vert}\frac{\mathbf{r}-\mathbf{R}_k}{\vert\mathbf{r}-\mathbf{R}_k\vert}.
\end{eqnarray}
The radial partial derivative $\rho_i$ of is finally calculated numerically using splines.

For a more thorough treatment of the topic of constrained DFT MD please consult reference \cite{Oberhofer09}.

\paragraph{Cutoff correction}
In order to avoid integrating over every real space gridpoint we introduced a cutoff $R_c$ in the generation of the weights. $R_c$ is chosen for each atom species such that the total reference density is smaller than $10^{-6}$. 

The action of the cutoff can be described by a Heaviside function $\theta$
\begin{equation}
\rho_k(\vert\mathbf{r}-\mathbf{R}_k\vert)\rightarrow\rho_k(\vert\mathbf{r}-\mathbf{R}_k\vert)\theta(R_c-\vert\mathbf{r}-\mathbf{R}_k\vert).
\end{equation}
Therefore the derivative of the reference density becomes
\begin{equation}
\rho_k^\prime(\vert\mathbf{r}-\mathbf{R}_k\vert)\rightarrow\rho_k^\prime(\vert\mathbf{r}-\mathbf{R}_k\vert)\theta(R_c-\vert\mathbf{r}-\mathbf{R}_k\vert)+\rho_k(\vert\mathbf{r}-\mathbf{R}_k\vert)\delta(R_c-\vert\mathbf{r}-\mathbf{R}_k\vert),
\end{equation}
with the Dirac $\delta$ function. Thus the full force splits up into 
\begin{equation}
\mathbf{F}_{\text{Full},k}=\mathbf{F}_{\text{Bias},k}+\mathbf{F}_{\text{Bound},k}.
\end{equation}
Here $\mathbf{F}_{\text{Bound},k}$ denotes forces due to the finite cutoff. They can be expressed by the following surface integral in spherical coordinates
\begin{eqnarray}
\mathbf{F}_{\text{Bound},k}=-V_c \rho_k(R_c) R_c \int \frac{\rho(R_c,\vartheta,\varphi)G(R_c,\vartheta,\varphi)}{\sum \rho_i(R_c,\vartheta,\varphi)}R_c
\left(
\begin{array}{c}
\sin\vartheta\cos\varphi\\
\sin\vartheta\sin\varphi\\
\cos\vartheta
\end{array}
\right)\sin\vartheta \;d\vartheta d\varphi
\end{eqnarray}
As we do have a Cartesian grid we need to perform the integral as an integral over a thin shell in
real space. $R_c$ times the vector is just the position vector of a point on the surface $(x,y,z)$
and the surface element can be expressed in Cartesian coordinates as \begin{equation}
\sin\vartheta\;d\vartheta d\varphi = \sgn(z) \frac{y\;dx dz-x\;dy dz}{R_c(R_c^2-z^2)}
\end{equation}

\subsubsection{Transition Matrix Element calculation}
The following technical description of CDFT matrix element calculations can also be found, together with some test calculation and an investigation of the influence of the choice of the weight function on the results, in ref.~\cite{Oberhofer10acie}.

In order to calculate the electronic transition matrix element we first need to calculate two constrained wave functions $\phi_A$ and $\phi_B$ and write down the Hamiltonian in the non-orthogonal constrained basis spanned by these two states. 

\paragraph{Hamiltonian in the non-orthogonal constrained basis}
\begin{equation}
\mathbb{H}_\text{no}=\left(\begin{array}{c c} H_{BB} & H_{BA}\\ H_{AB} & H_{AA}\end{array}\right),
\end{equation}
where $A$ and $B$ denote the two states, respectively. Here the diagonal elements are just given by the respective energies of donor and acceptor due to:
\begin{eqnarray}
H_{BB}&=&\langle \phi_B \vert H \vert \phi_B \rangle=E_B \\
H_{AA}&=&\langle \phi_A \vert H \vert \phi_A \rangle=E_A,
\end{eqnarray}
while the off-diagonal elements are determined by:
\begin{eqnarray}
H_{BA}&=&F_B \langle \phi_B \vert \phi_A \rangle -V_c^B \langle \phi_B \vert W\vert \phi_A \rangle\\
&=& F_B S_{BA} -V_c^B W_{BA}\\
H_{AB}&=&F_B S_{AB} -V_c^A W_{AB}.
\end{eqnarray}
Here we also introduced the overlap matrix element $S_{AB}$ and the weight matrix element $W_{AB}$. From now on we will only consider one of the off-diagonal elements as the calculation of the other one is analogous.
Approximating the wave functions of the two states by slater determinants built from Kohn-Sham orbitals we can write in a plane wave basis:
\begin{eqnarray}
S_{AB}=\langle \phi_B \vert \phi_A \rangle= \det \left[\sum_G (c_i^A(G))^\star c_j^B(G)\right]=\det \Phi_{ij},
\end{eqnarray}
where the $c_{i,j}^{A,B}(G)$ denote the plane wave coefficients of the respective wave function for
their respective electronic states $i$ and $j$ and $\Phi$ denotes the full overlap matrix. Note that
the diagonal elements of $\mathbb{S}$ are equal to one due to the normalisation of the two states.
The off-diagonal $\mathbb{W}$ matrix element is given by:
\begin{equation}
\label{eq::wab}
W_{AB}=\langle \phi_B \vert W \vert \phi_A \rangle=N\sum_{i=1}^N \sum_{j=1}^N \langle \phi^i_B \vert W \vert \phi^j_A \rangle  (-1)^{i+j} \det \Phi_{(i,j)},
\end{equation}
with $\Phi_{(i,j)}$ being the $i,j$th minor of $\Phi$ and the integral given by
\begin{equation}
\label{eq::gspacetransme}
\langle \phi^i_B \vert W \vert \phi^j_A \rangle=\sum_{G,G^\prime}(c_i^A(G))^\star c_j^B(G) \tilde{W}(G-G^\prime).
\end{equation}
Note that Equ.~\ref{eq::gspacetransme} actually constitutes a convolution in G-space. Therefore the
most efficient way to calculate the elements $\langle \phi^i_B \vert W \vert \phi^j_A \rangle$ is by
transforming Equ.~\ref{eq::gspacetransme} back into real-space where the convolution becomes a
simple multiplication. Additionally CPMD already stores the wavefunctions in real-space (REAL SPACE
WFN KEEP which is activated automatically by the CDFT HDA calculation) and thus saves us the
necessary FFTs.  Calculating every minor determinant of the overlap matrix in Equ.~(\ref{eq::wab})
would be rather costly, therefore we note that the second part of Equ.~(\ref{eq::wab}) is actually
the cofactor matrix $C$ of $\Phi$.
\begin{equation}
(-1)^{i+j} \det \Phi_{(i,j)}=C_{ij},
\end{equation}
which can be calculated using the Laplace Expansion of the inverse of $\Phi$:
\begin{equation}
C^T=\Phi^{-1}. [\det(\Phi) I ]
\end{equation}
Thus we only have to calculate the inverse and the determinant of $\Phi$ instead of all the minors.
The diagonal elements of $\mathbb{W}$ are just the constraint values $N_c$ for the two states.
Having performed this calculation we symmetrise $\mathbb{H}_\text{no}$ in order to correct for inaccuracies of the approximations we made.

\paragraph{Full diabatic non-orthogonal Hamiltonian}
However, the matrix $\mathbb{H}_\text{no}$ is not the full Hamiltonian as it neglects the fact that
the two diabatic wavefunctions form a non-orthogonal basis ($S_{AB}\neq 0$). Nevertheless, we can
construct the full Hamiltonian using the overlap matrix element $S_{AB}$: \begin{equation}
\mathbb{H}_\text{full}=\frac{1}{1-S_{AB}^2}\left(\begin{array}{c c} H_{BB}-S_{AB}H_{AB} & H_{BA}-S_{AB}H_{AA}\\ H_{AB}-S_{AB}H_{BB} & H_{AA}-S_{AB}H_{BA}\end{array}\right),
\end{equation}

\paragraph{Diabatic orthogonal Hamiltonian}
In order to compare our matrix elements with other methods an experiments we follow the procedure of Wu and Van Voorhis and first solve the two-dimensional generalised eigenvalue problem
\begin{equation}
\mathbb{W} \mathbb{V}=\mathbb{S}\mathbb{V}\mathbb{L},
\end{equation}
where $\mathbb{V}$ is the matrix of generalised eigenstates and $\mathbb{L}$ is the diagonal matrix of generalised eigenvalues.
To calculate the diabatic orthogonal Hamiltonian we then perform the similarity transformation
\begin{equation}
\mathbb{H}_\text{diab}=\mathbb{V}^{-1} \mathbb{H}_\text{full} \mathbb{V}.
\end{equation}
The off-diagonal elements of $\mathbb{H}_\text{diab}$ --or their average if the two states have been very different-- are then the desired transition matrix element.

\paragraph{Adiabatic Hamiltonian}
The adiabatic energies of ground and excited state $\varepsilon_{g,e}$ and their mixing coefficients
$\mathbf{x}_{g,e}$ can be calculated by simply diagonalising the diabatic Hamiltonian (in
principle either one as a similarity transform does not change the eigenvalues of a matrix, here we
use the full Hamiltonian) \begin{equation}
 \mathbb{H}_\text{full}\mathbf{x}_{g,e}=\varepsilon_{g,e}\mathbf{x}_{g,e}
\end{equation}
With the matrix $\mathbb{X}$ of mixing coefficients we could then in principle also calculate the adiabatic states themselves via
\begin{equation}
\label{eq::adiabS}
\left( \begin{array}{c} \psi^\text{ad}_a\\ \psi^\text{ad}_b\end{array}\right)=\mathbb{X} \left( \begin{array}{c} \psi_A\\ \psi_B\end{array}\right)
\end{equation}


\paragraph{Projection on reference states}
Unfortunately, CDFT only constrains a total charge density, thus sometimes one may get a spurious
delocalization of the orbitals over both donor and acceptor leading to a too large overlap $S_{AB}$
of the two diabatic states. One possible way to overcome this is to project the wavefunctions on
non-interacting reference states and transform the diabatic Hamiltonian into this basis.

The reference states are constructed from four wavefunctions $\Phi_{D^-},\Phi_{D^+},$ $\Phi_{A^+},\Phi_{A^-}$ describing donor and acceptor in the charge states corresponding to the two diabatic states
\begin{eqnarray}
\psi_a^0 &=& \Phi_{D^-}\times \Phi_{A^+} \nonumber\\
\psi_b^0 &= &\Phi_{D^+}\times \Phi_{A^-}.
\end{eqnarray}
Then we project the adiabatic states - Equ.~\ref{eq::adiabS} - onto the reference states
\begin{equation}
\label{eq::projection}
\left( \begin{array}{c} \psi^\prime_a\\ \psi^\prime_b\end{array}\right)=\left(
\begin{array}{c c}
 \langle \psi^\text{ad}_a \vert \psi^0_a\rangle & \langle \psi^\text{ad}_b \vert \psi^0_a\rangle\\
 \langle \psi^\text{ad}_a \vert \psi^0_b\rangle & \langle \psi^\text{ad}_b \vert \psi^0_b\rangle
\end{array} \right) \left( \begin{array}{c} \psi^\text{ad}_a\\ \psi^\text{ad}_b\end{array}\right) = 
\mathbb{K} \left( \begin{array}{c} \psi^\text{ad}_a\\ \psi^\text{ad}_b\end{array}\right).
\end{equation}
These, so called ``dressed'', states are not yet orthogonal which we rectify using the L\"owdin scheme
\begin{equation}
\left( \begin{array}{c} \psi_1\\ \psi_2\end{array}\right)=\mathbb{S}^{-1/2} \left( \begin{array}{c} \psi^\prime_a\\ \psi^\prime_b\end{array}\right),
\end{equation}
with the elements of the L\"owdin matrix $\mathbb{S}$ defined as $\mathbb{S}_{j,k}=\langle
\psi_j^\prime \vert \psi_k^\prime \rangle$ where $j,k \in {a,b}$. Combining this with
Equ.~\ref{eq::projection} we can then express the new projected diabatic states in terms of the
adiabatic constrained states: \begin{equation}
\left( \begin{array}{c} \psi_1\\ \psi_2\end{array}\right)=\mathbb{S}^{-1/2} \mathbb{K} \left( \begin{array}{c} \psi^\text{ad}_a\\ \psi^\text{ad}_b\end{array}\right).
\end{equation}
The similarity transformation from the adiabatic Hamiltonian matrix $\mathbb{H}_\text{ad}$ to the projected diabatic Hamiltonian $\mathbb{H}^\text{p}_\text{diab}$ is then simply
\begin{equation}
\mathbb{H}^\text{p}_\text{diab}= (\mathbb{S}^{-1/2} \mathbb{K}) \mathbb{H}_\text{ad} (\mathbb{S}^{-1/2} \mathbb{K})^\text{T}.
\end{equation}

\subsection{Fragment Orbital DFT (FO-DFT)}
The basic idea of FO-DFT, as formulated by Senthilkumar, Grozema, Bickelhaupt, and Siebbeles
\cite{Senthilkumar03}, is to construct the two reference states from calculations of the
non-interacting isolated donor and acceptor groups, respectively. This way one avoids spurious
delocalization of electrons over both groups, but also neglects polarisation effects of the donor on
the acceptor and vice versa. Thus the diabatic states - and therefore the matrix element $H_{AB}$ -
will in general be slightly different from those calculated with CDFT.

With these diabatic reference states the transition matrix element is then in principle given by:
\begin{equation}
H_{AB}= \left\langle \psi^D_\text{HOMO}\left\vert \mathcal{H}_\text{KS} \right\vert\psi^A_\text{HOMO}\right\rangle,
\end{equation}
where $\mathcal{H}_\text{KS}$ denotes the Kohn-Sham Hamiltonian and $\psi^{D,A}_\text{HOMO}$ the
highest occupied molecular orbitals (HOMO's) of donor and acceptor, respectively. Here we make the
assumption that we can - to a sufficient degree of accuracy - approximate the acceptors LUMO (lowest
unoccupied molecular orbital) by its HOMO in a system with one more electron. Successful
applications of our implementation of FO-DFT can be found in ref.\cite{Oberhofer10acie}.

\subsubsection{FODFT with CPMD}
An FO-DFT calculation in CPMD consists of six different calls to CPMD:
\begin{description}
\item[1 and 2:] the respective wavefunction optimisation of donor and acceptor, on positions as they
would be if they were combined (\textbf{CENTER MOLECULE OFF}). Note that here the calculated
acceptor state should be the desired acceptor state plus one electron, which is necessary because
CPMD does not calculate Kohn-Sham matrix elements for unoccupied orbitals. These two calls are the
only time consuming ones in the procedure.  \item[3 and 4:] the diagonalization of the two states in
order to get their respective Kohn-Sham Orbitals (KOHN-SHAM ENERGIES).
\item[5:] the combination and orthogonalisation of both wavefunctions with keywords COMBINE WAVEFUNCTION and one of the orthogonalisation schemes (ORTHOGONALIZATION LOWDIN of GRAM-SCHMIDT).
\item[6:] the calculation of the Kohn-Sham matrix for the combined and orthogonalised states (keyword KSHAM).
\end{description}


If both donor and acceptor atoms are set the constraint value $N_\text{c}$ denotes as usual the charge difference between donor and acceptor. If NDON$=0$, $N_\text{c}$ is the desired charge of the acceptor.

\paragraph{TODO's and usage WARNINGS}
\begin{itemize}
\item WARNING: don't use diagonalization schemes (e.g. Lanczos) during the MD because it breaks the force calculation.
\item WARNING: don't use non-orthogonal supercells, only SYMMETRY 0, 1 or 8 should be used.
\item WARNING: CDFT and FODFT are not implemented for QM/MM calculations, yet.
\end{itemize}

%%%%%%%%%%%%%%%%%%%%%%%%%%%%%%%%%%%%%%%%%%%%%%%%%%%%%%%%%%%%%%%%%%%%%%%%
% QM/MM documentation %%%%%%%%%%%%%%%%%%%%%%%%%%%%%%%%%%%%%%%%%%%%%%%%%%
%%%%%%%%%%%%%%%%%%%%%%%%%%%%%%%%%%%%%%%%%%%%%%%%%%%%%%%%%%%%%%%%%%%%%%%%
\subsection{CPMD/Gromos QM/MM Calculations}\label{sec:qmmm}

\subsubsection{General Overview}
\label{sec:qmmm-overview}
An additional interface code ({\bf MM\_Interface} folder) and an adapted
classical force field code  \cite{gromos96} ({\bf Gromos} folder) are
needed to run CPMD in fully Hamiltonian hybrid QM/MM
mode\cite{qmmm02}. To use this code a {\it Gromos license} is required
and therefore it is {\bf not} included in the standard CPMD code. The
interface code and the adapted classical force field code can be
obtained by directly contacting the CPMD developers.  

To create a makefile for compilation of a QM/MM enabled CPMD binary,
you have to copy  the two above folders (or create symbolic links to
them) in the CPMD source directory and then to add the \texttt{-qmmm}
flag when executing the \texttt{config.sh} script (see section
\ref{installation}).  The resulting binary can be used for normal CPMD 
runs as well as for QM/MM simulations.
% FIXME: add contact and web reference here (again).
%

\subsubsection{Input files for QM/MM CPMD}
\label{sec:qmmm-input}
A QM/MM run requires a modified CPMD input file, some additional
input files, and creates the normal CPMD output file and
some new ones. The input file consists of a standard CPMD input
with with the \refkeyword{QMMM} keyword in the \&CPMD section, a
modified \&ATOMS section and a mandatory \&QMMM section. Furthermore
three files for the classical code are needed (coordinates, topology and
input file). These can be taken from previous fully classical
simulations and have to be in Gromos format. Topologies and
coordinates files created with the Amber\cite{amber7} package are
also supported. A converter to Gromos format\cite{gromos96} is available.

\subsubsection{Starting a QM/MM run}
\label{sec:qmmm-start}
To start a QM/MM simulation, you first do a simulation of your
system with a regular classical MD-code to get an equilibrated
configuration. The tricky part in this is usually the treatment
of (metal-)ion or special molecules, that are not parameterized
withing a given force field but are in the active center of your
molecule (one of the prominent reasons why you want to do a QM/MM
run in the first place). It is usually easiest to keep that part
rigid throughout the equilibration, until after you have defined
the QM-subsystem.

Starting from the classically equilibrated structure, you have to
create a topology, a coordinate and an input file in Gromos format
(either by using the Gromos tools or a converter). Now you need to
define your QM system by assigning pseudopotentials to selected
atoms in your CPMD input file (see \ref{sec:defining-qm-system}).

You can now start to continue the classical equilibration with CPMD
using \refkeyword{MOLECULAR DYNAMICS} CLASSICAL. Please note, that
there are several special constraints available to ease the transition
in case of strong interactions within the QM part or between the QM
and the MM part. Finally, a wavefunction optimization
(either directly or via \refkeyword{QUENCH} BO) and a normal
\refkeyword{MOLECULAR DYNAMICS} CP or BO can be performed.
%FIXME: some more/comments need to be added here.
% CAVEATS: - must use AMBER keyword if using converted amber topology.
%          - must not have isolated positively charged MM-group in
%            QM-box -> spill-out
%          - must use rigid water when equilibrated with rigid water
%            -> amber2gromos (>= v0.5) will detect rigid water if name WAT.
%            -> waters for QM can be converted to solute via
%               the FLEXIBLE WATER keyword (use matching BONDTYPE for amber).
%            -> for all flexible water run already equilibration with
%               flexible water and rename residues to something else as WAT
%            -> when using older amber2gromos must convert manually
%               (not recommended).

\subsubsection{Defining internal Gromos array dimensions}
\label{sec:defining-arraysizes}
One rather new feature of this QM/MM interface code is the
\refkeyword{ARRAYSIZES ... END ARRAYSIZES} block in the \&QMMM section which
allows to change the internal array dimensions of the Gromos
part dynamically. Previously one had to change some include
files and recompile everything to adapt the code for a different
system.

These settings have to be consistent during a series of calculations,
or else you may not be able to read your restart files correctly.
% FIXME: add some text about the NAT/NSX issue
% comment about what happens when changing MAXNAT
% i.e. what the warnings mean and what it will lead to.
% comment about the need to keep MAXATT low for large systems
%
%
%   QUANTUM SYSTEM:
%   NAX:           6
%   NSX:           3
%   FULL SYSTEM:
%   NAX:          20
%   NSX:         120
%
%

\subsubsection{Defining the QM system}
\label{sec:defining-qm-system}

For a QM/MM calculation a subset of atoms are selected from the
classical restart and then for this QM part an isolated system
(SYMMETRY 0) calculation is performed. The supercell size has
to follow the requirements of the various Poisson solvers, as
listed in the hints section (\ref{hints:symm0}).

If not otherwise specified, the QM system (atoms and wavefunction)
is always re-centered in the given supercell (the current offset of
the QM cell is recorded in the file MM\_CELL\_TRANS).

The quantum atoms are specified in the \&ATOMS section similar to normal
CPMD calculations. Instead of explicit coordinates one has to provide
the atom index as given in the Gromos topology and coordinates files.


\subsubsection{List of keywords in the \&QMMM section}

\textbf{\underline{Mandatory keywords:}}\\[-1cm]

\keyword{COORDINATES}{}{}{}{\&QMMM}
  \desc{On the next line the name of a Gromos96 format coordinate file
  has to be given. Note, that this file must match the corresponding
  input and topology files. Note, that in case of hydrogen capping, this
  file has to be modified to also contain the respective dummy hydrogen atoms.
}

\keyword{INPUT}{}{}{}{\&QMMM}
\desc{On the next line the name of a Gromos input file has to be
  given. A short summary of the input file syntax and some keywords
  are in section \ref{sec:qmmm-gromos-inp}.
  Note, that it has to be a correct input file, even though
  many options do not apply for QM/MM runs.}

\keyword{TOPOLOGY}{}{}{}{\&QMMM}
\desc{On the next line the name of a Gromos topology
  file has to be given. Regardless of the force field,
  this topology file has to be in Gromos format\cite{gromos96}.
  Topologies created with Amber % or Gromacs (Gromos/OPLS-forcefield)
  can be converted using the respective conversion tools shipped
  with the interface code.
  A short summary of the topology file syntax and some keywords
  are in section \ref{sec:qmmm-gromos-inp}.}

\vspace{1cm}\textbf{\underline{Other keywords:}}\\[-1cm]

\spekeyword{ADD\_HYDROGEN}{}{}{}{\&QMMM}{ADD-HYDROGEN}
\desc{ This keyword is used to add hydrogens to the QM system if a
  united atom topology is used (like in Gromos). On the next line
  the number of atoms to be ``hydrogenized'' has to be given and in
  the line following that, the corresponding gromos atom numbers.
  A number of hydrogens consistent with the hybridization of the
  ``hydrogenized'' carbons are added.}

\keyword{AMBER}{}{}{}{\&QMMM}
\desc{An Amber functional form for the classical force field is
  used. In this case coordinates and topology files as obtained by
  Amber have to be converted in Gromos format just for input/read
  consistency. This is done with the tool amber2gromos availabe with
  the CPMD/QMMM package.\\
  This keyword is mutually exclusive with the \refkeyword{GROMOS}
  keyword (which is used by default).}

\keyword{ARRAYSIZES ... END ARRAYSIZES}{}{}{}{\&QMMM}
\desc{Parameters for the dimensions of various internal arrays
  can be given in this block. The syntax is one label and the
  according dimension per line. The suitable parameters can
  be estimated using the script \texttt{estimate\_gomos\_size}
  bundled with the QM/MM-code distribution. Example:}
\begin{verbatim}
 ARRAYSIZES
   MAXATT 20
   MAXAA2 17
   MXEX14 373
 END ARRAYSIZES
\end{verbatim}

\keyword{BOX TOLERANCE}{}{}{}{\&QMMM}
\desc{The value for the box tolerance is read from the next line.
In a QM/MM calculation the size of the QM-box is fixed and
the QM-atoms must not come to close to the walls of this box. On top of
always recentering the QM-box around the center of the distribution of
the atoms, CPMD prints a warning message to the output when the
distribution extends too much to fit into the QM-box properly anymore.
This value may need to be adjusted to the requirements of the Poisson
solver used (see section \ref{hints:symm0}).\\
{\bf Default} value is 8~a.u.}


\keyword{BOX WALLS}{}{}{}{\&QMMM}
\desc{
  The thickness parameter for soft, reflecting QM-box walls
  is read from the next line. This keyword allows to reverse the
  momentum of the particles (${\bf p}_I \rightarrow -{\bf p}_I$)
  when they reach the walls of the simulation supercell similar to
  the full quantum case, but acting along all the three directions
  $x,y,z$.
  In the case this keyword is used in the \&QMMM section,QM  particles
  are reflected back in the QM box. Contrary to the normal procedure of
  re-centering the QM-box, a soft, reflecting confinement potential
  is applied if atoms come too close to the border of the QM
  box~\cite{box-walls}.
  It is highly recommended to also use \refkeyword{SUBTRACT} COMVEL
  in combination with this feature. {\bf NOTE:} to have your QM-box
  properly centered, it is best to run a short MD with this feature
  turned off and then start from the resulting restart with the soft
  walls turned on. Since the reflecting walls reverse the sign of
  the velocities, ${\bf p}_I \to -{\bf p}_I$ ($I$ = QM atom
  index), be aware that this options affects the momentum conservation
  in your QM subsystem. \\
  This feature is {\bf disabled by default}}

\keyword{CAPPING}{}{}{}{\&QMMM}
\desc{Add (dummy) hydrogen atoms to the QM-system to saturate
  dangling bonds when cutting between MM- and QM-system. This needs
  a special pseudopotential entry in the \&ATOMS section (see section
  \ref{sec:qmmm-cut-bonds} for more details).}

\spekeyword{CAP\_HYDROGEN}{}{}{}{\&QMMM}{CAP-HYDROGEN}
\desc{same as \refkeyword{CAPPING}.}
%
%\keyword{CHARGE...}{}{}{}{\&QMMM}
%\desc{
% FIXME: this sets some charge groups for the QM atom, but for what?
%          ELSE IF(INDEX(LINE,'CHARGE').NE.0) THEN
%            read(iunit,*)N_CG
%            read(iunit,*)(atom_qm_cg(i),i=1,N_CG)
%            read(iunit,*)q_rest,lambda
%}

\keyword{ELECTROSTATIC COUPLING}{[LONG RANGE]}{}{}{\&QMMM}
\desc{The electrostatic interaction of the quantum system with the
  classical system is explicitly kept into account for all classical
  atoms  at a  distance $r \leq $~\refspekeyword{RCUT\_NN}{RCUT-NN} from any
  quantum atom and for all the MM  atoms at a distance of
  \refspekeyword{RCUT\_NN}{RCUT-NN}~$< r \leq$~\refspekeyword{RCUT\_MIX}{RCUT-MIX}
  and a charge larger than $0.1 e_0$ (NN atoms).\\

  MM-atoms with a charge smaller than $0.1 e_0$ and a distance of
  \refspekeyword{RCUT\_NN}{RCUT-NN}~$< r \leq$~\refspekeyword{RCUT\_MIX}{RCUT-MIX}
  and all MM-atoms with
  \refspekeyword{RCUT\_MIX}{RCUT-MIX}~$< r \leq$~\refspekeyword{RCUT\_ESP}{RCUT-ESP}
  are coupled  to the QM system by a ESP coupling Hamiltonian (EC atoms).\\

  If the additional \texttt{LONG RANGE} keyword is specified, the
  interaction of the QM-system with the rest of the classical atoms is
  explicitly kept into account via interacting with a multipole
  expansion for the QM-system up to quadrupolar order. A file
  named \texttt{MULTIPOLE} is produced.

  If \texttt{LONG RANGE} is omitted the quantum system is coupled to the
  classical atoms not in the NN-area and in the EC-area list via the
  force-field charges.\\

  If the keyword \texttt{ELECTROSTATIC COUPLING} is omitted, all
  classical atoms are coupled to the quantum system by the force-field
  charges (mechanical coupling).\\

  The files INTERACTING.pdb, TRAJECTORY\_INTERACTING, MOVIE\_INTERACTING,
  TRAJ\_INT.dcd, and ESP (or some of them) are created. The list of NN and
  EC atoms is updated every 100 MD steps. This can be changed using the
  keyword \refkeyword{UPDATE LIST}.\\

  The default values for the cut-offs are
  RCUT\_NN=RCUT\_MIX=RCUT\_ESP=10 a.u..
  These values can be changed by the keywords
  \refspekeyword{RCUT\_NN}{RCUT-NN},
  \refspekeyword{RCUT\_MIX}{RCUT-MIX},
  and \refspekeyword{RCUT\_ESP}{RCUT-ESP}
  with $r_{nn} \leq r_{mix} \leq r_{esp}$.}

\keyword{ESPWEIGHT}{}{}{}{\&QMMM}
\desc{The ESP-charg fit weighting parameter is read from the next line.\\
   {\bf Default} value is $0.1 e_0$.}

\keyword{EXCLUSION}{\{GROMOS,LIST\{NORESP\}\}}{}{}{\&QMMM}
\desc{Specify charge interactions that should be excluded
   from the QM/MM Hamiltonian. With the additional flag GROMOS,
   the exclusions from the Gromos topology are used. With the
   additional flag LIST, an explicit list is read from following lines.
   The format of that list has the number of exclusions in the first
   line and then the exclusions listed in pairs of numbers of the QM
   atom and the MM atom in Gromos ordering; the optional flag NORESP
   in this case requests usage of MM point charges for the QM atoms 
   instead of the D-RESP charges (default).}


\keyword{FLEXIBLE WATER}{[ALL,BONDTYPE]}{}{}{\&QMMM}
\desc{Convert some solven water molecules into solute molecules and
  thus using a flexible potential.\\
  With the BONDTYPE flag, the three bond potentials (OH1, OH2, and H1H2)
  can be given as index in the BONDTYPE section of the Gromos topology
  file. Note that the {\bf non-bonded} parameters are taken from the
  SOLVENATOM section of the \refkeyword{TOPOLOGY} file.
  {\bf Default} is to use the values: 35, 35, 41.\\
  With the additional flag ALL this applies to all solvent water
  molecules, otherwise on the next line the number of flexible water
  molecules has to be given with the Gromos index numbers of their
  respective Oxygen atoms on the following line(s).\\
  On successful conversion a new, adapted topology file, MM\_TOPOLOGY,
  is written that has to be used with the \refkeyword{TOPOLOGY}
  keyword for subsequent restarts. Also the \refkeyword{INPUT} file
  has to be adapted: in the SYSTEM section the number of solvent
  molecules has to be reduced by the number of converted molecules,
  and in the SUBMOLECULES section the new solute atoms have to be
  added accordingly.\\ Example:}
\begin{verbatim}
     FLEXIBLE WATER BONDTYPE
      4 4 5
      26
        32   101   188   284   308   359   407   476   506   680
       764   779   926  1082  1175  1247  1337  1355  1607  1943
      1958  1985  2066  2111  2153  2273
\end{verbatim}

%_FM[
\keyword{FORCEMATCH ... END FORCEMATCH}{}{}{}{\&QMMM}
\desc{Input block for the QM/MM forcematching. A general description is given in section \ref{sec:forcematch-desc}. \\
   \textbf{READ REF FORCES [FILE,COVALENT]}\\
Flag to read the QM/MM reference forces directly from the file FM\_REF\_FORCES, i.e. no QM/MM SPs
are computed. \textbf{Default}: false. An alternative file name can be specified on the next line
with the option \textbf{FILE}. With the option \textbf{COVALENT} covalent forces are read from the
file FM\_REF\_COVFORCES. \\ \textbf{READ REF TRAJ [FILE]} \\
Read reference trajectory from file TRAJECTORY\_REF (or set the \textbf{FILE} option to read a non-default file name from the next line) with a given stride and compute single points on the respective frames. \\
\textbf{RESTART SP} \\
If in a previous force matching run not all of the SPs could be computed (e.g. limited wall time)
this flag indicates cpmd to restart the SP calculations. The FM\_REF* files from the previous run
have to be present and they will be appended. With this option make sure that the frames contained
in the already existing FM\_REF* files are consistent. \textbf{Default}: false. \\ \textbf{READ REF
STRIDE} \\
Stride to apply when reading the TRAJECTORY\_REF file is read from the next line. Default=1, i.e. every frame is used for the SP calculations. \\
\textbf{TOPOL OUT} \\
Filename for the final topology file. \textbf{Default}: FMATCH.top. \\
\textbf{INITWF [OFF]} \\
Generate an initial guess for the wfkt for the SP calculations based on AOs (default). With the \textbf{OFF} option the wfkt of the previous frame is used as an intial guess.\\
\textbf{CHARGES [ONLY,NO],[FIX]} \\
Charge fitting is on by default and can be switched off with the \textbf{NO} option. In this case
the charges from the initial topology will not be modified. \textbf{ONLY} will let the program stop
after the charge fitting and the other parameters are not updated. \\ With the \textbf{FIX} option
target values for the restraints in the charge fitting on specific atoms can be specified by the
user. Usually the charges are restraint to the respective Hirshfeld values. On the next line the
number of charges to be fixed has to be given and then the corresponding number of lines with:
gromos index \ \ \ \ charge. \\
\textbf{MV} \\
Weight on the potential in the charge fitting. \textbf{Default}=0.1.\\
\textbf{WF} \\
Weight on the field in the charge fitting. \textbf{Default}=0.0. \\
\textbf{WQ INDIVIDUAL} \\
Weights on the charge restraints can be given individually here. From the next line the total number of individual weights is read. Then the lines with: gromos index \ \ \ \ weight.\\
\textbf{WQ GENERAL} \\
The weight for all the charge restraints that were not specified by individual weights can be given on the next line. \textbf{Default}=0.1. \\
\textbf{WTOT} \\
Weight of the total charge contribution in the charge fitting. \textbf{Default}=1.0E7.}
% I have to split the desc to avoid problems with the page break!
\desc{
\textbf{EQUIV} \\
Specify equivalent atoms. Syntax: \\

\texttt{EQUIV} \\
\texttt{  n\_equiv \\
atom1 \ \ atom4 \\
atom1 \ \  atom3 \\
... \\
atom5 \ \ atom7} \\

There are \texttt{\textit{n\_equiv}} equivalencies specified (\texttt{n\_equiv} lines are read from the input). For each pair of equivalencies the gromos indexes have to be specified on one separate line. The lower index has to be given first!\\
If an atom is equivalent to more then one other atom. E.g. atom1,atom3 and atom4 are equivalent. Then this has to be encoded by: \\

\texttt{atom1 \ \ atom3 \\
atom1 \ \  atom4} \\

and not by: \\

\texttt{atom1 \ \ atom4 \\
atom3 \ \ atom4 }  \\

where atom1 has a lower gromos index then atom3 and atom3 has a lower one then atom4. Per default no equivalencies are assumed. \\
\textbf{OPT FC ONLY}\\
Serves as a flag to remove the equilibrium values of the bonded interactions from the list of fitted parameters. I. e. only force constants are fitted for the bonded interactions.\\
\textbf{NO BONDS} \\
Do not fit bonds. \textbf{Default}=false. \\
\textbf{NO ANGLES} \\
Do not fit angles. \textbf{Default}=false. \\
\textbf{NO DIHEDRALS} \\
Do not fit dihedrals. \textbf{Default}=false. \\
\textbf{NO IMPROPERS} \\
Do not fit improper dihedrals. \textbf{Default}=false. \\
\textbf{MAXITER} \\
Give on the next line the maximal number of iterations for the non-linear fitting procedure of the bonded interactions. \textbf{Default}=500.\\
\textbf{COMPUTE RMS [NO]}\\
Per default the RMS on the forces is computed after the fitting has been completed. Switch it off with the \textbf{NO} option.  
   Example:}
\begin{verbatim}
 FORCEMATCH
   READ REF TRAJ FILE
     TRAJECTORY_REF
   READ REF STRIDE
     10
   WV
   1.0
   WOT
   1000000.0
   WQ GENERAL
   0.1    
 END FORCEMATCH
\end{verbatim}
%_FM]

\keyword{GROMOS}{}{}{}{\&QMMM}
\desc{A Gromos functional form for the classical force field is used
  (this is the default).\\
  This keyword is mutually exclusive with the \refkeyword{AMBER}
  keyword.}

\keyword{HIRSHFELD}{[ON,OFF]}{}{}{\&QMMM}
\desc{With this option, restraints to Hirshfeld charges~\cite{Hirshfeld77}
  can be turned on or off\\
   {\bf Default} value is ON.}

\keyword{MAXNN}{}{}{}{\&QMMM}
\desc{Then maximum number of NN atoms, i.e. the number of atoms
  coupled to the QM system via \refkeyword{ELECTROSTATIC COUPLING}
  is read from the next line. (Note: This keyword was renamed from MAXNAT
  in older versions of the QM/MM interface code to avoid confusion
  with the MAXNAT keyword in the \refkeyword{ARRAYSIZES ... END ARRAYSIZES}
  block.)\\
   {\bf Default} value is 5000.}

\keyword{NOSPLIT}{}{}{}{\&QMMM}
\desc{If the program is run on more than one node,
 the MM forces calculation is performed on all nodes.
 Since the MM part is not parallelized, this is mostly useful for systems
 with a small MM-part and for runs using only very few nodes. Usually
 the QM part of the calculation needs the bulk of the cpu-time in the QM/MM.\\
  This setting is the default. See also under \refkeyword{SPLIT}.}

\spekeyword{RCUT\_NN}{}{}{}{\&QMMM}{RCUT-NN}
\desc{The cutoff distance for atoms in the nearest neighbor region
  from the QM-system ($ r \leq r_{nn}$) is read from the next line.
  (see \refkeyword{ELECTROSTATIC COUPLING} for more details).\\
  {\bf Default} value is 10~a.u.}

\spekeyword{RCUT\_MIX}{}{}{}{\&QMMM}{RCUT-MIX}
\desc{The cutoff distance for atoms in the intermediate region
  ($r_{nn} < r \leq r_{mix}$) is read from the next line.
  (see \refkeyword{ELECTROSTATIC COUPLING} for more details).\\
  {\bf Default} value is 10~a.u.}

\spekeyword{RCUT\_ESP}{}{}{}{\&QMMM}{RCUT-ESP}
\desc{The cutoff distance for atoms in the ESP-area ($r_{mix} < r \leq r_{esp}$)
  is read from the next line. (see \refkeyword{ELECTROSTATIC COUPLING}
  for more details).\\
  {\bf Default} value is 10~a.u.}

\keyword{RESTART TRAJECTORY}{[FRAME \{num\},FILE '\{fname\}',REVERSE]}{}{}{\&QMMM}
\desc{Restart the MD with coordinates and velocities from a previous
  run. With the additional flag FRAME followed by the frame number the
  trajectory frame can be selected. With the flag FILE followed by
  the name of the trajectory file, the filename can be set (Default is
  TRAJECTORY). Finally the flag REVERSE will reverse the sign of the
  velocities, so the system will move backwards from the selected point
  in the trajectory.}

\keyword{SAMPLE INTERACTING}{[OFF,DCD]}{}{}{\&QMMM}
\desc{The sampling rate for writing a trajectory of the interacting subsystem
  is read from the next line. With the additional keyword OFF or a
  sampling rate of 0, those trajectories are not written.
  The coordinates of the atoms atoms contained in the file INTERACTING.pdb
  are written, in the same order, on the file TRAJECTORY\_INTERACTING
  every.  If the \refkeyword{MOVIE} output is turned on, a file
  MOVIE\_INTERACTING is written as well.  With the
  additional keyword DCD the file TRAJ\_INT.dcd is also written to.
  if the sampling rate is negative, then \textbf{only} the TRAJ\_INT.dcd
  is written.
\\
  {\bf Default} value is 5 for MD calculations and OFF for others.}

\keyword{SPLIT}{}{}{}{\&QMMM}
\desc{If the program is run on more than one node,
  the MM forces calculation is performed on a separate node.
  This is mostly useful for systems with a large MM-part and runs with
  many nodes where the accumulated time used for the classical part has
  a larger impact on the performance than losing one node for the (in
  total) much more time consuming QM-part.\\
 {\bf Default} is \refkeyword{NOSPLIT}.}

\keyword{TIMINGS}{}{}{}{\&QMMM}
\desc{Display timing information about the various parts of the
 QM/MM interface code in the output file. Also a file \texttt{TIMINGS}
 with even more details is written. This option is off by {\bf default}.}

\keyword{UPDATE LIST}{}{}{}{\&QMMM}
\desc{On the next line the number of MD steps between updates of the
  various lists of atoms for \refkeyword{ELECTROSTATIC COUPLING}
  is given. At every list update a file INTERACTING\_NEW.pdb is
  created (and overwritten).\\

  {\bf Default} value is 100.}

\keyword{VERBOSE}{}{}{}{\&QMMM}
\desc{The progress of the QM/MM simulation is reported more verbosely
  in the output. This option is off by {\bf default}.}

\keyword{WRITE LOCALTEMP}{[STEP \{nfi\_lt\}]}{}{}{\&QMMM}
\desc{The Temperatures of the QM subsystem, the MM solute (without
  the QM atoms) and the solvent (if present) are calculated
  separately and written to the standard output and a file \texttt{QM\_TEMP}.
  The file has 5 columns containing the QM temperature, the MM temperature,
  the solvent temperature (or 0.0 if the solvent is part of the solute),
  and the total temperature in that order.
  With the optional parameters STEP followed by an integer, this is
  done only every \texttt{nfi\_lt} timesteps.}
%
%%%%%%%%%%%%%%%%5
% FIXME: AK 2005/07/18 this should be made obsolete and
                % the functionaliy combined with RHOPRI (if needed).
%WRITE {DENSITY,POTENTIAL} [STRIDE n_stride] [STEP n_step]
%
% The electronic density (resp. the external potential) on the quantum
% grid at step n_step is written in cube format on the file DENSITY.cube
% (resp. POTENTIAL.cube) with a stride n_stride on the real space
% grid. Defaults are n_step=1 n_stride=1. This keyword can be used only
% if the keyword ELECTROSTATIC COUPLING is specified.

\subsubsection{Keywords in the Gromos Input and Topology files}
\label{sec:qmmm-gromos-inp}
For a detailed description of the Gromos file formats please
have a look at the Gromos documentation\cite{gromos96}. Note, that
not all keyword are actually active in QM/MM simulations, but the
files still have to be syntactically correct.
Both, the input and the topology file are structured in sections
starting with a keyword in the first column and ending with the
keyword END. Lines starting with a pound sign '\#' may contain
comments and are ignored. Both files are required to have a
TITLE section as the first section. The rest can be in almost
any order. Here is a short list of some important flags and
their meaning.

\begin{description}
\item[Gromos Input File:]~\\
  \begin{description}
  \item[TITLE]~\\
    Text that identifies this input file.
    Will be copied into the CPMD output.
  \item[SYSTEM]~\\
    This section contains two integer numbers. The first
    is the number of (identical) solute molecules (NPM) and the
    second the number of (identical) solvent molecules (NSM).
  \item[BOUNDARY]~\\
    This section defines the classical simulation cell.
    It contains 6 numbers. The first (NTB) defines the type
    of boundary conditions ( NTB $<0$ means truncated octahedron boundary
    conditions, NTB=0 vacuum, and NTB $>0$ rectangular boundary
    conditions).\\
    The next three numbers (BOX(1..3)) define the size of the classical
    cell. The fifth number (BETA) is the angle between the x- and z-axes
    and the last number usually determines whether the cell dimensions
    are taken from the input file (NRDBOX=0) or from the BOX section
    of the \refkeyword{COORDINATES} file (NRDBOX=1), but is ignored
    for QM/MM simulations.\\
    Note: that even for vacuum simulations valid simulation cell
    sizes must be provided.
  \item[PRINT]~\\
    This section determines how often some properties are monitored.
    Here only the first number (NTPR) matters, as it determines the
    number of MD steps between printing the various energies to the
    CPMD output.\\
    Note many old Gromos input files created by the amber2gromos program
    default to NTPR=1, which makes the CPMD output huge.
%  \item[SHAKE]~\\
% does this work???
  \item[SUBMOLECULES]~\\
    Defines number of submolecules in the solute. The first number
    is the number of submolecules followed by the index number of
    the last atom of each submolecule. The last number must be identical
    to the number of atoms in the solute.
  \item[FORCE]~\\
    Contains two groups of numbers, that controls the various force
    component and the partitioning of the resulting energies. The first
    group of 1/0 flags turn the various force components on or off.
    The second group defines energy groups (the first number is the
    number of groups followed by the index number of the last atom
    in each group). The last number must be identical
    to the number of all atoms.
%  \item[LATSUM]~\\
% ewald sum parameters. are they needed here???
  \end{description}

\item[Gromos Topology File:]~\\
  \begin{description}
  \item[TITLE]~\\
    Text that identifies this topology file.
    Will be copied into the CPMD output.

  \item[ATOMTYPENAME]~\\
    This section contains the number of classical atom types (NRATT)
    followed by the respective labels, one per line. Note that
    the \refkeyword{ARRAYSIZES ... END ARRAYSIZES} MAXATT must be large enough
    to accomodate all defined atom types.
  \item[RESNAME]~\\
    This section contains the number of residues in the solute (NRAA2)
    followed by the respective residue names. % ARRAYSIZE?
  \item[SOLUTEATOM]~\\
    This section defines the number (NRP) and sequence of atoms in the
    solute, their names, residue numbers, non-nonded interaction codes,
    masses, charges, charge groups and their full + scaled 1-4 exclusions.
  \item[BONDTYPE]~\\
    This section contains the list of parameters for bonded
    interactions. You have to pick the two matching entries from this
    list for the O-H and H-H potential, when using the 
    \refkeyword{FLEXIBLE WATER} keyword to convert solvent water back
    into solute (e.g. to included them into the QM part).
  \item[SOLVENTATOM]~\\
    This section defines the number of atoms (NRAM) in the solvent
    and their respective names, non-bonded interactions types, masses,
    and charges.
  \item[SOLVENCONTSTR]~\\
    This section defines the number (NCONS) and parameters for the
    distance constraints, that are used to keep the solvent rigid.
  \end{description}
\item
\end{description}

\subsubsection{Files generated by the interface code}
\label{sec:qmmm-files}
\begin{itemize}

\item {\bf QMMM\_ORDER}\\
The first line specifies the total number of atoms (NAT) and the number of quantum
atoms (NATQ). The subsequent NAT lines contain, for every atom, the gromos atom number,
the internal CPMD atom number, the CP species number isp and the number in the list
of atoms for this species NA(isp). The quantum atoms are specified in the first NATQ lines.

\item {\bf CRD\_INI.grm}\\
Contains the positions of all atoms in the first frame of the simulation in
Gromos extended format.

\item {\bf CRD\_FIN.grm}\\
Contains the positions of all atoms in the last frame of the simulation in
Gromos extended format.

\item {\bf INTERACTING.pdb}\\
% FIXME: this is _not_ pdb format!
Contains (in pdb format) all the QM atoms and all the MM atoms in the electrostatic
coupling NN list. The 5-th column in this file specifies the gromos atom number as
defined in the topology file and in the coordinates file. The 10-th column specifies
the CPMD atom number as in the TRAJECTORY file. The quantum atoms are labelled by
the residue name QUA.

\item {\bf INTERACTING\_NEW.pdb}\\
The same as INTERACTING.pdb, but it is created if the file INTERACTING.pdb
is detected in the current working directory of the CPMD run.

\item {\bf TRAJECTORY\_INTERACTING}\\
Contains the coordinates and the velocities (in TRAJECTORY format) of the atoms
listed in INTERACTING.pdb. The format is the same as in the files TRAJECTORY and MOVIE,
hence frames belonging to different runs are separated by the line <<<<< NEW DATA >>>>>.

The atoms in this file do not necessarily coincide with the NN atoms, that are written
at every update of the pair list in the file INTERACTING\_NEW.pdb.


\item {\bf MOVIE\_INTERACTING}\\
The MOVIE-like file corresponding to TRAJECTORY\_INTERACTING.

%POTENTIAL.cube
%Created if the keyword WRITE POTENTIAL is specified. Contains, in cube format,
%thepotential due to the MM atoms on the quantum grid.
%
%DENSITY.cube
%Created if the keyword WRITE DENSITY is specified. Contains, in cube format,
%the electronic density.
%

\item {\bf ESP}\\
Contains the ESP charges of the QM atoms in CPMD order (the corresponding Gromos
numbers can be found in QMMM\_ORDER). The first column is the frame number.


\item {\bf EL\_ENERGY}\\
Contains the electrostatic interaction energy. First column: frame number.
Second column: total electrostatic interaction energy.
Other columns: interaction energy of the NN atoms with the QM system;
interaction energy with the ESP coupled atoms;
multipolar interaction energy;
electrostatic interacion energy evaluated using
the classical force field charges for the QM atoms.

\item {\bf MULTIPOLE}\\
Contains, for every frame (specified in the first column), the three components of
the dipol D(ix) and the five independent components of the quadrupole Q(ix,jx) of
the quantum system in a.u.  The order is: D(1),D(2),D(3),Q(1,1),Q(2,2),Q(1,2),Q(1,3),Q(2,3).

\item {\bf MM\_CELL\_TRANS}\\
Contains, the Trajectory of the re-centering offset for the QM-box. The first column
is the frame number (NFI) followed by the x-, y-, and z-component of the cell-shift vector.
%
%CHJ
%????
\end{itemize}

\subsubsection{Hydrogen Capping vs. Link Atoms}
\label{sec:qmmm-cut-bonds}
Whenever the QM/MM-boundary cuts through an existing bond, special
care has to be taken to make sure that the electronic structure of
the QM-subsystem is a good representation of an all-QM calculation
and also the structure in the boundary region is preserved. So far,
two methods methods are available to do this: using special link-atom
pseudopotentials and hydrogen capping.

{\large Link Atom}\\
The simplest way is to use a link-atom pseudopotential. In the
simplest case, this would be a scaled down pseudopotential with the
required valence change (e.g. ZV=1 when cutting through a single
carbon-carbon bond). However in this case it is required to constrain
the distance between the link atom and the (full-QM) neighbor atom
to the (full-QM) equilibrium distance, to preserve the electronic
structure in the center of the QM subsystem. You should be aware of
the fact, that this is a rather crude approximation and that the
constraint will create a small imbalance in the forces between the QM and
MM subsystems, that can result in a drift in the total energy, if the
length of the constraint is badly chosen.

A more rigorous approach would be to use an optimized pseudopotential
constructed with the method described in ref.~\cite{opt-ecp04}, that
should take care of the need for the constraint.

{\large Hydrogen Capping}\\
An alternative way would be to use the \refkeyword{CAPPING} flag in
order to introduce additional (dummy) hydrogen atoms to saturate the
dangling bonds. These capping hydrogen atoms have to be hidden from the
MM hamiltonian so the Gromos \refkeyword{INPUT} and
\refkeyword{TOPOLOGY} files have to be modified for subsequent runs, in
addition to adding an explicit \refkeyword{EXCLUSION} LIST to the cpmd
input. The whole procedure is a bit complicated so here is a short
protocol of the required steps.

\begin{itemize}
\item[1a] Set up a normal QM/MM run as for using link atoms with the
  additional keyword \refkeyword{CAPPING} and instead of the link-atom
  potential use a hydrogen potential with the additional flag
  \texttt{ADD\_H}. Note that you have to provide the correct number of
  hydrogens, but no atom index number.
\item[1b] Run a short MD (a couple of steps) and use the resulting
  \texttt{CRD\_FIN.grm} file (under a different name) in the
  \refkeyword{COORDINATES} section.
\item[2a] Modify the Gromos input file to match the new coordinate file.
  \begin{itemize}
  \item Increase the number of atoms per solute molecule in the
    SUBMOLECULES section.
  \item increase the total number of atoms in the FORCE section.
  \end{itemize}
\item[2b] Modify the Gromos topology file to match the new coordinate
  file.
  \begin{itemize}
  \item Add a DUM atom type at the end of the ATOMTYPENAME section if
    not already present (and increase NRATT accordingly).
  \item if you have added a new atom type, you have to add the
    corresponding entries in the LJPARAMETERS section as well.
    Since the capping hydrogen atoms should be invisible from the MM
    Hamiltonian all Lennard-Jones parameters are set to 0.0 for those
    new entries. This section is a triangular matrix, so you have to
    add NRATT lines (and increase NRATT2 accordingly).
  \item Add new residues named DUM (one for each capping hydrogen) to
    the RESNAME section (and increase NRAA2).
  \item In the SOLUTEATOM section you have to increase NRP and add the
     dummy hydrogens at the end of the solute.
     The structure of the entry is:
     \verb|<atom nr> <residue nr> <atom type name> <vdw type index> <mass> <charge> 1 0|
     and a single '0' on the next line. Use a mass of 1.008 and a charge of 0.000.
  \end{itemize}
\item [2c] Modify the CPMD input file.
  \begin{itemize}
  \item Make sure that the \refkeyword{TOPOLOGY}, \refkeyword{INPUT},
    and \refkeyword{COORDINATES} keywords in the \&QMMM section match
    the newly created or modified files.
  \item Add the capping hydrogens to the \&ATOMS section as normal QM
    atoms, but add the DUMMY flag to the pseudopotential line.
  \item Build an \refkeyword{EXCLUSION} LIST entry that lists for each
    capping atom the respective QM/MM atoms pairs that should be
    excluded from the electrostatic coupling (all other MM interactions
    are set to zero already in the topology file). For consistency only
    full charge groups should be excluded. In the supplementary material
    should be a script genexcl.tcl which can help you in building that
    list (it needs the modified Gromos coordinate and topology file as
    well as the QMMM\_ORDER file as input).
  \item Update the \refkeyword{ARRAYSIZES ... END ARRAYSIZES} entry to match
    the new topology.
  \end{itemize}
\end{itemize}

With the three modified files you should be able to run a regular QM/MM
run. Note, that you may have to update the exclusion list occasionally,
depending on your system and that you should pick the bond(s) to cut very
carefully.


\subsubsection{What type of QM/MM calculations are available?}
\label{sec:what-type-qmmm}
The QM/MM interface only supports a subset of the functionality of CPMD.
Please note, that although there are some tests and warnings included
into the source code, not every job that runs without a warning will
be automatically correct. So far, the interface code requires the use of
norm-conserving pseudopotentials. Tested and supported job types are:
\refkeyword{MOLECULAR DYNAMICS} (CLASSICAL, CP and BO),
\refkeyword{OPTIMIZE WAVEFUNCTION},
\refkeyword{KOHN-SHAM ENERGIES},
and \refkeyword{ELECTRONIC SPECTRA}.
Supported are closed shell systems as well as \refkeyword{LSD} and
\refkeyword{LOW SPIN EXCITATION} calculations.

\refkeyword{OPTIMIZE GEOMETRY} is experimental and currently
supports optimization of the QM atom positions only. Use of the
linear scaling geometry optimizer (\refkeyword{LBFGS}) is highly
recommended and the currently also the default.

\refkeyword{PROPERTIES} calculations with QM/MM are experimental.
Most properties (WF projection, population analysis, localization)
that only need the plain QM wavefunction work.

\refkeyword{LINEAR RESPONSE} calculations are currently at an
twofold experimental status. Both, the isolated system setup
(\refkeyword{SYMMETRY} 0) and the QM/MM coupling of the response
calculations itself are not yet fully tested.

Options that are known to be incompatible with \refkeyword{QMMM} are
\refkeyword{VIBRATIONAL ANALYSIS}, \refkeyword{PATH INTEGRAL}, and all
calculations that require a wavefunction optimization via a
diagonalization method at some point.

%_FM[
\subsubsection{QM/MM Force Matching}
\label{sec:forcematch-desc}
This tool allows the automated (re)parametrization of classical force fields from QM/MM reference
calculations via a force matching protocol as published in \cite{FM-maurer}. Thereby only MM
parameters among the atoms comprised in the QM subsystem are reparametrized. In this first release
VdW parameters are excluded from the optimization and kept constant. Fitting of these parameters
will be a feature of a future release. \\
The jobs requires a QM/MM reference TRAJECTORY file and the corresponding gromos topology, input and 
coordinate files and the cpmd input that were used to generate the reference TRAJECTORY file.
 The initial topology can be a reasonable guess and will be refined during the actual
 force matching procedure. Currently the QM/MM reference trajectory 
 has to be generated prior to the force matching job. \\
The actual forcematching job is invoked by the \refkeyword{FORCEMATCH} in \&CPMD and the
\refkeyword{FORCEMATCH ... END FORCEMATCH} block in \&QMMM. Besides, the \&CPMD and \&QMMM sections should contain sensible keywords and parameters for high quality reference forces (e.g. convergence orbitals 1.0d-7). \\
The parametrization protocol consists of a three-step process: First, the reference trajectory is
read with a given stride. On each of the selected frames QM/MM reference forces (BO) are calculated.
The forces on the atoms of the QM subsystem are stored (FM\_REF\_FORCES) along with the Hirshfeld
charges (FM\_REF\_CHJ) as well as the electrostatic potential and field on the nearby MM atoms
(FM\_REF\_PIP). Second, a set of atomic point charges that reproduce the electrostatic potential and
forces that the QM system exerts on the surrounding classical atoms is derived. Third, the nonbonded
contributions, computed with the charges obtained in the second step and given Lennard-Jones
parameters, are subtracted from the total reference forces on the QM atoms. The remaining forces are
assumed to be derived from bonded interactions. The parameters for bonded interactions (torsions,
bending and bonds) are thus adjusted in order to reproduce the remaining forces. See reference
\cite{FM-maurer} for details. An updated topology file FMATCH.top is written at the end of the run.
To check the quality of the fitting procedure a section with the absolute and relative force RMS per
atom is printed at the end to standard output. \\ Files generated by the force matching code (Some
of the following default filenames can be changed via the respective keywords in the FORCEMATCH
block):
\begin{itemize}
\item \textbf{FMATCH.top} \\
      Updated topology file at the end of the job.
\item \textbf{FM\_REF\_CHJ} \\
      Hirshfeld charges on the QM atoms from the reference force calculations. Format: Two lines
per frame. First line contains frame index from the original reference trajectory file. Second line
gives the Hirshfeld charges on the QM atoms in cpmd ordering.  \item \textbf{FM\_REF\_PIP} \\
      Electrostatic potential and field on the NN atoms from the reference force calculations.
\item \textbf{FM\_REF\_FORCES} \\            
      For each frame extracted from the reference TRAJECTORY file and for which QM/MM forces were
calculated, the QM/MM forces on the QM atoms are dumped into this file. One line per frame with the
original frame index and the number of QM atoms. Then for each QM atom in cpmd ordering:\\ atom
index,x,y,z,fx,fy,fz            
\end{itemize}
All force matching related information written to standard output are labeled with '  fm '. After
the initialization the reference trajectory file is parsed and the line 'fm extracting total number
of frames for SPs' is printed. For each frame you should find the following lines: 'frame number',
'computing reference forces' and 'Total nr. of iterations:'. The beginning of the second part of the
force matching protocol is marked with the line 'Reading values for charge fitting from file'. At
the end the RMS deviation of the charges, electrostatic potential and field are printed. The third
part starts with 'Will now loop over reference frames again' to compute the non-bonded interactions.
After 'Done with classical loop' the covalent parameters are fitted and you can monitor the change
of the absolute and relative RMS deviation from the reference covalent forces during the
optimization. 'Optimization successful' indicates the end of the fitting of the covalent parameters,
FMATCH.top is written and, finally, the total (non-bonded plus covalent) forces are calculated with
the updated topology to get the RMS deviation of the total force. The force matching related output
ends with a block containing 'computing RMS per atom'.
%_FM]

%
%
%%%%%%%%%%%%%%%%%%%%%%%%%%%%%%%%%%%%%%%%%%%%%%%%%%%%%%%%%%%%%%%%%%%%%%%%
% Gromacs QM/MM documentation %%%%%%%%%%%%%%%%%%%%%%%%%%%%%%%%%%%%%%%%%%
%%%%%%%%%%%%%%%%%%%%%%%%%%%%%%%%%%%%%%%%%%%%%%%%%%%%%%%%%%%%%%%%%%%%%%%%
\subsection{Gromacs/CPMD QM/MM Calculations}\label{sec:gmxqmmm}
As of version 3.3 the Gromacs\cite{gmx3} classical MD code contains
a generic API for QM/MM calculations. So far this API has been used to
QM/MM interfaces for GAMESS, Gaussian, and \dots CPMD. Unlike in the
Gromos/CPMD QM/MM interface code (see above) the main MD driver is in
the classical code. The Gromacs/CPMD interface code is based on the
EGO/CPMD interface and was developed by Pradip Kumar Biswas in the
group of Valentin Gogonea at Cleveland State University.
For additional information and downloads see
\htref{http://comppsi.csuohio.edu/groups/qmmm.html}{http://comppsi.csuohio.edu/groups/qmmm.html},
and the respective publication~\cite{gmxqmmm}.

\subsubsection{Technical Introduction}
The whole interface code is divided into two parts: one part is
embedded in the Gromacs code and the other in CPMD. Both, the modified
Gromacs and the CPMD codes are compiled independently and communicate
via files in the current working directory.

Since the Gromacs code acts as the driver, you first have to set up a regular
Gromacs classical MD simulation in the usual way by building/providing
a .pdf/.gro file and a .top file. Before running grompp, you also need
to create an index file (usually named index.ndx) that lists the atoms
of the QM subsystem and provide further parameters for the CPMD calculation
like the size of QM-simulation box, plane-wave cutoff for CPMD, Coulomb cutoff,
if any, etc (for details, see the rgmx script in the QM/MM examples).
%Grompp switches off the QM/MM Coulomb interactions and creates the environment
%for energy minimization or mdrun (as per the case).

During mdrun, the interface is controlled by two function  calls:\\
a) init\_cpmd() prepares the ground for the QM/MM interface. It sets the flags
for the QM and MM atoms finds LINK atoms from the topology and prepares temporary
structures to process the QM/MM data etc.\\
b) call\_cpmd() first creates the CPMD input file "CPMD\_inp.run" using a template
"CPMD\_inp.tmpl" and then kickstarts the CPMD code via a "fork/exec" or "system" call.
The interface is set to use "fork". If system call is preferred, you need to set
the defined variable NOFORK to 1. call\_cpmd() gets forces and energy from CPMD
and appends them to Gromacs structures. Gromacs then moves the atoms and while
evaluating the forces, calls CPMD again (this is the QM/MM loop). Thus this interface
essentially performs a QM/MM Born-Oppenheimer MD simulation.\\

\subsubsection{Compilation of Gromacs}
In its present state, you need to use CFLAGS = -DGMX\_QMMM\_CPMD
in configure to include the CPMD interface code into Gromacs.
In the adapted Gromacs package, a script "build" is provided in the gromacs folder
that takes care of the QM/MM configuration and compilation. Please adapt
as needed.

\subsubsection{Execution of QM/MM runs}
It is like running Gromacs with the additional needs are given by:

a) having an index.ndx file specifying the QM atoms (see example index.ndx file).
You can create a usual Gromacs index.ndx file and then append to it the QM
group.

b) specifying other QM informations like planewave cutoff, qmbox size etc
in the grompp setup (for the mdp file).

c) having a CPMD input file template "CPMD\_inp.tmpl" where essential keywords
for CPMD run need to be mentioned. "INTERFACE GMX" is essential for QMMM; it
ensures a single-point calculation inside CPMD each time it is invoked. Inside the
interface, all the QM \& MM atoms are translated in such a way that the QM system
be at the center of the QM box. Thus the keyword "MOLECULE CENTER OFF" is required
to avoid any further movements of the QM atoms.

Right now the \refkeyword{ODIIS} minimizer and \refkeyword{PCG} minimizer
(including PCG MINIMIZE) are allowed to be used inside CPMD.
There also is a hybrid scheme where for the MD first step it will  use the
"PCG MINIMIZE" but for all subsequent steps it will use the faster ODIIS
minimizer.). Other sections of CPMD input structures need to be kept as usual
though the final values for the CELL size and CUTOFF will be those provided
by you in the mdp file.

\subsubsection{QM/MM Examples}
Example inputs for a H${}_2$O-dimer and an ethane molecule are bundled with the modified
gromacs distribution from. \\
\htref{http://comppsi.csuohio.edu/groups/qmmm.html}{http://comppsi.csuohio.edu/groups/qmmm.html},

% FIXME:
%\subsubsection{Alternative QM/MM interfaces}
%Ego interface, use ego interface for PES scanning driver.
%%% QM/(P)MM Interface
\subsection{QM/(P)MM Interface to IPHIGENIE}
The interface of CPMD to the PMM-MD program IPHIGENIE\cite{Schwoerer2013,Schwoerer2015} (\url{https://sourceforge.net/projects/iphigenie})
supersedes the 
CPMD interface of Eichinger et al.\ \cite{egoqmmm} to the MD program EGO.\cite{ego1}
It provides a Hamiltonian coupling to polarizable MM force fields (PMM).
Here, similar to the Eichinger implementation and the derived CPMD/Gromacs coupling\cite{gmxqmmm},
CPMD is used only to compute energies and forces 
but the propagation of the equations of motion is done
externally by IPHIGENIE. 

The implementation is based on a single executable, which links
CPMD and the interface routines as libraries to the IPHIGENIE MD executable 'iffi'.
For CPMD-3.17 a patch is provided on \url{www.cpmd.org} with example configurations 
for the compilation. For CPMD-4.0 and later the interface to IPHIGENIE is provided as a module.
Here, using the configure.sh script with the option {\tt -iphigenie}  generates
a makefile which builds the interface library along with a regular cpmd.x executable.

For further instructions on compiling and example inputs for QM/PMM runs please consult
the main iphigenie page \url{https://sourceforge.net/projects/iphigenie} and the
associated Wiki pages.

\subsection{CPMD on parallel computers}

There are three different parallel strategies implemented in CPMD.
The actual strategy has to be chosen at compile time.
\begin{itemize}
   \item Shared memory parallelization\\[8pt]
         This strategy uses the OpenMP library.
         The code is compiled {\em without} the {\bf PARALLEL}
         preprocessor flag and compilation and linking need the
         corresponding OpenMP flags (dependent on compiler). \\
         Depending on the overhead of the OpenMP system implementation
         good speedups can be achieved for small numbers of
         processors (typically 4 to 8). The advantages of this
         version of the code are small additional memory usage and it
         can be used in non-dedicated CPU environments.

   \item Distributed memory parallelization\\[8pt]
         This is the standard parallelization scheme used in CPMD
         based on MPI message passing library.\\
         The single processor version of this code typically
         shows an overhead of ca. 10\% with respect to the optimal
         serial code. This overhead is due to additional copy and
         sort operations during the FFTs.\\
         All the basic system data and many matrices of the size
         of the number of electrons are replicated on all
         processors. This leads to considerable additional memory
         usage (calculated as the sum over the memory of all
         processors compared to the memory needed on a single
         processor). For large systems distributed over many
         processors the replicated data can dominate the memory usage.\\
         The efficiency of the parallelization depends on the
         calculated system (e.g. cutoff and number of electrons) and
         the hardware platform, mostly latency and bandwidth of the
         communication system. The most important bottleneck in the
         distributed memory parallelization of CPMD is the
         load-balancing problem in the FFT. The real space grids
         are distributed over the first dimension alone (see line
         {\tt REAL SPACE MESH:} in the output. As the mesh sizes
         only vary between 20 (very small systems, low cutoffs) and
         300 (large systems, high cutoff) we have a rather coarse grain
         parallelization. To avoid load unbalance the number of processors
         should be a divisor of the mesh size. It is therefore clear that even
         for large systems no speedup can be achieved beyond 300 processors.
         A solution to this problem is provided with the keyword
         \refspekeyword{CP GROUPS}. This technique, together with optimal mapping,
         allow to scale from thousands to millions of cores on modern
         supercomputers such as IBM BG/Q (in particular when running HFX calculations).  To learn more about the
         distributed memory parallelization of CPMD consult
         D. Marx and J. Hutter, "Modern Methods and Algorithms of Quantum Chemistry",
         Forschungszentrum J\"ulich, NIC Series, Vol. 1 (2000), 301-449.
         For recent developments and for a perspective see \htref{http://www.cpmd.org}{http://www.cpmd.org}.

         When selecting {\bf NSTBLK} for  \refkeyword{BLOCKSIZE STATES} it is important
         to take into account the granularity of the problem at hand. For example,
         in cases where the number of  \refkeyword{STATES} is smaller than the total number of
                   the available processors, one must choose a value for {\bf NSTBLK} such
         that only a subgroup of the processors participate in the distributed
         linear algebra calculations. The same argument is also relevant when
         the number of  \refkeyword{STATES} is only moderately larger than the number of processors.

   \item Mixed shared/distributed memory parallelization\\[8pt]
         The two parallelization schemes described above are implemented
         in such a way that they don't interfere. Therefore it is easy
         to combine them if this is supported by hardware (shared/distributed
         memory architecture) and software (libraries). Since the MPI
         parallelization is very efficient for a small to medium number of
         nodes and all modern MPI libraries are able to take advantage
         from shared memory communication, using the mixed
         shared/distributed memory parallelization is of most use if
         you run a job on a large number of SMP nodes, when the
         distributed memory parallelization has reached its scalability
         limit (see above). To learn more about the mixed parallelization
         scheme of CPMD consult \cite{mixed}.
\end{itemize}

As with all general statements, these are only guidelines. The only
way to get reliable information is to run benchmarks with the system
you want to calculate.


\subsection{Using the new xc driver and assembling functionals}
The new \refspekeyword{XC\_DRIVER}{XC DRIVER} offers more flexibility compared to \textbf{OLDCODE} and
\textbf{NEWCODE}. The minimal setup for the use of, \emph{e.g.{}}, the BLYP xc functional:
\begin{verbatim}
&DFT
 XC_DRIVER
 FUNCTIONAL GGA_XC_BLYP
&END
\end{verbatim}
is equivalent to specifying
\begin{verbatim}
&DFT
 XC_DRIVER
 FUNCTIONAL GGA_X_BLYP GGA_C_LYP
&END
\end{verbatim}
or, in a more verbose way,
\begin{verbatim}
&DFT
 XC_DRIVER
 FUNCTIONAL GGA_X_BLYP GGA_C_LYP
 SCALES     1.0        1.0
 LIBRARY    CP         CP
&END
\end{verbatim}

If not explicitly stated, the xc driver will always use the internal CP functional library. The use of 
libxc functionals can be requested by using the keyword \refkeyword{LIBRARY}, and both libxc and internal
CP functionals may be used simultaneously. The following example combines a particular parametrisation of VWN
available in libxc with Slater exchange computed within the internal CP library.
\begin{verbatim}
&DFT
 XC_DRIVER
 FUNCTIONAL LDA_X LDA_C_VWN_3
 LIBRARY    CP    LIBXC
&END
\end{verbatim}
If not specified, all functionals will contribute equally to the total xc energy and the potential. A scaling factor
for every functional can be introduced by using \refkeyword{SCALES}. A manual setup equivalent to
HYB\_GGA\_XC\_CAM\_B3LYP, but using different attenuation parameters $\alpha=0.19, \beta=0.66, \mu=0.330$ would read:
\begin{verbatim}
&DFT
 XC_DRIVER
 FUNCTIONAL GGA_X_B88 GGA_C_LYP LDA_C_VWN
 SCALES     1.00      0.81      0.19
 COULOMB ATTENUATION
 0.190  0.660  0.330
&END
\end{verbatim}

It is also possible to use a linear response kernel different from the one used to generate the ground-state orbitals.
In this case, duplicates of the aforementionned exist with the prefix \textbf{KERNEL\_}.
Note that is currently not possible to combine different forms of the Coulomb operator between the kernel and the ground-state
functional (\emph{i.e.{}} it is not possible to use the combination of B3LYP and CAM-B3LYP; however, combinations between GGA and any
(screened) hybrid are possible).
\begin{verbatim}
&DFT
 XC_DRIVER
 FUNCTIONAL       GGA_X_B88 GGA_C_LYP LDA_C_VWN_3
 LIBRARY          CP        CP        LIBXC 
 SCALES           0.80      0.81      0.19
 HFX_SCALE        0.20
 LR_KERNEL        GGA_X_B88 GGA_C_LYP LDA_C_VWN
 KERNEL_LIBRARY   CP        CP        CP
 KERNEL_SCALES    0.75      0.81      0.19
 KERNEL_HFX_SCALE 0.25
&END
\end{verbatim}


\clearpage
%---------------------------------------------------------------------
\section{Questions and Answers}\label{faq}

The following section is a slightly edited collection of questions and
answers from the cpmd mailing list, cpmd-list@cpmd.org.

\subsection{How to Report Problems}
Up front a few remarks on how to report problems (and how to respond),
so that the chances to solve the problem (permanently)
are as high as possible.

If you have compilation problems, please always state what version of
CPMD you are trying to compile and what kind of machine you are using,
i.e. what operating system, what compiler (particularly important on
linux machines), which compilation flags, and what libraries you are
using. Best you include the first part of your makefile (up to `End of
Personal Configuration', please \textbf{don't} post the whole makefile)
as this contains most of the required information. Also include the
\textbf{relevant} part of the make output (again, the full output
usually is very long and rarely needed).

If you have problems with a specific calculation, please include your
input and the output of the run, so that others can try to reproduce
the error. Again, please state the version of CPMD you are using and
the platform you are running on.

% MPB: Links broken...
% A good general guide on how to report bugs can be found at:\\
% \htref{http://freshmeat.net/articles/view/149/}{http://freshmeat.net/articles/view/149/},\\
% the corresponding guide for people responding can be found at:\\
% \htref{http://freshmeat.net/articles/view/1082/}{http://freshmeat.net/articles/view/1082/}.\\
% Another useful article about how to ask questions the smart way is at:\\
A useful article about how to ask questions the smart way is at:\\
\htref{http://www.catb.org/~esr/faqs/smart-questions.html}{http://www.catb.org/\~{}esr/faqs/smart-questions.html}
%%%%%%%%%%%%%%%%%%%%%%%%%%%%%%%%%%%%%%%%%%%%%%%%%%%%%%%%%%%%%%%%%%%%%%%%

%%%%%%%%%%%%%%%%%%%%%%%%%%%%%%%%%%%%%%%%%%%%%%%%%%%%%%%%%%%%%%%%%%%%%%%%
\subsection{Explanation of Warnings and Error Messages}
\label{sec:expl-warn-error}

If execution of CPMD aborts with an error message, please do not only
consult the standard output, but also the \textbf{LocalError} files which
are produced in the run directory. They may contain more valuable information
of the problem at hand.\\~\\

\iffaqsum
FAQ on warnings and errors:
\begin{itemize}
\item{\reffaqquestion{crash}{The code crashes without leaving any error message.}}
\item{\reffaqquestion{wannier}{WANNIER CODE WARNING: GRADIENT FOR RESTA
FUNCTIONAL}\\(during CP molecular dynamics)}
\item{\reffaqquestion{xcpp}{WARNING! XC FUNCTIONALS INCONSISTENT}}
\item{\reffaqquestion{lsdgeoopt}{STOPGM! STACK OF MAIN CALLS} (during geometry
optimization)}
\item{\reffaqquestion{spline}{Warning! Spline region smaller than maximum
G-value}\\(during geometry optimization with variable cell)}
\item{\reffaqquestion{gorder}{GORDER| PROGRAMING ERROR. INFORM THE
PROGRAMER} (during startup of WAVEFUNCTION or GEOMETRY optimization)}
\end{itemize}
\fi

%%%%%%%%%%
%! Martin
\faqquestion{crash}
My CPMD run suddenly aborted without any error message. What happened?

\faqanswer
CPMD will only print error messages to the output file in some very
specific cases. Usually, if an error is encountered, every task
writes to a file called \emph{LocalError-X-X-X.log} (where $x$ denotes
a task ID), where you will find an \emph{error message}, the 
\emph{procedure} and \emph{file name} where the error occured, as well as a 
\emph{call stack} (although the latter may sometimes be incomplete). 
In case the error message is empty, you may have to consult the source
code directly.

If no \emph{LocalError} files are produced, the processes may have
exited due to an insufficient amount of RAM available for a single task.

%! End

\faqquestion{wannier}
Could anybody explain to me the follwing error in the cpmd output
during CP Molecular Dynamics runs with the flag
\refkeyword{WANNIER WFNOUT} LIST DENSITY?

\begin{verbatim}
WANNIER CODE WARNING: GRADIENT FOR RESTA FUNCTIONAL IS GMAX=0.118E-02
\end{verbatim}
Does it mean any serious error in the calculation?

\faqanswer
The default spreadfunctional used in CPMD is the Vanderbilt type. At the
end of the calculation the convergence with respect to the Resta type
functional is also checked.  For large cells both should be converged at
the same time.  However, for typical application this is not the case
and you get the warning.  This is not serious and you can ignore it.
%jgh
%%%%%%%%%%

\faqquestion{xcpp}
A warning message appeared in the output file:
\begin{verbatim}
 !!!!!!!!!!!!!!!!!!!!!!!!!!!!!!!!!!!!!!!!!!!!!!!!!!!!!!!!!!!!!!!!!
 !WARNING! XC FUNCTIONALS INCONSISTENT FOR h.pp
 !!!!!!!!!!!!!!!!!!!!!!!!!!!!!!!!!!!!!!!!!!!!!!!!!!!!!!!!!!!!!!!!!
\end{verbatim}

Does it mean that my pseudopotential file is wrong? I used fhi98PP
to create this pp file, and just recast it. So if it is wrong, what
more should I do to the output file of fhi98PP?

\faqanswer
It means that the XC functional used to generate the pseudo potential
and the functional which you want to use in CPMD are not the same. This
could be more or less serious depending on the specific situation:
Some of the XC functionals in CPMD are
just minor variations of each other, e.g. the LDA part may evaluated
using Perdew-Wang, Perdew-Zunger, VWN or the Pade-interpolation
formula. If this is the case, the resulting error is usually small.
However, if your pseudo potential was generated \emph{e.g.{}} with PBE
and you try to use it with the BLYP functional, the error might become
non-negligible. You may compare the two functionals (pp \& CPMD-input),
and judge for yourself if the difference is significant or not. \\
For \textbf{hybrid} and \textbf{meta} functionals, there are usually
no pseudo potentials available. In these cases, it is common practice to use
the pseudo potentials of related GGA xc functionals (\emph{e.g.{}} PBE for TPSS,
BLYP for CAM-B3LYP, etc.{}), or, if there is no `close relative', LDA
(this is possible \emph{e.g.{}} in combination with the Minnesota xc functionals).
% apsi / MPB
%%%%%%%%%%
% MPB: Outdated (new stopping, new functionals, new everything) 
%   \faqquestion{lsdgeoopt}
%   When I am running a GEOMETRY OPTIMIZATION using the keywords LSD and
%   MULTIPLICITY, I get the following output:
%   \begin{verbatim}
%    ================================================================
%    =                  GEOMETRY OPTIMIZATION                       =
%    ================================================================
%    NFI      GEMAX       CNORM           ETOT        DETOT      TCPU
%    EWALD SUM IN REAL SPACE OVER  1* 1* 1 CELLS
%    STOPGM! STACK OF MAIN CALLS:
%    STOPGM! CALL    FORCEDR
%    STOPGM! CALL     FORCES
%    STOPGM! CALL    VOFRHOB
%    STOPGM! CALL      GCLSD
%   
%    PROGRAM STOPS IN SUBROUTINE LSD_GGAX| NOT PROGRAMMED
%   \end{verbatim}
%   I would like to know how to solve this problem?
%   
%   \faqanswer
%   This simply means that the LSD version of this specific
%   functional has not been implemented (yet). Feel free to
%   implement it yourself and submit a patch.
%   
%   Possible solutions are:\\
%   \begin{enumerate}
%   \item you implement it into CPMD (see file lsd\_func.F)\\
%   \item you switch to the PBE functional, which is
%      the modern variant of this functional and is
%      implemented both for spin-restricted and unrestricted cases.
%   \end{enumerate}
%   %jgh
%   %%%%%%%%%%
                
\faqquestion{spline}
I am trying to optimize a crystal structure (both ion positions and
cell volume) using CPMD and get the warning message:\\
Warning! Spline region smaller than maximum G-value.\\
The optimization seems to converge nicely but what does this warning
mean/imply?

\faqanswer
This in fact touches several points. The following applies
also to other variable cell calculations with CPMD (e.g. constant
pressure simulations).

\begin{itemize}
\item Pseudopotential functions in CPMD are calculated on a
   radial grid in G-space and then used in spline interpolations.
   This speeds up variable cell calculations considerably.
   The maximum grid point is given by the cutoff. With the
   keyword
\begin{verbatim}
   SPLINE RANGE
     x.xx
\end{verbatim}
   you can enlarge the grid to x.xx times the cutoff.
   Also, you should make sure, that the number of spline points is
   large enough. Older version of CPMD defaulted to as little as 501.
   This is at the lower limit for accuracy with a fixed cell.
   Especially if you have high cutoffs it is better to increase
   this value, e.g.
\begin{verbatim}
   SPLINE POINTS
     2500
\end{verbatim}
   or larger. The current default (5000) should be large enough.

\item  In a variable cell calculation, CPMD uses a constant
   number of plane waves. Therefore if your cell
   contracts the cutoff increases, if the cell gets larger
   the cutoff decreases. So if you have the first case
   the spline interpolation needs points above the original
   cutoff and you get a warning.
   Depending on the amount of change in the cell you expect
   a value for the \refkeyword{SPLINE} \textbf{RANGE} of 2--5 is needed.

\item Coming back to the constant number of plane waves.
   If your cell gets larger the effective cutoff decreases.
   This may have very undesirable effects and it is better
   to define the plane waves on a box larger than any box
   you anticipate the simulation will reach.
   In order to not have to start with an unreasonable cell
   you can define the plane waves not with the actual box
   but with a reference cell, use the keyword
\begin{verbatim}
   REFERENCE CELL
     a b c xa xb xc
\end{verbatim}

\item However, you really want to do a constant cutoff
   calculation, not a constant number of plane waves.
   For technical reasons this is not possible and in
   principle you should do the calculation at a high
   enough cutoff in order that the calculation is
   converged all along the simulation path (with the
   slight changes in cutoff).

   To avoid these very high cutoffs the group in Trieste
   came up with a method that allows to perform pseudo
   constant cutoff calculations.
   This method is implemented in CPMD
   (keyword \refkeyword{CONSTANT CUTOFF}) and explained
   in the paper~\cite{bernasconi95}
\end{itemize}
%jgh

\faqquestion{gorder}
I was optimizing wavefunctions for my system. After a successful run I
modified CELL VECTORS. This latter calculation crashed with the
following error either when I restarted (only wavefunctions) or if I
started from scratch.
\begin{verbatim}
 GORDER| PROGRAMING ERROR. INFORM THE PROGRAMER
\end{verbatim}
Is there something wrong with the code?

\faqanswer
This is a known problem in CPMD.
The message comes from a test, which probes whether the G vectors are in a
``safe'' order, namely such that after a restart with a different number
of processes or on a different machine the results agree. Usually this
error only occurs in large systems with a high cut-off energy and/or
large unit cells, i.\,e. where one gets lots of close-lying G vectors.

There are two possible workarounds:
\begin{enumerate}
\item Slightly change your computational box in one dimension
   e.\,g. from $10.000000$ to $10.000001$.
   This helps some times.
\item
The check is not 100\,\% accurate.
This means by just ignoring the message you will
most likely get correct results. The error would
only appear in restarts where you could see a small
inconsistency in energy in the first step. Final
results should not be affected (except for MD if
you do restarts).

To avoid the stop, comment out the two lines
at the end of file loadpa\_utils.mod.F90 of the form
\begin{verbatim}
        CALL STOPGM('GORDER','ERROR IN G-VEC ORDERING (NHG)')
\end{verbatim}

However, be sure to check that the results are reasonable.
\end{enumerate}
%AK,apsi,jgh,wjq
%%%%%%%%%%

\subsection{Pseudopotentials}
\label{sec:FAQPP}

\iffaqsum
FAQ on pseudopotentials:
\begin{itemize}
\item{\reffaqquestion{whichPP}{Which type of pseudopotentials to use?}}
\item{\reffaqquestion{whichLMAX}{Which is the correct value of LMAX?}}
\end{itemize}
\fi

\faqquestion{whichPP}
I'm confused about how to select a pseudopotential type (Troullier-Martins,
Goedecker, etc.). What makes one choose say a Goedecker potential instead
of a Vanderbilt potential?

\faqanswer
The choice of a pseudopotential in CPMD calculations depends on
needs, available resources and taste.

Troullier-Martins norm-conserving pseudopotentials are probably the
most-commonly used type of pseudopotentials in CPMD calculations.
They work very well for non-problematic elements and they are quite
easy to create (note, that it is also easy to create a bad pseudopotential).
When using the Kleinman-Bylander separation, one also has to be
careful to avoid so called ghost states (e.g. many transition metals
need LOC=$l$ with $l$ being an angular momentum smaller than default value
which is highest).

Goedecker pseudopotentials are stored in an analytical form, that
ensures the separability, but they usually need a higher (sometimes
much higher) plane wave cutoff than the corresponding Troullier-Martins
counterparts. Also the creation procedure is more complicated,
but there is a very large library of tested
pseudopotentials (mostly LDA but also some GGA pseudopotentials).

Vanderbilt pseudopotentials have the advantage of needing a much
reduced plane wave cutoff. The drawback is, that
only a limited subset of the functionality in CPMD is actually implemented
with uspps (MD, wavefunction/geometry optimization and related stuff and
only at the gamma point and you have to make sure, that your real space
grid is tight enough).
Also due to sacrificing norm-conservation for softer
pseudopotentials, your wavefunction has very limited meaning, so
that not all features available for norm-conserving pseudopotentials
can actually be easily implemented or implemented at all.

For some elements it can
be rather difficult to generate good (i.e. transferable) pseudopotentials,
so you should always check out the available literature.

\faqquestion{whichLMAX}
How do I choose the correct value of LMAX?

\faqanswer
If you use a Vanderbilt or Goedecker type potential only the format of
the LMAX-line has to be valid. The actual value is read from the
pseudopotential file and the value in the input file will be ignored.
It is highly recommended to still use values that make sense, in case you want
to do a quick test with a numerical (Troullier-Martins) pseudopotential.

Generally, the highest  possible value of LMAX depends on the highest
angular momentum for which a ``channel'' was created in the pseudopotential.
In the pseudopotential file you can see this from the number of coloumns
in the \&POTENTIAL section. The first is the radius, the next ones are
the orbital angular momenta (s, p, d, f,\ldots).
As an example you can determine a potential for carbon using f-electrons
and set LMAX=F. Since the f-state is not occupied in this case there is
very little advantage but it costs calculation time. In short, you can use values
as high as there is data in the pseudopotential file, but you don't have to
if it is not needed by the physics of the problem.

A fact that causes confusion is that Hamann's code for pseudopotential
generation always produces output for the d-channel, even if you only
request channels s and p. You should be cautious if $r_{c}$ and the
energy eigenvalue of the p- and d-channels are equal. In most of these
cases LMAX=P should be used.

\subsection{File Formats and Interpretation of Data}
\label{sec:file-formats}

\iffaqsum
FAQ on interpretation of data:
\begin{itemize}
\item{\reffaqquestion{etot}{Why is my total energy so much
    different from a Gaussian calculation?}}
\item{What is the meaning of the \reffaqquestion{energies}{energies} reported
in a molecular dynamics simulation?}
\item{What do \reffaqquestion{gnmax}{GNMAX, GNORM, and CNSTR} mean in a
geometry optimization?}
\item{All \reffaqquestion{ir-int}{IR intensities are zero} in VIB.LOG when I
try to calculate the IR of NH$_4^+$ ion by CPMD}
\item{Why can some atoms move far away from the central box in a long MD
simulation of \reffaqquestion{liquidpbcs}{bulk liquid using periodic boundary
conditions}?}
\item{Why did the \reffaqquestion{eneele}{electron energy continuously increase}
during the MD simulation of my metal?}
\item{What is the meaning of the files \reffaqquestion{raman}{APT,
    POLARIZATION and POLARIZABILITY} resulting from my linear response calculation of Raman spectra?}
\item{What is the meaning of \reffaqquestion{dipole}{columns 2 to 7 in the
DIPOLE file}?}
\item{How do I use the \reffaqquestion{restart}{binary RESTART file across
different platforms} with different encoding?}
\end{itemize}
\fi

\faqquestion{etot}
Why is my total energy so much different from a Gaussian calculation?

\faqanswer
With CPMD you are using \textbf{pseudopotentials} to describe the atoms.
Since the total energy describes only the interactions between the
pseudocores and the valence electrons (and \textbf{some} core electrons
in the case of so-called semi-core pseudopotentials), you are missing
the contribution of the core electrons and the full core charges of a
regular all-electron calculation.
\textbf{Energy differences} between two configurations, on the other
hand, should be comparable, provided you use the same number of atoms,
the same plane wave cutoff, the same pseudopotentials, and the same
supercell geometry in the CPMD calculation.

\faqquestion{energies}
In a molecular dynamics simulation, CPMD prints out a list of
energies for each integration step. Does anyone know the meaning
of the individual values.

\faqanswer
Some explanations to the energy terms:

\begin{description}
\item[EKINC] fictitious kinetic energy of the electrons in a.u.
           this quantity should oscillate but not increase during a
           simulation.
\item[TEMPP] Temperature of the ions, calculated from the kinetic
           energy of the ions (EKIONS).
\item[EKS]   Kohn-Sham energy (the equivalent of the potential energy
           in classical MD).
\item[ECLASSIC] = EKS + EKIONS
\item[EHAM]     = ECLASSIC + EKINC.
           Hamiltonian energy, this is the conserved
           quantity, depending on the time step and
           the electron mass, this might oscillate but
           should not drift.
\item[DIS]    mean square displacement of the ions with respect to
           the initial positions. Gives some information on the diffusion.
\end{description}

You can modify the list of individual energies to be displayed
with the \refkeyword{PRINT ENERGY} keyword.
%jgh
%%%%%%%%%%

\faqquestion{gnmax}
What do GNMAX, GNORM and CNSTR mean in a geometry optimization?

\faqanswer
These are abbreviations for the following quantities:
\begin{description}
\item[GNMAX] max${}_{I,a}$ ($|F_{Ia}| $) = largest absolute
  component ($a=x,y,z$) of the force on any atom $I$.
\item[GNORM] $\left< F_{I}^2\right>_{I}$ = average force on the atoms $I$
\item[CNSTR] max${}_{I,a} { F^{constr}_{Ia} }$ = largest absolute
  component ($a=x,y,z$) of force due to constraints on any atom $I$.
\end{description}
% apsi
%%%%%%%%%%


\faqquestion{ir-int}
I found all the IR intensities in VIB.log file are zero when I try to
calculate the IR of NH${}_4^+$ ion by CPMD.

\begin{verbatim}
 Harmonic frequencies (cm**-1), IR intensities (KM/Mole),
 Raman scattering activities (A**4/AMU), Raman depolarization ratios,
 reduced masses (AMU), force constants (mDyne/A) and normal coordinates:
                     1                      2                      3
                    ?A                     ?A                     ?A
 Frequencies --   142.9800               188.9340               237.2614
 Red. masses --     0.0000                 0.0000                 0.0000
 Frc consts  --     0.0000                 0.0000                 0.0000
 IR Inten    --     0.0000                 0.0000                 0.0000
 Raman Activ --     0.0000                 0.0000                 0.0000
 Depolar     --     0.0000                 0.0000                 0.0000
 Atom AN      X      Y      Z        X      Y      Z        X      Y      Z
   1   7     0.00   0.00   0.00     0.00   0.00   0.00     0.00   0.00   0.00
   2   1     0.00  -0.35  -0.50    -0.35   0.00   0.00    -0.50   0.00   0.00
   3   1     0.00  -0.35   0.50    -0.35   0.00   0.00     0.50   0.00   0.00
   4   1     0.00   0.35   0.00     0.35   0.00   0.50     0.00  -0.50   0.00
   5   1     0.00   0.35   0.00     0.35   0.00  -0.50     0.00   0.50   0.00
                     4                      5                      6
\end{verbatim}

\faqanswer
That's not a problem of your calculation. The keyword
\respkeyword{VIBRATIONAL ANALYSIS} does not calculate intensities.
The calculation of intensities is currently not possible in CPMD.
The intensities in the 'VIBx.log' files are arbitrarily set to zero.
The entries have to be there so that visualisation programs, that are
able to read output of the Gaussian program, can be also used to
visualize the CPMD results.
%
% hlanger
%%%%%%%%%%

\faqquestion{liquidpbcs}
I am trying to simulate a bulk liquid in
CPMD and supposing that periodic
boundary conditions are built into the program.
But after several thousand MD steps, I found some
particles are far away from the central simulation box.

Why it is so if periodic boundary conditions (PBC) on
particle coordinates are imposed in all three
directions?

\faqanswer
If you are not using the
\begin{verbatim}
SYMMETRY
  0
\end{verbatim}
options your calculations are actually using periodic boundary
conditions (PBC). PBC are imposed within CPMD for
all calculations. However, the particle positions
are not folded back to the original computational
box. The reason for this is that most people prefer
to have ``smooth'' trajectories without jumps of
particles. This allows for easier tracking of
special particles and nicer graphics. In addition
it is easy (with a little script) to apply PBC
afterwards yourself, if needed.
%
% jgh
%%%%%%%%%%


\faqquestion{eneele}
I am trying to simulate a bulk sodium and
I found electron energy is increasing
continuously and it is in the range of 0.07 a.u. at
the end of 20000 steps.

\faqanswer
Sodium is a metal, and therefore missing an important
feature that allows for stable CP dynamics: the band gap.
Using Nos\'{e} thermostats (on electrons and ions) it
might still be possible to perform meaningful CP
simulations~\cite{Blochl92}.

The choice of parameters for the thermostats, however,
will be nontrivial, highly system dependent and require
extensive testing.
Without thermostats you will have strong coupling between
electronic degrees of freedom and ionic degrees of freedom.
Adiabaticity is not maintained and a steady increase of the
fictitious kinetic energy will occur.
%
% jgh
%%%%%%%%%%


\faqquestion{raman}
I have computed RAMAN by LINEAR RESPONSE, and get three files: APT,
POLARIZATION and POLARIZABILITY with lots of data in these files. I
want to know the meaning of the data, please give me some answer in
detail. 

\faqanswer
The POLARIZABILITY file simply contains the polarizability tensor of the
whole system in atomic units. The POLARIZATION file contains the
total dipole moment (electronic + ionic) of the
whole system in atomic units. As for the file APT, it contains the
atomic polar tensors for each atom in the system.  The atomic polar
tensor is the derivative of the forces on the atoms with respect to an
applied external electric field. Equivalently it is, from a Maxwell
relation, the derivative of the total dipole of the system with respect
to the nuclei positions. It is thus an important ingredient of the
calculation of infrared spectra intensities, for example used in an
harmonic approximation. The trace of this tensor is the so-called Born
charge of the considered atom.  The data is arranged in the following
order (still in a.u.): the APT tensor is
$\frac{\mathrm{d}F_{I,i}}{\mathrm{d}E_{j}}$ where $F_{I,i}$ is the force
on atom I along $i=x,y,z$ and $E_j$ is the electric field along
$j=x,y,z$. $(I,i)$ are the indices of the $3N$ atoms lines in the APT
file, one atom after the other, and $j$ is the column index in the APT
file.
%Rodolphe Vuilleumier

\faqquestion{dipole}
I was wondering what columns 2 to 7 in the DIPOLE file correspond to?
When I run CPMD v.3.5.1, columns 2 to 4 come out identical to columns
5 to 7 respectively. When I run with CPMD v.3.4.1, the columns
come out different. Is there an explanation for this?

\faqanswer
Columns 2 to 4 in the DIPOLE file are the electronic
contribution to the dipole moment, columns 5 to 7 are
the total (electronic + ionic) dipole moment.
All dipole moments are divided by the volume of the box.

In CPMD version 3.5.1 we have changed the reference point
of the calculation. Now the reference point is chosen such
that the ionic contribution is zero and the electronic
contribution minimal (=total dipole). This avoids
a problem that occasionally was seen in older versions.
The electronic dipole is calculated modulo(2$\pi$/L).
Now if the electronic dipole became too large, because
the ionic contribution was large (bad choice of reference
point) the total dipole made jumps of 2$\pi$.
%jgh

\faqquestion{restart}
As you know, the cpmd RESTART file is saved as binary.
But I want to change it to ASCII and vice versa, because I use several
machines of different architecture, for example COMPAQ, IBM, and
LINUX machine. Please help me with any comments.

\faqanswer
The code to read and write the RESTART file is in the files rv30\_utils.mod.F90
and wv30\_utils.mod.F90. Feel free to implement an ASCII version of the restart,
but be aware that the file will be \textbf{huge}.

But you may not need to do that. Let's say you decide to use big-endian
binary encoding (this is what e.g. IBM, Sun and SGI machines do
natively).\\
With Compaq machines there is a compiler flag, -convert,
which you could set to big\_endian (we only have here
linuxalpha, but the compaq compiler should be essentially the same).

On a Linux PC you can use the use the -Mbyteswapio or
the -byteswapio flag, if you have the PGI compiler.

For the Intel compiler (ifc/ifort/efc) you simply set the environment
variable F\_UFMTENDIAN to big (i.e.\\'export F\_UFMTENDIAN=big'\\if you are in
a bourne/korn shell and\\'setenv F\_UFMTENDIAN big'\\if you are in a (t)csh).

Now even your cpmd executables will read and write big-endian
restart files.

Check your compiler documentation for more details (search for endian).
%AK


\subsection{Input Parameter Values}
\label{sec:input-param-valu}

\iffaqsum
FAQ on input parameters:
\begin{itemize}
\item{If I set the keyword \reffaqquestion{systematomrestart}{RESTART
WAVEFUNCTION COORDINATES}, would I have to
write the \&SYSTEM and \&ATOM section again?}
\item{Could anybody tell me \reffaqquestion{cutoff}{how to choose the energy
cutoff in \&SYSTEM section}?}
\item{I have a problem with \reffaqquestion{unocc}{visualising unoccupied
orbitals}. Either I get only occupied orbitals, or, if I add one empty state
when optimizing wavefunction the program never reaches convergence.}
\item{Is there any way to force \reffaqquestion{dumpdensity}{CPMD to dump
DENSITY files every N steps of a molecular dynamics run}?}
\item{\reffaqquestion{bandstruct}{How do I calculate a Band structure with
CPMD}?}
\item{I want to collide fast atoms against surfaces. Why are the
\reffaqquestion{velocities}{initial velocities given using the VELOCITIES
keyword} not considered in the calculation?}
\item{I want to to run CPMD with \reffaqquestion{gaussianbasis}{basis sets
equivalent to Gaussian} 6-31+G(d) and 6-311+G(2d,p). How do i set up the
\&BASIS section?}
\item{\reffaqquestion{newdft}{How do I add support for a new functional?}}
\end{itemize}
\fi

\faqquestion{systematomrestart}
If I set the keyword RESTART WAVEFUNCTION COORDINATES, would I have to
write the \&SYSTEM and \&ATOM section again?


\faqanswer
Yes, you have to include the \&SYSTEM and \&ATOM sections even if you
are restarting. If you write RESTART COORDINATES, the coordinates in the
RESTART file override the ones in the input. RESTART WAVEFUNCTION alone
does not select the coordinates in the RESTART file, but does use those
in the \&ATOMS section.
%
% srb
%%%%%%%%%%

\faqquestion{cutoff}
Could anybody tell me how to choose the energy cutoff in \&SYSTEM section?

\faqanswer
The best way to choose the cutoff for CPMD calculations
is by running first a series of tests. Select a test system
and a representative quantity (bond length, reaction energy,
etc.), perform a series of calculations with increasing
cutoff, pick the lowest cutoff with satisfactory results.

It's always a good idea to make checks at some critical points
of the calculations by increasing the cutoff. See also section \ref{hints:cutoff}.
%jgh
%%%%%%%%%%

\faqquestion{unocc}
I have a problem with visualising unoccupied orbitals. When I use
\refkeyword{RHOOUT}~BANDS or \refkeyword{CUBEFILE}~ORBITALS after the
wavefunction optimization I get only occupied orbitals.
If I add one empty state when optimizing wavefunction
the program never reaches convergence.

\faqanswer
The most efficient way to calculate unoccupied orbitals
is to first optimize the occupied orbitals  and then
restart the calculation using the run option
\begin{verbatim}
 KOHN-SHAM ENERGIES
  n
\end{verbatim}
where n ist the number of unoccupied orbitals. This
will diagonalize the Kohn-Sham Potential (defined
by the occupied orbitals alone).

To test if everything goes fine, you can check
the total energy printed at the beginning of this
job, it should be exactly the one at the end of the
optimization. In addition, if you don't change the
default convergence criteria, the number of
converged Kohn-Sham states should be equal to
the number of occupied states in the first step.
%
% jgh
%%%%%%%%%%

\faqquestion{dumpdensity}
Is there any way to force CPMD to dump DENSITY files every N
steps of molecular dynamics run instead (or except) of the end of
the job?

\faqanswer
Short of modifying the source code, you could set the parameter
\refkeyword{RESTFILE} to a large number and than have CPMD
write a restart file every N steps via the \refkeyword{STORE} keyword.
Now you rename each restart in turn from RESTART.\# to RESTART and do
a single step calculation using the \refkeyword{RESTART} keyword without
the LATEST modifier which will write the DENSITY file (or run
a \refkeyword{PROPERTIES} job using \refkeyword{CUBEFILE} DENSITY to get
the cube file directly).

\faqquestion{bandstruct}
How do I calculate a Band structure with CPMD?
   To calculate a band structure with CPMD,
You first calculate the correct density for your system with
a Monkhorst-Pack Mesh.

\faqanswer
Then you use:
OPTIMIZE WAVEFUNCTIONS
with MAXSTEP 1 (no self-consistency)
and RESTART DENSITY.

In the section KPOINTS, you should use for instance a bcc:
\begin{verbatim}
   KPOINTS BANDS
        51   0   0   0      0   0   1             Gamma to H
        51   0   0   1      0  .5  .5             H to N
        51   0  .5  .5     .5  .5  .5             N to P
        51  .5  .5  .5      0   0   0             P to Gamma
        51   0   0   0     .5  .5   0             Gamma to N
        51   0   0   1     .5  .5  .5             H to P
        0    0 0 0  0 0 0
\end{verbatim}

You say that you want 51 points from (0,0,0) and (0,0,1) and
so on. The last line with many zeros is to stop.

If the memory of your computer is not enough, you can add in
the line KPOINTS the option BLOCK=50 that means you want to
have only 50 kpoints in memory. This options worked some time ago.
%tdeutsch

\faqquestion{velocities}
   I've been recently trying to use the VELOCITIES keyword in a molecular
dynamics run. I want to collide fast atoms against surfaces. Despite the
code seems to read the input velocities properly, when the run starts the
initial velocities are always the same (apparently coming from a
thermal distribution), no matter what is the velocity you specify for the
                incoming atom. I'm not using QUENCH IONS, so I don't understand why the
input initial velocities are not considered in the calculation.

\faqanswer
There is no straightforward way in CPMD to achieve what you
want. I suggest to follow this procedure:\\

1) Run a single step of MD with the following set up
\begin{verbatim}
MOLECULAR DYNAMICS
MAXSTEP
 1
RESTART WAVEFUNCTION COORDINATES
TEMPERATURE
  300   <- or whatever your surface should be
\end{verbatim}

This generates RESTART and GEOMETRY files.
Now edit the GEOMETRY file to change the velocities
of the particles according to your experiment.
Now restart the MD with the options
\begin{verbatim}
MOLECULAR DYNAMICS
MAXSTEP
 1000
RESTART WAVEFUNCTION COORDINATES VELOCITIES GEOFILE
QUENCH ELECTRONS
\end{verbatim}
The effect of this is:
IONIC coordinates and velocities are read from GEOMETRY,
ELECTRON wavefunctions and velocities are read from RESTART,
ELECTRON velocities are set to zero.
% jgh
%%%%%%%%%%

\faqquestion{gaussianbasis}
I want to to run CPMD with basis sets equivalent to Gaussian
6-31+G(d) and 6-311+G(2d,p). How do I set up the \&BASIS section?

\faqanswer
You should be able to construct
inputs from the description in this manual (see section~\ref{input:basis}).

Please note, that the basis set generated from the \&BASIS
section is used in CPMD for two purposes:
\begin{enumerate}
\item Analyzing orbitals:\\
   We usually use the atomic pseudo-wavefunctions to analyze
   the orbitals from CPMD. The 6-31G type Gaussian basis
   sets are for all electron calculations. Don't expect
   very good results when analyzing wavefunctions from a
   pseudopotential calculation.
\item Generating orbitals for an initial guess:\\
   By default we use a Slater minimal basis. In most
   cases the effort to produce a better initial guess
   using ``better'' wavefunctions does not pay off.
\end{enumerate}
%
% xc_driver
%%%%%%%%%%

\faqquestion{newdft}
How do I add support for a new functional?

\faqanswer
The exchange-correlation functionals and scratch variables
are set and handled in the module
cp\_xc\_utils.mod.F90. The individual functionals are grouped in modules
(cp\_gga\_exchange\_utils.mod.F90 etc.{}) which are in turn used in
cp\_xc\_utils.mod.F90. If you wish to implement higher-order analytical derivatives
for LR-TDDFT, cp\_dxc\_utils.mod.F90, and the cp\_dgga\dots routines may be of interest. \\
The inclusion of hybrids will require additional modifications to dftin\_utils.mod.F90 and
cpfunc\_func\_init (in cp\_xc\_utils.mod.F90). The variables for
range separation are currently still found in func.mod.F90 and the range separated exact exchange
is handled in initclust\_utils.mod.F90.
%
% Minnesota
%%%%%%%%%%

\faqquestion{minnesotaconv}
I am trying to use the Minnesota functionals (M06, M11, \dots), but the results I get are most erratic,
or they require very high values for the \refkeyword{CUTOFF}.
What is wrong?

\faqanswer
The Minnesota functionals require unusually dense integration grids. Simply increasing the cutoff will
introduce numerical noise and make the wavefunction optimisation unstable. This issue can be overcome by
increasing the \refkeyword{DUAL} up to 8-12. A detailled assessment of these effects along with recommendations
for appropriate values can be found in the following reference: M.P. Bircher, P. Lopez-Tarifa and U. Rothlisberger:
\emph{J.\ Chem.\ Theory Comput.}, \textbf{15} (1), 557 (2019). DOI: 10.1021/acs.jctc.8b00897

\faqquestion{minnesotacomp}
Results obtained from the Minnesota functionals (M05, M06, M08, M11) seem well converged, but exhibit significant differences with
respect to Gaussian bases. Is there something wrong?

\faqanswer
Due to their highly flexible functional form, the Minnesota functionals are extremely sensitive to the
choice and flexibility of the underlying basis set. Since Gaussian functions always impose a certain rigidity,
results obtained in Gaussian basis sets may occasionally differ by more than just chemical accuracy from the
plane wave results. This effect is not limited to plane waves, it can also be observed - to a lesser extent -
when comparing plane waves and Slater functions, and can be rather pronounced when comparing Gaussian and Slater functions.
A detailled study of basis set effects is provided in the following article: M.P. Bircher, P. Lopez-Tarifa and U. Rothlisberger:
\emph{J.\ Chem.\ Theory Comput.}, \textbf{15} (1), 557 (2019). DOI: 10.1021/acs.jctc.8b00897
%
% Minnesota
%%%%%%%%%%

\faqquestion{hfxslow}
I am running a calculation with exact exchange, but it is exceedingly slow compared to a GGA or MGGA run.
Can this issue be resolved?

\faqanswer
The implementation of exact (or Hartree-Fock) exchange in CPMD is highly efficient and parallel. If the number of
available nodes is limited, increasing the value in \refspekeyword{HFX\_BLOCK\_SIZE}{HFX BLOCK SIZE} can substantially speed up
the calculation. In the presence of ample computational resources, the calculation can be sped up by using
\refspekeyword{CP\_GROUPS}{CP GROUPS}.

In isolated systems, the calculation can be sped up by up to an additional order of magnitude by using the coordinate-scaling
scheme by Bircher and Rothlisberger. For details, see \refkeyword{SCEX}.


%
% AK
%%%%%%%%%%

\cleardoublepage
%---------------------------------------------------------------------
%

\section*{References}
%With section* by default, no entry in the table of contents
\addcontentsline{toc}{section}{References}
%Rename Reference name
\renewcommand{\refname}{}

\begin{thebibliography}{999}

%The empty line is needed after each bibitem for hyperref

\bibitem{codata2006}
  P.J.~Mohr, B.N.~Taylor, and D.B.~Newell,
  {\em The 2006 CODATA Recommended Values of the 
  Fundamental Physical Constants},
  Web Version 5.1, 2007.

\bibitem{Allen87}
    M.P.~Allen and D.J.~Tildesley,
    {\em Computer Simulations of Liquids},
    Clarendon Press, Oxford, 1987.

\bibitem{CP85}
    R.~Car and M.~Parrinello,
    Phys.~Rev.~Lett. {\bf 55}, 2471 (1985).

\bibitem{Vanderbilt}
    D.~Vanderbilt, Phys.~Rev.~B {\bf 41}, 7892 (1990).

\bibitem{Galli91} G.~Galli and M.~Parrinello, 
{\it Computer Simulation in Materials Science}, in {\em Proc.\ NATO ASI}, Eds.\ M.~Meyer and V.~Pontikis, 
    Kluwer, 1991.

\bibitem{Nose84}
    S.~Nos\'e, J.~Chem.~Phys. {\bf 81}, 511 (1984),
    Mol.~Phys. {\bf 52}, 255 (1984).

\bibitem{Hoover85}
    W.~G.~Hoover, Phys.~Rev.~A {\bf 31}, 1695 (1985).

\bibitem{KS}
    W.~Kohn and L.~J.~Sham, Phys.~Rev. {\bf 140}, A1133, (1965).

\bibitem{Marx94}
    D. Marx and M. Parrinello,
    Z.~Phys.~B (Rapid Note) {\bf 95}, 143 (1994).

\bibitem{Marx96}
    D. Marx and M. Parrinello,
    J.~Chem.~Phys. {\bf 104}, 4077 (1996).

\bibitem{parrah}
    M.~Parrinello and A. Rahman,
    J.~Appl.~Phys. {\bf 52}, 7182 (1981).

\bibitem{parrah2}
    M.~Parrinello and A. Rahman,
    Phys.~Rev.~Lett. {\bf 45}, 1196 (1980).

\bibitem{goe_pp}
    S.~Goedecker, M.~Teter, and J.~Hutter, Phys.~Rev.~B {\bf 54}, 1703 (1996).\\
    C.~Hartwigsen, S.~Goedecker, and J.~Hutter, Phys.~Rev.~B {\bf 58}, 3641 (1998).

\bibitem{Alavi94}
    A.~Alavi, J.~Kohanoff, M.~Parrinello, and D.~Frenkel,
    Phys.~Rev.~Lett. {\bf 73}, 2599 (1994).

\bibitem{Tuckerman96}
    M.~E. Tuckerman, D. Marx, M.~L. Klein, and M. Parrinello,
    J.~Chem.~Phys. {\bf 104}, 5579 (1996).

\bibitem{egoqmmm} M.~Eichinger, P.~Tavan, J.~Hutter, and M.~Parrinello,
    J.~Chem.~Phys. {\bf 110}, 10452 (1999).

\bibitem{LinRes}
    H.~B.~Callen and R.~F.~Greene, Phys.~Rev. {\bf 86}, 702 (1952).\\
    R.~Kubo, J.~Phys.~Soc.~Japan {\bf 12}, 570 (1957). 

\bibitem{Marzari97}
    N.~Marzari and D.~Vanderbilt,
    Phys.~Rev.~B {\bf 56}, 12847 (1997).

\bibitem{qmmm02}
    A.~Laio, J.~VandeVondele, and U.~R\"othlisberger
    J.~Chem.~Phys. {\bf 116}, 6941 (2002).

\bibitem{qmmm03}
    A.~Laio, J.~VandeVondele, and U.~R\"othlisberger
    J.~Phys.~Chem.~B {\bf 106}, 7300 (2002).

\bibitem{qmmm04}
    M.~Boero, {\it Reactive Simulations for Biochemical Processes}
    in {\em Atomic-Scale Modeling of Nanosystems and
    Nanostructured materials} p.\~81-98, Springer,
    Berlin Heidelberg, 2010. ISBN 978-3-642-04650-6

\bibitem{apdsmp} A. Putrino, D. Sebastiani, and M. Parrinello,
     J.~Chem.~Phys. {\bf 113}, 7102 (2000).

\bibitem{apmp} A. Putrino and M. Parrinello,
     Phys.~Rev.~Lett. {\bf 88}, 176401 (2002).

\bibitem{dsmp} D. Sebastiani and M. Parrinello,
     J.~Phys.~Chem.~A {\bf 105}, 1951 (2001).

\bibitem{mimp} M. Iannuzzi and M. Parrinello,
      Phys.~Rev.~B {\bf 64}, 233104 (2002).

\bibitem{LSCAL}
    S.~R. Billeter, A.~Curioni, and W.~Andreoni,
    Comput.~Mat.~Sci. {\bf 27}, 437 (2003).

\bibitem{alaio} A. Laio and M. Parrinello,
      Proc.~Natl~ Acad.~Sci.~USA {\bf 20}, 12562 (2002).

\bibitem{iannuzzi}  M. Iannuzzi, A. Laio, and M. Parrinello,
      Phys.~Rev.~Lett. {\bf 90}, 238302 (2003).

\bibitem{Hockney70}
    R.~W.~Hockney, Methods Comput.~Phys. {\bf 9}, 136 (1970).

\bibitem{Elstner}
   M. Elstner, P. Hobza, T. Frauneheim, S. Suhai, and
   E. Kaxiras,  J.~Chem.~Phys. {\bf 114}, 5149 (2001).

\bibitem{Frank98}
    I. Frank, J. Hutter, D. Marx, and M. Parrinello,
    J.~Chem.~Phys. {\bf 108}, 4060 (1998).

\bibitem{xcder}
    D. Egli and S.~R. Billeter,
    Phys.~Rev.~B {\bf 69}, 115106 (2004).

\bibitem{gmxqmmm} P.~K.~Biswas, V.~Gogonea,
    J.~Chem.~Phys. {\bf 123},164114 (2005).
    The corresponding modifications to the Gromacs code are available at
\htref{http://comppsi.csuohio.edu/groups/}{http://comppsi.csuohio.edu/groups/}

\bibitem{bgl} J. Hutter and A. Curioni,
     ChemPhysChem {\bf 6}, 1788-1793 (2005).

\bibitem{mixed} J. Hutter and A. Curioni,
     Parallel Computing {\bf 31}, 1 (2005).

\bibitem{shock}
    E.J.~Reed, L.E.~Fried and J.D.~Joannopoulos,
    Phys.~Rev.~Lett. {\bf 90}, 235503 (2003).

\bibitem{GrimmJCP2003}
    S. Grimm, C. Nonnenberg, and I. Frank, J.~Chem.~Phys. {\bf 119}, 11574 (2003).

\bibitem{Grimme06}
    S.~Grimme, J.~Comp.~Chem. {\bf 27}, 1787 (2006).

\bibitem{GrimmeD3}
    S. Grimme, J. Antony, S. Ehrlich, and S. Krieg, J.~ Chem.~ Phys. {\bf 132}, 154104 (2010).

\bibitem{be_cur}
C. Bekas and A. Curioni
Comp.~Phys.~Comm.{\bf 181}, 1057 (2010)

\bibitem{ober}
H ~Oberhofer and J.~Blumberger,
J.~Chem.~Phys. {\bf 131}, 064101 (2009)

\bibitem{ceriotti}
M.~Ceriotti, G.~Bussi and M.~Parrinello,
Phys.~Rev.~Lett.{\bf 102} , 02061 (2009)

\bibitem{Ceriotti10}
    Ceriotti, M. and Bussi, G. and Parrinello, M.,
    J.~Chem.~Th.~Comput. {\bf 6}, 1170 (2010).

\bibitem{HSE06a}
J.~Heyd, G.~E.~Scuseria, and M.~Ernzerhof,
J.~Chem.~Phys. {\bf 118}, 8207 (2003);
J.~Chem.~Phys. {\bf 124}, 21990624(E) (2006).

\bibitem{HSE06b}
A.~V.~Krukau, O.~A.~Vydrov, A.~F.~Izmaylov, and G.~E.~Scuseria,
J.~Chem.~Phys. {\bf 125}, 224106 (2006).

\bibitem{taver}
I. ~Tavernelli, B.~Curchod and U.~R\"othlisberger
J.~Chem.~Phys. {\bf 131}, 196101 (2009).

\bibitem{taver1} 
I.~Tavernelli
Phys.~Rev.~B  {\bf 73}, 094204,(2006).

\bibitem{Kerker}
    G.~P.~Kerker, Phys.~Rev.~B {\bf 23}, 3082 (1981).

\bibitem{Eijnden06}
    L.~Maragliano, A.~Fischer, E.~Vanden-Eijnden, and G.~Ciccotti, 
    J.~Chem.~Phys. {\bf 125}, 024106 (2006).

\bibitem{taver_eh}
    I.~Tavernelli, U.F.~Rohrig, U.~R\"othlisberger
    Mol.~Phys. {\bf 103}, 963 (2005).

\bibitem{acm0}
    M. Ernzerhof, {\it Density Functionals: Theory and Applications},
    in {\em Lecture Notes in Physics}, vol. 500, Eds. D.~P. Joubert,
    Springer--Verlag, Berlin, 1998.

\bibitem{adamo2000}
    C. Adamo, A. di Matteo, and V. Barone, {\it From Classical Density
    Functionals to Adiabatic Connection Methods. The State of the
    Art.} in {\em Advances in Quantum Chemistry}, Vol. 36, Academic Press
    (2000).

\bibitem{acm1}
    A.~D. Becke, J.~Chem.~Phys. {\bf 104} 1040 (1996).

\bibitem{acm3}
    A.~D. Becke, J.~Chem.~Phys. {\bf 98} 5648 (1993).

\bibitem{Becke88}
    A.~D.~Becke,
    Phys.~Rev.~A {\bf 38}, 3098 (1988).

\bibitem{Berendsen84}
    H.~J.~C.~Berendsen, J.~P.~M.~Postma, W.~F.~van~Gunsteren, 
    A.~DiNola, and J.~R.~Haak,
    J.~Chem.~Phys. {\bf 81}, 3684 (1984).

\bibitem{Fletcher80}
    R.~Fletcher,
    {\em Practical Methods of Optimizations}, vol. 1, 
    Wiley,: New York, 1980.

\bibitem{Frenzel2017}
    J.~Frenzel, B.~Meyer, D.~Marx, 
    {\em Bicanonical Ab Initio Molecular Dynamics for Open Systems}
    J. Chem. Theory Comput., Just Accepted Manuscript 
    (2017), 
    DOI: 10.1021/acs.jctc.7b00263 .

\bibitem{Johnson88}
    D.~D. Johnson, Phys.~Rev.~B {\bf 38}, 12807 (1988).

\bibitem{Cao93}
  (a) J. Cao and G.~A. Voth,
    J.~Chem.~Phys. {\bf 99}, 10070 (1993);
  (b) J. Cao and G.~A. Voth,
    J.~Chem.~Phys. {\bf 100}, 5106 (1994).

\bibitem{Martyna96}
  (a) G.~J. Martyna,
    J.~Chem.~Phys. {\bf 104}, 2018 (1996);
  (b) J. Cao and G.~J. Martyna,
    J.~Chem.~Phys. {\bf 104}, 2028 (1996).

\bibitem{aicmd}
    D. Marx, M.~E. Tuckerman, and G.~J. Martyna,
    Comput.~Phys.~Commun. {\bf 118}, 166 (1999).

\bibitem{Hirshfeld77}
    F.~L.~Hirshfeld,
    Theoret.~Chim.~Acta {\bf 44}, 129 (1977).

\bibitem{Cox84}
    S.~R.~Cox and P.~A.~~Kollman,
    J.~Comput.~Chem. {\bf 5}, 129 (1984).

\bibitem{solve} M. Boero, M. Parrinello, K. Terakura, T. Ikeshoji, and C.~C. Liew,
     Phys.~Rev.~Lett. {\bf 90}, 226403 (2003).

\bibitem{solve2} M. Boero, J.~Phys.~Chem.~A {\bf 111}, 12248 (2007).

\bibitem{bernasconi95}
    M. Bernasconi, G.~L. Chiarotti, P. Focher, S. Scandalo, E. Tosatti,
    and M. Parrinello, J.~Phys.~Chem.~Solids, {\bf 56} 501 (1995).

\bibitem{xray} M. Cavalleri, M. Odelius, A. Nilsson and L. G. Pettersson,
     J.~Chem.~Phys. {\bf 121}, 10065 (2004).

\bibitem{nonadiabatic}
    S.~R. Billeter and A.~Curioni,
    J.~Chem.~Phys. {\bf 122}, 034105 (2005).

\bibitem{davidson75} E.~R. Davidson,
     J.~Comput.~Phys. {\bf 17}, 87 (1975).

\bibitem{Fosdick66}
    L.~D. Fosdick and H.~F. Jordan,
    Phys.~Rev. {\bf 143}, 58 (1966).

\bibitem{vdb_berry}
     R.~Resta, Ferroelectrics {\bf 136}, 51 (1992). \\
     R.~D. King--Smith and D. Vanderbilt, Phys.~Rev.~B {\bf 47}, 1651 (1993). \\
     R.~Resta, Europhys.~ Lett. {\bf 22}, 133 (1993).

\bibitem{resta}
     R.~Resta, Rev.~Mod.~Phys. {\bf 66}, 899 (1994). \\
     R.~Resta, Phys.~Rev.~Lett. {\bf 80}, 1800 (1998). \\
     R.~Resta, Phys.~Rev.~Lett. {\bf 82}, 370 (1999).

\bibitem{PSil99}
    P.~L.~Silvestrelli, Phys.~Rev.~B {\bf 59}, 9703 (1999).

\bibitem{berghold} G. Berghold, C.~J. Mundy, A.~H. Romero, J. Hutter, and M. Parrinello,
      Phys.~Rev.~B {\bf 61}, 10040 (2000).

\bibitem{distrib.lanczos.07}
    C.~Bekas, A.~Curioni and W.~Andreoni,
    Parallel Computing {\bf 34}, 441 (2008). 

\bibitem{tddft_all}
     E. Runge and E.~K.~U. Gross, Phys.~Rev.~Lett. {\bf 52}, 997 (1984). \\
     M.~E. Casida, in {\em Recent Advances in Density Functional Methods}, Vol. 1,
     edited by D.~P. Chong (World Scientific, Singapore, 1995). \\
     M.~E. Casida, in {\em Recent Developments and Applications of Modern
     Density Functional Theory}, Theoretical and Computational Chemistry, Vol. 4,
     edited by J.~M. Seminario (Elsevier, Amsterdam, 1996). \\
     C. Jamorski, M.~E. Casida, and D.~R. Salahub, J.~Chem.~Phys. {\bf 104}, 5134 (1996). \\
     S.~J.~A. van Gisbergen, J.~G. Snijders, and E.~J. Baerends, J.~Chem.~Phys. {\bf 103}, 9347 (1996). \\
     R. Bauernschmitt and R. Ahlrichs, Chem.~Phys.~Lett. {\bf 256}, 454 (1996). \\
     K.~B. Wiberg, R.~E. Stratmann, and M.~J. Frisch, Chem.~Phys.~Lett. {\bf 297}, 60 (1998). \\
     R.~E. Stratmann, G.~E. Scuseria, and M.~J. Frisch, J.~Chem.~Phys. {\bf 109}, 8218 (1998). \\
     A.~G\"orling, H.~H. Heinze, S.~Ph. Ruzankin, M. Staufer, and N. R\"osch,
     J.~Chem.~Phys. {\bf 110}, 2785 (1999). \\
     R. Bauernschmitt, M. H\"aser, O. Treutler, and R. Ahlrichs, Chem.~Phys.~Lett. {\bf 264}, 573 (1997).

\bibitem{tddft_pw} J.~Hutter,
    J.~Chem.~Phys. {\bf 118}, 3928 (2003).

\bibitem{Becke90}
    A.~D.~Becke and K.~E.~Edgecombe,
    J.~Chem.~Phys. {\bf 92}, 5397 (1990).

\bibitem{silsav}
    B.~Silvi and A.~Savin,
    Nature {\bf 371}, 683 (1994).

\bibitem{marx-savin-97}
    D. Marx and A. Savin,
    Angew.~Chem.~Int.~Ed.~Engl. {\bf 36}, 2077 (1997).

\bibitem{homeofelf}
    Check the ELF homepage
    \htref{http://www.cpfs.mpg.de/ELF/}%
    {http://www.cpfs.mpg.de/ELF/}
    for lots of useful information in particular on how ELF
    should be interpreted.

\bibitem{Kohut96}
    M.~Kohut and A.~Savin,
    Int.~J.~Quant.~Chem. {\bf 60}, 875--882 (1996)

\bibitem{tavernelli2010}
    I.~Tavernelli, B.~F.~E.~Curchod, and U.~Rothlisberger,
    Phys.~Rev.~ A. {\bf 81}, 052508 (2010).

\bibitem{surfhop} N.~L.~Doltsinis, D.~Marx,
    Phys.~Rev.~Lett. {\bf 88}, 166402 (2002).

\bibitem{PSil}
    P.~L.~Silvestrelli, A.~Alavi, M.~Parrinello, and D.~Frenkel,
    Phys.~Rev.~Lett. {\bf 77}, 3149 (1996).

\bibitem{mbaops}
    M.~Boero, A.~Oshiyama, P.~L.~Silvestrelli, and K.~Murakami,
    Appl.~Phys.~Lett. {\bf 86}, 201910 (2005).

\bibitem{lsets}
    S.~R. Billeter and D.~Egli,
    J.~Chem.~Phys. {\bf 125}, 224103 (2006).

\bibitem{Csaszar84}
    P.~Cs\'asz\'ar and P.~Pulay,
    J.~Mol.~Struct. {\bf 114} 31 (1984).

\bibitem{Rossi2014}
    M.~Rossi, M.~Ceriotti, and D.~E.~Manolopoulos, 
    J.~Chem.~Phys. {\bf 140}, 234116 (2014).

\bibitem{Perdew92}
    J.~P.~Perdew, J.~A.~Chevary, S.~H.~Vosko, K.~A.~Jackson,
    M.~R.~Pederson, D.~J.~Singh, and C.~Fiolhais, 
    Phys.~Rev.~B {\bf 46}, 6671 (1992);
    {\it Erratum} Phys.~Rev.~B {\bf 48}, 4978 (1993).

\bibitem{Perdew96}
    J.~P.~Perdew, K.~Burke, and M.~Ernzerhof,  
    Phys.~Rev.~Lett. {\bf 77}, 3865 (1996).

\bibitem{Zhang98}
    Y.~Zhang and W.~Yang, Phys.~Rev.~Lett. {\bf 80}, 890 (1998).

\bibitem{Handy98}
    A.D. Boese, N.L. Doltsinis, N.C. Handy, and M. Sprik,
    J.~Chem.~Phys. {\bf 112}, 1670 (2000).

\bibitem{Optx}
    N.~C.~Handy and A.~J.~Cohen, 
    J.~Chem.~Phys. {\bf 116}, 5411 (2002).

\bibitem{Perdew07}
    J.~P. Perdew, A. Ruzsinszky, G.~I. Csonka, O.~A. Vydrov, G.~E. Scuseria,
    L.~A. Constantin, X. Zhou, and K. Burke,
    Phys.~Rev.~Lett. {\bf 100}, 136406 (2008).

\bibitem{Perdew86}
    J.~P.~Perdew,
    Phys.~Rev.~B {\bf 33}, 8822 (1986);
    {\it Erratum} Phys.~Rev.~B {\bf 34}, 7406 (1986).

\bibitem{Lee88}
    C.~Lee, W.~Yang and R.~G.~Parr,
    Phys.~Rev.~B {\bf 37}, 785 (1988).

\bibitem{Tuckerman94}
    M.~E.~Tuckerman and M.~Parrinello, J.~Chem.~Phys.
    {\bf 101}, 1302 (1994); {\it ibid.} {\bf 101}, 1316 (1994).

\bibitem{Fischer92}
    T.~H.~Fischer and J.~Alml\"of,
    J.~Phys.~Chem. {\bf 96}, 9768 (1992).

\bibitem{Schlegel84}
    H.~B.~Schlegel, Theo.~Chim.~Acta {\bf 66}, 333 (1984).

\bibitem{ego1} M.~Eichinger, H.~Heller, and H.~Grubm\"uller,
   in
   {\em Workshop on Molecular Dynamics on Parallel Computers},
   p. 154-174, Singapore 912805, World Scientific, (2000).

\bibitem{ego2} M.~Eichinger, H.~Grubm\"uller, H.~Heller, and P.~Tavan,
    J.~Comp.~Chem. {\bf 18}, 1729 (1997).

\bibitem{gmx3}
    E.~Lindahl, B.~Hess, and D. van der Spoel,
    J.~Mol.~Mod. {\bf 7}, 306 (2001).

\bibitem{Liu89}
    D.~C. Liu and J.~Nocedal,
    Math.~Prog. {\bf 45}, 503 (1989).

\bibitem{Perdew81}
    J.~P.~Perdew and A.~Zunger,
    Phys.~Rev.~B {\bf 23}, 5048 (1981).

\bibitem{Vosko80}
    S.~H.~Vosko, L.~Wilk and M.~Nusair,
    Can.~J.~Phys. {\bf 58}, 1200 (1980).

\bibitem{Perdew91}
    J.~P.~Perdew and Y.~Wang, 
    Phys.~Rev.~B {\bf 45}, 13244 (1991).

\bibitem{Ceperley80}
    D.~M.~Ceperley and B.~J.~Alder,
    Phys.~Rev.~Lett.{\bf 45}, 566 (1980).

\bibitem{Hutter94b}
    J.~Hutter, M.~Parrinello, and S.~Vogel,
    J.~Chem.~Phys. {\bf 101}, 3862 (1994).

\bibitem{ABDY}
    I. Tavernelli,
    Phys. Rev. A {\bf 87}, 042501 (2013).

\bibitem{HutterIP}
    J.~Hutter, M.~E. Tuckerman, and M.~Parrinello.
    J.~Chem.~Phys. {\bf 102}, 859 (1995).

\bibitem{cafes02}
    J.~VandeVondele, and U.~R\"othlisberger
    J.~Phys.~Chem.~B {\bf 106}, 203 (2002).

\bibitem{Hutter94a}
    J.~Hutter, H.~P.~L\"uthi, and M.~Parrinello,
    Comput.~Mat.~Sci. {\bf 2}, 244 (1994).

\bibitem{Martyna94}
    G.~J.~Martyna, D.~J.~Tobias, and M.~L.~Klein,
    J.~Chem.~Phys. {\bf 101}, 4177 (1994).

\bibitem{tps}
     C. Dellago, P.~G. Bolhuis, F.~S. Csajka, D.~J. Chandler, J.~Chem.~Phys. {\bf 108}, 1964 (1998). \\
     P.~G. Bolhuis, C. Dellago, D.~J. Chandler, Faraday Discuss. {\bf 110}, 42 (1998).

\bibitem{Martyna99} G.J. Martyna and M. E. Tuckerman,
    J.~Chem.~Phys. {\bf 110}, 2810 (1999).

\bibitem{Davidson67}
    E.~R.~Davidson, J.~Chem.~Phys. {\bf 46}, 3320 (1967);
    K.~R.~Roby, Mol.~Phys. {\bf 27}, 81 (1974);
    R.~Heinzmann and R.~Ahlrichs, Theoret.~Chim.~Acta (Berl.)
    {\bf 42}, 33 (1976);
    C.~Ehrhardt and R.~Ahlrichs, Theoret.~Chim.~Acta (Berl.)
    {\bf 68}, 231 (1985).

\bibitem{Powell71}
    M.~J.~D. Powell, Math.~Prog. {\bf 1}, 26 (1971).

\bibitem{Banerjee85}
    A.~Banerjee, N.~Adams, J.~Simons, and R.~Shepard,
    J.~Phys.~Chem. {\bf 89}, 52 (1985).

\bibitem{Turner99}
    A.~J. Turner, V.~Moliner and I.~H. Williams,
    Phys.~Chem.~Chem.~Phys. {\bf 1}, 1323 (1999).

\bibitem{Craig2004}
    R. Craig and D.~E.~Manolopoulos, 
    J.~Chem.~Phys. {\bf 121}, 3368 (2004).

\bibitem{gromos96}
    W.~F.~van~Gunsteren, S.~R.~Billeter, A.~A.~Eising, P.~H.~H\"unenberger,
    P.~Kr\"uger, A.~E.~Mark, W.~R.~P.~Scott, I.~G.~Tironi,
    Biomolecular Simulation: The GROMOS96 Manual and User Guide;
    Vdf Hochschulverlag AG an der ETH~Z\"urich: Z\"urich, 1996.

\bibitem{amber7}
    D.~A.~Case, D.~A.~Pearlman, J.~W.~Caldwell, T.~E.~Cheatham~III, J.~Wang,
    W.~S.~Ross, C.~L.~Simmerling, T.~A.~Darden, K.~M.~Merz, R.~V.~Stanton,
    A.~L.~Cheng, J.~J.~Vincent, M.~Crowley, V.~Tsui, H.~Gohlke, R.~J.~Radmer,
    Y.~Duan, J.~Pitera, I.~Massova, G.~L.~Seibel, U.~C.~Singh, P.~K.~Weiner,
    and P.~A.~Kollman, AMBER~7 (2002), University of California, San Francisco.

\bibitem{qmmm05}
    F.~J.~Momay,
    J.~Phys.~Chem. {\bf 82}, 592 (1978).

\bibitem{qmmm06}
    C.~I.~Bayly, P.~Cieplak, W.~D.~Cornell and P.~A.~Kollman,
    J.~Phys.~Chem. {\bf 97}, 10269 (1993).

\bibitem{Slater51}
    J.~C.~Slater, Phys.~Rev. {\bf 81}, 385 (1951).

\bibitem{Laasonen93}
    K.~Laasonen, M.~Sprik, M.~Parrinello and R.~Car,
    J.~Chem.~Phys. {\bf 99}, 9081 (1993).

\bibitem{SOC} F. Franco de Carvalho, B.F.E. Curchod, T. Penfold, I. Tavernelli,
     J. Chem. Phys.  {\bf 140}, 144103 (2014).

\bibitem{SSIC} J. VandeVondele and M. Sprik,
     Phys.~Chem.~Chem.~Phys. {\bf 7}, 1363 (2005).

\bibitem{dna_sic} F.~L. Gervasio, M. Boero, and M. Parrinello,
     Angew.~Chem.~Int.~Ed. {\bf 45}, 5606 (2006).

\bibitem{TDDFT-SH}
    E.~Tapavicza, I.~Tavernelli, and U.~Rothlisberger,
    Phys.~Rev.~Lett. {\bf 98}, 023001 (2007).\\
    I.~Tavernelli, E.~Tapavicza, and U.~Rothlisberger, 
    J.~Mol.~ Struct.~: THEOCHEM {\bf 914}, 22 (2009).

\bibitem{Landman} R.~N.~Barnett and U.~Landman,
    Phys.~Rev.~B {\bf 48}, 2081 (1993).

\bibitem{knoll-marx-00} 
    L. Knoll and D. Marx,
    Eur.~Phys.~J.~D {\bf 10}, 353 (2000).

\bibitem{Liberatore2018} 
   E.~Liberatore, R.~Meli, and U.~Rothlisberger,
   J.~Chem.~Theory~Comput., {\bf 14}, 2834 (2018).

\bibitem{mooij:99}
    W.~T.~M. Mooij, F.~B. van Duijneveldt,J.~G.~C.~M. van Duijneveldt-van de Rijdt,
    and B.~P. van Eijck, J.~Phys.~Chem~A {\bf 103}, 9872 (1999).

\bibitem{williams-vdw:06}
    R.~W.~Williams and D.~Malhotra, Chem.~Phys. {\bf 327}, 54 (2006).

\bibitem{KB}
    L.~Kleinman and D.~M.~Bylander,
    Phys.~Rev.~Lett. {\bf 48}, 1425 (1982).

\bibitem{NLCC}
    S.~G.~Louie, S.~Froyen, and M.~L.~Cohen,
    Phys.~Rev.~B {\bf 26}, 1738 (1982). 

\bibitem{BHS}
    G.~B.~Bachelet, D.~R.~Hamann and M.Schl\"uter,
    Phys.~Rev.~B {\bf 26}, 4199 (1982).

\bibitem{SGS}
    X.~Gonze, R.~Stumpf and M.~Scheffler,
    Phys.~Rev.~B {\bf 44}, 8503 (1991);
    R.~Stumpf and M.~Scheffler,
    Research Report of the Fritz-Haber-Institut (1990).

\bibitem{TM}
    N.~Troullier and J.~L.~Martins,
    Phys.~Rev.~B {\bf 43}, 1993 (1991).

\bibitem{GTH}
    S.~Goedecker, M.~Teter, and J.~Hutter,
    Phys.~Rev.~B {\bf 54}, 1703 (1996).

\bibitem{Sprik98a}
    M. Sprik, Faraday Discuss.\ {\bf 110}, 437 (1998).

\bibitem{Sprik98b} M.~Sprik and G.~Ciccotti,
    J.~Chem.~Phys. {\bf 109}, 7737 (1998).

\bibitem{cco}
    K.~Kamiya, M.~Boero, M.~Tateno, K.~Shiraishi, and A.~Oshiyama,
    J.~Am.~Chem.~Soc. {\bf 129}, 9663 (2007).

\bibitem{Frenkel02}
    D. Frenkel and B. Smit,
    {\em Understanding Molecular Simulation},
    Academic Press, San Diego, 2002.

\bibitem{ffmp} F. Filippone and M. Parrinello,
     Chem.~Phys.~Lett. {\bf 345}, 179 (2001).

\bibitem{fukui} K. Fukui, Science {\bf 217}, 747 (1982).

\bibitem{fukui2} W. Yang and R.~G. Parr,
     Proc.~Natl.~Acad.~Sci.~USA {\bf 82}, 6723 (1985).

\bibitem{fukui3} E. Chamorro, F. De Proft and P. Geerlings,
     J.~Chem.~Phys. {\bf 123}, 084104 (2005).

\bibitem{micheletti} C.~Micheletti,  A.~Laio, and M.~Parrinello,
      Phys.~Rev.~Lett. {\bf 92}, 170601 (2004).

\bibitem{gerlaio} A.~Laio and F.~L.~Gervasio,
      Rep.~Prog.~Phys. {\bf 71}, 126601 (2008).

\bibitem{michele} A.~Barducci, M.~Bonomi, and M.~Parrinello,
      Wiley Interdisciplinary Reviews - Computational Molecular Science
      {\bf 1}, 826 (2011).

\bibitem{andras} A. Stirling, M. Iannuzzi, A. Laio, and M. Parrinello,
       ChemPhysChem {\bf 5}, 1558 (2004).

\bibitem{gervasio} A. Laio, A. Rodriguez-Fortea, F.~L. Gervasio, M. Ceccarelli,
       and M. Parrinello, J.~Phys.~Chem.~B {\bf 109}, 6714 (2005).

\bibitem{iannuzzi2} M. Iannuzzi and M. Parrinello,
         Phys.~Rev.~Lett. {\bf 93}, 025901 (2004).

\bibitem{sergey} S. Churakov, M. Iannuzzi, and M. Parrinello,
       J.~Phys.~Chem.~B {\bf 108} 11567 (2004).

\bibitem{scwo} M. Boero, T. Ikeshoji, C.~C. Liew, K. Terakura and M. Parrinello,
      J.~Am.~Chem.~Soc. {\bf 126}, 6280 (2004).

\bibitem{ikeda} T. Ikeda, M. Hirata and T. Kimura,
      J.~Chem.~Phys. {\bf 122}, 244507 (2005).

\bibitem{rna} M. Boero, M. Tateno, K. Terakura, and A. Oshiyama,
     J.~Chem.~Theor.~Comput. {\bf 1}, 925 (2005).

\bibitem{Nair-jacs-08} N.~N. Nair, E. Schreiner, and D. Marx,
     J.~Am.~Chem.~Soc. {\bf 130}, 14148 (2008).

\bibitem{mb11} M.~Boero, J.~Phys.~Chem.~B {\bf 115}, 12276 (2011).

\bibitem{Raiteri05} P.~Raiteri, A.~Laio, F.~L.~Gervasio, C.~Micheletti and M.~Parrinello,
      J.~Phys.~Chem.~B {\bf 110}, 3533 (2005).

\bibitem{Nair-inside-08} N.~N.~Nair, E.~Schreiner and D.~Marx
      inSiDE {\bf 6}, 30 (2008).\\
      http://inside.hlrs.de/htm/Edition$\_02\_08$/article$\_09$.html

\bibitem{chandler} C. Dellago, P.~G. Bolhuis, F.~S.Csajka, and D. Chandler,
      J.~Chem.~Phys. {\bf 108} 1964 (1998).

\bibitem{Wu05}
    Q.~Wu and T.~{Van~Voorhis},
    Phys.~Rev.~A     {\bf 72}, 024502 (2005).

\bibitem{Wu06jcp}
    Q.~Wu and T.~{Van~Voorhis},
    J.~Chem.~Phys.  {\bf 125}, 164105 (2006).

\bibitem{Wu06jctc}
    Q.~Wu and T.~{Van~Voorhis},
    J.~Chem.~Theory.~Comput. {\bf 2}, 765 (2006).

\bibitem{Wu06jpca}
    Q.~Wu and T.~{Van~Voorhis},
    J.~Phys.~Chem.~A  {\bf 110}, 9212 (2006).

\bibitem{Wu07}
    Q.~Wu and T.~{Van~Voorhis},
    J.~Chem.~Phys.    {\bf 127}, 164119 (2007).

\bibitem{Oberhofer09}
    H.~Oberhofer and J.~Blumberger,
    J.~Chem.~Phys. {\bf 131}, 64101 (2009).

\bibitem{Oberhofer10acie}
    H.~Oberhofer and J.~Blumberger,
    Angew.~Chem.~Int.~Ed. {\bf 49}, 3631 (2010).

\bibitem{Senthilkumar03}
    Senthilkumar, K. and Grozema, F.~C. and Bickelhaupt, F.~M. and Siebbeles, L. D.~A.,
    J.~Chem.~Phys. {\bf 119}, 9809 (2003).

\bibitem{box-walls}
    M.~Boero, T.~Ikeda, E.~Ito and K.~Terakura,
    J.~Am.~Chem.~Soc. {\bf 128}, 16798 (2006)

\bibitem{opt-ecp04}
    O.~A.~v.~Lilienfeld, D.~Sebastiani, I.~Tavernelli, and U.~Rothlisberger,
    Phys.~Rev.~Lett. {\bf 93}, 153004 (2004).

\bibitem{FM-maurer}
    P. Maurer, A. Laio, H.~W. Hugosson, M.~C. Colombo, and U. R\"othlisberger,
    J.~Chem.~Theor.~Comput. {\bf 3}, 628 (2007).  

\bibitem{Blochl92}
       P. Bl{\"o}chl and M. Parrinello,
      Phys.~Rev.~B {\bf 45}, 9413 (1992).

\bibitem{psil1}
      P.~L.~Silvestrelli, Phys.~Rev.~Lett.~{\bf 100}, 053002 (2008);
       J.~Phys.~Chem.~A {\bf 113}, 5224 (2009).

\bibitem{psil2}
     A.~Ambrosetti and P.~L.~Silvestrelli, Phys.~Rev.~B {\bf 85}, 073101 (2012).

\bibitem{psil3}
     P.~L.~Silvestrelli and A.~Ambrosetti, J.~Chem.~Phys.{\bf 150}, 164109 (2019).

\bibitem{molphy}
     L.~V.~Slipchenko and M.~S.~Gordon, 
     Mol.~Phys.~{\bf 107}, 999 (2009).

\bibitem{heat1}
    T.~Luo and J.~R.~Lloyd, J.~Heat Transfer {\bf 130}, 122403 (2008).

\bibitem{heat2}
    G.~Wu and B.~Li, Phys.~Rev.~B {\bf 76}, 085424 (2007).

\bibitem{Schwoerer2013}
M.~Schw\"{o}rer, B.~Breitenfeld, P.~Tr\"{o}ster, S.~Bauer, K.~Lorenzen, P.~Tavan, and G.~Mathias,
 J.\ Chem.\ Phys., {\bf 138},  244103  (2013).

\bibitem{Schwoerer2015}
M.~Schw\"{o}rer, K.~Lorenzen, G.~Mathias, and P.~Tavan,
 J.\ Chem.\ Phys., {\bf 142}   (2015).

\bibitem{bircher_rothlisberger18b}
M.P.~Bircher and U.~Rothlisberger,
 J.\ Phys.\ Chem.\ Lett., {\bf 9} (14), 3886 (2018).

\bibitem{bircher_rothlisberger19}
M.P.~Bircher, P.~Lopez-Tarifa and U.~Rothlisberger,
 J.\ Chem.\ Theory Comput., {\bf 15} (1), 557 (2019).

%-> References correctly reordered until this point.

%\bibitem{LeSar}
%   R. Le Sar, J.~Phys.~Chem. {\bf 88}, 4272 (1984);
%   R. Le Sar, J.~Chem.~Phys. {\bf 86}, 1485 (1987).

\end{thebibliography}
%
%---------------------------------------------------------------------

%
%Index follows...
%
\cleardoublepage

%Redefine the environment
\renewenvironment{theindex}%
{\newpage%
\section*{Index}%
\addcontentsline{toc}{section}{Index}%
\begin{multicols}{2}%
\par\bigskip%
\begin{list}%
{}%Default item
{%Set up the different geometrical parameters
\setlength{\leftmargin}{3em}
\setlength{\itemindent}{0pt}
\setlength{\parsep}{0pt}
\setlength{\itemsep}{0pt}
\setlength{\itemindent}{-3em}}
}%definition of \begin{index}
{\end{list}\end{multicols}}%definition of \end{index}

\printindex

%---------------------------------------------------------------------
%
\end{document}
